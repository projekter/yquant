% BEGIN_FOLD Drawing wires
% extends the wire of register #1. Assumes a node called yquantbox is set up, and the \pgfshapeclippathhorzresult was set up appropriately for this node.
\protected\def\yquant@circuit@extendwire#1#2{%
   \begingroup%
      \ifx*#2%
         \edef\wirexpos{\the\dimen0}%
      \else%
         \pgfpointanchor{yquantbox}{#2}%
         \edef\wirexpos{\the\pgf@x}%
      \fi%
      \ifdim\wirexpos>0pt % may be negative for init gates
         \yquant@register@get@typeywire{#1}\wiretype\wireypos\wirelast%
         \edef\wirexprevpos{\expandafter\@firstoffour\wirelast}%
         \ifnum\wiretype=\yquant@register@type@none%
            % the clippings of the previous operator will for sure not be needed, but the type might be turned into an active one, so we need the last clipping.
            \yquant@register@set@lastwire{#1}{%
               {\wirexprevpos}{\wirexpos}{}%
               {\unexpanded\expandafter{\pgfshapeclippathhorzresult}}%
            }%
         \else%
            % append the previous `last' clipping to the old list and insert the new one
            \yquant@register@set@lastwire{#1}{%
               {\wirexprevpos}{\wirexpos}%
               {\unexpanded\expandafter\expandafter\expandafter{%
                   \expandafter\@thirdandfourthoffour\wirelast%
                }%
               }%
               {\unexpanded\expandafter{\pgfshapeclippathhorzresult}}%
            }%
         \fi%
      \fi%
   \endgroup%
}

% finishes the wires of all registers
\protected\def\yquant@circuit@endwires{%
   \expandafter\expandafter\expandafter\yquant@register@get@maxxrange%
      \expandafter\expandafter\expandafter\yquant@circuit@endwires@x%
      \expandafter\expandafter\expandafter1%
      \expandafter\expandafter\expandafter{\csname\yquant@prefix registers\endcsname}%
   % to have a symmetric situation, we extend again one separation at the end, unless this is supposed to be seamless and we don't have outputs (for seamless circuits with outputs, extend - since this extension will be between last register and output)
   \ifyquant@env@seamless{%
      \expandafter\unless\expandafter\ifx\csname\yquant@prefix outputs\endcsname\relax%
         \dimdef\yquant@circuit@endwires@x{\yquant@circuit@endwires@x+\yquant@config@operator@sep}%
      \fi%
   }{%
      \dimdef\yquant@circuit@endwires@x{\yquant@circuit@endwires@x+\yquant@config@operator@sep}%
   }%
   \let\yquant@circuit@endwires@finalize=\relax%
   \yquant@for \yquant@circuit@endwires@i := 1 to \csname\yquant@prefix registers\endcsname {%
      % we only extend the wire if it does not come from the outer circuit - this one would be responsible for the extension.
      \ifcsname\yquant@prefix registermap@\yquant@circuit@endwires@i\endcsname%
         \xifinlistcs\yquant@circuit@endwires@i{\yquant@prefix inonly}{%
            % however, the wire is to be discarded after this circuit
            \edef\storedleft{\the\pgf@picminx}%
            \yquant@draw@wire\yquant@circuit@endwires@i1%
            \global\pgf@picminx=\storedleft%
            \eappto\yquant@circuit@endwires@finalize{%
               \yquant@register@set@type%
                  \yquant@circuit@endwires@i\noexpand\yquant@register@type@none%
            }%
         }\relax%
      \else%
         \yquant@draw@wire\yquant@circuit@endwires@i1%
      \fi%
   }%
}

% outputs the wire according to its previous instructions and prepares for a change in the wire style
\protected\def\yquant@circuit@flushwire#1{%
   \yquant@draw@wire{#1}0%
   \begingroup%
      \yquant@register@get@lastwire{#1}\wirelast%
      \yquant@register@set@lastwire{#1}{%
         {\expandafter\@secondoffour\wirelast}{\expandafter\@secondoffour\wirelast}{}%
         {\unexpanded\expandafter\expandafter\expandafter{%
             \expandafter\@fourthoffour\wirelast%
          }}%
      }%
   \endgroup%
}
% END_FOLD

% BEGIN_FOLD Drawing control lines
% populates a drawing macro with the current control line with style #1 at position #2. Assumes a node called yquantbox is set up, and the \pgfshapeclippathvertresult was set up appropriately for this node. At the first call, \yquant@circuit@extendcontrolline@cmd must be \let to \empty and \yquant@circuit@extendcontrolline@prev to \relax.
\protected\def\yquant@circuit@extendcontrolline#1#2{%
   \begingroup%
      \tikzset{/yquant/every control line}%
      \expandafter%
   \endgroup%
   \eappto\yquant@circuit@extendcontrolline@clip{%
      \unexpanded\expandafter{\pgfshapeclippathvertresult}%
   }%
   \expandafter\@tempdima\the\pgflinewidth%
   \ifcase#1%
      % no control (or to a discarded target, which we don't do)
   \or%
      % qubit
      \pgfpointanchor{yquantbox}{center}%
      \unless\ifx\yquant@circuit@extendcontrolline@prev\relax%
         \eappto\yquant@circuit@extendcontrolline@cmd{%
            \expandafter\@secondofthree\yquant@circuit@extendcontrolline@prev%
            -- (#2,\the\pgf@y)%
         }%
      \fi%
      \yquant@circuit@extendcontrolline@store{#2}%
   \or%
      % cbit
      \pgfpointanchor{yquantbox}{center}%
      \unless\ifx\yquant@circuit@extendcontrolline@prev\relax%
         \eappto\yquant@circuit@extendcontrolline@cmd{%
            \expandafter\@firstofthree\yquant@circuit@extendcontrolline@prev%
            -- (\the\dimexpr#2-2\@tempdima\relax,\the\pgf@y)%
            \expandafter\@thirdofthree\yquant@circuit@extendcontrolline@prev%
            -- (\the\dimexpr#2+2\@tempdima\relax,\the\pgf@y)%
         }%
      \fi%
      \yquant@circuit@extendcontrolline@store{#2}%
   \or%
      % quantum-bundle (very unusual, but perhaps for transversal operations?)
      \pgfpointanchor{yquantbox}{center}%
      \unless\ifx\yquant@circuit@extendcontrolline@prev\relax%
         \eappto\yquant@circuit@extendcontrolline@cmd{%
            \expandafter\@firstofthree\yquant@circuit@extendcontrolline@prev%
            -- (\the\dimexpr#2-2\@tempdima\relax,\the\pgf@y)%
            \expandafter\@secondofthree\yquant@circuit@extendcontrolline@prev%
            -- (\the\dimexpr#2\relax,\the\pgf@y)%
            \expandafter\@thirdofthree\yquant@circuit@extendcontrolline@prev%
            -- (\the\dimexpr#2+2\@tempdima\relax,\the\pgf@y)%
         }%
      \fi%
      \yquant@circuit@extendcontrolline@store{#2}%
   \else%
      \PackageError{yquant.sty}{Invalid control line type `#1'}%
                   {yquant encountered an internal error.}%
   \fi%
}

\protected\def\yquant@circuit@extendcontrolline@store#1{%
   \edef\yquant@circuit@extendcontrolline@prev{%
      {(\the\dimexpr#1-2\@tempdima\relax,\the\pgf@y)}%
      {(\the\dimexpr#1\relax,\the\pgf@y)}%
      {(\the\dimexpr#1+2\@tempdima\relax,\the\pgf@y)}%
   }%
}

% populates a drawing macro with the multi operation connector at position #2. Assumes a node called yquantbox is set up, and the \pgfshapeclippathvertresult was set up appropriately for this node. At the first call, \yquant@circuit@extendmultiline@cmd must be \let to \empty and \yquant@circuit@extendmultiline@prev to \relax.
\protected\def\yquant@circuit@extendmultiline#1{%
   \begingroup%
      \tikzset{/yquant/every multi line}%
      \expandafter%
   \endgroup%
   \eappto\yquant@circuit@extendmultiline@clip{%
      \unexpanded\expandafter{\pgfshapeclippathvertresult}%
   }%
   \expandafter\@tempdima\the\pgflinewidth%
   \pgfpointanchor{yquantbox}{center}%
   \unless\ifx\yquant@circuit@extendmultiline@prev\relax%
      \eappto\yquant@circuit@extendmultiline@cmd{%
         \yquant@circuit@extendmultiline@prev -- (#1,\the\pgf@y)%
      }%
   \fi%
   \edef\yquant@circuit@extendmultiline@prev{%
      (\the\dimexpr#1\relax,\the\pgf@y)%
   }%
}
% END_FOLD

\newif\ifyquant@circuit@operator@hasControls%
% sets up the relevant variables associated with an operator
% #1: positive controls
% #2: negative controls
% #3: targets
\protected\def\yquant@circuit@operator#1#2#3{%
   % convert all names to indices
   % targets
   \yquant@register@get@ids{#3}%
   \let\yquant@circuit@operator@targets=\yquant@register@get@ids@list%
   \let\yquant@circuit@operator@mintarget=\yquant@register@get@ids@min%
   \let\yquant@circuit@operator@maxtarget=\yquant@register@get@ids@max%
   \let\yquant@circuit@operator@numtarget=\yquant@register@get@ids@count%
   % make sure to reset this for subcircuits
   \yquant@circuit@operator@hasControlsfalse%
   % For the targets, multi-qubit registers might have been allowed, but certainly not for
   % the controls!
   \yquant@register@get@allowmultifalse%
   % positive controls
   \yquant@register@get@ids{#1}%
   \let\yquant@circuit@operator@pctrls=\yquant@register@get@ids@list%
   \let\yquant@circuit@operator@minpctrl=\yquant@register@get@ids@min%
   \let\yquant@circuit@operator@maxpctrl=\yquant@register@get@ids@max%
   \let\yquant@circuit@operator@numpctrl=\yquant@register@get@ids@count%
   \ifnum\yquant@register@get@ids@count>0 %
      \yquant@circuit@operator@hasControlstrue%
   \fi%
   % negative controls
   \yquant@register@get@ids{#2}%
   \let\yquant@circuit@operator@nctrls=\yquant@register@get@ids@list%
   \let\yquant@circuit@operator@minnctrl=\yquant@register@get@ids@min%
   \let\yquant@circuit@operator@maxnctrl=\yquant@register@get@ids@max%
   \let\yquant@circuit@operator@numnctrl=\yquant@register@get@ids@count%
   \ifnum\yquant@register@get@ids@count>0 %
      \yquant@circuit@operator@hasControlstrue%
   \fi%
   % determine the qubits spanned
   \yquant@min\yquant@circuit@operator@minctrl%
      \yquant@circuit@operator@mintarget%
      \yquant@circuit@operator@minpctrl%
      \yquant@circuit@operator@minnctrl%
      \relax%
   \yquant@max\yquant@circuit@operator@maxctrl%
      \yquant@circuit@operator@maxtarget%
      \yquant@circuit@operator@maxpctrl%
      \yquant@circuit@operator@maxnctrl%
      \relax%
}

% creates a subcircuit. Parameter registers must be filled in \yquant@circuit@subcircuit@params.
% #1: style
\protected\def\yquant@circuit@subcircuit#1{%
   \numdef\yquant@circuit@subcircuit@id{\yquant@env+1}%
   \listcsxadd{\yquant@prefix subcircuits}{\yquant@circuit@subcircuit@id}%
   \begingroup%
      \tikzset{/yquant/every operator, #1,%
               /yquant/this operator, /yquant/operators/this subcircuit box/.style={}}%
      % execute the subcircuit
      \expandafter%
      \yquant@env@begin@noarg%
         \yquant@lang@attr@value% this contains the subcircuit's content
         \expandafter\unless\expandafter\ifx\csname\yquant@prefix parameters\endcsname\empty%
            \PackageError{yquant.sty}{Invalid subcircuit parameters count}%
                         {Too many parameters given.}%
         \fi%
      \yquant@env@end%
   \endgroup%
}

% BEGIN_FOLD Helpers for operator callbacks
% turn a wire into a different type
\def\yquant@circuit@settype#1{%
   \yquant@circuit@flushwire{#1}%
   \yquant@register@set@type{#1}{\yquant@circuit@settype@to}%
}

\protected\def\yquant@circuit@settype@prepare#1{%
   \yquant@register@set@type{#1}{\yquant@circuit@settype@to}%
}

\protected\long\def\yquant@circuit@setstyle#1#2{%
   \yquant@circuit@flushwire{#1}%
   \yquant@register@set@style{#1}{#2}%
}

\protected\long\def\yquant@circuit@addstyle#1#2{%
   \yquant@circuit@flushwire{#1}%
   \yquant@register@set@style{#1}{\yquant@register@get@style{#1},#2}%
}

% performs an alignment of all registers specified in the argument; that is, the next operation on any of the listed registers will be after the maximum position of all of them
% #1: arbitrary register list
\protected\def\yquant@circuit@align#1{%
   \csxappto{\yquant@prefix draw}{%
      \yquant@draw@hspace{#1}{0pt}%
   }%
}

% introduces a horizontal skip (= invisible operator of given width) among the registers; that is, those registers are first aligned, then skipped by the given amount.
% #1: arbitrary register list
% #2: skip width
\protected\def\yquant@circuit@hspace#1#2{%
   \csxappto{\yquant@prefix draw}{%
      \yquant@draw@hspace{#1}{#2}%
   }%
}

% applies an action to wires a list of registers and causes them to be redrawn
% #1: action for prepare
% #2: action for draw
% #3: arbitrary register list
% #4: parameter(s)
\protected\def\yquant@circuit@actonwires#1#2#3#4{%
   \begingroup%
      \let\tmp=\empty%
      \def\do##1{%
         % We do not extend the wire: a register that is discarded somewhere does not make
         % sense, only right after some application (which is supposed to already have
         % extended the wire appropriately).
         \appto\tmp{#2{##1}#4}%
         #2{##1}#4%
      }%
      \dolistloop{#3}%
      \csxappto{\yquant@prefix draw}{\tmp}%
   \endgroup%
}

% sets the output of wires
% #1: arbitrary register list
\protected\def\yquant@circuit@output{%
   \csxappto{\yquant@prefix outputs}%
            {\yquant@circuit@output@prepare%
               {\unexpanded\expandafter{\yquant@lang@attr@value}}%
               {\unexpanded\expandafter{\yquant@attrs@remaining}}%
               {\unexpanded\expandafter{\yquant@circuit@operator@targets}}%
               {\yquant@circuit@operator@mintarget}{\yquant@circuit@operator@maxtarget}%
               {\yquant@circuit@operator@numtarget}%
            }%
}

\protected\def\yquant@circuit@output@prepare#1#2#3#4#5#6{%
   \def\yquant@attrs@remaining{#2}%
   \def\yquant@circuit@operator@targets{#3}%
   \def\yquant@circuit@operator@mintarget{#4}%
   \def\yquant@circuit@operator@maxtarget{#5}%
   \def\yquant@circuit@operator@numtarget{#6}%
   \yquant@circuit@operator@hasControlsfalse%
   \let\yquant@circuit@operator@pctrls=\empty%
   \def\yquant@circuit@operator@minpctrl{2147483647}%
   \def\yquant@circuit@operator@maxpctrl{0}%
   \let\yquant@circuit@operator@numpctrl=\yquant@circuit@operator@maxpctrl%
   \let\yquant@circuit@operator@nctrls=\empty%
   \let\yquant@circuit@operator@minnctrl=\yquant@circuit@operator@minpctrl%
   \let\yquant@circuit@operator@maxnctrl=\yquant@circuit@operator@maxpctrl%
   \let\yquant@circuit@operator@numnctrl=\yquant@circuit@operator@numpctrl%
   \let\yquant@circuit@operator@minctrl=\yquant@circuit@operator@mintarget%
   \let\yquant@circuit@operator@maxctrl=\yquant@circuit@operator@maxtarget%
   \yquant@prepare{#1}{}%
}
% END_FOLD