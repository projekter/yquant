\def\yquant@register@type@none{0}
\def\yquant@register@type@q{1}
\def\yquant@register@type@c{2}
\def\yquant@register@type@qs{3}

% Internally, we store the register information in a single macro. Indeed, this is almost always faster than using multiple macros (at least in the way implemented here), apart from changing a single type of information, which is almost as fast.
% BEGIN_FOLD Constructor: create single register and cleanup environment

% #1: type
% #2: x position
% #3: y position
% #4: wire start positions
\protected\def\yquant@register@define#1#2{%
   \csnumgdef{\yquant@prefix registers}{\csname\yquant@prefix registers\endcsname+1}%
   \csxdef{\yquant@prefix register@\csname\yquant@prefix registers\endcsname}{%
      {#1}% type
      {#2}% x pos
      {\yquant@config@register@minimum@height}% height; at the end, the y position
      {}% wire start positions
   }%
   \yquant@cleanup@csadd{\yquant@prefix register@\csname\yquant@prefix registers\endcsname}%
}
% END_FOLD

% BEGIN_FOLD Converter: convert a register name to its id
% store the number corresponding to register #2 into #1
\protected\def\yquant@register@get@id#1#2{%
   \ifcsname\yquant@prefix registername@#2\endcsname%
      \letcs#1{\yquant@prefix registername@#2}%
   \else%
      \ifcsname\yquant@prefix registerlast@#2\endcsname%
         \letcs#1{\yquant@prefix registername@#2[0]}%
      \else%
         \yquant@register@get@id@lazycreate{#2}%
         \letcs#1{\yquant@prefix registers}%
      \fi%
   \fi%
}

% same as before, but resolves to the last register in a set if no index was specificed
\protected\def\yquant@register@get@id@last#1#2{%
   \ifcsname\yquant@prefix registername@#2\endcsname%
      \letcs#1{\yquant@prefix registername@#2}%
   \else%
      \ifcsname\yquant@prefix registerlast@#2\endcsname%
         \letcs#1{\yquant@prefix registerlast@#2}%
      \else%
         \yquant@register@get@id@lazycreate{#2}%
         \letcs#1{\yquant@prefix registers}%
      \fi%
   \fi%
}

% In the lazy mode, we create all unknown registers on-the-fly.
\def\yquant@register@get@id@lazycreate#1{%
   \ifyquant@env@lazy%
      % In principle, we could just call \yquant@lang@@qubit. However, if access to a vector register is desired and a _part_ of this does already exist (though not the requested index), we only create the missing registers.
      \yquant@register@get@id@lazycreate@parse#1[;
   \else%
      \PackageError{yquant.sty}{Register `#1' unknown}%
         {You referred to an unknown register.
          Declare registers first using `qubit' and friends.}%
   \fi%
}

\def\yquant@register@get@id@lazycreate@parse#1[#2;{%
   \if\relax\detokenize{#2}\relax%
      \yquant@register@get@id@lazycreate@do#1[0][;%
   \else%
      \yquant@register@get@id@lazycreate@do#1[#2;%
   \fi%
}

\protected\def\yquant@register@get@id@lazycreate@do#1[#2]#3[;{%
   % parse length
   \if\relax\detokenize{#3}\relax%
      \yquant@langhelper@validate\len\count{#2}%
      \numdef\len{\len+1}%
   \else%
      \PackageError{yquant.sty}{Invalid register name}%
         {Register names must not contain `[' apart from register length indication.}%
   \fi%
   \edef\reg{\trim@spaces{#1}}%
   % Was this register already defined?
   \ifcsname\yquant@prefix registername@\reg[0]\endcsname%
      \ifnum%
         \csname\yquant@prefix registerlast@\reg\endcsname<%
         \csname\yquant@prefix registers\endcsname\relax%
         \PackageError{yquant.sty}%
            {On-the-fly enlargement only possible for the last register in a circuit}%
            {Tried to enlarge vector register `\reg' to \len\space qubits, but it is no longer the last register. Enlargement would lead to scattering.}%
      \fi%
      \numdef\idx{\csname\yquant@prefix registerlast@\reg\endcsname-%
                  \csname\yquant@prefix registername@\reg[0]\endcsname+1}%
      \unless\ifnum\idx<\len\relax%
         \PackageError{yquant.sty}{Internal inconsistency detected}%
            {Tried to create a register on-the-fly that already existed.}%
      \fi%
   \else%
      \def\idx{0}%
      \yquant@cleanup@csadd{\yquant@prefix registerlast@\reg}%
   \fi%
   \begingroup%
      % pre-set y position
      \yquant@for \idx := \idx to \numexpr \len -1\relax {%
         \yquant@register@define%
            \yquant@register@type@q%
            {0pt}%
         % Prepare to shipout
         \csxappto{\yquant@prefix draw}{%
            \yquant@lang@create@draw{\csname\yquant@prefix registers\endcsname}%
                                    {0pt}%
                                    {qubit}%
                                    {}%
                                    {}%
         }%
         \global\csletcs{\yquant@prefix registername@\reg[\idx]}%
                        {\yquant@prefix registers}%
         \yquant@cleanup@csadd{\yquant@prefix registername@\reg[\idx]}%
      }%
   \endgroup%
   \global\csletcs{\yquant@prefix registerlast@\reg}{\yquant@prefix registers}%
}

\newif\ifyquant@register@get@allowmulti
% converts names to indices
% #1: comma-separated string of names or ranges
%     range format: startname-endname
%     Both startname and endname may be omitted, denoting the very first and last
%     register, respectively
%     If \ifyquant@register@get@allowmulti is set to true, multi-qubit registers are
%     allowed, which are surrounded by parentheses. They may not appear in ranges.
% \yquant@register@get@ids@list, \yquant@register@get@ids@min, \yquant@register@get@ids@max and \yquant@register@get@ids@count will be set appropriately
\protected\def\yquant@register@get@ids#1{%
   \begingroup%
      \let\yquant@register@get@ids@list=\empty%
      \count0=2147483647 %
      \count2=0 %
      \count4=0 %
      \let\ifinmulti=\iffalse%
      \ifyquant@register@get@allowmulti%
         \def\do##1{%
            \ifinmulti%
               \yquant@register@get@ids@checkinmulti##1)\yquant@sep%
            \else%
               \ifyquant@firsttoken({##1}{%
                  \begingroup%
                     \csletcs{ifinmulti}{iftrue}%
                     \let\yquant@register@get@ids@list=\empty%
                     \count0=2147483647 %
                     \count2=0 %
                     \count4=0 %
                     \expandafter%
                        \yquant@register@get@ids@checkinmulti\@gobble##1)\yquant@sep%
               }{%
                  \yquant@register@get@ids@checkrange##1-\yquant@sep%
               }%
            \fi%
         }%
      \else%
         \def\do##1{%
            \yquant@register@get@ids@checkrange##1-\yquant@sep%
         }%
      \fi%
      \docsvlist{#1}%
      \global\let\yquant@register@get@ids@list=\yquant@register@get@ids@list%
      \xdef\yquant@register@get@ids@min{\the\count0}%
      \xdef\yquant@register@get@ids@max{\the\count2}%
      \xdef\yquant@register@get@ids@count{\the\count4}%
   \endgroup%
}

\def\yquant@register@get@ids@checkinmulti#1)#2\yquant@sep{%
   \yquant@register@get@ids@checkrange#1-\yquant@sep%
   \ifstrequal{#2}{)}{%
      % End of multi
      \edef\process{%
            \endgroup%
         \noexpand\listadd\noexpand\yquant@register@get@ids@list{%
            \noexpand\yquant@register@multi%
               {\the\count0}{\the\count2}{\the\count4}%
               {\yquant@register@get@ids@list}%
         }%
         \noexpand\ifnum\count0>\the\count0\space%
            \count0=\the\count0\space%
         \noexpand\fi%
         \noexpand\ifnum\count2<\the\count2\space%
            \count2=\the\count2\space%
         \noexpand\fi%
         \advance\count4 by 1 %
      }%
      \process%
   }{%
      % Continuation of multi
      \if\relax\detokenize{#2}\relax%
      \else%
         \PackageError{yquant.sty}{Invalid register name}%
                      {Register names must not contain brackets.}%
      \fi%
   }%
}

\def\yquant@register@get@ids@checkrange#1-#2\yquant@sep{%
   \if\relax\detokenize{#2}\relax%
      % the string does not contain a dash
      \yquant@register@get@ids@single{#1}%
   \else%
      % this is a range argument
      \yquant@register@get@ids@range#1-#2\yquant@sep%
   \fi%
}

\def\yquant@register@get@ids@single#1{%
   \yquant@register@get@id\first{#1}%
   \yquant@register@get@id@last\last{#1}%
   \yquant@register@get@ids@do%
}

\protected\def\yquant@register@get@ids@range#1-#2-\yquant@sep{%
   \if\relax\detokenize{#1}\relax%
      \def\first{1}%
   \else%
      \yquant@register@get@id\first{#1}%
   \fi%
   \if\relax\detokenize{#2}\relax%
      \letcs\last{\yquant@prefix registers}%
   \else%
      \yquant@register@get@id@last\last{#2}%
   \fi%
   \yquant@register@get@ids@do%
}

\protected\def\yquant@register@get@ids@do{%
   \ifnum\first<\last\relax%
      \ifnum\first<\count0 %
         \count0=\first\relax%
      \fi%
      \ifnum\last>\count2 %
         \count2=\last\relax%
      \fi%
      \advance\count4 by \numexpr\last-\first+1\relax%
   \else%
      \ifnum\last<\count0 %
         \count0=\last\relax%
      \fi%
      \ifnum\first>\count2 %
         \count2=\first\relax%
      \fi%
      \advance\count4 by \numexpr\first-\last+1\relax%
   \fi%
   \yquant@for \i := \first to \last {%
      \listeadd\yquant@register@get@ids@list{\i}%
   }%
}

\let\yquant@register@multi=\empty%
% END_FOLD

% BEGIN_FOLD Getters: extract the requested information from the register with given id
\def\yquant@register@get@type#1{%
   \expandafter\expandafter\expandafter%
   \yquant@register@get@type@aux\csname\yquant@prefix register@#1\endcsname%
}

\def\yquant@register@get@type@aux#1#2#3#4{#1}

\def\yquant@register@get@x#1{%
   \expandafter\expandafter\expandafter%
   \yquant@register@get@x@aux\csname\yquant@prefix register@#1\endcsname%
}

\def\yquant@register@get@x@aux#1#2#3#4{#2}

\def\yquant@register@get@height#1{%
   \expandafter\expandafter\expandafter%
   \yquant@register@get@height@aux\csname\yquant@prefix register@#1\endcsname%
}

\def\yquant@register@get@height@aux#1#2#3#4{#3}

% The y parameter get macros exist in two forms: The protected one is used during the storage to the draw macro; it should never be executed. The env environment then maps them to their proper expandable forms.
\protected\def\yquant@register@get@@protected{%
   \PackageError%
      {yquant.sty}%
      {Execution of a get-y macro in an illegal context}%
      {get-y macros may not be called unless the yquant environment ends.}%
}

\let\yquant@register@get@y=\yquant@register@get@@protected
\let\yquant@register@y=\yquant@register@get@@protected
% This plays a similar role: It must never be expanded before the tikz command is invoked (because it does not exist).
\let\nodepart=\yquant@register@get@@protected

% getting the y position is the same as getting the height parameter, since we will reuse it at the end.
\let\yquant@register@get@y@unprotected=\yquant@register@get@height

% y distance between two registers.
\def\yquant@register@get@ydist#1#2{%
   \expandafter\yquant@abs\expandafter%
      {\the\dimexpr\yquant@register@get@y{#2}-\yquant@register@get@y{#1}\relax}%
}

% y position of the current register
\def\yquant@register@y@unprotected{0pt}%

\protected\def\yquant@register@get@y@@expandable{%
   \let\yquant@register@get@y=\yquant@register@get@y@unprotected%
   \let\yquant@register@y=\yquant@register@y@unprotected%
}

\def\yquant@register@get@lastwire#1{%
   \expandafter\expandafter\expandafter%
   \yquant@register@get@lastwire@aux\csname\yquant@prefix register@#1\endcsname%
}

\def\yquant@register@get@lastwire@aux#1#2#3#4{#4}

% Set #1 to the maximum x value found between #2 and #3
\protected\def\yquant@register@get@maxxrange#1#2#3{%
   \begingroup%
      \dimen0=-16383.99999pt %
      \yquant@for \yquant@register@get@maxx@i := #2 to #3 {%
         \dimen2=\yquant@register@get@x{\yquant@register@get@maxx@i}\relax%
         \ifdim\dimen0<\dimen2 %
            \dimen0=\dimen2 %
         \fi%
      }%
      \expandafter%
   \endgroup%
   \expandafter\def\expandafter#1\expandafter{\the\dimen0}%
}

% Set #1 to the maximum x value found in the list #2
\protected\def\yquant@register@get@maxxlist#1#2{%
   \begingroup%
      \dimen0=-16383.99999pt %
      \def\do##1{%
         \dimen2=\yquant@register@get@x{##1}\relax%
         \ifdim\dimen0<\dimen2 %
            \dimen0=\dimen2 %
         \fi%
      }%
      \dolistloop{#2}%
      \expandafter%
   \endgroup%
   \expandafter\def\expandafter#1\expandafter{\the\dimen0}%
}
% END_FOLD

% BEGIN_FOLD Setters: change register information
\protected\long\def\yquant@register@set@@aux#1#2#3{%
   \long\xdef#1{\expandafter#2#1{#3}}%
}

\protected\def\yquant@register@set@type#1{%
   \expandafter\yquant@register@set@@aux%
   \csname\yquant@prefix register@#1\endcsname\yquant@register@set@type@aux%
}

\long\def\yquant@register@set@type@aux#1#2#3#4#5{%
   {#5}{#2}{#3}{#4}%
}

\protected\def\yquant@register@set@x#1{%
   \expandafter\yquant@register@set@@aux%
   \csname\yquant@prefix register@#1\endcsname\yquant@register@set@x@aux%
}

\long\def\yquant@register@set@x@aux#1#2#3#4#5{%
   {#1}{#5}{#3}{#4}%
}

\protected\def\yquant@register@update@height#1#2{%
   \ifdim#2>\yquant@register@get@height{#1} %
      \expandafter\yquant@register@set@@aux%
      \csname\yquant@prefix register@#1\endcsname\yquant@register@set@height@aux{#2}%
   \fi%
}

\protected\def\yquant@register@set@y#1{%
   \expandafter\yquant@register@set@@aux%
   \csname\yquant@prefix register@#1\endcsname\yquant@register@set@height@aux%
}

% Set the currently used register
\protected\def\yquant@register@use#1{%
   \edef\yquant@register@y{\yquant@register@get@y{#1}}%
}

\long\def\yquant@register@set@height@aux#1#2#3#4#5{%
   {#1}{#2}{#5}{#4}%
}

\protected\def\yquant@register@set@lastwire#1{%
   \expandafter\yquant@register@set@@aux%
   \csname\yquant@prefix register@#1\endcsname\yquant@register@set@lastwire@aux%
}

\long\def\yquant@register@set@lastwire@aux#1#2#3#4#5{%
   {#1}{#2}{#3}{#5}%
}
% END_FOLD