\def\yquant@register@type@none{0}
\def\yquant@register@type@q{1}
\def\yquant@register@type@c{2}
\def\yquant@register@type@qs{3}

\def\yquant@register@type@tostring#1{%
   \ifcase#1 %
      nobit%
   \or%
      qubit%
   \or%
      cbit%
   \or%
      qubits%
   \else%
      \PackageError{yquant.sty}{Internal inconsistency detected}%
                   {Unknown wire type: `#1'.}%
   \fi%
}

\protected\def\yquant@register@type@fromstring#1#2{%
   \expandafter\ifstrequal\expandafter{#1}{qubit}%
      {\let#2=\yquant@register@type@q}%
      {%
         \expandafter\ifstrequal\expandafter{#1}{cbit}%
         {\let#2=\yquant@register@type@c}%
         {%
            \expandafter\ifstrequal\expandafter{#1}{qubits}%
            {\let#2=\yquant@register@type@qs}%
            {%
               \PackageError{yquant.sty}{Invalid bit type: `#1'}%
                  {Use one of `qubit', `cbit', or `qubits'.}%
            }%
         }%
      }%
}

% Internally, we store the register information in a single macro. Indeed, this is almost always faster than using multiple macros (at least in the way implemented here), apart from changing a single type of information, which is almost as fast.
% BEGIN_FOLD Constructor: create single register and cleanup environment

% #1: type
% #2: x position
% #3: name
% #4: index
\protected\def\yquant@register@define#1#2#3#4{%
   \csnumgdef{\yquant@prefix registers}%
             {\csname\yquant@prefix registers\endcsname+1}%
   \csxdef{\yquant@prefix register@\csname\yquant@prefix registers\endcsname}{%
      {#1}% type
      {#2}% x pos
      {{\yquant@config@register@minimum@height}%
       {\yquant@config@register@minimum@depth}%
       {}}% height, depth, and multi-space parts; at the end, the y position
      {{0pt}{0pt}{}{}}% wire start positions and clipping
      {}% wire style
   }%
   \global\csletcs{\yquant@prefix registername@#3[#4]}{\yquant@prefix registers}%
   \csxdef{\yquant@prefix registerhigh@#3}{#4}%
   \yquant@cleanup@csadd{\yquant@prefix register@\csname\yquant@prefix registers\endcsname}%
   \yquant@cleanup@csadd{\yquant@prefix registername@#3[#4]}%
   \ifnum0=#4\relax%
      \yquant@cleanup@csadd{\yquant@prefix registerhigh@#3}%
   \fi%
}

% in a subcircuit, create a new register that is only an alias for an existing register in the outer circuit.
% #5: id in the outer circuit
\protected\def\yquant@register@alias#1#2#3#4#5{%
   \csnumgdef{\yquant@prefix registers}%
             {\csname\yquant@prefix registers\endcsname+1}%
   \ifcsname\yquant@parent registermap@#5\endcsname%
      % the parent is already an alias itself, pass this on transparently
      \global\csletcs%
         {\yquant@prefix registermap@\csname\yquant@prefix registers\endcsname}%
         {\yquant@parent registermap@#5}%
   \else%
      \csxdef{\yquant@prefix registermap@\csname\yquant@prefix registers\endcsname}%
             {\yquant@parent register@#5}%
   \fi%
   % we still store height, depth, and multi-space parts. To keep the same macros, we create a full register structure, but "abuse" the type component to contain the direct parent
   \csxdef{\yquant@prefix register@\csname\yquant@prefix registers\endcsname}{%
      {{\expandafter\yquant@register@alias@extract\yquant@parent}{#5}}% direct parent (circuit id, register id)
      {}% unused
      {{\yquant@config@register@minimum@height}%
       {\yquant@config@register@minimum@depth}%
       {}}% height, depth, and multi-space parts
      {}% unused
      {}% unused
   }%
   \global\csletcs{\yquant@prefix registername@#3[#4]}{\yquant@prefix registers}%
   \csxdef{\yquant@prefix registerhigh@#3}{#4}%
   \yquant@cleanup@csadd{\yquant@prefix register@\csname\yquant@prefix registers\endcsname}%
   \yquant@cleanup@csadd{\yquant@prefix registermap@\csname\yquant@prefix registers\endcsname}%
   \yquant@cleanup@csadd{\yquant@prefix registername@#3[#4]}%
   \ifnum0=#4\relax%
      \yquant@cleanup@csadd{\yquant@prefix registerhigh@#3}%
   \fi%
   \expandafter\yquant@register@set@@aux%
      \csname\csname\yquant@prefix registermap@\csname\yquant@prefix registers\endcsname\endcsname\endcsname%
      \yquant@register@set@typex@aux%
      {{#1}{#2}}%
}

\def\yquant@register@alias@extract yquant@env#1@{#1}%
% END_FOLD

% BEGIN_FOLD Converter: convert a register name to its id
% store the number corresponding to register #2 into #1
\protected\def\yquant@register@get@id#1#2{%
   \ifcsname\yquant@prefix registername@#2\endcsname%
      \letcs#1{\yquant@prefix registername@#2}%
   \else%
      \yquant@register@get@id@lazycreate{#2}%
      \letcs#1{\yquant@prefix registers}%
   \fi%
}

% same as before, but resolves to the _index_ of the last register
\protected\def\yquant@register@get@id@high#1#2{%
   \unless\ifcsname\yquant@prefix registerhigh@#2\endcsname%
      \yquant@register@get@id@lazycreate{#2}%
   \fi%
   \letcs#1{\yquant@prefix registerhigh@#2}%
}

% In the lazy mode, we create all unknown registers on-the-fly.
\def\yquant@register@get@id@lazycreate#1{%
   \ifyquant@env@lazy%
      % In principle, we could just call \yquant@lang@@qubit. However, if access to a vector register is desired and a _part_ of this does already exist (though not the requested index), we only create the missing registers.
      \yquant@register@get@id@lazycreate@parse#1[;%
   \else%
      \PackageError{yquant.sty}{Register `#1' unknown}%
         {You referred to an unknown register.
          Declare registers first using `qubit' and friends.}%
   \fi%
}

\def\yquant@register@get@id@lazycreate@parse#1[#2;{%
   \ifstrempty{#2}{%
      \yquant@register@get@id@lazycreate@do#1[0][;%
   }{%
      \yquant@register@get@id@lazycreate@do#1[#2;%
   }%
}

\protected\def\yquant@register@get@id@lazycreate@do#1[#2]#3[;{%
   \begingroup%
      % parse length
      \ifstrempty{#3}{%
         \yquant@langhelper@validate\len\count{#2}%
         \numdef\len{\len+1}%
      }{%
         \PackageError{yquant.sty}{Invalid register name}%
            {Register names must not contain `[' apart from register length indication.}%
      }%
      \edef\reg{\trim@spaces{#1}}%
      % Was this register already defined?
      \ifcsname\yquant@prefix registerhigh@\reg\endcsname%
         \numdef\idx{\csname\yquant@prefix registerhigh@\reg\endcsname+1}%
         \unless\ifnum\idx<\len\relax%
            \PackageError{yquant.sty}{Internal inconsistency detected}%
               {Tried to create a register on-the-fly that already existed.}%
         \fi%
      \else%
         \def\idx{0}%
      \fi%
      % pre-set y position
      \yquant@for \idx := \idx to \numexpr \len -1\relax {%
         \yquant@register@define%
            \yquant@register@type@q%
            {0pt}%
            \reg\idx%
         % Prepare to shipout
         \csxappto{\yquant@prefix draw}{%
            \yquant@lang@create@draw{\csname\yquant@prefix registers\endcsname}%
                                    {0pt}%
                                    {qubit}%
                                    0%
                                    {}%
                                    {}%
         }%
      }%
   \endgroup%
}

\newif\ifyquant@register@get@allowmulti
% converts names to indices
% #1: comma-separated string of names or ranges
%     range format: startname-endname
%     Both startname and endname may be omitted, denoting the very first and last
%     register, respectively
%     If \ifyquant@register@get@allowmulti is set to true, multi-qubit registers are
%     allowed, which are surrounded by parentheses. They may not appear in ranges.
% \yquant@register@get@ids@list, \yquant@register@get@ids@min, \yquant@register@get@ids@max and \yquant@register@get@ids@count will be set appropriately
\begingroup%
\catcode`[=\active%
\catcode`(=\active%
\protected\gdef\yquant@register@get@ids#1{%
   \begingroup%
      \let\yquant@register@get@ids@list=\empty%
      \count0=2147483647 % minimal id
      \count2=0 % maximal id
      \count4=0 % number of total registers
      \ifblank{#1}{}{%
         \let\ifinmulti=\iffalse%
         \let\ifallowmain=\iffalse%
         \let\do=\yquant@register@get@ids@outerlist%
         \begingroup%
            \catcode`[=\active%
            \catcode`(=\active%
            \def[##1]{\yquant@register@get@ids@@index{##1}}%
            \def(##1){\yquant@register@get@ids@@multi{##1}}%
            % this is also invoked by other languages which might have completely redefined all catcodes (qasm redefines the backslash), so don't just add \noexpand at the end of the \scantokens...
            \everyeof{\noexpand}%
            \endlinechar=-1 %
            \edef\yquant@register@get@ids@retokenized{\scantokens{#1}}%
            \expandafter%
         \endgroup%
         \expandafter\docsvlist\expandafter{\yquant@register@get@ids@retokenized}%
      }%
      \global\let\yquant@register@get@ids@list=\yquant@register@get@ids@list%
      \xdef\yquant@register@get@ids@min{\the\count0}%
      \xdef\yquant@register@get@ids@max{\the\count2}%
      \xdef\yquant@register@get@ids@count{\the\count4}%
   \endgroup%
}
\endgroup

\protected\def\yquant@register@get@ids@@index{}
\protected\def\yquant@register@get@ids@@multi{}

\def\yquant@register@get@ids@outerlist#1{%
   \ifyquant@firsttoken\yquant@register@get@ids@@multi{#1}{%
      \unless\ifyquant@register@get@allowmulti%
         \PackageError{yquant.sty}{Multi-register gate not allowed}%
                      {The selected gate can only be used in a single-register context.}%
      \fi%
      \yquant@register@get@ids@multi#1\yquant@sep%
   }{%
      \yquant@register@get@ids@checkrange#1-\yquant@sep%
   }%
}

\protected\def\yquant@register@get@ids@multi\yquant@register@get@ids@@multi#1\yquant@sep{%
   \begingroup%
      \let\yquant@register@get@ids@list=\empty%
      \count0=2147483647 % minimal id
      \count2=0 % maximal id
      \count4=0 % number of total registers
      \count6=-1 % index of main register
      \let\ifinmulti=\iftrue%
      \let\ifallowmain=\iftrue%
      \let\do=\yquant@register@get@ids@multilist%
      \docsvlist{#1}%
      \ifnum\count6=-1 %
         \count6=\count0 %
      \fi%
      \yquant@register@multi@splitparts%
      \edef\process{%
            \endgroup%
         \noexpand\listadd\noexpand\yquant@register@get@ids@list{%
            \noexpand\yquant@register@multi%
               {\the\count0}{\the\count2}{\the\count4}%
               {\yquant@register@get@ids@list}%
         }%
         \noexpand\ifnum\count0>\the\count0\space%
            \count0=\the\count0\space%
         \noexpand\fi%
         \noexpand\ifnum\count2<\the\count2\space%
            \count2=\the\count2\space%
         \noexpand\fi%
         \advance\count4 by 1 %
      }%
      \process%
}

\def\yquant@register@get@ids@multilist#1{%
   \yquant@register@get@ids@checkrange#1-\yquant@sep%
}

\def\yquant@register@get@ids@checkrange#1-#2\yquant@sep{%
   \ifstrempty{#2}{%
      % the string does not contain a dash
      \yquant@register@get@ids@norange{#1}%
   }{%
      % this is a range argument
      \yquant@register@get@ids@range#1-#2\yquant@sep%
   }%
}

\def\yquant@register@get@ids@norange#1{%
   % the register is not used within a range, which gives us full freedom with respect to sub-index selections
   \yquant@register@get@ids@norange@checkindex%
      #1%
      \yquant@register@get@ids@@index%
      \yquant@sep%
}

\protected\def\yquant@register@get@ids@norange@checkindex#1\yquant@register@get@ids@@index#2\yquant@sep{%
   \edef\current{\trim@spaces{#1}}%
   \ifallowmain%
      \expandafter\ifyquant@firsttoken\expandafter*\expandafter{\current}{%
         \edef\current{%
            \expandafter\expandafter\expandafter\trim@spaces%
            \expandafter\expandafter\expandafter{%
               \expandafter\@gobble\current%
            }%
         }%
         % catch the minimal index
         \edef\yquant@register@get@ids@norange@checkindex@setmain{%
            \count6=\count2%
            \ifnum\count0<\the\count0 %
               \count0=\the\count0 %
            \fi%
         }%
         \count0=2147483647 %
         \csletcs{ifallowmain}{iffalse}%
      }{%
         \let\yquant@register@get@ids@norange@checkindex@setmain=\relax%
      }%
   \else%
      \let\yquant@register@get@ids@norange@checkindex@setmain=\relax%
   \fi%
   \ifstrempty{#2}{%
      % the string does not contain a sub-index; we add the full register
      \yquant@register@get@id\first{\current[0]}%
      \letcs\high{\yquant@prefix registerhigh@\current}%
      \yquant@register@get@id\last{\current[\high]}%
      \ifnum\first<\count0 %
         \count0=\first\relax%
      \fi%
      \ifnum\last>\count2 %
         \count2=\last\relax%
      \fi%
      \advance\count4 by \numexpr\high+1\relax%
      \yquant@for \i := 0 to \high {%
         \yquant@register@get@id\idx{\current[\i]}%
         \listeadd\yquant@register@get@ids@list{\idx}%
      }%
   }{%
      \yquant@register@get@ids@norange@index#2%
   }%
   \yquant@register@get@ids@norange@checkindex@setmain%
}

\protected\def\yquant@register@get@ids@norange@index#1\yquant@register@get@ids@@index{%
   \let\olddo=\do%
   % the string contains sub-indices; since we are not in a range, multi-indices may still be allowed
   \ifinmulti%
      \let\do=\yquant@register@get@ids@subindex@nomulti%
   \else%
      \let\do=\yquant@register@get@ids@subindex@allowmulti%
   \fi%
   \docsvlist{#1}%
   \let\do=\olddo%
}

\def\yquant@register@get@ids@subindex@nomulti#1{%
   \yquant@register@get@ids@subindex@checkrange#1-\yquant@sep%
}

\def\yquant@register@get@ids@subindex@allowmulti#1{%
   \ifyquant@firsttoken\yquant@register@get@ids@@multi{#1}{%
      \unless\ifyquant@register@get@allowmulti%
         \PackageError{yquant.sty}{Multi-register gate not allowed}%
                      {The selected gate can only be used in a single-register context.}%
      \fi%
      \yquant@register@get@ids@subindex@multi#1\yquant@sep%
   }{%
      \yquant@register@get@ids@subindex@checkrange#1-\yquant@sep%
   }%
}

\protected\def\yquant@register@get@ids@subindex@multi\yquant@register@get@ids@@multi#1\yquant@sep{%
   \begingroup%
      \let\yquant@register@get@ids@list=\empty%
      \count0=2147483647 % minimal id
      \count2=0 % maximal id
      \count4=0 % number of total registers
      \count6=-1 % index of main register
      \let\ifinmulti=\iftrue%
      \let\ifallowmain=\iftrue%
      \let\do=\yquant@register@get@ids@subindex@nomulti%
      \docsvlist{#1}%
      \ifnum\count6=-1 %
         \count6=\count0 %
      \fi%
      \yquant@register@multi@splitparts%
      \edef\process{%
            \endgroup%
         \noexpand\listadd\noexpand\yquant@register@get@ids@list{%
            \noexpand\yquant@register@multi%
               {\the\count0}{\the\count2}{\the\count4}%
               {\yquant@register@get@ids@list}%
         }%
         \noexpand\ifnum\count0>\the\count0\space%
            \count0=\the\count0\space%
         \noexpand\fi%
         \noexpand\ifnum\count2<\the\count2\space%
            \count2=\the\count2\space%
         \noexpand\fi%
         \advance\count4 by 1 %
      }%
      \process%
}

\def\yquant@register@get@ids@subindex@checkrange#1-#2\yquant@sep{%
   \ifstrempty{#2}{%
      % the string does not contain a dash, this is a single sub-item
      \edef\idx{\trim@spaces{#1}}%
      \expandafter\yquant@register@get@ids@subindex@norange\expandafter{\idx}%
   }{%
      % this is a range argument
      \yquant@register@get@ids@subindex@range#1-#2\yquant@sep%
   }%
}

\protected\def\yquant@register@get@ids@subindex@norange#1{%
   \ifallowmain%%
      \ifyquant@firsttoken*{#1}{%
         \edef\idx{\current[\expandafter\trim@spaces\expandafter{\@gobble#1}]}%
         \expandafter\yquant@register@get@id\expandafter\idx\expandafter{\idx}%
         \count6=\idx%
         \csletcs{ifallowmain}{iffalse}%
      }{%
         \yquant@register@get@id\idx{\current[#1]}%
      }%
   \else%
      \yquant@register@get@id\idx{\current[#1]}%
   \fi%
   \ifnum\idx<\count0 %
      \count0=\idx\relax%
   \fi%
   \ifnum\idx>\count2 %
      \count2=\idx\relax%
   \fi%
   \advance\count4 by 1\relax%
   \listeadd\yquant@register@get@ids@list{\idx}%
}

\protected\def\yquant@register@get@ids@subindex@range#1-#2-\yquant@sep{%
   \ifallowmain%
      \ifyquant@firsttoken*{#1}{%
         \expandafter\ifblank\expandafter{\@gobble#1}{%
            \def\first{0}%
         }{%
            \expandafter\yquant@langhelper@validate\expandafter\first\expandafter\count\expandafter{\@gobble#1}%
         }%
         \yquant@register@get@id\idx{\current[\first]}%
         \count6=\idx%
         \csletcs{ifallowmain}{iffalse}%
      }{%
         \ifblank{#1}{%
            \def\first{0}%
         }{%
            \yquant@langhelper@validate\first\count{#1}%
         }%
      }%
   \else%
      \ifblank{#1}{%
         \def\first{0}%
      }{%
         \yquant@langhelper@validate\first\count{#1}%
      }%
   \fi%
   \ifallowmain%
      % in #1, all initial spaces are gobbled automatically, but not in #2
      \expandafter\ifyquant@firsttoken\expandafter*\expandafter{\empty#2}{%
         \expandafter\ifblank\expandafter{\@gobble#2}{%
            \yquant@register@get@id@high\last\current%
         }{%
            \expandafter\yquant@langhelper@validate\expandafter\last\expandafter\count\expandafter{\@gobble#2}%
         }%
         \yquant@register@get@id\idx{\current[\last]}%
         \count6=\idx%
         \csletcs{ifallowmain}{iffalse}%
      }{%
         \ifblank{#2}{%
            \yquant@register@get@id@high\last\current%
         }{%
            \yquant@langhelper@validate\last\count{#2}%
         }%
      }%
   \else%
      \ifblank{#2}{%
         \yquant@register@get@id@high\last\current%
      }{%
         \yquant@langhelper@validate\last\count{#2}%
      }%
   \fi%
   \yquant@for \i := \first to \last {%
      \yquant@register@get@ids@subindex@norange\i%
   }%
}

\protected\def\yquant@register@get@ids@range#1-#2-\yquant@sep{%
   % being a range between two registers, those must be uniquely identifiable, i.e. either a single sub-indexed part of a vector register, or no vector specification at all.
   \ifallowmain%
      \ifyquant@firsttoken*{#1}{%
         \expandafter\ifblank\expandafter{\@gobble#1}{%
            \def\first{1}%
         }{%
            \expandafter\yquant@register@get@ids@range@getfirst\@gobble#1\yquant@register@get@ids@@index\yquant@sep%
         }%
         \count6=\first%
         \csletcs{ifallowmain}{iffalse}%
      }{%
         \ifblank{#1}{%
            \def\first{1}%
         }{%
            \yquant@register@get@ids@range@getfirst#1\yquant@register@get@ids@@index\yquant@sep%
         }%
      }%
   \else%
      \ifblank{#1}{%
         \def\first{1}%
      }{%
         \yquant@register@get@ids@range@getfirst#1\yquant@register@get@ids@@index\yquant@sep%
      }%
   \fi%
   % it does not make sense to allow to set the main part on the second half of the range, since a range over registers (not indices) is automatically visual, so this range will be contiguous.
   \ifblank{#2}{%
      \letcs\last{\yquant@prefix registers}%
   }{%
      \yquant@register@get@ids@range@getlast#2\yquant@register@get@ids@@index\yquant@sep%
   }%
   \ifnum\first<\last\relax%
      \ifnum\first<\count0 %
         \count0=\first\relax%
      \fi%
      \ifnum\last>\count2 %
         \count2=\last\relax%
      \fi%
      \advance\count4 by \numexpr\last-\first+1\relax%
   \else%
      \ifnum\last<\count0 %
         \count0=\last\relax%
      \fi%
      \ifnum\first>\count2 %
         \count2=\first\relax%
      \fi%
      \advance\count4 by \numexpr\first-\last+1\relax%
   \fi%
   \yquant@for \i := \first to \last {%
      \listeadd\yquant@register@get@ids@list{\i}%
   }%
}

\def\yquant@register@get@ids@range@getfirst#1\yquant@register@get@ids@@index#2\yquant@sep{%
   \ifstrempty{#2}{%
      \yquant@register@get@id\first{\trim@spaces{#1}[0]}%
   }{%
      \yquant@register@get@ids@range@get\first#1\yquant@register@get@ids@@index#2%
   }%
}

\protected\def\yquant@register@get@ids@range@getlast#1\yquant@register@get@ids@@index#2\yquant@sep{%
   \ifstrempty{#2}{%
      \yquant@register@get@id@high\last{\trim@spaces{#1}}%
      \yquant@register@get@id\last{\trim@spaces{#1}[\last]}%
   }{%
      \yquant@register@get@ids@range@get\last#1\yquant@register@get@ids@@index#2%
   }%
}

\def\yquant@register@get@ids@range@get#1#2\yquant@register@get@ids@@index#3\yquant@register@get@ids@@index{%
   \yquant@register@get@id#1{\trim@spaces{#2}[\trim@spaces{\trim@spaces#3}]}%
}

\let\yquant@register@multi=\empty%
\protected\def\yquant@register@multi@contiguous{}%

% splits \yquant@register@get@ids@list into a list of contiguous parts
\protected\def\yquant@register@multi@splitparts{%
   \begingroup%
      \let\registers=\yquant@register@get@ids@list%
      \yquant@sort@list\registers\yquant@sort@ascending%
      \let\newlist=\empty%
      \count2=-1 %
      \def\do##1{%
         \ifnum\count2=-1 %
            \count2=##1 %
            \count4=\count2 %
         \else%
            \advance\count2 by 1 %
            \ifnum##1>\count2 %
               % this is a discontiguous change (we disallow duplicate entries)
               \eappto\newlist{%
                  \yquant@register@multi@contiguous%
                     {\the\count4}{\the\numexpr\count2-1\relax}%
                     {\ifnum\count6<\count2 %
                         \ifnum\count4>\count6 %
                            0%
                         \else%
                            1%
                         \fi%
                      \else%
                         0%
                      \fi}%
               }%
               \count2=##1 %
               \count4=\count2 %
            \fi%
         \fi%
      }%
      \dolistloop\registers%
      % we disallow empty lists
      \eappto\newlist{%
         \yquant@register@multi@contiguous%
            {\the\count4}{\the\count2}%
            {\ifnum\count6>\count2 %
                0%
             \else%
                \ifnum\count4>\count6 %
                   0%
                \else%
                   1%
                \fi%
             \fi}%
      }%
      \expandafter%
   \endgroup%
   \expandafter\def\expandafter\yquant@register@get@ids@list\expandafter{%
      \newlist%
   }%
}

% splits \yquant@register@get@ids@list into a list of discontiguous parts
\protected\def\yquant@register@multi@splitparts@sepall{%
   \begingroup%
      \let\registers=\yquant@register@get@ids@list%
      \yquant@sort@list\registers\yquant@sort@ascending%
      \let\newlist=\empty%
      \def\do##1{%
         \eappto\newlist{%
            \yquant@register@multi@contiguous%
               {##1}{##1}%
               {\ifnum##1=\count6 %
                   1%
                \else%
                   0%
                \fi%
               }%
         }%
      }%
      \dolistloop\registers%
      \expandafter%
   \endgroup%
   \expandafter\def\expandafter\yquant@register@get@ids@list\expandafter{%
      \newlist%
   }%
}

\protected\def\yquant@register@multi@splitparts@sort{%
   \yquant@sort@list\yquant@register@get@ids@list\yquant@sort@ascending%
}

\protected\def\yquant@register@get@multiassingle{%
   \yquant@register@get@allowmultitrue%
   \let\yquant@register@multi@splitparts=\yquant@register@multi@splitparts@sepall%
   \preto\yquant@attrs@remaining{/yquant/operator/multi as single,}%
}

\protected\def\yquant@register@get@multiaslist{%
   \yquant@register@get@allowmultitrue%
   \let\yquant@register@multi@splitparts=\yquant@register@multi@splitparts@sort%
}

\let\yquant@register@multi@reset@original@splitparts=\yquant@register@multi@splitparts%
\protected\def\yquant@register@reset@multi{%
%   \yquant@register@get@allowmultifalse% not needed, as \yquant@circuit@operator automatically resets it
   \let\yquant@register@multi@splitparts=\yquant@register@multi@reset@original@splitparts%
}
% END_FOLD

% BEGIN_FOLD Getters: extract the requested information from the register with given id
\def\yquant@register@get@type#1{%
   \ifcsname\yquant@prefix registermap@#1\endcsname%
      \expandafter\expandafter\expandafter%
      \yquant@register@get@type@aux\csname\csname\yquant@prefix registermap@#1\endcsname\endcsname%
   \else%
      \expandafter\expandafter\expandafter%
      \yquant@register@get@type@aux\csname\yquant@prefix register@#1\endcsname%
   \fi%
}

\def\yquant@register@get@parent#1{%
   \ifcsname\yquant@prefix registermap@#1\endcsname%
      \expandafter\expandafter\expandafter%
      \yquant@register@get@type@aux\csname\yquant@prefix register@#1\endcsname%
   \else%
      \PackageError{yquant.sty}{Internal inconsistency detected}%
                   {Tried to get parent of a non-connected wire.}%
   \fi%
}

\def\yquant@register@get@type@aux#1#2#3#4#5{#1}

\def\yquant@register@get@x#1{%
   \ifcsname\yquant@prefix registermap@#1\endcsname%
      \expandafter\expandafter\expandafter%
      \yquant@register@get@x@aux\csname\csname\yquant@prefix registermap@#1\endcsname\endcsname%
   \else%
      \expandafter\expandafter\expandafter%
      \yquant@register@get@x@aux\csname\yquant@prefix register@#1\endcsname%
   \fi%
}

\def\yquant@register@get@x@aux#1#2#3#4#5{#2}

\def\yquant@register@get@height#1{%
   \expandafter\expandafter\expandafter%
   \yquant@register@get@height@aux\csname\yquant@prefix register@#1\endcsname%
}

\def\yquant@register@peek@height#1#2{%
   \expandafter\expandafter\expandafter%
   \yquant@register@get@height@aux\csname yquant@env#1@register@#2\endcsname%
}

\def\yquant@register@get@height@aux#1#2#3#4#5{\@firstofthree#3}

\def\yquant@register@get@depth#1{%
   \expandafter\expandafter\expandafter%
   \yquant@register@get@depth@aux\csname\yquant@prefix register@#1\endcsname%
}

\def\yquant@register@peek@depth#1#2{%
   \expandafter\expandafter\expandafter%
   \yquant@register@get@depth@aux\csname yquant@env#1@register@#2\endcsname%
}

\def\yquant@register@get@depth@aux#1#2#3#4#5{\@secondofthree#3}

\def\yquant@register@get@multispace#1{%
   \expandafter\expandafter\expandafter%
   \yquant@register@get@multispace@aux\csname\yquant@prefix register@#1\endcsname%
}

\def\yquant@register@get@multispace@aux#1#2#3#4#5{\@thirdofthree#3}

% The y parameter get macros exist in two forms: The protected one is used during the storage to the draw macro; it should never be executed. The env environment then maps them to their proper expandable forms.
\protected\def\yquant@register@get@y{%
   \PackageError%
      {yquant.sty}%
      {Execution of a get-y macro in an illegal context}%
      {get-y macros may not be called unless the yquant environment ends.}%
}

% This plays a similar role: It must never be expanded before the tikz command is invoked (because it does not exist).
\protected\def\nodepart{}

\def\yquant@register@get@y@unprotected#1{%
   \ifcsname\yquant@prefix registermap@#1\endcsname%
      \expandafter\expandafter\expandafter%
      \yquant@register@get@y@aux\csname\csname\yquant@prefix registermap@#1\endcsname\endcsname%
   \else%
      \expandafter\expandafter\expandafter%
      \yquant@register@get@y@aux\csname\yquant@prefix register@#1\endcsname%
   \fi%
}

\def\yquant@register@get@y@aux#1#2#3#4#5{#3}

% y distance between two registers.
\def\yquant@register@get@ydist#1#2{%
   \expandafter\yquant@abs\expandafter%
      {\the\dimexpr\yquant@register@get@y{#2}-\yquant@register@get@y{#1}\relax}%
}

\protected\def\yquant@register@get@y@@expandable{%
   \let\yquant@register@get@y=\yquant@register@get@y@unprotected%
}

\def\yquant@register@get@lastwire#1{%
   \ifcsname\yquant@prefix registermap@#1\endcsname%
      \expandafter\@firstoftwo%
   \else%
      \expandafter\@secondoftwo%
   \fi%
   {%
      \expandafter\expandafter\expandafter%
      \yquant@register@get@lastwire@aux\csname\csname\yquant@prefix registermap@#1\endcsname\endcsname%
   }{%
      \expandafter\expandafter\expandafter%
      \yquant@register@get@lastwire@aux\csname\yquant@prefix register@#1\endcsname%
   }%
}

\protected\def\yquant@register@get@lastwire@aux#1#2#3#4#5#6{%
   \def#6{#4}%
}

\def\yquant@register@get@style#1{%
   \ifcsname\yquant@prefix registermap@#1\endcsname%
      \expandafter\expandafter\expandafter%
      \yquant@register@get@style@aux\csname\csname\yquant@prefix registermap@#1\endcsname\endcsname%
   \else%
      \expandafter\expandafter\expandafter%
      \yquant@register@get@style@aux\csname\yquant@prefix register@#1\endcsname%
   \fi%
}

\def\yquant@register@get@style@aux#1#2#3#4#5{#5}

\def\yquant@register@get@typeywire#1{%
   \ifcsname\yquant@prefix registermap@#1\endcsname%
      \expandafter\@firstoftwo%
   \else%
      \expandafter\@secondoftwo%
   \fi%
   {%
      \expandafter\expandafter\expandafter%
      \yquant@register@get@typeywire@aux\csname\csname\yquant@prefix registermap@#1\endcsname\endcsname%
   }{%
      \expandafter\expandafter\expandafter%
      \yquant@register@get@typeywire@aux\csname\yquant@prefix register@#1\endcsname%
   }%
}

\protected\def\yquant@register@get@typeywire@aux#1#2#3#4#5#6#7#8{%
   \def#6{#1}%
   \def#7{#3}%
   \def#8{#4}%
}%

% Set #1 to the maximum x value found between #2 and #3
\protected\def\yquant@register@get@maxxrange#1#2#3{%
   \begingroup%
      \dimen0=-16383.99999pt %
      \yquant@for \yquant@register@get@maxx@i := #2 to #3 {%
         \dimen2=\yquant@register@get@x{\yquant@register@get@maxx@i}\relax%
         \ifdim\dimen0<\dimen2 %
            \dimen0=\dimen2 %
         \fi%
      }%
      \expandafter%
   \endgroup%
   \expandafter\def\expandafter#1\expandafter{\the\dimen0}%
}

% Set #1 to the maximum x value found in the list #2
\protected\def\yquant@register@get@maxxlist#1#2{%
   \begingroup%
      \dimen0=-16383.99999pt %
      \def\do##1{%
         \dimen2=\yquant@register@get@x{##1}\relax%
         \ifdim\dimen0<\dimen2 %
            \dimen0=\dimen2 %
         \fi%
      }%
      \dolistloop{#2}%
      \expandafter%
   \endgroup%
   \expandafter\def\expandafter#1\expandafter{\the\dimen0}%
}
% END_FOLD

% BEGIN_FOLD Setters: change register information
\protected\long\def\yquant@register@set@@aux#1#2#3{%
   \xdef#1{\expandafter#2#1{#3}}%
}

\protected\def\yquant@register@set@type#1{%
   \ifcsname\yquant@prefix registermap@#1\endcsname%
      \expandafter\@firstoftwo%
   \else%
      \expandafter\@secondoftwo%
   \fi%
   {%
      \expandafter\yquant@register@set@@aux%
         \csname\csname\yquant@prefix registermap@#1\endcsname\endcsname%
   }{%
      \expandafter\yquant@register@set@@aux%
      \csname\yquant@prefix register@#1\endcsname%
   }%
   \yquant@register@set@type@aux%
}

\long\def\yquant@register@set@type@aux#1#2#3#4#5#6{%
   {#6}\unexpanded{{#2}{#3}{#4}{#5}}%
}

\protected\def\yquant@register@set@x#1{%
   \ifcsname\yquant@prefix registermap@#1\endcsname%
      \expandafter\@firstoftwo%
   \else%
      \expandafter\@secondoftwo%
   \fi%
   {%
      \expandafter\yquant@register@set@@aux%
         \csname\csname\yquant@prefix registermap@#1\endcsname\endcsname%
   }{%
      \expandafter\yquant@register@set@@aux%
      \csname\yquant@prefix register@#1\endcsname%
   }%
   \yquant@register@set@x@aux%
}

\long\def\yquant@register@set@x@aux#1#2#3#4#5#6{%
   \unexpanded{{#1}}{#6}\unexpanded{{#3}{#4}{#5}}%
}

\long\def\yquant@register@set@typex@aux#1#2#3#4#5#6{%
   #6\unexpanded{{#3}{#4}{#5}}%
}

\protected\def\yquant@register@update@height#1#2{%
   \ifdim#2>%
      \expandafter\expandafter\expandafter%
         \yquant@register@get@height@aux\csname\yquant@prefix register@#1\endcsname%
      \relax%
      \expandafter\yquant@register@set@@aux%
         \csname\yquant@prefix register@#1\endcsname%
         \yquant@register@set@height@aux{#2}%
   \fi%
}

\protected\def\yquant@register@set@height@remote#1#2#3{%
   \expandafter\yquant@register@set@@aux%
      \csname yquant@env#1@register@#2\endcsname%
      \yquant@register@set@height@aux{#3}%
}

\long\def\yquant@register@set@height@aux#1#2#3#4#5#6{%
   \unexpanded{{#1}{#2}}%
   {{#6}%
    {\unexpanded\expandafter{\@secondofthree#3}}%
    {\unexpanded\expandafter{\@thirdofthree#3}}}%
   \unexpanded{{#4}{#5}}%
}

\protected\def\yquant@register@update@depth#1#2{%
   \ifdim#2>%
      \expandafter\expandafter\expandafter%
         \yquant@register@get@depth@aux\csname\yquant@prefix register@#1\endcsname%
      \relax%
      \expandafter\yquant@register@set@@aux%
         \csname\yquant@prefix register@#1\endcsname%
         \yquant@register@set@depth@aux{#2}%
   \fi%
}

\protected\def\yquant@register@set@depth@remote#1#2#3{%
   \expandafter\yquant@register@set@@aux%
      \csname yquant@env#1@register@#2\endcsname%
      \yquant@register@set@depth@aux{#3}%
}

\long\def\yquant@register@set@depth@aux#1#2#3#4#5#6{%
   \unexpanded{{#1}{#2}}%
   {{\unexpanded\expandafter{\@firstofthree#3}}%
    {#6}%
    {\unexpanded\expandafter{\@thirdofthree#3}}}%
   \unexpanded{{#4}{#5}}%
}

\protected\def\yquant@register@update@multispace#1#2#3{%
   \unless\ifnum#1<#2 %
      \PackageError{yquant.sty}{Internal inconsistency detected}%
                   {Setting multispace part in wrong order.}%
   \fi%
   \begingroup%
      \edef\tmp{%
         \expandafter\expandafter\expandafter%
         \yquant@register@get@multispace@aux\csname\yquant@prefix register@#1\endcsname%
      }%
      \let\old=\tmp%
      \yquant@list@eupdateorinsert\tmp{#2}{#3}%
      \unless\ifx\old\tmp%
         \expandafter\yquant@register@set@@aux%
            \csname\yquant@prefix register@#1\endcsname%
            \yquant@register@set@multispace@aux\tmp%
      \fi%
   \endgroup%
}

\long\def\yquant@register@set@multispace@aux#1#2#3#4#5#6{%
   \unexpanded{{#1}{#2}}%
   {{\unexpanded\expandafter{\@firstofthree#3}}%
    {\unexpanded\expandafter{\@secondofthree#3}}%
    {#6}}%
   \unexpanded{{#4}{#5}}%
}

\protected\def\yquant@register@inject@extents#1#2{%
   % enlarge the last subcircuit that was created previously by height #1 and depth #2.
   % We enforce subcircuits to have registers, so no check needed.
   \expandafter\yquant@register@set@@aux%
      \csname yquant@env\the\yquant@env @register@1\endcsname%
      \yquant@register@inject@height@aux{#1}%
   \expandafter\yquant@register@set@@aux%
      \csname yquant@env\the\yquant@env @register@\csname yquant@env\the\yquant@env @registers\endcsname\endcsname%
      \yquant@register@inject@depth@aux{#2}%
}

\long\def\yquant@register@inject@height@aux#1#2#3#4#5#6{%
   \unexpanded{{#1}{#2}}%
   {{\the\dimexpr\@firstofthree#3+#6\relax}%
    {\unexpanded\expandafter{\@secondofthree#3}}%
    {\unexpanded\expandafter{\@thirdofthree#3}}}%
   \unexpanded{{#4}{#5}}%
}

\long\def\yquant@register@inject@depth@aux#1#2#3#4#5#6{%
   \unexpanded{{#1}{#2}}%
   {{\unexpanded\expandafter{\@firstofthree#3}}%
    {\the\dimexpr\@secondofthree#3+#6\relax}%
    {\unexpanded\expandafter{\@thirdofthree#3}}}%
   \unexpanded{{#4}{#5}}%
}

\protected\def\yquant@register@set@y#1{%
   \ifcsname\yquant@prefix registermap@#1\endcsname%
      \expandafter\@firstoftwo%
   \else%
      \expandafter\@secondoftwo%
   \fi%
   {%
      \expandafter\yquant@register@set@@aux%
         \csname\csname\yquant@prefix registermap@#1\endcsname\endcsname%
   }{%
      \expandafter\yquant@register@set@@aux%
      \csname\yquant@prefix register@#1\endcsname%
   }%
   \yquant@register@set@y@aux%
}

\long\def\yquant@register@set@y@aux#1#2#3#4#5#6{%
   \unexpanded{{#1}{#2}}{#6}\unexpanded{{#4}{#5}}%
}

\protected\def\yquant@register@set@lastwire#1{%
   \ifcsname\yquant@prefix registermap@#1\endcsname%
      \expandafter\@firstoftwo%
   \else%
      \expandafter\@secondoftwo%
   \fi%
   {%
      \expandafter\yquant@register@set@@aux%
         \csname\csname\yquant@prefix registermap@#1\endcsname\endcsname%
   }{%
      \expandafter\yquant@register@set@@aux%
      \csname\yquant@prefix register@#1\endcsname%
   }%
   \yquant@register@set@lastwire@aux%
}

\long\def\yquant@register@set@lastwire@aux#1#2#3#4#5#6{%
   \unexpanded{{#1}{#2}{#3}}{#6}\unexpanded{{#5}}%
}

\protected\def\yquant@register@set@typelastwire#1{%
   \ifcsname\yquant@prefix registermap@#1\endcsname%
      \PackageError{yquant.sty}{Internal inconsistency detected}%
                   {Setting type and lastwire was requested for a subcircuit.}%
   \fi%
   \expandafter\yquant@register@set@@aux%
      \csname\yquant@prefix register@#1\endcsname%
      \yquant@register@set@typelastwire@aux%
}

\long\def\yquant@register@set@typelastwire@aux#1#2#3#4#5#6{%
   {\@firstoftwo#6}\unexpanded{{#2}{#3}}{\@secondoftwo#6}\unexpanded{{#5}}%
}

\protected\def\yquant@register@set@style#1{%
   \ifcsname\yquant@prefix registermap@#1\endcsname%
      \expandafter\@firstoftwo%
   \else%
      \expandafter\@secondoftwo%
   \fi%
   {%
      \expandafter\yquant@register@set@@aux%
         \csname\csname\yquant@prefix registermap@#1\endcsname\endcsname%
   }{%
      \expandafter\yquant@register@set@@aux%
      \csname\yquant@prefix register@#1\endcsname%
   }%
   \yquant@register@set@style@aux%
}

\long\def\yquant@register@set@style@aux#1#2#3#4#5#6{%
   \unexpanded{{#1}{#2}{#3}{#4}}{#6}%
}
% END_FOLD