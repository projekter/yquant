% BEGIN_FOLD Attributes
\yquant@langhelper@declare@attr{%
   value/.store in=\yquant@lang@attr@value,%
   after/.code={%
      \yquant@register@get@ids{#1}%
      \let\yquant@lang@attr@after=\yquant@register@get@ids@list%
   },%
   type/.store in=\yquant@lang@attr@type,%
   ancilla/.code={%
      \undef\yquant@lang@attr@input%
      \undef\yquant@lang@attr@output%
   },%
   in/.code={%
      \let\yquant@lang@attr@input=\relax%
      \undef\yquant@lang@attr@output%
   },%
   out/.code={%
      \undef\yquant@lang@attr@input%
      \let\yquant@lang@attr@output=\relax%
   },%
   inout/.code={%
      \let\yquant@lang@attr@input=\relax%
      \let\yquant@lang@attr@output=\relax%
   },%
   frameless/.code={%
      \pgfkeysalso{/yquant/operators/subcircuit/frameless}%
      \appto\yquant@attrs@remaining{,/yquant/operators/subcircuit/frameless}%
   },%
   seamless/.code={%
      \pgfkeysalso{/yquant/operators/subcircuit/seamless}%
      \appto\yquant@attrs@remaining{,/yquant/operators/subcircuit/seamless}%
   },%
   direct control/.code={%
      \let\yquant@lang@attr@directcontrol=\relax%
   },%
   indirect control/.code={%
      \undef\yquant@lang@attr@directcontrol%
   },%
   name mangling/.code={%
      \pgfkeysalso{/yquant/operators/subcircuit/name mangling={#1}}%
      \appto\yquant@attrs@remaining{,/yquant/operators/subcircuit/name mangling={#1}}%
   }%
}
\yquant@langhelper@declare@attr@global{%
   name/.store in=\yquant@lang@attr@name,%
   overlay/.code={%
      \ifstrequal{#1}{true}{%
         \yquant@lang@attr@overlay@multitrue%
         \yquant@lang@attr@overlay@heighttrue%
         \yquant@lang@attr@overlay@depthtrue%
      }{%
         \ifstrequal{#1}{false}{%
            % Why??? This is the default and should always be reset automatically!
            \yquant@lang@attr@overlay@multifalse%
            \yquant@lang@attr@overlay@heightfalse%
            \yquant@lang@attr@overlay@depthfalse%
         }{%
            \pgfqkeys{/yquant/global attrs/overlay}{#1}%
         }%
      }%
   },%
   overlay/.default=true,%
   overlay/multi/.is if=yquant@lang@attr@overlay@multi,%
   overlay/m/.forward to=/yquant/global attrs/overlay/multi,%
   overlay/height/.is if=yquant@lang@attr@overlay@height,%
   overlay/ht/.forward to=/yquant/global attrs/overlay/height,%
   overlay/h/.forward to=/yquant/global attrs/overlay/height,%
   overlay/depth/.is if=yquant@lang@attr@overlay@depth,%
   overlay/dp/.forward to=/yquant/global attrs/overlay/depth,%
   overlay/d/.forward to=/yquant/global attrs/overlay/depth,%
   overlay/single/.code={\pgfkeysalso{%
      /yquant/global attrs/overlay/height={#1},%
      /yquant/global attrs/overlay/depth={#1}%
   }},%
   overlay/single/.default=true,%
   overlay/s/.forward to=/yquant/global attrs/overlay/single,%
}
\newif\ifyquant@lang@attr@overlay@multi%
\newif\ifyquant@lang@attr@overlay@height%
\newif\ifyquant@lang@attr@overlay@depth%

\protected\def\yquant@lang@reset@attrs{%
   \undef\yquant@lang@attr@value%
   \undef\yquant@lang@attr@after%
   \undef\yquant@lang@attr@type%
   \yquant@lang@reset@attrs@inputoutput%
   \let\yquant@lang@attr@name=\empty%
   \yquant@lang@attr@overlay@multifalse%
   \yquant@lang@attr@overlay@heightfalse%
   \yquant@lang@attr@overlay@depthfalse%
}

\protected\def\yquant@lang@reset@attrs@inputoutput{%
   \undef\yquant@lang@attr@input%
   \undef\yquant@lang@attr@output%
}

\protected\def\yquant@lang@reset@attrs@inputoutput@subcircuit{%
   \let\yquant@lang@attr@input=\relax%
   \let\yquant@lang@attr@output=\relax%
}
% END_FOLD

% BEGIN_FOLD Declaration of registers
\yquant@langhelper@declare@command@create{nobit}\yquant@lang@@nobit
\yquant@langhelper@setup@attrs{nobit}{}{ancilla,out}
\def\yquant@lang@@nobit#1{%
   \let\yquant@lang@create@type=\yquant@register@type@none%
   \def\yquant@lang@create@style{initial}% there is no separate style, just duplicate
   \ifdefined\yquant@lang@attr@value%
      \PackageError{yquant.sty}{Placeholder initialization must not have value}%
                   {You must not provide a description for an invisible register.}
   \else%
      \let\yquant@lang@attr@value=\empty%
   \fi%
   \yquant@lang@create@parse@name#1[;%
}

\yquant@langhelper@declare@command@create{qubit}\yquant@lang@@qubit
\yquant@langhelper@setup@attrs{qubit}{}{after,value,ancilla,in,out,inout}
\def\yquant@lang@@qubit#1{%
   \let\yquant@lang@create@type=\yquant@register@type@q%
   \def\yquant@lang@create@style{qubit}%
   \unless\ifdefined\yquant@lang@attr@value%
      \let\yquant@lang@attr@value=\yquant@config@register@default@name%
   \fi%
   \yquant@lang@create@parse@name#1[;%
}

\yquant@langhelper@declare@command@create{cbit}\yquant@lang@@cbit
\yquant@langhelper@setup@attrs{cbit}{}{after,value,ancilla,in,out,inout}
\def\yquant@lang@@cbit#1{%
   \let\yquant@lang@create@type=\yquant@register@type@c%
   \def\yquant@lang@create@style{cbit}%
   \unless\ifdefined\yquant@lang@attr@value%
      \let\yquant@lang@attr@value=\yquant@config@register@default@name%
   \fi%
   \yquant@lang@create@parse@name#1[;%
}

\yquant@langhelper@declare@command@create{qubits}\yquant@lang@@qubits
\yquant@langhelper@setup@attrs{qubits}{}{after,value,ancilla,in,out,inout}
\def\yquant@lang@@qubits#1{%
   \let\yquant@lang@create@type=\yquant@register@type@qs%
   \def\yquant@lang@create@style{qubits}%
   \unless\ifdefined\yquant@lang@attr@value%
      \let\yquant@lang@attr@value=\yquant@config@register@default@name%
   \fi%
   \yquant@lang@create@parse@name#1[;%
}

\def\yquant@lang@create@parse@name#1[#2;{%
   \ifstrempty{#2}{%
      \yquant@lang@create@do#1[1][;%
   }{%
      \yquant@lang@create@do#1[#2;%
   }%
}

\protected\def\yquant@lang@create@do#1[#2]#3[;{%
   % parse length
   \ifstrempty{#3}{%
      \yquant@langhelper@validate\len\count{#2}%
      \ifnum\len<1 %
         \PackageError{yquant.sty}{Invalid register length}%
                      {Valid register lengths are integers greater or equal to one.}%
      \fi%
   }{%
      \PackageError{yquant.sty}{Invalid register name}%
         {Register names must not contain `[' apart from register length indication.}%
   }%
   \edef\reg{\trim@spaces{#1}}%
   % we allow for scattering, so check whether the register already exists
   \ifcsname\yquant@prefix registerhigh@\reg\endcsname%
      \ifyquant@firsttoken+{#2}{%
         \numdef\idx{\csname\yquant@prefix registerhigh@\reg\endcsname+1}%
         \numdef\len{\len+\idx}%
      }{%
         \PackageError{yquant.sty}{Duplicate registers}%
            {Register `\reg' was already defined. Use `\reg[+#2]' with explicit plus symbol to indicate that you want to enlarge the register on purpose.}%
      }%
   \else%
      \def\idx{0}%
   \fi%
   % define text macros
   \ifnum\len=1 %
      \let\regidx=\reg%
   \else%
      \def\regidx{\reg[\idx]}%
   \fi%
   \numdef\yquant@circuit@operator@mintarget{\csname\yquant@prefix registers\endcsname+1}%
   % create the registers
   \begingroup%
      % if we have the after attribute, we start with a discarded wire
      \ifdefined\yquant@lang@attr@after%
         \ifdefined\yquant@lang@attr@input%
            \PackageError{yquant.sty}{An input register cannot be created with `after` attribute}{}%
         \fi%
         \let\yquant@lang@create@type=\yquant@register@type@none%
      \fi%
      \yquant@for \idx := \idx to \numexpr \len -1\relax {%
         \yquant@prepare@create\reg\idx\yquant@lang@create@type%
      }%
   \endgroup%
   % gather details about the created registers
   \letcs\yquant@circuit@operator@maxtarget{\yquant@prefix registers}%
   \numdef\yquant@circuit@operator@numtarget{\yquant@circuit@operator@maxtarget-\yquant@circuit@operator@mintarget+1}%
   \edef\yquant@circuit@operator@targets{%
      \yquant@list@range%
         \yquant@circuit@operator@mintarget%
         \yquant@circuit@operator@maxtarget%
   }%
   \let\yquant@circuit@operator@pctrls=\empty%
   \def\yquant@circuit@operator@minpctrl{2147483647}%
   \def\yquant@circuit@operator@maxpctrl{0}%
   \let\yquant@circuit@operator@numpctrl=\yquant@circuit@operator@maxpctrl%
   \let\yquant@circuit@operator@nctrls=\empty%
   \let\yquant@circuit@operator@minnctrl=\yquant@circuit@operator@minpctrl%
   \let\yquant@circuit@operator@maxnctrl=\yquant@circuit@operator@maxpctrl%
   \let\yquant@circuit@operator@numnctrl=\yquant@circuit@operator@numpctrl%
   \let\yquant@circuit@operator@minctrl=\yquant@circuit@operator@mintarget%
   \let\yquant@circuit@operator@maxctrl=\yquant@circuit@operator@maxtarget%
   % if we have the after attribute, we fake an align gate
   \ifdefined\yquant@lang@attr@after%
      \edef\jointindices{\yquant@circuit@operator@targets\yquant@lang@attr@after}%
      \yquant@circuit@align\jointindices%
      % and we also need to change the wire after the alignment
      \let\yquant@circuit@settype@to=\yquant@lang@create@type%
      \let\yquant@prepare@callback@prepare=\yquant@circuit@settype@prepare%
      \let\yquant@prepare@callback@draw=\yquant@circuit@settype%
      \expandafter\@thirdofthree%
   \else%
      \expandafter\@firstofone%
   \fi%
   % unless there is no name, no text and no after attribute, we now need an init gate
   {%
      \ifx\yquant@lang@attr@value\empty%
         \expandafter\@firstofone%
      \else%
         \expandafter\@secondoftwo%
      \fi%
   }%
   {%
      \ifx\yquant@lang@attr@name\empty%
         \expandafter\@gobble%
      \else%
         \expandafter\@firstofone%
      \fi%
   }%
   {%
      % there are no multi inits in this context
      \preto\yquant@attrs@remaining{internal/move label,}%
      \def\yquant@config@operator@minimum@width{0pt}%
      \unless\ifx\yquant@lang@attr@value\empty%
         % make sure to immediately remove the "clear" marker again if we have a text
         \yquant@for \i := \yquant@circuit@operator@mintarget to \yquant@circuit@operator@maxtarget {%
            \yquant@register@execclear@lastgate{\i}{init}%
         }%
      \fi%
      \expandafter\yquant@prepare%
         \expandafter{\yquant@lang@attr@value}%
         {/yquant/every label, /yquant/every initial label,%
          /yquant/every \yquant@lang@create@style\space label,%
          \ifnum\yquant@compat<2 \ifdefined\yquant@lang@attr@input /yquant/every input label\fi\fi}%
      \unless\ifx\yquant@lang@attr@name\empty%
         \ifnum\len=1 %
            \csxappto{\yquant@prefix draw}%
                     {\yquant@draw@alias{\yquant@lang@attr@name}}%
         \fi%
      \fi%
   }%
}
% END_FOLD

% BEGIN_FOLD Define own gates
\def\yquantdefinegate#1{%
   \ifcsname yquant@lang@#1\endcsname%
      \PackageError{yquant.sty}{Gate redefined}%
                   {The gate `#1' already exists. Use \string\yquantredefinegate\space if you really want to redefine it.}%
   \fi%
   \@ifnextchar[{\yquantdefinegate@i{#1}}%
                {\yquantdefinegate@i{#1}[/yquant/operators/every custom gate]}%
}

% while this command is provided to redefine gates, we do not clear the attributes that were once assigned to the previous one, so gate redefinition is discouraged!
\def\yquantredefinegate#1{%
   \unless\ifcsname yquant@lang@#1\endcsname%
      \PackageError{yquant.sty}{Unknown gate redefined}%
                   {The gate `#1' is unknown and cannot be redefined. Use \string\yquantdefinegate\space to define it.}%
   \fi%
   \@ifnextchar[{\yquantdefinegate@i{#1}}%
                {\yquantdefinegate@i{#1}[/yquant/operators/every custom gate]}%
}

\protected\long\def\yquantdefinegate@i#1[#2]#3{%
   \begingroup%
      \yquant@prepare@ifs@set%
      % usually, we will not be in a tikzpicture here, so all the commands that abbreviate some path operation are undefined!
      \tikz@installcommands%
      % While we want the content to be expanded, protect the most likely TikZ commands - the same ones that we usually substitute in \yquant@env@substikz.
      \protected\def\path{}%
      \let\scoped=\path%
      \let\scope=\path%
      \let\endscope=\path%
      \let\stopscope=\path%
      \protected@edef\yquantdefinegate@do{%
         \endgroup%
         \noexpand\pgfkeys{/yquant/operators/every #1/.code={%
            \yquant@config@operator@subcircuit@mangling@set{%
               \yquant@config@operator@subcircuit@mangling%
            }%
            \noexpand\pgfkeysalso{\unexpanded{#2}}%
         }}%
         \yquant@langhelper@declare@command%
            {#1}%
            \yquant@register@get@multiaslist%
            {%
               \let\noexpand\yquant@lang@attr@value=\expandafter\noexpand\csname yquant@lang@@#1\endcsname%
               \yquant@prepare@subcircuit{/yquant/operators/every #1}%
            }%
         % This does not clear the attributes for redefines, but makes at least sure nothing is marked as required that should not be.
         \yquant@langhelper@setup@attrs{#1}{}{}%
         % Now define the gate's content as a macro
         \def\expandafter\noexpand\csname yquant@lang@@#1\endcsname{%
            #3%
         }%
      }%
   \yquantdefinegate@do%
}

\def\yquantdefinebox#1{%
   \ifcsname yquant@lang@#1\endcsname%
      \PackageError{yquant.sty}{Gate redefined}%
                   {The gate `#1' already exists. Use \string\yquantredefinebox\space if you really want to redefine it.}%
   \fi%
   \yquantdefinebox@{#1}{}%
}

\def\yquantdefinemultibox#1{%
   \ifcsname yquant@lang@#1\endcsname%
      \PackageError{yquant.sty}{Gate redefined}%
                   {The gate `#1' already exists. Use \string\yquantredefinemultibox\space if you really want to redefine it.}%
   \fi%
   \yquantdefinebox@{#1}\yquant@register@get@allowmultitrue%
}

\def\yquantredefinebox#1{%
   \unless\ifcsname yquant@lang@#1\endcsname%
      \PackageError{yquant.sty}{Unknown gate redefined}%
                   {The gate `#1' is unknown and cannot be redefined. Use \string\yquantdefinebox\space to define it.}%
   \fi%
   \yquantdefinebox@{#1}{}%
}

\def\yquantredefinemultibox#1{%
   \unless\ifcsname yquant@lang@#1\endcsname%
      \PackageError{yquant.sty}{Unknown gate redefined}%
                   {The gate `#1' is unknown and cannot be redefined. Use \string\yquantdefinemultibox\space to define it.}%
   \fi%
   \yquantdefinebox@{#1}\yquant@register@get@allowmultitrue%
}

\def\yquantdefinebox@#1#2{%
   \@ifnextchar[{\yquantdefinebox@i{#1}{#2}}%
                {\yquantdefinebox@i{#1}{#2}[/yquant/operators/every box]}%
}

\protected\long\def\yquantdefinebox@i#1#2[#3]#4{%
   \pgfkeys{/yquant/operators/every #1/.style={#3}}%
   \begingroup%
      \yquant@prepare@ifs@set%
      \protected@edef\yquantdefinebox@do{%
         \endgroup%
         \yquant@langhelper@declare@command%
            {#1}%
            {\unexpanded{#2}}%
            {\yquant@prepare{#4}{/yquant/operators/every #1}}%
      }%
      \yquantdefinebox@do%
   \yquant@langhelper@setup@attrs{#1}{}{}%
}
% END_FOLD

% BEGIN_FOLD Box registers
% all-purpose box with arbitrary text
\yquant@langhelper@declare@command%
   {box}%
   \yquant@register@get@allowmultitrue%
   {%
      \expandafter\yquant@prepare%
         \expandafter{\yquant@lang@attr@value}%
         {/yquant/operators/every box}%
   }
\yquant@langhelper@setup@attrs{box}{value}{}

% Hadamard
\yquantdefinebox{h}{$H$}

% Pauli X (or NOT)
\yquantdefinebox{x}[/yquant/operators/every pauli]{$X$}

% Pauli Y
\yquantdefinebox{y}[/yquant/operators/every pauli]{$Y$}

% Pauli Z
\yquantdefinebox{z}[/yquant/operators/every pauli]{$Z$}

% sub-circuit: This is a nested circuit.
\yquant@langhelper@declare@command%
   {subcircuit}%
   \yquant@register@get@multiaslist%
   {%
      \yquant@prepare@subcircuit{/yquant/operators/every subcircuit}%
   }
\yquant@langhelper@setup@attrs{subcircuit}{value}{frameless,seamless,name mangling}
% END_FOLD

% BEGIN_FOLD other geometric shapes
% phase
\yquant@langhelper@declare@command%
   {phase}%
   {}%
   {%
      \edef\cmd{%
         \yquant@prepare%
            {}%
            {/yquant/operators/every phase, label={\unexpanded\expandafter{\yquant@lang@attr@value}}}%
      }%
      \cmd%
   }
\yquant@langhelper@setup@attrs{phase}{value}{}%

% two-qubit controlled x (symmetric notation)
\yquant@langhelper@declare@command%
   {xx}%
   \yquant@register@get@multiassingle%
   {%
      \yquant@prepare%
         {}%
         {/yquant/operators/every xx}%
   }
\yquant@langhelper@setup@attrs{xx}{}{}

% two-qubit controlled phase (symmetric notation)
\yquant@langhelper@declare@command@uncontrolled%
   {zz}%
   \yquant@register@get@multiassingle%
   {%
      \yquant@prepare%
         {}%
         {/yquant/operators/every zz}%
   }
\yquant@langhelper@setup@attrs{zz}{}{}

% slash (pseudo-operator, alternative indication for a bundle)
\yquant@langhelper@declare@command@uncontrolled%
   {slash}%
   {}%
   {%
      % temporarily squeeze most into the separation
      \pgfkeys{%
         /yquant/operator/minimum width=0pt,%
      }%
      \preto\yquant@attrs@remaining{internal/squeeze slash,}%
      \yquant@prepare%
         {}%
         {/yquant/operators/every slash}%
   }
\yquant@langhelper@setup@attrs{slash}{}{}

% swap
\yquant@langhelper@declare@command%
   {swap}%
   \yquant@register@get@multiassingle
   {%
      \yquant@prepare%
         {}%
         {/yquant/operators/every swap}%
   }
\yquant@langhelper@setup@attrs{swap}{}{}

% not
\yquant@langhelper@declare@command%
   {not}%
   {}%
   {%
      \yquant@prepare%
         {}%
         {/yquant/operators/every not}%
   }
\yquant@langhelper@setup@attrs{not}{}{}
% alias to cnot
\yquant@langhelper@declare@command@alias{cnot}{not}
\yquant@langhelper@setup@attrs{cnot}{}{}

% measure
\yquant@langhelper@declare@command@uncontrolled%
   {measure}%
   \yquant@register@get@allowmultitrue%
   {%
      \ifdefined\yquant@lang@attr@type%
         \yquant@register@type@fromstring\yquant@lang@attr@type\yquant@circuit@settype@to%
      \else%
         \let\yquant@circuit@settype@to=\yquant@register@type@c%
      \fi%
      \let\yquant@prepare@callback@prepare=\yquant@circuit@settype@prepare%
      \let\yquant@prepare@callback@draw=\yquant@circuit@settype%
      \unless\ifcsname yquant@lang@attr@value\endcsname%
         \let\yquant@lang@attr@value=\empty%
      \fi%
      \ifdefined\yquant@lang@attr@directcontrol%
         % direct control means that we must defer this operator until the next that has this/these targets as positive controls.
         \expandafter\yquant@prepare@injection%
            \expandafter{\yquant@lang@attr@value}%
            {/yquant/operators/every measure}%
      \else%
         \expandafter\yquant@prepare%
            \expandafter{\yquant@lang@attr@value}%
            {/yquant/operators/every measure}%
      \fi%
   }
\yquant@langhelper@setup@attrs{measure}{}{value,type,direct control,indirect control}

% measure (dmeter)
\yquant@langhelper@declare@command@uncontrolled%
   {dmeter}%
   \yquant@register@get@allowmultitrue%
   {%
      \ifdefined\yquant@lang@attr@type%
         \yquant@register@type@fromstring\yquant@lang@attr@type\yquant@circuit@settype@to%
      \else%
         \let\yquant@circuit@settype@to=\yquant@register@type@c%
      \fi%
      \let\yquant@prepare@callback@prepare=\yquant@circuit@settype@prepare%
      \let\yquant@prepare@callback@draw=\yquant@circuit@settype%
      \unless\ifcsname yquant@lang@attr@value\endcsname%
         \let\yquant@lang@attr@value=\empty%
      \fi%
      \expandafter\yquant@prepare%
         \expandafter{\yquant@lang@attr@value}%
         {/yquant/operators/every dmeter}%
   }
\yquant@langhelper@setup@attrs{dmeter}{}{value,type}
% END_FOLD

% BEGIN_FOLD miscellaneous
\yquant@langhelper@declare@command@uncontrolled%
   {barrier}%
   \yquant@register@get@allowmultitrue%
   {%
      \yquant@prepare%
         {}%
         {/yquant/operators/every barrier}%
   }%
\yquant@langhelper@setup@attrs{barrier}{}{}

\yquant@langhelper@declare@command@uncontrolled%
   {correlate}%
   {%
      % do not call \yquant@register@get@multiassingle, we do not want to install a different multi line style!
      \yquant@register@get@allowmultitrue%
      \let\yquant@register@multi@splitparts=\yquant@register@multi@splitparts@sepall%
   }{%
      \yquant@prepare%
         {}%
         {/yquant/operators/every wave}%
   }
\yquant@langhelper@setup@attrs{correlate}{}{}

\yquant@langhelper@declare@command@uncontrolled%
   {align}%
   {}%
   {%
      \yquant@circuit@align\yquant@circuit@operator@targets%
   }
\yquant@langhelper@setup@attrs{align}{}{}

\yquant@langhelper@declare@command@uncontrolled%
   {hspace}%
   {%
      \yquant@langhelper@validate\amount\dimen\yquant@lang@attr@value%
   }{%
      \yquant@circuit@hspace\yquant@circuit@operator@targets\amount%
   }
\yquant@langhelper@setup@attrs{hspace}{value}{}

\yquant@langhelper@declare@command@uncontrolled%
   {discard}%
   {}%
   {%
      \let\yquant@circuit@settype@to=\yquant@register@type@none%
      \yquant@circuit@actonwires%
         \yquant@circuit@settype@prepare%
         \yquant@circuit@settype%
         \yquant@circuit@operator@targets%
         {}%
   }
\yquant@langhelper@setup@attrs{discard}{}{}

\yquant@langhelper@declare@command@uncontrolled%
   {init}%
   {%
      \yquant@register@get@allowmultitrue%
      % we will count how many registers contain the "clean" flag, and only if this is equal to the number of targets, we apply the shift.
      \count8=0 %
   }%
   {%
      \ifdefined\yquant@lang@attr@type%
         \yquant@register@type@fromstring\yquant@lang@attr@type\yquant@circuit@settype@to%
      \else%
         % We don't know which type is desired. Scan all target registers and use the first wire that is available as a type.
         \let\yquant@circuit@settype@to=\yquant@register@type@none%
         \forlistloop\yquant@lang@@init@loop\yquant@circuit@operator@targets%
         \ifx\yquant@circuit@settype@to\yquant@register@type@none%
            % now we don't have a clue; assume it's a qubit
            \let\yquant@circuit@settype@to=\yquant@register@type@q%
         \fi%
         \edef\yquant@lang@attr@type{%
            \yquant@register@type@tostring\yquant@circuit@settype@to%
         }%
      \fi%
      \let\yquant@prepare@callback@prepare=\yquant@circuit@settype@prepare%
      \let\yquant@prepare@callback@draw=\yquant@circuit@settype%
      \let\yquant@prepare@multi=\yquant@prepare@multiinit%
      \ifnum\count8=\numexpr\yquant@circuit@operator@maxctrl-\yquant@circuit@operator@minctrl+1\relax%
         \def\yquant@config@operator@minimum@width{0pt}%
         \preto\yquant@attrs@remaining{internal/move label,}%
         \edef\cmd{%
            \yquant@prepare%
               {\unexpanded\expandafter{\yquant@lang@attr@value}}%
               {/yquant/every label, /yquant/every initial label,%
                /yquant/every \yquant@lang@attr@type\space label}%
         }%
      \else%
         \edef\cmd{%
            \yquant@prepare%
               {\unexpanded\expandafter{\yquant@lang@attr@value}}%
               {/yquant/every label, /yquant/every \yquant@lang@attr@type\space label}%
         }%
      \fi%
      \cmd%
   }
\yquant@langhelper@setup@attrs{init}{value}{type}

\protected\def\yquant@lang@@init@loop#1{%
   \ifyquant@firsttoken\yquant@register@multi{#1}{%
      \let\yquant@register@multi@contiguous=\yquant@lang@@init@multi@@extract%
      \@fifthoffive#1%
      % we should reset multi@contiguous to the original command; but this is really just a placeholder. As long as it is \protected, everything is fine.
   }{%
      \edef\yquant@circuit@settype@to{\yquant@register@get@type{#1}}%
   }%
   \unless\ifx\yquant@circuit@settype@to\yquant@register@type@none%
      \expandafter\listbreak%
   \fi%
}

\protected\def\yquant@lang@@init@multi@@extract#1#2#3{%
   \yquant@for \yquant@i := #1 to #2 {%
      \edef\yquant@circuit@settype@to{\yquant@register@get@type\yquant@i}%
      \unless\ifx\yquant@circuit@settype@to\yquant@register@type@none%
         \expandafter\yquant@for@break%
      \fi%
   }%
}

\yquant@langhelper@declare@command@uncontrolled%
   {output}%
   \yquant@register@get@allowmultitrue%
   \yquant@circuit@output%
\yquant@langhelper@setup@attrs{output}{value}{}

\yquant@langhelper@declare@command@uncontrolled%
   {inspect}%
   \yquant@register@get@allowmultitrue%
   {%
      \expandafter\yquant@prepare%
         \expandafter{\yquant@lang@attr@value}%
         {/yquant/operators/every inspect}%
   }
\yquant@langhelper@setup@attrs{inspect}{value}{}

\yquant@langhelper@declare@command@uncontrolled%
   {settype}%
   {}%
   {%
      \yquant@register@type@fromstring\yquant@lang@attr@value\yquant@circuit@settype@to%
      \yquant@circuit@actonwires%
         \yquant@circuit@settype@prepare%
         \yquant@circuit@settype%
         \yquant@circuit@operator@targets%
         {}%
   }
\yquant@langhelper@setup@attrs{settype}{value}{}

\ifnum\yquant@compat<2 %
   \def\yquant@lang@setwire{%
      \PackageWarning{yquant.sty}{`setwire' gate is deprecated as of yquant 0.1.2. Use `settype' instead.}%
      \yquant@lang@settype%
   }
   \yquant@langhelper@setup@attrs{setwire}{value}{}
\fi

\yquant@langhelper@declare@command@uncontrolled%
   {setstyle}%
   {}%
   {%
      \yquant@circuit@actonwires%
         \@gobbletwo%
         \yquant@circuit@setstyle%
         \yquant@circuit@operator@targets%
         {{\yquant@lang@attr@value}}%
   }
\yquant@langhelper@setup@attrs{setstyle}{value}{}

\yquant@langhelper@declare@command@uncontrolled%
   {addstyle}%
   {}%
   {%
      \yquant@circuit@actonwires%
         \@gobbletwo%
         \yquant@circuit@addstyle%
         \yquant@circuit@operator@targets%
         {{\yquant@lang@attr@value}}%
   }
\yquant@langhelper@setup@attrs{addstyle}{value}{}
% END_FOLD