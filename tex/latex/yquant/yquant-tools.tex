% from TeXbook, appendix D
\protected\def\yquant@futurenonspacelet#1{%
   \def\yquant@futurenonspacelet@cs{#1}%
   \afterassignment\yquant@futurenonspacelet@i\let\yquant@futurenonspacelet@next= %
}

\def\yquant@futurenonspacelet@i{%
   \expandafter\futurelet\yquant@futurenonspacelet@cs\yquant@futurenonspacelet@ii%
}

\def\yquant@futurenonspacelet@ii{%
   \expandafter\ifx\yquant@futurenonspacelet@cs\@sptoken%
      \expandafter\yquant@futurenonspacelet@iii%
   \else%
      \expandafter\yquant@futurenonspacelet@next%
   \fi%
}

\def\yquant@futurenonspacelet@iii{%
   \afterassignment\yquant@futurenonspacelet@i%
   \let\@eattoken= %
}

% a bit faster than nested \@firstoftwo/\@secondoftwo
% note \@thirdofthree is defined in the latex kernel already.
\long\def\@firstofthree#1#2#3{#1}%
\long\def\@secondofthree#1#2#3{#2}%
\long\def\@firstoffour#1#2#3#4{#1}%
\long\def\@secondoffour#1#2#3#4{#2}%
\long\def\@thirdoffour#1#2#3#4{#3}%
\long\def\@fourthoffour#1#2#3#4{#4}%
\long\def\@thirdandfourthoffour#1#2#3#4{#3#4}%
\long\def\@secondandthirdoffive#1#2#3#4#5{{#2}{#3}}
\long\def\@firstoffive#1#2#3#4#5{#1} % unused
\long\def\@secondoffive#1#2#3#4#5{#2}
\long\def\@thirdoffive#1#2#3#4#5{#3}
\long\def\@fourthoffive#1#2#3#4#5{#4} % unused
\long\def\@fifthoffive#1#2#3#4#5{#5}

% Loop #1 from min(#2, #3) to max(#2, #3), executing #4
\protected\def\yquant@for #1:=#2to#3#{%
   \yquant@for@aux#1{#2}{#3}%
}

% Loop #1 from max(#2, #3) down to min(#2, #3), executing #4
\protected\def\yquant@fordown #1:=#2downto#3#{%
   \yquant@fordown@aux#1{#2}{#3}%
}

\long\def\yquant@for@aux#1#2#3#4{%
   \ifnum#2<#3\relax%
      \numdef#1{#2}%
      % to allow for things like \yquant@for \i := \i to ..., expand the boundaries
      \expandafter\yquant@for@loop\expandafter#1\expandafter{\the\numexpr#3+1\relax}{#4}%
   \else%
      \numdef#1{#3}%
      \expandafter\yquant@for@loop\expandafter#1\expandafter{\the\numexpr#2+1\relax}{#4}%
   \fi%
}

\long\def\yquant@fordown@aux#1#2#3#4{%
   \ifnum#2>#3\relax%
      \numdef#1{#2}%
      % to allow for things like \yquant@for \i := \i to ..., expand the boundaries
      \expandafter\yquant@fordown@loop\expandafter#1\expandafter{\the\numexpr#3-1\relax}{#4}%
   \else%
      \numdef#1{#3}%
      \expandafter\yquant@fordown@loop\expandafter#1\expandafter{\the\numexpr#2-1\relax}{#4}%
   \fi%
}

\long\def\yquant@for@loop#1#2#3{%
   \loop%
      \ifnum#1<#2\relax%
         #3%
         \numdef#1{#1+1}%
   \repeat%
}

\long\def\yquant@fordown@loop#1#2#3{%
   \loop%
      \ifnum#1>#2\relax%
         #3%
         \numdef#1{#1-1}%
   \repeat%
}

\def\yquant@for@break{%
   \fi%
   \iffalse%
}

% Def #1 to be the minimum of #2, ... until \relax
\protected\def\yquant@min#1{%
   \def#1{2147483647}%
   \def\yquant@min@var{#1}%
   \yquant@min@loop%
}

\def\yquant@min@loop#1{%
   \unless\ifx#1\relax\relax%
      \ifnum#1<\yquant@min@var\relax%
         \expandafter\edef\yquant@min@var{#1}%
      \fi%
      \expandafter\yquant@min@loop%
   \fi%
}

% Def #1 to be the maximum of #2, ... until \relax
\protected\def\yquant@max#1{%
   \def#1{-2147483647}%
   \def\yquant@max@var{#1}%
   \yquant@max@loop%
}

\def\yquant@max@loop#1{%
   \unless\ifx#1\relax\relax%
      \ifnum#1>\yquant@max@var\relax%
         \expandafter\edef\yquant@max@var{#1}%
      \fi%
      \expandafter\yquant@max@loop%
   \fi%
}

% Cleanup global tokens after environment
\protected\def\yquant@cleanup@csadd#1{%
   \csxappto{\yquant@prefix cleanup}{\expandafter\noexpand\csname#1\endcsname}%
}

\def\yquant@cleanup#1#2{%
   \global\undef#1%
   \unless\ifx|#2%
      \expandafter\yquant@cleanup\expandafter#2%
   \fi%
}

% Executes #3 if #1 (single token!) is equal (\ifx) to the first token of #2, and #4 else.
\def\ifyquant@firsttoken#1#2{%
   % First check whether #2 is present at all...
   \ifstrempty{#2}{%
      \expandafter\@secondoftwo%
   }{%
      \ifyquant@firsttoken@aux#1#2\yquant@sep%
   }%
}

\def\ifyquant@firsttoken@aux#1#2#3\yquant@sep{%
   \ifx#1#2%
      \expandafter\expandafter\expandafter\@firstoftwo%
   \else%
      \expandafter\expandafter\expandafter\@secondoftwo%
   \fi%
}

% Executes #3 if #1 begins with #2, and #4 else - non-expandable
\protected\def\ifyquant@beginswith#1#2{%
   \def\ifyquant@beginswith@##1#2##2\yquant@end{%
      \ifstrempty{##1}%
   }%
   \ifyquant@beginswith@#1#2\yquant@end%
}

% absolute value of a dimension
\def\yquant@abs#1{%
   \ifdim#1<0pt %
      \the\dimexpr-\dimexpr#1\relax\relax%
   \else%
      #1%
   \fi%
}

% Sortlist related macros.
\newcount\yquant@sort@count

\protected\def\yquant@sort@clear{%
   % Probably cleanup used macros?
   \yquant@sort@count=0 %
}

\protected\def\yquant@sort@eadd#1{%
   \csedef{yquant@sort@item\the\yquant@sort@count}{#1}%
   \advance \yquant@sort@count by 1 %
}

% Perform quicksort on the stored sortlist.
% #1: compare macro that expands to ##3 if its second argument is strictly larger than its first or to ##4 else
\protected\def\yquant@sort#1{%
   \let\yquant@sort@cmp=#1%
   \expandafter\yquant@sort@aux\expandafter0\expandafter{\the\numexpr\yquant@sort@count-1\relax}%
}

% Returns the first item in the unordered sortlist when compared according to #1 and stores it in #2.
\protected\def\yquant@sort@findfirst#1#2{%
   \ifnum\yquant@sort@count<1 %
      \let#2=\empty%
   \else%
      \letcs#2{yquant@sort@item0}%
      \count0=1 %
      \loop%
         \ifnum\count0<\yquant@sort@count\relax%
            \expandafter\expandafter\expandafter\expandafter\expandafter\expandafter\expandafter#1%
               \expandafter\expandafter\expandafter\expandafter\expandafter\expandafter\expandafter{%
               \expandafter\expandafter\expandafter#2%
               \expandafter\expandafter\expandafter}%
               \expandafter\expandafter\expandafter{%
                  \csname yquant@sort@item\the\count0\endcsname%
               }%
               \relax{%
               \letcs#2{yquant@sort@item\the\count0}%
            }%
            \advance\count0 by 1 %
      \repeat%
   \fi%
}

\def\yquant@sort@ascending#1#2{%
   \ifnum#2>#1 %
      \expandafter\@firstoftwo%
   \else%
      \expandafter\@secondoftwo%
   \fi%
}

\protected\def\yquant@sort@aux#1#2{%
   \ifnum#1<#2\relax%
      \yquant@sort@divide{#1}{#2}%
      \edef\cmd{%
         \noexpand\yquant@sort@aux{#1}{\the\numexpr\count0-1\relax}%
         \noexpand\yquant@sort@aux{\the\numexpr\count0+1\relax}{#2}%
      }%
      \cmd%
   \fi%
}

\def\iftrue@hidden{\iftrue}%
\def\iffalse@hidden{\iffalse}%
\protected\def\yquant@sort@divide#1#2{%
   \count0=#1\relax% i
   \count2=#2\relax% j
   \advance\count2 by -1 %
   \letcs\yquant@sort@pivot{yquant@sort@item#2}%
   \loop%
      % search an item from the left that is larger or equal to the pivot
      {% protect the outer loop from finding \repeat
         \loop%
            \ifnum\count0<#2\relax%
               \expandafter\expandafter\expandafter\expandafter\expandafter\expandafter\expandafter\yquant@sort@cmp%
                  \expandafter\expandafter\expandafter\expandafter\expandafter\expandafter\expandafter{%
                  \expandafter\expandafter\expandafter\yquant@sort@pivot%
                  \expandafter\expandafter\expandafter}%
                  \expandafter\expandafter\expandafter{%
                     \csname yquant@sort@item\the\count0\endcsname%
                  }{%
                     \expandafter\iffalse@hidden%
                  }{%
                     \advance\count0 by 1 %
                     \expandafter\iftrue@hidden%
                  }%
            \else%
               \expandafter\iffalse@hidden%
            \fi%
         \repeat%
         \expandafter%
      }%
      \expandafter\count\expandafter0\expandafter=\the\count0\relax%
      % search an item from the right that is small than the pivot
      {% protect the outer loop from finding \repeat
         \loop%
            \ifnum\count2>#1\relax%
               \expandafter\expandafter\expandafter\expandafter\expandafter\expandafter\expandafter\yquant@sort@cmp%
                  \expandafter\expandafter\expandafter\expandafter\expandafter\expandafter\expandafter{%
                  \expandafter\expandafter\expandafter\yquant@sort@pivot%
                  \expandafter\expandafter\expandafter}%
                  \expandafter\expandafter\expandafter{%
                     \csname yquant@sort@item\the\count2\endcsname%
                  }{%
                     \advance\count2 by -1 %
                     \expandafter\iftrue@hidden%
                  }{%
                     \expandafter\iffalse@hidden%
                  }%
            \else%
               \expandafter\iffalse@hidden%
            \fi%
         \repeat%
         \expandafter%
      }%
      \expandafter\count\expandafter2\expandafter=\the\count2\relax%
      \ifnum\count0<\count2 %
         % swap item i <> item j
         \letcs\tmp{yquant@sort@item\the\count0}%
         \csletcs{yquant@sort@item\the\count0}{yquant@sort@item\the\count2}%
         \cslet{yquant@sort@item\the\count2}\tmp%
      \fi%
      \ifnum\count0<\count2 %
   \repeat%
   \letcs\tmp{yquant@sort@item\the\count0}%
   \csletcs{yquant@sort@item\the\count0}{yquant@sort@item#2}%
   \cslet{yquant@sort@item#2}\tmp%
}

% Add an internal etoolbox list to the sorted items
\def\yquant@sort@addlist#1{%
   \forlistloop\yquant@sort@addlist@aux#1%
}

\protected\def\yquant@sort@addlist@aux#1{%
   \csdef{yquant@sort@item\the\yquant@sort@count}{#1}%
   \advance\yquant@sort@count by 1 %
}

% Sorts an internal etoolbox list #1 using macro #2
\protected\def\yquant@sort@list#1#2{%
   \begingroup%
      \yquant@sort@count=0 %
      \yquant@sort@addlist#1%
      \yquant@sort#2%
      \let#1=\empty%
      \count0=0 %
      \loop%
         \ifnum\count0<\yquant@sort@count%
            \expandafter\expandafter\expandafter\listadd%
            \expandafter\expandafter\expandafter#1%
            \expandafter\expandafter\expandafter{%
               \csname yquant@sort@item\the\count0\endcsname%
            }%
            \advance\count0 by 1 %
      \repeat%
      \expandafter%
   \endgroup%
   \expandafter\def\expandafter#1\expandafter{#1}%
}

\protected\def\yquant@sort@dolistloop{%
   \count0=0 %
   \loop%
      \ifnum\count0<\yquant@sort@count%
         \expandafter\expandafter\expandafter\do%
         \expandafter\expandafter\expandafter{%
            \csname yquant@sort@item\the\count0\endcsname%
         }%
         \advance\count0 by 1 %
   \repeat%
}

\begingroup
\catcode`\|=3
\catcode`\&=3
\gdef\yquant@list@delim{|}

\protected\gdef\yquant@list@dequeue#1#2{%
   \expandafter\ifblank\expandafter{#1}{%
      \let#2=\empty%
   }{%
      \expandafter\yquant@list@dequeue@i#1\etb@lst@q@end{#1}{#2}\def%
   }%
}%

\protected\gdef\yquant@list@dequeue@i#1|#2\etb@lst@q@end#3#4#5{%
   \def#4{#1}%
   #5#3{#2}%
}

\protected\gdef\yquant@list@gdequeue#1#2{%
   \expandafter\ifblank\expandafter{#1}{%
      \let#2=\empty%
   }{%
      \expandafter\yquant@list@dequeue@i#1\etb@lst@q@end{#1}{#2}\gdef%
   }%
}

\protected\gdef\yquant@list@eupdateorinsert#1#2#3{%
   % the list items are of the form <number>:<value>. If an item is present where <number> = #2, then set its <value> to #3, if #3 is greater. If no item is present, add it.
   \begingroup%
      \def\etb@tempa##1|#2:##2|##3&{%
         \endgroup%
         \ifstrempty{##3}{%
            \eappto#1{#2:#3|}%
         }{%
            \ifdim##2<#3 %
               \yquant@list@ereplace#1{#2:##2}{#2:#3}%
            \fi%
         }%
      }%
   \expandafter\etb@tempa\expandafter|#1|#2:#3|&%
}

\protected\gdef\yquant@list@ereplace#1#2#3{%
   \begingroup%
      \def\etb@tempa##1|#2|##2&{%
         \endgroup%
         \ifstrempty{##1}{%
            \edef#1{%
               \unexpanded{##1}|#3|\unexpanded{##2}%
            }%
         }{%
            \edef#1{%
               \unexpanded\expandafter{\@gobble##1}|#3|\unexpanded{##2}%
            }%
         }%
      }%
   \expandafter\etb@tempa\expandafter|#1&
}

% expands to the contents of a list ranging from #1 to #2 (ascending)
\gdef\yquant@list@range#1#2{%
   \ifnum#1>#2 %
      \expandafter\@firstoftwo%
   \else%
      \expandafter\@secondoftwo%
   \fi{}{%
      #1|%
      \expandafter\yquant@list@range\expandafter{\the\numexpr#1+1\relax}{#2}%
   }%
}
\endgroup

% performs #3 if #1 is a subset of #2; else, performs #4
\protected\def\ifyquant@registersubset#1#2{%
   \begingroup%
      \let\yquant@register@multi=\@fourthoffour
      \def\yquant@register@multi@contiguous##1##2##3{\yquant@list@range{##1}{##2}}%
      \edef\listA{#1}%
      \edef\listB{#2}%
      \let\ifsuccess=\iftrue%
      \forlistloop\yquant@registersubset@loop\listA%
      \expandafter%
   \endgroup%
   \ifsuccess%
      \expandafter\@firstoftwo%
   \else%
      \expandafter\@secondoftwo%
   \fi%
}

\protected\def\yquant@registersubset@loop#1{%
   \ifinlist{#1}\listB\relax{%
      \let\ifsuccess=\iffalse%
      \listbreak%
   }%
}

\def\ifyquant@OR#1#2{%
   #1%
      \expandafter\@firstoftwo%
   \else%
      #2%
         \expandafter\expandafter\expandafter\@firstoftwo%
      \else%
         \expandafter\expandafter\expandafter\@secondoftwo%
      \fi%
   \fi%
}

% #1 is a pgf soft path. We extract the maximum x position at the y position specified in #2 and assign it to \dimen0, which is translated to the user coordinate system.
\protected\def\yquant@softpath@extractmaxxat#1#2{%
   \begingroup%
      \dimen0=-16000pt %
      \dimen2=#2 %
      \pgftransforminvert%
      \let\pgfsyssoftpath@movetotoken=\yquant@softpath@extractmaxxat@moveto%
      \let\pgfsyssoftpath@linetotoken=\yquant@softpath@extractmaxxat@lineto%
      \let\pgfsyssoftpath@curvetosupportatoken=\yquant@softpath@extractmaxxat@curveto%
      \let\pgfsyssoftpath@rectcornertoken=\yquant@softpath@extractmaxxat@rectto%
      \let\pgfsyssoftpath@closepath=\@gobbletwo%
      % the specialroundtoken (undocumented) is \@gobbletwo by default.
      #1%
      \expandafter%
   \endgroup%
   \expandafter\dimen\expandafter0\expandafter=\the\dimen0  %
}

\protected\def\yquant@softpath@extractmaxxat@update#1{%
   \ifdim\dimen0<#1 %
      \dimen0=#1 %
   \fi%
}

\protected\def\yquant@softpath@extractmaxxat@moveto#1#2{%
   \pgfpointtransformed{\pgfqpoint{#1}{#2}}%
   \dimen4=\pgf@x %
   \dimen6=\pgf@y %
}

\protected\def\yquant@softpath@extractmaxxat@lineto#1#2{%
   \pgfpointtransformed{\pgfqpoint{#1}{#2}}%
   \ifyquant@OR{\ifdim\dimen4>\dimen0 }{\ifdim\pgf@x>\dimen0 }{%
      \ifdim\dimen6=\dimen2 %
         \yquant@softpath@extractmaxxat@update{\dimen4}%
      \else%
         \ifdim\dimen6<\dimen2 %
            \unless\ifdim\pgf@y<\dimen2 %
               \expandafter\yquant@softpath@extractmaxxat@update\expandafter{\the\dimexpr%
                  \dimen4+% x0
                  \dimexpr\pgf@x-\dimen4\relax*% (x1-x0)
                  \dimexpr\dimen2-\dimen6\relax/\dimexpr\pgf@y-\dimen6\relax% (y-y0)/(y1-y0)
               \relax}%
            \fi%
         \else%
            \unless\ifdim\pgf@y>\dimen2 %
               \expandafter\yquant@softpath@extractmaxxat@update\expandafter{\the\dimexpr%
                  \dimen4+% x0
                  \dimexpr\pgf@x-\dimen4\relax*% (x1-x0)
                  \dimexpr\dimen2-\dimen6\relax/\dimexpr\pgf@y-\dimen6\relax% (y-y0)/(y1-y0)
               \relax}%
            \fi%
         \fi%
      \fi%
   }\relax%
   \dimen4=\pgf@x%
   \dimen6=\pgf@y%
}

\protected\def\yquant@softpath@extractmaxxat@curveto@checkx{%
   % \dimen11 holds our only candidate for t. Is it within the curve?
   \unless\ifdim\dimen11<0pt %
      \unless\ifdim\dimen11>1pt %
         % it is. \dimen4: x0, \pgf@xa: xa, \pgf@xb: xb, \pgf@xc: x1
         \begingroup%
            \dimen12=\dimexpr1pt-\dimen11\relax% 1 - t
            \dimen13=\dimexpr\dimen11*\dimen11/65535\relax% t^2
            \dimen14=\dimexpr\dimen12*\dimen12/65535\relax% (1 - t)^2
            \dimen255=\dimexpr\dimen13*\dimen11/65535*\pgf@xc/65535+% t^3 x1
                              3\dimen13*\dimen12/65535*\pgf@xb/65535+% t^2(1 - t) xb
                              \dimen14*\dimen12/65535*\dimen4/65535+% (1 - t)^3 x0
                              3\dimen11*\dimen14/65535*\pgf@xa/65535% 3t(1 - t)^2 xa
                      \relax%
            \expandafter%
         \endgroup%
         \expandafter\yquant@softpath@extractmaxxat@update\expandafter{\the\dimen255}%
      \fi%
   \fi%
}

\protected\def\yquant@softpath@extractmaxxat@curveto#1#2\pgfsyssoftpath@curvetosupportbtoken#3#4\pgfsyssoftpath@curvetotoken#5#6{%
   % There's really no good way to do this apart from solving the Bézier curve (a third-order polynomial). Let's do it. (Yes, this is inefficient, but if someone substitutes the rectangular box of a subcircuit by a more fancy design, this is not our fault).
   % Parametrized by t, the x coordinates of the curve are
   % x0 + 3 (xa - x0) t + 3 (x0 - 2xa + xb) t^2 + (3xa - 3xb + x1 - x0) t^3
   % where x0 = \dimen4 (the moveto point), xa = #1, xb = #3, x1 = #5.
   % Likewise for y:
   %       y0 = \dimen6 (the moveto point), ya = #2, yb = #4, y1 = #6.
   \pgfpointtransformed{\pgfqpoint{#1}{#2}}%
   \pgf@xa=\pgf@x%
   \pgf@ya=\pgf@y%
   \pgfpointtransformed{\pgfqpoint{#3}{#4}}%
   \pgf@xb=\pgf@x%
   \pgf@yb=\pgf@y%
   \pgfpointtransformed{\pgfqpoint{#5}{#6}}%
   \pgf@xc=\pgf@x%
   \pgf@yc=\pgf@y%
   % We first solve the third-order polynomial for t using the y value, then plug it back into the x value.
   % TODO: this is accurate to approx. 3 digits. Can this be improved by reformulating Cardanos formula to involve less divisions?
   \begingroup%
      % We need so may dimensions that we break with TeX's convention for their use.
      % for the multiplications with and divisions by dimensions, we exploit that eTeX fuses muldiv to 64 bits. Further note that each dimension has a scaling factor of 65535 for sp<->pt conversion. This is why don't factor out divisions (which would be more efficient, but not give the benefit of 64bit accuracy).
      % a = 3(ya - yb) + (y1 - y0)
      \dimen1=\dimexpr3\pgf@ya-3\pgf@yb+\pgf@yc-\dimen6\relax%
      \ifdim\dimen1=0pt %
         % this is only a quadratic curve!
         % b = 3(y0 - 2ya + yb)
         \dimen3=\dimexpr3\dimexpr\dimen6-2\pgf@ya+\pgf@yb\relax*65535\relax%
         % c: 3(ya - y0)
         \dimen5=\dimexpr3\dimexpr\pgf@ya-\dimen6\relax*65535\relax%
         % d: y0 - <desired y>
         \dimen7=\dimexpr\dimexpr\dimen6-\dimen2\relax*65535\relax%
         % check the discriminant of the equation
         \dimen8=\dimexpr\dimen3*\dimen3/65535-4\dimen3*\dimen7/65535\relax%
         \unless\ifdim\dimen8<0pt%
            % there are two potential candidates, (-c +- sqrt(c^2 - 4b d))/2b
            \pgfmathsqrt@{\the\dimen8\@gobbletwo}%
            \dimen11=\dimexpr\dimexpr-\dimen5+\pgfmathresult pt\relax*65535/%
                             \dimexpr2\dimen3\relax\relax%
            \yquant@softpath@extractmaxxat@curveto@checkx%
            \dimen11=\dimexpr\dimexpr-\dimen5-\pgfmathresult pt\relax*65535/%
                             \dimexpr2\dimen3\relax\relax%
            \yquant@softpath@extractmaxxat@curveto@checkx%
         \fi%
      \else%
         % We will simplify by directly dividing all coefficients by a
         % b = 3(y0 - 2ya + yb)
         \dimen3=\dimexpr3\dimexpr\dimen6-2\pgf@ya+\pgf@yb\relax*65535/\dimen1\relax%
         % c: 3(ya - y0)
         \dimen5=\dimexpr3\dimexpr\pgf@ya-\dimen6\relax*65535/\dimen1\relax%
         % d: y0 - <desired y>
         \dimen7=\dimexpr\dimexpr\dimen6-\dimen2\relax*65535/\dimen1\relax%
         % Note that now our a value (\dimen1) is no longer needed, it is one.
         % check the discriminant of the equation
         % Q = (3c - b^2)/9
         \dimen8=\dimexpr\dimexpr3\dimen5-\dimen3*\dimen3/65535\relax/9\relax%
         % R = (9bc - 27d - 2b^3)/54 = bc/6 - d/2 - b^3/27
         \dimen9=\dimexpr\dimen3*\dimen5/393210-% 6*65535
                          .5\dimen7-%
                          \dimen3*\dimen3/65535*\dimen3/1769445% 27*65535
                  \relax%
         % D = Q^3 + R^2
         \dimen10=\dimexpr\dimen8*\dimen8/65535*\dimen8/65535+\dimen9*\dimen9/65535\relax%
         \ifdim\dimen10>0pt %
            % only one real root: y_1 = S + T - b/3a
            % S = cbrt(R + sqrt(Q^3 + R^2))
            % T = cbrt(R - sqrt(Q^3 + R^2))
            \pgfmathsqrt@{\the\dimen10\@gobbletwo}%
            \dimen12=\dimexpr\dimen9+\pgfmathresult pt\relax%
            \dimen13=\dimexpr\dimen9-\pgfmathresult pt\relax%
            \ifdim\dimen12>0pt %
               \pgfmathpow@{\the\dimen12\@gobbletwo}{.3333333333}%
               \dimen11=\pgfmathresult pt %
            \else%
               \pgfmathpow@{\the\dimexpr-\dimen12\relax\@gobbletwo}{.3333333333}%
               \dimen11=-\pgfmathresult pt %
            \fi%
            \ifdim\dimen13>0pt %
               \pgfmathpow@{\the\dimen13\@gobbletwo}{.3333333333}%
               \dimen11=\dimexpr\dimen11+\pgfmathresult pt-.33333333333\dimen3\relax%
            \else%
               \pgfmathpow@{\the\dimexpr-\dimen13\relax\@gobbletwo}{.3333333333}%
               \dimen11=\dimexpr\dimen11-\pgfmathresult pt-.33333333333\dimen3\relax%
            \fi%
            \yquant@softpath@extractmaxxat@curveto@checkx%
         \else%
            \ifdim\dimen10=0pt %
               % easiest case, three real roots, two of which are equal:
               % y_1 = 2cbrt(R) - b/3a
               % y_2, x_3 = -cbrt(R) - b/3a
               \ifdim\dimen9>0pt %
                  \pgfmathpow@{\the\dimen9\@gobbletwo}{.3333333333}%
                  \dimen15=\pgfmathresult pt %
               \else%
                  \pgfmathpow@{\the\dimexpr-\dimen9\relax\@gobbletwo}{.3333333333}%
                  \dimen15=-\pgfmathresult pt %
               \fi%
               \dimen11=\dimexpr2\dimen15-.33333333333\dimen3\relax%
               \yquant@softpath@extractmaxxat@curveto@checkx%
               % check the next candidate
               \dimen11=\dimexpr-\dimen15-.33333333333\dimen3\relax%
               \yquant@softpath@extractmaxxat@curveto@checkx%
            \else%
               % nastiest case, three distinct real roots which we can find only by taking a complex-valued cube root.
               % p + i q = cbrt(R + i sqrt(|D|))
               \pgfmathsqrt@{\the\dimexpr-\dimen10\relax\@gobbletwo}%
               \dimen10=\pgfmathresult pt %
               % Let us first find the absolute value
               \dimen12=\dimexpr\dimen9*\dimen9/65535+\dimen10*\dimen10/65535\relax%
               \pgfmathpow@{\the\dimen12\@gobbletwo}{.1666666667}%
               \dimen12=\pgfmathresult pt%
               % then we need 1/3 the argument of R + i sqrt(|D|).
               \pgfmathatantwo@{\the\dimen10\@gobbletwo}{\the\dimen9\@gobbletwo}%
               \dimen13=.3333333333\dimexpr\pgfmathresult pt\relax%
               % and then the real and imaginary parts as cosine and sine.
               \pgfmathcos@{\the\dimen13\@gobbletwo}%
               \dimen14=\dimexpr\pgfmathresult\dimen12\relax%
               \pgfmathsin@{\the\dimen13\@gobbletwo}%
               \dimen15=\dimexpr\pgfmathresult\dimen12\relax%
               % Now the candidates are
               % y_1 = 2p - b/3a
               % y_2 = -p - sqrt(3)q - b/3a
               % y_3 = -p + sqrt(3)q - b/3a
               \dimen11=\dimexpr2\dimen14-.33333333333\dimen3\relax%
               \yquant@softpath@extractmaxxat@curveto@checkx%
               \dimen11=\dimexpr-\dimen14-1.732050808\dimen15-.33333333333\dimen3\relax%
               \yquant@softpath@extractmaxxat@curveto@checkx%
               \dimen11=\dimexpr-\dimen14+1.732050808\dimen15-.33333333333\dimen3\relax%
               \yquant@softpath@extractmaxxat@curveto@checkx%
            \fi%
         \fi%
      \fi%
      % Now after all these calculations, \dimen0 was updated within the group. Make available outside.
      \expandafter%
   \endgroup%
   \expandafter\dimen\expandafter0\expandafter=\the\dimen0  %
   \dimen4=\pgf@xc %
   \dimen6=\pgf@yc %
}

\protected\def\yquant@softpath@extractmaxxat@rectto#1#2\pgfsyssoftpath@rectsizetoken#3#4{%
   % #1: lower left x, #2: lower left y, #3: width, #4: height
   \pgfpointtransformed{\pgfqpoint{#1}{#2}}%
   \pgf@xa=\pgf@x%
   \pgf@ya=\pgf@y%
   \pgfpointtransformed{\pgfqpoint{\dimexpr#1+#3\relax}{\dimexpr#2+#4\relax}}%
   % (\pgf@xa, \pgf@ya) one corner, (\pgf@x, \pgf@y) other corner
   \ifdim\pgf@y>\pgf@ya %
      \unless\ifdim\pgf@ya>\dimen2 %
         \unless\ifdim\pgf@y<\dimen2 %
            \ifdim\pgf@x>\pgf@xa %
               \yquant@softpath@extractmaxxat@update\pgf@x%
            \else%
               \yquant@softpath@extractmaxxat@update\pgf@xa%
            \fi%
         \fi%
      \fi%
   \else%
      \unless\ifdim\pgf@ya<\dimen2 %
         \unless\ifdim\pgf@y>\dimen2 %
            \ifdim\pgf@x>\pgf@xa %
               \yquant@softpath@extractmaxxat@update\pgf@x%
            \else%
               \yquant@softpath@extractmaxxat@update\pgf@xa%
            \fi%
         \fi%
      \fi%
   \fi%
}