\newcount\yquant@draw@@currentcontroltype%

\protected\long\def\yquant@draw@create#1#2#3#4#5#6{%
   \begingroup%
      \dimdef\wireypos{\yquant@register@get@y{#1}}%
      \ifcsname\yquant@prefix xshift\endcsname%
         \dimdef\createxpos{#2+\csname\yquant@prefix xshift\endcsname}%
      \else%
         \dimdef\createxpos{#2}%
      \fi%
      \ifstrempty{#5}{%
         % For empty labels, we still put the node at the appropriate position as it may needs to be referenced, but we will not let it effect the bounding box (so that the left end is not shifted), and we don't need an inner separation, so that the label is truely just a coordinate. However, we manually take care of extending the vertical bounding box appropriately, so that if this wire is the last or first one, contains no text and no gates, it is not clipped away.
         \path[overlay]%
            (\createxpos, \wireypos)%
            coordinate[name prefix=, name suffix=, name=yquantbox];%
         \let\pgfshapeclippathhorzresult=\empty%
         \tikzset{/yquant/every wire}%
         \pgfpointanchor{yquantbox}{east}%
         % 2 \pgflinewidth is the separation of cbit and qubits lines, and .5 for the line itself
         \ifdim\dimexpr\pgf@y-2.5\pgflinewidth\relax<\pgf@picminy%
            \global\pgf@picminy=\dimexpr\pgf@y-2.5\pgflinewidth\relax%
         \fi%
         \ifdim\dimexpr\pgf@y+2.5\pgflinewidth\relax>\pgf@picmaxy%
            \global\pgf@picmaxy=\dimexpr\pgf@y+2.5\pgflinewidth\relax%
         \fi%
      }{%
         \edef\cmd{%
            \noexpand\path%
               (\createxpos, \wireypos)%
               node[/yquant/every label, /yquant/every initial label,%
                    /yquant/every #3 label\ifx1#4, /yquant/every input label\fi,%
                    name prefix=, name suffix=, name=yquantbox]%
                   {\unexpanded{#5}};%
            \ifcsname\yquant@prefix registermap@#1\endcsname%
               \pgfshapeclippath{yquantbox}%
                                {/yquant/every label, /yquant/every initial label,%
                                 /yquant/every #3 label\ifx1#4, /yquant/every input label\fi}%
            \fi%
         }%
         \cmd%
      }%
      \ifcsname\yquant@prefix registermap@#1\endcsname%
         % if this is an alias, the creation is just an extension
         \if\yquant@register@type@none\yquant@register@get@type{#1}%
            \ifstrequal{#3}{initial}{%
               \yquant@circuit@extendwire{#1}{east}%
            }{%
               % however, our inner wire is present, while the outer wire was discarded.
               \tikzset{/yquant/every wire}%
               \pgfpointanchor{yquantbox}{east}%
               \yquant@register@set@lastwire{#1}{%
                  {\the\pgf@x}{\the\pgf@x}{}%
                  {\unexpanded\expandafter{\pgfshapeclippathhorzresult}}%
               }%
            }%
         \else%
            \yquant@circuit@extendwire{#1}{east}%
         \fi%
      \else%
         % set the wire style to have the correct \pgflinewidth available (we don't allow individual line widths for different types of wires)
         \tikzset{/yquant/every wire}%
         \pgfpointanchor{yquantbox}{east}%
         \ifstrequal{#3}{initial}{%
            \let\newtype=\yquant@register@type@none%
         }{%
            \yquant@register@type@fromstring{#3}\newtype%
         }%
         \yquant@register@set@typelastwire{#1}{\newtype{{\the\pgf@x}{\the\pgf@x}{}{}}}%
      \fi%
      \ifstrempty{#6}\relax{%
         \pgfnodealias{\tikz@pp@name{#6}}{yquantbox}%
      }%
   \endgroup%
}

\protected\long\def\yquant@draw@group#1#2#3#4#5#6{%
   \begingroup%
      \yquant@register@get@maxxlist\yquant@draw@@x{#6}%
      \ifx F#2%
         \yquant@draw@@currentcontroltype=0 %
      \else%
         \yquant@draw@@currentcontroltype=\yquant@register@get@type{#2}\relax%
      \fi%
      \let\yquant@circuit@extendcontrolline@cmd=\empty%
      \let\yquant@circuit@extendcontrolline@prev=\relax%
      \let\yquant@circuit@extendcontrolline@clip=\empty%
      \let\yquant@circuit@extendmultiline@total=\empty%
      \def\yquant@draw@@width{#1}%
      \def\yquant@draw@@style{#4}%
      \def\yquant@draw@@content{#5}%
      \def\yquant@draw@@idx@content{0}%
      \def\yquant@draw@@idx@pcontrol{0}%
      \def\yquant@draw@@idx@ncontrol{0}%
      % If the quotes library is loaded, activate it. (else, this is by default \relax)
      \tikz@enable@node@quotes%
      \yquant@config@operator@position@moveleftfalse%
      \yquant@config@operator@position@advancetrue%
      \yquant@set{#3}%
      \ifyquant@config@operator@position@moveleft%
         \ifcsname\yquant@prefix xshift\endcsname%
            \dimdef\yquant@draw@@x{\csname\yquant@prefix xshift\endcsname}%
            \letcs\newx{\yquant@prefix xshift}%
         \else%
            \dimdef\yquant@draw@@x{0pt}%
            \def\newx{1sp}%
         \fi%
         \ifyquant@config@operator@position@advance%
            \ifdim\newx=0pt %
               \def\newx{1sp}%
            \fi%
            \forlistloop\yquant@draw@group@advance{#6}%
         \fi%
      \else%
         % 1sp: special marker = no gates, zero position, but initial text present
         \ifdim\yquant@draw@@x=1sp %
            % just add the separation, drop this special 1sp marker
            \let\yquant@draw@@x=\yquant@config@operator@sep%
         \else%
            \ifdim\yquant@draw@@x>1sp %
               \dimdef\yquant@draw@@x{%
                  \yquant@draw@@x+\yquant@config@operator@sep%
               }%
            \else%
               % there was no text and no operation before. For seamless circuits, start right at zero; else, add separation
               \ifyquant@env@seamless{%
                  \def\yquant@draw@@x{0pt}%
               }{%
                  \let\yquant@draw@@x=\yquant@config@operator@sep%
               }%
            \fi%
         \fi%
         \ifcsname\yquant@prefix xshift\endcsname%
            \dimdef\yquant@draw@@x{\csname\yquant@prefix xshift\endcsname+\yquant@draw@@x}%
         \fi%
         \ifyquant@config@operator@position@advance%
            \dimdef\newx{\yquant@draw@@x+#1}%
            \forlistloop\yquant@draw@group@advance{#6}%
         \fi%
         \dimdef\yquant@draw@@x{\yquant@draw@@x+.5\dimexpr#1\relax}%
      \fi%
}

\protected\def\yquant@draw@group@advance#1{%
   \yquant@register@set@x{#1}\newx%
}

\protected\def\yquant@draw@endgroup#1#2#3{%
      \unless\ifx F#1%
         \yquant@draw@cline%
      \fi%
      \ifcase#3\relax%
      \or%
         \yquant@draw@alias@ctrl{#2}n%
      \or%
         \yquant@draw@alias@ctrl{#2}p%
      \or%
         \yquant@draw@alias@ctrl{#2}p%
         \yquant@draw@alias@ctrl{#2}n%
      \or%
         \yquant@draw@alias{#2}%
      \or%
         \yquant@draw@alias{#2}%
         \yquant@draw@alias@ctrl{#2}n%
      \or%
         \yquant@draw@alias{#2}%
         \yquant@draw@alias@ctrl{#2}p%
      \or%
         \yquant@draw@alias{#2}%
         \yquant@draw@alias@ctrl{#2}p%
         \yquant@draw@alias@ctrl{#2}n%
      \fi%
      \yquant@circuit@extendmultiline@total%
   \endgroup%
}

\protected\def\yquant@draw@single#1#2{%
   \let\idx=\yquant@draw@@idx@content%
   \edef\cmd{%
      \noexpand\path (\yquant@draw@@x, \yquant@register@get@y{#1})%
         node[/yquant/every operator, \yquant@draw@@style, /yquant/this operator,%
              name prefix=, name suffix=, name=yquantbox]%
         {\unexpanded\expandafter{\yquant@draw@@content}};%
      \pgfshapeclippath{yquantbox}%
                       {/yquant/every operator, \yquant@draw@@style,%
                        /yquant/this operator}%
   }%
   \cmd%
   \yquant@circuit@extendwire{#1}{center}%
   \expandafter\yquant@circuit@extendcontrolline\expandafter%
      {\the\yquant@draw@@currentcontroltype}\yquant@draw@@x%
   % check for empty name parameter
   \ifstrempty{#2}\relax{%
      \pgfnodealias{\tikz@pp@name{#2}}{yquantbox}%
   }%
   \numdef\yquant@draw@@idx@content{\yquant@draw@@idx@content+1}%
}

\protected\def\yquant@draw@multi#1#2#3#4#5{%
   \let\idx=\yquant@draw@@idx@content%
   \edef\yquant@draw@multi@@name{#5}%
   \def\yquant@draw@@idx@multipart{0}%
   \let\yquant@circuit@extendmultiline@cmd=\empty%
   \let\yquant@circuit@extendmultiline@prev=\relax%
   \let\yquant@circuit@extendmultiline@clip=\empty%
   \let\yquant@register@multi@contiguous=\yquant@draw@multi@contiguous%
   #4%
   \ifnum\yquant@draw@@idx@multipart>1 %
      % make sure also the first split part is available via the "-s0" suffix
      \unless\ifx\yquant@draw@multi@@name\empty%
         \pgfnodealias{\tikz@pp@name{\yquant@draw@multi@@name-s0}}%
                      {\tikz@pp@name{\yquant@draw@multi@@name}}%
      \fi%
      \yquant@draw@mline@prep%
   \fi%
   \numdef\yquant@draw@@idx@content{\yquant@draw@@idx@content+1}%
}

\protected\def\yquant@draw@multi@contiguous#1#2#3{%
   % We need to somehow extract the y radius
   \edef\cmd{%
      \noexpand\path (\yquant@draw@@x, \the\dimexpr.5\dimexpr%
                         \yquant@register@get@y{#1}+\yquant@register@get@y{#2}\relax%
                      \relax)%
         node[/yquant/every operator, \yquant@draw@@style, /yquant/this operator,%
              /yquant/operator/multi main=\ifnum#3=1 true\else false\fi\unless\ifnum#1=#2 ,%
              y radius/.expanded=\the\dimexpr.5\dimexpr\yquant@register@get@ydist{#1}{#2}\relax\relax+%
                     .5*\noexpand\pgfkeysvalueof{/tikz/y radius}\fi,%
              name prefix=, name suffix=, name=yquantbox]%
            {\unexpanded\expandafter{\yquant@draw@@content}};%
      \pgfshapeclippath{yquantbox}%
                       {/yquant/every operator, \yquant@draw@@style,%
                        /yquant/this operator}%
   }%
   \cmd%
   \yquant@for \i := #1 to #2 {%
      \yquant@circuit@extendwire\i{center}%
   }%
   \yquant@circuit@extendmultiline\yquant@draw@@x%
   \expandafter\yquant@circuit@extendcontrolline\expandafter%
      {\the\yquant@draw@@currentcontroltype}\yquant@draw@@x%
   \unless\ifx\yquant@draw@multi@@name\empty%
      \ifnum\yquant@draw@@idx@multipart=0 %
         \pgfnodealias{\tikz@pp@name{\yquant@draw@multi@@name}}{yquantbox}%
      \else%
         \pgfnodealias{\tikz@pp@name{\yquant@draw@multi@@name-s\yquant@draw@@idx@multipart}}{yquantbox}%
      \fi%
   \fi%
   \numdef\yquant@draw@@idx@multipart{\yquant@draw@@idx@multipart+1}%
}

\protected\def\yquant@draw@multiinit#1#2#3#4#5{%
   \let\idx=\yquant@draw@@idx@content%
   \@tempdima=-.5\dimexpr\yquant@config@register@sep\relax%
   \dimdef\yquant@draw@multiinit@@min{\yquant@register@get@y{#1}-\@tempdima}%
   \dimdef\yquant@draw@multiinit@@max{\yquant@register@get@y{#2}+\@tempdima}%
   \dimdef\yquant@draw@multiinit@@total{%
      \yquant@draw@multiinit@@max-\yquant@draw@multiinit@@min%
   }%
   \def\pgfdecorationsegmentaspect{0}%
   \let\yquant@register@multi@contiguous=\yquant@draw@multiinit@contiguous%
   \let\pgfdecorationsegmentfromto=\empty%
   #4%
   \edef\pgfdecorationsegmentfromto{\expandafter\@gobble\pgfdecorationsegmentfromto}%
   % We need to somehow extract the y radius
   \edef\cmd{%
      \noexpand\path (\yquant@draw@@x, \the\dimexpr.5\dimexpr%
                         \yquant@draw@multiinit@@min+\yquant@draw@multiinit@@max\relax%
                      \relax)%
         node[/yquant/every operator, \yquant@draw@@style, /yquant/every multi label,%
              /yquant/this operator,%
              y radius=\yquant@abs{\the\dimexpr.5\dimexpr\yquant@draw@multiinit@@total\relax\relax},%
              name prefix=, name suffix=, name=yquantbox]%
            {\unexpanded\expandafter{\yquant@draw@@content}};%
      \pgfshapeclippath{yquantbox}%
                       {/yquant/every operator, \yquant@draw@@style,%
                        /yquant/every multi label, /yquant/this operator}%
   }%
   \cmd%
   \yquant@for \i := #1 to #2 {%
      \yquant@circuit@extendwire\i{center}%
   }%
   % no control line (init doesn't allow for those, so just save the no-op), no multi line
   % check for empty name parameter
   \ifstrempty{#5}\relax{%
      \pgfnodealias{\tikz@pp@name{#5}}{yquantbox}%
   }%
   \numdef\yquant@draw@@idx@content{\yquant@draw@@idx@content+1}%
}

\protected\def\yquant@draw@multiinit@contiguous#1#2#3{%
   \edef\yquant@draw@multiinit@@from{%
      \pgfmath@tonumber{\dimexpr%
         \dimexpr\yquant@register@get@y{#1}-\@tempdima-\yquant@draw@multiinit@@min\relax*65536/%
         \dimexpr\yquant@draw@multiinit@@total\relax%
      \relax}%
   }%
   \edef\yquant@draw@multiinit@@to{%
      \pgfmath@tonumber{\dimexpr%
         \dimexpr\yquant@register@get@y{#2}+\@tempdima-\yquant@draw@multiinit@@min\relax*65536/%
         \dimexpr\yquant@draw@multiinit@@total\relax%
      \relax}%
   }%
   \eappto\pgfdecorationsegmentfromto{,%
      \yquant@draw@multiinit@@from-\yquant@draw@multiinit@@to%
   }%
   % We need to decide where to put the brace arch.
   \ifdim\yquant@draw@multiinit@@from pt<.5pt %
      \ifdim\yquant@draw@multiinit@@to pt>.5 pt%
         % This segment covers the true 1/2 position, take it
         \def\pgfdecorationsegmentaspect{.5}%
      \else%
         % We are not there yet, so the end of this segment is the closest we can get to the mid so far
         \edef\pgfdecorationsegmentaspect{\yquant@draw@multiinit@@to}%
      \fi%
   \else%
      % We are already beyond the mid...
      \ifdim\pgfdecorationsegmentaspect pt<.5pt %
         % ...but we did not find an ideal position yet
         \ifdim\dimexpr\yquant@draw@multiinit@@from pt-.5pt\relax<%
            \dimexpr.5pt-\pgfdecorationsegmentaspect pt\relax%
            % this one is closer to the mid than anything found before
            \edef\pgfdecorationsegmentaspect{\yquant@draw@multiinit@from}%
         \fi%
      \fi%
   \fi%
}

\protected\long\def\yquant@draw@output@single#1#2{%
   \let\idx=\yquant@draw@@idx@content%
   \edef\cmd{%
      \noexpand\path (\yquant@circuit@endwires@x, \yquant@register@get@y{#1})%
         node[\ifnum\yquant@compat>1 /yquant/every operator,\fi%
              /yquant/every output,%
              /yquant/every \yquant@register@type@tostring{\yquant@register@get@type{#1}} output,%
              \yquant@draw@@style, /yquant/this operator,%
              name prefix=, name suffix=, name=yquantbox]%
         {\unexpanded\expandafter{\yquant@draw@@content}};%
      \ifdefined\yquant@parent%
         \pgfshapeclippath{yquantbox}%
                          {\ifnum\yquant@compat>1 /yquant/every operator,\fi%
                           /yquant/every output,%
                           /yquant/every \yquant@register@type@tostring{\yquant@register@get@type{#1}} output,%
                           \yquant@draw@@style,%
                           /yquant/this operator}%
      \fi%
   }%
   \cmd%
   % only extend for subcircuits
   \ifdefined\yquant@parent%
      \yquant@circuit@extendwire{#1}{center}%
   \fi%
   % check for empty name parameter
   \ifstrempty{#2}\relax{%
      \pgfnodealias{\tikz@pp@name{#2}}{yquantbox}%
   }%
   \numdef\yquant@draw@@idx@content{\yquant@draw@@idx@content+1}%
}

\protected\long\def\yquant@draw@output@multi#1#2#3#4#5{%
   \let\idx=\yquant@draw@@idx@content%
   \@tempdima=-.5\dimexpr\yquant@config@register@sep\relax%
   \dimdef\yquant@draw@multiinit@@min{\yquant@register@get@y{#1}-\@tempdima}%
   \dimdef\yquant@draw@multiinit@@max{\yquant@register@get@y{#2}+\@tempdima}%
   \dimdef\yquant@draw@multiinit@@total{%
      \yquant@draw@multiinit@@max-\yquant@draw@multiinit@@min%
   }%
   \def\pgfdecorationsegmentaspect{0}%
   \let\yquant@register@multi@contiguous=\yquant@draw@multiinit@contiguous%
   \let\pgfdecorationsegmentfromto=\empty%
   #4%
   \edef\pgfdecorationsegmentfromto{\expandafter\@gobble\pgfdecorationsegmentfromto}%
   % We need to somehow extract the y radius
   \edef\cmd{%
      \noexpand\path (\yquant@circuit@endwires@x, \the\dimexpr.5\dimexpr%
                         \yquant@draw@multiinit@@min+\yquant@draw@multiinit@@max\relax%
                      \relax)%
         node[\ifnum\yquant@compat>1 /yquant/every operator, /yquant/every output,\fi%
              \yquant@draw@@style, /yquant/every multi output,%
              /yquant/this operator,%
              y radius=\yquant@abs{\the\dimexpr.5\dimexpr\yquant@draw@multiinit@@total\relax\relax},%
              name prefix=, name suffix=, name=yquantbox]%
            {\unexpanded\expandafter{\yquant@draw@@content}};%
      \ifdefined\yquant@parent%
         \pgfshapeclippath{yquantbox}%
                          {\ifnum\yquant@compat>1 /yquant/every operator, /yquant/every output,\fi%
                           \yquant@draw@@style, /yquant/every multi output,%
                           /yquant/this operator}%
      \fi%
   }%
   \cmd%
   % only extend for subcircuits
   \ifdefined\yquant@parent%
      \yquant@for \i := #1 to #2 {%
         \yquant@circuit@extendwire\i{center}%
      }%
   \fi%
   % check for empty name parameter
   \ifstrempty{#5}\relax{%
      \pgfnodealias{\tikz@pp@name{#5}}{yquantbox}%
   }%
   \numdef\yquant@draw@@idx@content{\yquant@draw@@idx@content+1}%
}

\newbox\yquant@draw@subcircuit@box

\protected\def\yquant@draw@subcircuit@nodecallback#1{%
   \ifstrequal{#1}{yquantbox}\relax{%
      \listcsxadd{\yquant@prefix draw@subcircuit@nodelist}{#1}%
   }%
}

\protected\long\def\yquant@draw@subcircuit@prepare#1#2{%
   \let\idx=\yquant@draw@@idx@content%
   % In order to wrap the inner circuit in a proper box operator and clip outer paths appropriately (which was not possible yet, as we didn't know the exact vertical positions), we first place it within a box. During the setup time, we assumed that the subcircuit be placed at position #3; however, now, this has changed due to the additional box.
   % First, we anticipate the macro that is used by our subcircuit to store the node
   % names.
   \edef\yquant@draw@subcircuit@nodelist{yquant@env#1@draw@subcircuit@nodelist}%
   \global\cslet\yquant@draw@subcircuit@nodelist\empty%
   % We must globally store the leftmost position, as perhaps no wire was drawn yet - so we'll miss something in the global bounding box
   \unless\ifdefined\yquant@parent%
      \edef\yquant@draw@subcircuit@outermin{\the\pgf@picminx}%
   \fi%
   \pgfinterruptboundingbox%
      \let\yquant@parent=\yquant@prefix%
      \def\yquant@prefix{yquant@env#1@}%
      \ifstrempty{#2}{%
         % we make sure there are no conflicts by prefixing any named nodes in any case.
         \pgfkeys{/tikz/name prefix/.expanded={sub\yquant@prefix-}}%
         \let\pgf@nodecallback=\yquant@draw@subcircuit@nodecallback%
      }{%
         \pgfkeys{/tikz/name prefix/.expanded={\pgfkeysvalueof{/tikz/name prefix}#2-}}%
      }%
      \pgfkeys{/yquant/operators/this subcircuit box/.style={}}%
      \cslet{\yquant@prefix xshift}\yquant@draw@@x%
      \global\setbox\yquant@draw@subcircuit@box=\hbox to 0pt {{%
         % bypass 'overlay' option
         \pgf@relevantforpicturesizetrue%
         \pgfsys@beginpicture%
            % reset all styles to the expected defaults (similar, but extended to \pgfpicture, see pgf issue #870)
            \pgfsetcolor{.}%
            \pgfsetlinewidth{\pgflinewidth}%
            % for the others, pgf does not keep track of the state, so we cannot do much
            \pgfsetbuttcap%
            \pgfsetmiterjoin%
            \pgfsetmiterlimit{10}%
            \pgfsetdash{}{0pt}%
            \csname\yquant@prefix draw\endcsname%
            \ifdim\pgf@picmaxx=-16000pt %
               \global\pgf@picmaxx=0pt %
               \global\pgf@picminx=0pt %
               \global\pgf@picmaxy=0pt %
               \global\pgf@picminy=0pt %
            \fi%
            \ifyquantdebug%
               \pgf@relevantforpicturesizefalse%
               \draw[green] (current bounding box.north east) rectangle (current bounding box.south west);%
            \fi%
         \pgfsys@endpicture%
      }}%
      \global\setbox\yquant@draw@subcircuit@box=\hbox to \dimexpr\pgf@picmaxx-\pgf@picminx\relax {%
         \hskip-\pgf@picminx%
         \lower\pgf@picmaxy%
         \box\yquant@draw@subcircuit@box%
      }%
      \ht\yquant@draw@subcircuit@box=0pt%
      \dp\yquant@draw@subcircuit@box=\dimexpr\pgf@picmaxy-\pgf@picminy\relax%
      \expandafter%
   \endpgfinterruptboundingbox%
   \expandafter\edef\expandafter\yquant@draw@subcircuit@y\expandafter{%
      \the\dimexpr.5\pgf@picminy+.5\pgf@picmaxy\relax%
   }%
   \unless\ifdefined\yquant@parent%
      \global\pgf@picminx=\yquant@draw@subcircuit@outermin%
      \gundef\yquant@draw@subcircuit@outermin%
   \fi%
   \ifstrempty{#2}{%
      % However, if the outer node was not named, no access to the inner nodes is desired, so we delete all nodes again.
      \def\do##1{%
         \csgundef{pgf@sh@ns@##1}%
         \csgundef{pgf@sh@np@##1}%
         \csgundef{pgf@sh@nt@##1}%
         \csgundef{pgf@sh@pi@##1}%
         \csgundef{pgf@sh@ma@##1}%
      }%
      \dolistcsloop{\yquant@draw@subcircuit@nodelist}%
      \csgundef\yquant@draw@subcircuit@nodelist%
   }{%
      \ifcsname\yquant@prefix draw@subcircuit@nodelist\endcsname%
         \csxappto{\yquant@prefix draw@subcircuit@nodelist}%
                  {\csname\yquant@draw@subcircuit@nodelist\endcsname}%
      \fi%
   }%
}

\protected\long\def\yquant@draw@subcircuit@single#1#2#3{%
   \yquant@draw@subcircuit@prepare{#2}{#3}%
   \edef\cmd{%
      \noexpand\path (\yquant@draw@@x, \yquant@draw@subcircuit@y)%
         node[/yquant/every operator, \yquant@draw@@style,%
              /yquant/operators/every subcircuit box, /yquant/this operator,%
              /yquant/operators/this subcircuit box,%
              name prefix=, name suffix=, name=yquantbox]%
         {\box\yquant@draw@subcircuit@box};%
      \pgfshapeclippath{yquantbox}%
                       {/yquant/every operator, \yquant@draw@@style,%
                        /yquant/operators/every subcircuit box, /yquant/this operator,%
                        /yquant/operators/this subcircuit box}%
   }%
   \cmd%
   % see comment in draw@subcircuit@multi
   \yquant@softpath@extractmaxxat\pgfshapeclippathhorzresult{\yquant@register@get@y{#1}}%
   \let\pgfshapeclippathhorzresult=\empty%
   \yquant@circuit@extendwire{#1}*%
   \expandafter\yquant@circuit@extendcontrolline\expandafter%
      {\the\yquant@draw@@currentcontroltype}\yquant@draw@@x%
   % check for empty name parameter
   \ifstrempty{#3}\relax{%
      \pgfnodealias{\tikz@pp@name{#3}}{yquantbox}%
   }%
   \numdef\yquant@draw@@idx@content{\yquant@draw@@idx@content+1}%
}

\protected\long\def\yquant@draw@subcircuit@multi#1#2#3#4#5#6{%
   % there is no contiguous slicing for subcircuits, as they may have all kinds of wire operations that can extend beyond the individual slices, let alone ancillas
   \yquant@draw@subcircuit@prepare{#5}{#6}%
   % We need to somehow extract the y radius
   \edef\cmd{%
      \noexpand\path (\yquant@draw@@x, \yquant@draw@subcircuit@y)%
         node[/yquant/every operator, \yquant@draw@@style,%
              /yquant/operators/every subcircuit box, /yquant/this operator,%
              /yquant/operators/this subcircuit box,
              /yquant/operator/multi main=true,%
              name prefix=, name suffix=, name=yquantbox]%
            {\box\yquant@draw@subcircuit@box};%
      \pgfshapeclippath{yquantbox}%
                       {/yquant/every operator, \yquant@draw@@style,%
                        /yquant/operators/every subcircuit box, /yquant/this operator,%
                        /yquant/operators/this subcircuit box,%
                        /yquant/operator/multi main=true}%
   }%
   \cmd%
   % install the clippings - but only on wires that are visually between the first and list while not being part of the circuit.
   \let\nonaffectedpgfshapeclippathhorzresult=\pgfshapeclippathhorzresult%
   \yquant@for \i := #1 to #2 {%
      \xifinlist{\i}{#4}{%
         % Usually, we always begin with a wire from the center of the operator shape and clip the inner parts away. This can't be done here, as the wire needs to be drawn _inside_ of the outer box operator here. Instead of clipping against the clip path, we extract its maximum x position at the position of the wire (which is an overkill for simple shapes, but the allows to specify even more complicated ones) and place the wire at this position without clipping.
         % Note: this works very well for lines joining at perpendicular angles; but if the shape of the box is more fancy, while the position will be calculated correctly, the wire has a rectangular (or rounded, depending on the line cap) shape that is drawn on top of thw operator. While \yquant@softpath@extractmaxxat could without much effort determine exactly the segment of the path that corresponds to the rightmost line, we would then have to convert this single line into a closed path that fills the linewidth and clip against it to get proper joiners. Since most likely, no-one will ever need this, we don't do it. But file a feature request if desired.
         \yquant@softpath@extractmaxxat\nonaffectedpgfshapeclippathhorzresult%
                                       {\yquant@register@get@y\i}%
         \let\pgfshapeclippathhorzresult=\empty%
         \yquant@circuit@extendwire\i{*}%
      }{%
         \let\pgfshapeclippathhorzresult=\nonaffectedpgfshapeclippathhorzresult%
         \yquant@circuit@extendwire\i{center}%
      }%
   }%
   \expandafter\yquant@circuit@extendcontrolline\expandafter%
      {\the\yquant@draw@@currentcontroltype}\yquant@draw@@x%
   \ifstrempty{#6}\relax{%
      \pgfnodealias{\tikz@pp@name{#6}}{yquantbox}%
   }%
   \numdef\yquant@draw@@idx@content{\yquant@draw@@idx@content+1}%
}

\protected\def\yquant@draw@control#1#2#3{%
   \edef\cmd{%
      \noexpand\path (\yquant@draw@@x, \yquant@register@get@y{#2})%
         node[/yquant/every control, /yquant/every #1 control, /yquant/this control,%
              name prefix=, name suffix=, name=yquantbox]%
         {};%
      \pgfshapeclippath{yquantbox}%
                       {/yquant/every control, /yquant/every #1 control,%
                        /yquant/this control}%
   }%
   \cmd%
   \yquant@circuit@extendwire{#2}{center}%
   \yquant@draw@@currentcontroltype=\yquant@register@get@type{#2}\relax%
   \expandafter\yquant@circuit@extendcontrolline\expandafter%
      {\yquant@draw@@currentcontroltype}\yquant@draw@@x%
   % check for empty name parameter
   \ifstrempty{#3}\relax{%
      \pgfnodealias{\tikz@pp@name{#3}}{yquantbox}%
   }%
}

\protected\def\yquant@draw@pcontrol#1#2{%
   \let\idx=\yquant@draw@@idx@pcontrol%
   \yquant@draw@control{positive}{#1}{#2}%
   \numdef\yquant@draw@@idx@pcontrol{\yquant@draw@@idx@pcontrol+1}%
}

\protected\def\yquant@draw@ncontrol#1#2{%
   \let\idx=\yquant@draw@@idx@ncontrol%
   \yquant@draw@control{negative}{#1}{#2}%
   \numdef\yquant@draw@@idx@ncontrol{\yquant@draw@@idx@ncontrol+1}%
}

\protected\def\yquant@draw@cline{%
   \pgfscope%
      % install the clipping
      \pgfsyssoftpath@setcurrentpath\yquant@circuit@extendcontrolline@clip%
      % and invert it. It is sufficient to cover the current bounding box, as the wire will be drawn between existing operators.
      \ifyquantdebug%
         \pgfsetfillcolor{teal}%
         \pgfsetfillopacity{.3}%
         \pgfusepathqfill%
      \else%
         \begingroup%
            \pgftransformreset%
            \pgfpathrectanglecorners%
               {\pgfqpoint{\pgf@picminx}{\pgf@picminy}}%
               {\pgfqpoint{\pgf@picmaxx}{\pgf@picmaxy}}%
            \pgfseteorule% even-odd to properly invert the clipping
            \pgfusepathqclip%
         \endgroup%
      \fi%
      \edef\cmd{%
         \noexpand\path[/yquant/every control line]%
            \yquant@circuit@extendcontrolline@cmd;%
      }%
      \cmd%
   \endpgfscope%
}

\protected\def\yquant@draw@mline@prep{%
   \eappto\yquant@circuit@extendmultiline@total{%
      \yquant@draw@mline%
         {\unexpanded\expandafter{\yquant@circuit@extendmultiline@clip}}%
         {\yquant@circuit@extendmultiline@cmd}%
   }%
}

\protected\def\yquant@draw@mline#1#2{%
   \pgfscope%
      % install the clipping
      \def\pgfsyssoftpath@thepath{#1}%
      \pgfsyssoftpath@setcurrentpath\pgfsyssoftpath@thepath%
      % and invert it. It is sufficient to cover the current bounding box, as the wire will be drawn between existing operators.
      \ifyquantdebug%
         \pgfsetfillcolor{teal}%
         \pgfsetfillopacity{.3}%
         \pgfusepathqfill%
      \else%
         \begingroup%
            \pgftransformreset%
            \pgfpathrectanglecorners%
               {\pgfqpoint{\pgf@picminx}{\pgf@picminy}}%
               {\pgfqpoint{\pgf@picmaxx}{\pgf@picmaxy}}%
            \pgfseteorule% even-odd to properly invert the clipping
            \pgfusepathqclip%
         \endgroup%
      \fi%
      \path[/yquant/every multi line] #2;%
   \endpgfscope%
}

\protected\def\yquant@draw@alias#1{%
   \pgfnodealias{\tikz@pp@name{#1}}{\tikz@pp@name{#1-0}}%
}

\protected\def\yquant@draw@alias@ctrl#1#2{%
   \pgfnodealias{\tikz@pp@name{#1-#2}}{\tikz@pp@name{#1-#20}}%
}

\protected\def\yquant@draw@wire#1#2{%
   \begingroup%
      \yquant@register@get@typeywire{#1}\wiretype\wireypos\wirelast%
      \unless\ifnum\wiretype=\yquant@register@type@none%
         \edef\wirexprevpos{\expandafter\@firstoffour\wirelast}%
         \ifx0#2%
            \edef\wirexpos{\expandafter\@secondoffour\wirelast}%
         \else%
            \let\wirexpos=\yquant@circuit@endwires@x%
         \fi%
         \ifdim\wirexpos>\wirexprevpos %
            \edef\wirestyle{\noexpand\tikzset{%
               /yquant/this wire/.style={%
                  /yquant/every wire,%
                  /yquant/every \yquant@register@type@tostring\wiretype\space wire,%
                     \yquant@register@get@style{#1}%
               }, /yquant/this wire%
            }}%
            \wirestyle%
            % load all clippings
            \edef\wireclipping{%
               \unexpanded\expandafter\expandafter\expandafter{%
                  \expandafter\@thirdandfourthoffour\wirelast%
               }%
            }%
            \pgfscope%
               % install the clipping
               \pgfsyssoftpath@setcurrentpath\wireclipping%
               % invert the clipping
               \ifyquantdebug%
                  \pgfsetfillcolor{orange}%
                  \pgfsetfillopacity{.3}%
                  \pgfusepathqfill%
               \else%
                  % We need to access the current bounding box as well as other positions in the local coordinate frame. For this, transform the bounding box to the current frame (though this is expensive). Does this capture rotations correctly?
                  \begingroup%
                     \pgftransforminvert%
                     \pgfpointtransformednonlinear{\pgfqpoint{\pgf@picminx}{\pgf@picminy}}%
                     \global\@tempdima=\pgf@y%
                     \pgfpointtransformednonlinear{\pgfqpoint{\pgf@picmaxx}{\pgf@picmaxy}}%
                     \global\@tempdimb=\pgf@y%
                  \endgroup%
                  % To avoid rendering artifacts at all zoom levels with all renderers, we need to make the clipping region large. Let's try the current bounding box first.
                  % This may be insufficient if there no or a tiny wire label and only registers of a small height. In this case, take ten times the line width or at least 1cm, but don't let it affect the bounding box.
                  \ifdim\dimexpr\@tempdimb-\@tempdima\relax<10\pgflinewidth %
                     \@tempdima=\dimexpr\wireypos-5\pgflinewidth\relax%
                     \@tempdimb=\dimexpr\wireypos+5\pgflinewidth\relax%
                  \fi%
                  \ifdim\dimexpr\@tempdimb-\@tempdima\relax<1cm %
                     \@tempdima=\dimexpr\wireypos-5mm\relax%
                     \@tempdimb=\dimexpr\wireypos+5mm\relax%
                  \fi%
                  \pgfinterruptboundingbox%
                     \pgfpathrectanglecorners%
                        {\pgfqpoint{\dimexpr\wirexprevpos-2\pgflinewidth\relax}%
                                   {\@tempdima}}%
                        {\pgfqpoint{\dimexpr\wirexpos+2\pgflinewidth\relax}%
                                   {\@tempdimb}}%
                  \endpgfinterruptboundingbox%
                  \pgfseteorule% even-odd to properly invert the clipping
                  \pgfusepathqclip%
               \fi%
               % the clip inversion is left to the drawing commands (clip two \pgflinewidth more to avoid renderer artifacts)
               \csname yquant@draw@wire@\wiretype\endcsname{#1}%
            \endpgfscope%
         \fi%
      \fi%
   \endgroup%
}

% quantum wire
\protected\csdef{yquant@draw@wire@\yquant@register@type@q}#1{%
   \edef\cmd{%
      \noexpand\path [/yquant/this wire]%
         (\wirexprevpos,\wireypos) -- (\wirexpos,\wireypos);%
   }%
   \cmd%
}

% classical wire
\protected\csdef{yquant@draw@wire@\yquant@register@type@c}#1{%
   \edef\cmd{%
      \noexpand\path [/yquant/this wire]%
         (\wirexprevpos,\wireypos+2\pgflinewidth)--(\wirexpos,\wireypos+2\pgflinewidth)%
         (\wirexprevpos,\wireypos-2\pgflinewidth)--(\wirexpos,\wireypos-2\pgflinewidth);%
   }%
   \cmd%
}

% quantum-bundle
\protected\csdef{yquant@draw@wire@\yquant@register@type@qs}#1{%
   \edef\cmd{%
      \noexpand\path [/yquant/this wire]%
         (\wirexprevpos,\wireypos+2\pgflinewidth)--(\wirexpos,\wireypos+2\pgflinewidth)%
         (\wirexprevpos,\wireypos)--(\wirexpos,\wireypos)%
         (\wirexprevpos,\wireypos-2\pgflinewidth)--(\wirexpos,\wireypos-2\pgflinewidth);%
   }%
   \cmd%
}

\protected\def\yquant@draw@hspace#1#2{%
   \begingroup%
      \yquant@register@get@maxxlist\newx{#1}%
      \dimdef\newx{\newx+#2}%
      \dimen0=\newx%
      \let\pgfshapeclippathhorzresult=\empty%
      \forlistloop\yquant@draw@move@loop{#1}%
   \endgroup%
}

\protected\def\yquant@draw@move@loop#1{%
   \yquant@register@set@x{#1}\newx%
   \yquant@circuit@extendwire{#1}*%
}