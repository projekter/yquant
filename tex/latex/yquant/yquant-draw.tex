% BEGIN_FOLD Actual drawing at shipout
\newcount\yquant@draw@@currentcontroltype%

\protected\def\yquant@draw@group#1#2#3#4#5{%
   \begingroup%
      \ifcsname\yquant@prefix xshift\endcsname%
         \dimdef\yquant@draw@@x{#1+\csname\yquant@prefix xshift\endcsname}%
      \else%
         \def\yquant@draw@@x{#1}%
      \fi%
      \ifx F#2%
         \yquant@draw@@currentcontroltype=0 %
      \else%
         \yquant@draw@@currentcontroltype=\yquant@register@get@type{#2}\relax%
      \fi%
      \let\yquant@circuit@extendcontrolline@cmd=\empty%
      \let\yquant@circuit@extendcontrolline@prev=\relax%
      \let\yquant@circuit@extendcontrolline@clip=\empty%
      \let\yquant@circuit@extendmultiline@total=\empty%
      \yquant@langhelper@list@attrs%
      % If the quotes library is loaded, activate it. (else, this is by default \relax)
      \tikz@enable@node@quotes%
      \yquant@set{#3}%
      \def\yquant@draw@@style{#4}%
      \def\yquant@draw@@content{#5}%
      \def\yquant@draw@@idx@content{0}%
      \def\yquant@draw@@idx@pcontrol{0}%
      \def\yquant@draw@@idx@ncontrol{0}%
}

\protected\def\yquant@draw@endgroup#1#2#3{%
      \unless\ifx F#1%
         \yquant@draw@cline%
      \fi%
      \ifcase#3\relax%
      \or%
         \yquant@draw@alias@ctrl{#2}n%
      \or%
         \yquant@draw@alias@ctrl{#2}p%
      \or%
         \yquant@draw@alias@ctrl{#2}p%
         \yquant@draw@alias@ctrl{#2}n%
      \or%
         \yquant@draw@alias{#2}%
      \or%
         \yquant@draw@alias{#2}%
         \yquant@draw@alias@ctrl{#2}n%
      \or%
         \yquant@draw@alias{#2}%
         \yquant@draw@alias@ctrl{#2}p%
      \or%
         \yquant@draw@alias{#2}%
         \yquant@draw@alias@ctrl{#2}p%
         \yquant@draw@alias@ctrl{#2}n%
      \fi%
      \yquant@circuit@extendmultiline@total%
   \endgroup%
}

\protected\long\def\yquant@draw@single#1#2{%
   \let\idx=\yquant@draw@@idx@content%
   \edef\cmd{%
      \noexpand\path (\yquant@draw@@x, \yquant@register@get@y{#1})%
         node[/yquant/every operator, \yquant@draw@@style, /yquant/this operator,%
              name prefix=, name suffix=, name=yquantbox]%
         {\unexpanded\expandafter{\yquant@draw@@content}};%
      \pgfshapeclippath{yquantbox}%
                       {/yquant/every operator, \yquant@draw@@style,%
                        /yquant/this operator}%
   }%
   \cmd%
   \yquant@circuit@extendwire{#1}{center}%
   \expandafter\yquant@circuit@extendcontrolline\expandafter%
      {\the\yquant@draw@@currentcontroltype}\yquant@draw@@x%
   % check for empty name parameter
   \ifstrempty{#2}\relax{%
      \pgfnodealias{\tikz@pp@name{#2}}{yquantbox}%
   }%
   \numdef\yquant@draw@@idx@content{\yquant@draw@@idx@content+1}%
}

\protected\def\yquant@draw@multi#1#2#3#4#5{%
   \let\idx=\yquant@draw@@idx@content%
   \edef\yquant@draw@multi@@name{#5}%
   \def\yquant@draw@@idx@multipart{0}%
   \let\yquant@circuit@extendmultiline@cmd=\empty%
   \let\yquant@circuit@extendmultiline@prev=\relax%
   \let\yquant@circuit@extendmultiline@clip=\empty%
   \let\yquant@register@multi@contiguous=\yquant@draw@multi@contiguous%
   #4%
   \ifnum\yquant@draw@@idx@multipart>1 %
      % make sure also the first split part is available via the "-s0" suffix
      \unless\ifx\yquant@draw@multi@@name\empty%
         \pgfnodealias{\tikz@pp@name{\yquant@draw@multi@@name-s0}}%
                      {\tikz@pp@name{\yquant@draw@multi@@name}}%
      \fi%
      \yquant@draw@mline@prep%
   \fi%
   \numdef\yquant@draw@@idx@content{\yquant@draw@@idx@content+1}%
}

\protected\def\yquant@draw@multi@contiguous#1#2#3{%
   % We need to somehow extract the y radius
   \edef\cmd{%
      \noexpand\path (\yquant@draw@@x, \the\dimexpr.5\dimexpr%
                         \yquant@register@get@y{#1}+\yquant@register@get@y{#2}\relax%
                      \relax)%
         node[/yquant/every operator, \yquant@draw@@style, /yquant/this operator,%
              /yquant/operator/multi main=\ifnum#3=1 true\else false\fi\unless\ifnum#1=#2 ,%
              y radius/.expanded=\the\dimexpr.5\dimexpr\yquant@register@get@ydist{#1}{#2}\relax\relax+%
                     .5*\noexpand\pgfkeysvalueof{/tikz/y radius}\fi,%
              name prefix=, name suffix=, name=yquantbox]%
            {\unexpanded\expandafter{\yquant@draw@@content}};%
      \pgfshapeclippath{yquantbox}%
                       {/yquant/every operator, \yquant@draw@@style,%
                        /yquant/this operator}%
   }%
   \cmd%
   \yquant@for \i := #1 to #2 {%
      \yquant@circuit@extendwire\i{center}%
   }%
   \yquant@circuit@extendmultiline\yquant@draw@@x%
   \expandafter\yquant@circuit@extendcontrolline\expandafter%
      {\the\yquant@draw@@currentcontroltype}\yquant@draw@@x%
   \unless\ifx\yquant@draw@multi@@name\empty%
      \ifnum\yquant@draw@@idx@multipart=0 %
         \pgfnodealias{\tikz@pp@name{\yquant@draw@multi@@name}}{yquantbox}%
      \else%
         \pgfnodealias{\tikz@pp@name{\yquant@draw@multi@@name-s\yquant@draw@@idx@multipart}}{yquantbox}%
      \fi%
   \fi%
   \numdef\yquant@draw@@idx@multipart{\yquant@draw@@idx@multipart+1}%
}

\protected\def\yquant@draw@multiinit#1#2#3#4#5{%
   \let\idx=\yquant@draw@@idx@content%
   \@tempdima=-.5\dimexpr\yquant@config@register@sep\relax%
   \dimdef\yquant@draw@multiinit@@min{\yquant@register@get@y{#1}-\@tempdima}%
   \dimdef\yquant@draw@multiinit@@max{\yquant@register@get@y{#2}+\@tempdima}%
   \dimdef\yquant@draw@multiinit@@total{%
      \yquant@draw@multiinit@@max-\yquant@draw@multiinit@@min%
   }%
   \def\pgfdecorationsegmentaspect{0}%
   \let\yquant@register@multi@contiguous=\yquant@draw@multiinit@contiguous%
   \let\pgfdecorationsegmentfromto=\empty%
   #4%
   \edef\pgfdecorationsegmentfromto{\expandafter\@gobble\pgfdecorationsegmentfromto}%
   % We need to somehow extract the y radius
   \edef\cmd{%
      \noexpand\path[/yquant/every operator, \yquant@draw@@style,%
                     /yquant/every multi label, /yquant/this operator]%
         (\yquant@draw@@x, \yquant@draw@multiinit@@min) --%
         (\yquant@draw@@x, \yquant@draw@multiinit@@max)%
         node[name prefix=, name suffix=, name=yquantbox]%
            {\unexpanded\expandafter{\yquant@draw@@content}};%
   }%
   \cmd%
   % no wire extension (we are still at the initial position), no control line (init doesn't allow for those, so just save the no-op), no multi line
   % check for empty name parameter
   \ifstrempty{#5}\relax{%
      \pgfnodealias{\tikz@pp@name{#5}}{yquantbox}%
   }%
   \numdef\yquant@draw@@idx@content{\yquant@draw@@idx@content+1}%
}

\protected\def\yquant@draw@multiinit@contiguous#1#2#3{%
   \edef\yquant@draw@multiinit@@from{%
      \pgfmath@tonumber{\dimexpr%
         \dimexpr\yquant@register@get@y{#1}-\@tempdima-\yquant@draw@multiinit@@min\relax*65536/%
         \dimexpr\yquant@draw@multiinit@@total\relax%
      \relax}%
   }%
   \edef\yquant@draw@multiinit@@to{%
      \pgfmath@tonumber{\dimexpr%
         \dimexpr\yquant@register@get@y{#2}+\@tempdima-\yquant@draw@multiinit@@min\relax*65536/%
         \dimexpr\yquant@draw@multiinit@@total\relax%
      \relax}%
   }%
   \eappto\pgfdecorationsegmentfromto{,%
      \yquant@draw@multiinit@@from-\yquant@draw@multiinit@@to%
   }%
   % We need to decide where to put the brace arch.
   \ifdim\yquant@draw@multiinit@@from pt<.5pt %
      \ifdim\yquant@draw@multiinit@@to pt>.5 pt%
         % This segment covers the true 1/2 position, take it
         \def\pgfdecorationsegmentaspect{.5}%
      \else%
         % We are not there yet, so the end of this segment is the closest we can get to the mid so far
         \edef\pgfdecorationsegmentaspect{\yquant@draw@multiinit@@to}%
      \fi%
   \else%
      % We are already beyond the mid...
      \ifdim\pgfdecorationsegmentaspect pt<.5pt %
         % ...but we did not find an ideal position yet
         \ifdim\dimexpr\yquant@draw@multiinit@@from pt-.5pt\relax<%
            \dimexpr.5pt-\pgfdecorationsegmentaspect pt\relax%
            % this one is closer to the mid than anything found before
            \edef\pgfdecorationsegmentaspect{\yquant@draw@multiinit@from}%
         \fi%
      \fi%
   \fi%
}

\newbox\yquant@draw@subcircuit@box

\protected\def\yquant@draw@subcircuit@nodecallback#1{%
   \ifstrequal{#1}{yquantbox}\relax{%
      \listcsxadd{\yquant@prefix draw@subcircuit@nodelist}{#1}%
   }%
}

\protected\long\def\yquant@draw@subcircuit@prepare#1#2{%
   \let\idx=\yquant@draw@@idx@content%
   % In order to wrap the inner circuit in a proper box operator and clip outer paths appropriately (which was not possible yet, as we didn't know the exact vertical positions), we first place it within a box. During the setup time, we assumed that the subcircuit be placed at position #3; however, now, this has changed due to the additional box.
   % First, we anticipate the macro that is used by our subcircuit to store the node
   % names.
   \edef\yquant@draw@subcircuit@nodelist{yquant@env#1@draw@subcircuit@nodelist}%
   \global\cslet\yquant@draw@subcircuit@nodelist\empty%
   \pgfinterruptboundingbox%
      \let\yquant@parent=\yquant@prefix%
      \def\yquant@prefix{yquant@env#1@}%
      \ifstrempty{#2}{%
         % we make sure there are no conflicts by prefixing any named nodes in any case.
         \pgfkeys{/tikz/name prefix/.expanded={sub\yquant@prefix-}}%
         \let\pgf@nodecallback=\yquant@draw@subcircuit@nodecallback%
      }{%
         \pgfkeys{/tikz/name prefix/.expanded={\pgfkeysvalueof{/tikz/name prefix}#2-}}%
      }%
      \letcs\xmin{\yquant@prefix xmin}%
      \letcs\xmax{\yquant@prefix xmax}%
      \global\setbox\yquant@draw@subcircuit@box=\hbox to 0pt {{%
         % bypass 'overlay' option
         \pgf@relevantforpicturesizetrue%
         \pgfsys@beginpicture%
            % reset all styles to the expected defaults (similar, but extended to \pgfpicture, see pgf issue #870)
            \pgfsetcolor{.}%
            \pgfsetlinewidth{0.4pt}%
            \pgfsetbuttcap%
            \pgfsetmiterjoin%
            \pgfsetmiterlimit{10}%
            \pgfsetdash{}{0pt}%
            % The left outer position of our box will be \yquant@draw@@x-.5(xmax-xmin).
            % To compensate for, we perform a left shift of all commands that take explicit coordinates from the subcircuit.
            % The y positions, on the other hand, are exactly the ones as they should be integrated in the picture.
            \csdimdef{\yquant@prefix xshift}{\yquant@draw@@x-.5\dimexpr\xmax+\xmin\relax}%
            \csname\yquant@prefix draw\endcsname%
            \ifdim\pgf@picmaxx=-16000pt %
               \global\pgf@picmaxx=0pt %
               \global\pgf@picminx=0pt %
               \global\pgf@picmaxy=0pt %
               \global\pgf@picminy=0pt %
            \fi%
            \ifyquantdebug%
               \pgf@relevantforpicturesizefalse%
               \draw[green] (current bounding box.north east) rectangle (current bounding box.south west);%
            \fi%
         \pgfsys@endpicture%
      }}%
      \global\setbox\yquant@draw@subcircuit@box=\hbox to \dimexpr\xmax-\xmin\relax {%
         \hskip-\dimexpr\yquant@draw@@x-.5\dimexpr\xmax-\xmin\relax\relax%
         \lower\pgf@picmaxy%
         \box\yquant@draw@subcircuit@box%
      }%
      \ht\yquant@draw@subcircuit@box=0pt%
      \dp\yquant@draw@subcircuit@box=\dimexpr\pgf@picmaxy-\pgf@picminy\relax%
      \expandafter%
   \endpgfinterruptboundingbox%
   \expandafter\edef\expandafter\yquant@draw@subcircuit@y\expandafter{%
      \the\dimexpr.5\pgf@picminy+.5\pgf@picmaxy\relax%
   }%
   \ifstrempty{#2}{%
      % However, if the outer node was not named, no access to the inner nodes is desired, so we delete all nodes again.
      \def\do##1{%
         \csgundef{pgf@sh@ns@##1}%
         \csgundef{pgf@sh@np@##1}%
         \csgundef{pgf@sh@nt@##1}%
         \csgundef{pgf@sh@pi@##1}%
         \csgundef{pgf@sh@ma@##1}%
      }%
      \dolistcsloop{\yquant@draw@subcircuit@nodelist}%
      \csgundef\yquant@draw@subcircuit@nodelist%
   }{%
      \ifcsname\yquant@prefix draw@subcircuit@nodelist\endcsname%
         \csxappto{\yquant@prefix draw@subcircuit@nodelist}%
                  {\csname\yquant@draw@subcircuit@nodelist\endcsname}%
      \fi%
   }%
}

\protected\long\def\yquant@draw@subcircuit@single#1#2#3{%
   \yquant@draw@subcircuit@prepare{#2}{#3}%
   \edef\cmd{%
      \noexpand\path (\yquant@draw@@x, \yquant@draw@subcircuit@y)%
         node[/yquant/every operator, /yquant/operators/every subcircuit,%
              /yquant/operators/every subcircuit box, /yquant/this operator,%
              /yquant/operators/this subcircuit box,%
              name prefix=, name suffix=, name=yquantbox]%
         {\box\yquant@draw@subcircuit@box};%
      \pgfshapeclippath{yquantbox}%
                       {/yquant/every operator, /yquant/operators/every subcircuit,%
                        /yquant/operators/every subcircuit box, /yquant/this operator,%
                        /yquant/operators/this subcircuit box}%
   }%
   \cmd%
   % see comment in draw@subcircuit@multi
   \yquant@softpath@extractmaxxat\pgfshapeclippathhorzresult{\yquant@register@get@y{#1}}%
   \let\pgfshapeclippathhorzresult=\empty%
   \yquant@circuit@extendwire{#1}{*}%
   \expandafter\yquant@circuit@extendcontrolline\expandafter%
      {\the\yquant@draw@@currentcontroltype}\yquant@draw@@x%
   % check for empty name parameter
   \ifstrempty{#3}\relax{%
      \pgfnodealias{\tikz@pp@name{#3}}{yquantbox}%
   }%
   \numdef\yquant@draw@@idx@content{\yquant@draw@@idx@content+1}%
}

\protected\long\def\yquant@draw@subcircuit@multi#1#2#3#4#5#6{%
   % there is no contiguous slicing for subcircuits, as they may have all kinds of wire operations that can extend beyond the individual slices, let alone ancillas
   \yquant@draw@subcircuit@prepare{#5}{#6}%
   % We need to somehow extract the y radius
   \edef\cmd{%
      \noexpand\path (\yquant@draw@@x, \yquant@draw@subcircuit@y)%
         node[/yquant/every operator, /yquant/operators/every subcircuit,%
              /yquant/operators/every subcircuit box, /yquant/this operator,%
              /yquant/operators/this subcircuit box,
              /yquant/operator/multi main=true,%
              name prefix=, name suffix=, name=yquantbox]%
            {\box\yquant@draw@subcircuit@box};%
      \pgfshapeclippath{yquantbox}%
                       {/yquant/every operator, /yquant/operators/every subcircuit,%
                        /yquant/operators/every subcircuit box, /yquant/this operator,%
                        /yquant/operators/this subcircuit box,%
                        /yquant/operator/multi main=true}%
   }%
   \cmd%
   % install the clippings - but only on wires that are visually between the first and list while not being part of the circuit.
   \let\nonaffectedpgfshapeclippathhorzresult=\pgfshapeclippathhorzresult%
   \yquant@for \i := #1 to #2 {%
      \xifinlist{\i}{#4}{%
         % Usually, we always begin with a wire from the center of the operator shape and clip the inner parts away. This can't be done here, as the wire needs to be drawn _inside_ of the outer box operator here. Instead of clipping against the clip path, we extract its maximum x position at the position of the wire (which is an overkill for simple shapes, but the allows to specify even more complicated ones) and place the wire at this position without clipping.
         % Note: this works very well for lines joining at perpendicular angles; but if the shape of the box is more fancy, while the position will be calculated correctly, the wire has a rectangular (or rounded, depending on the line cap) shape that is drawn on top of thw operator. While \yquant@softpath@extractmaxxat could without much effort determine exactly the segment of the path that corresponds to the rightmost line, we would then have to convert this single line into a closed path that fills the linewidth and clip against it to get proper joiners. Since most likely, no-one will ever need this, we don't do it. But file a feature request if desired.
         \yquant@softpath@extractmaxxat\nonaffectedpgfshapeclippathhorzresult%
                                       {\yquant@register@get@y\i}%
         \let\pgfshapeclippathhorzresult=\empty%
         \yquant@circuit@extendwire\i{*}%
      }{%
         \let\pgfshapeclippathhorzresult=\nonaffectedpgfshapeclippathhorzresult%
         \yquant@circuit@extendwire\i{center}%
      }%
   }%
   \expandafter\yquant@circuit@extendcontrolline\expandafter%
      {\the\yquant@draw@@currentcontroltype}\yquant@draw@@x%
   \ifstrempty{#6}\relax{%
      \pgfnodealias{\tikz@pp@name{#6}}{yquantbox}%
   }%
   \numdef\yquant@draw@@idx@content{\yquant@draw@@idx@content+1}%
}

\protected\def\yquant@draw@control#1#2#3{%
   \edef\cmd{%
      \noexpand\path (\yquant@draw@@x, \yquant@register@get@y{#2})%
         node[/yquant/every control, /yquant/every #1 control, /yquant/this control,%
              name prefix=, name suffix=, name=yquantbox]%
         {};%
      \pgfshapeclippath{yquantbox}%
                       {/yquant/every control, /yquant/every #1 control,%
                        /yquant/this control}%
   }%
   \cmd%
   \yquant@circuit@extendwire{#2}{center}%
   \yquant@draw@@currentcontroltype=\yquant@register@get@type{#2}\relax%
   \expandafter\yquant@circuit@extendcontrolline\expandafter%
      {\yquant@draw@@currentcontroltype}\yquant@draw@@x%
   % check for empty name parameter
   \ifstrempty{#3}\relax{%
      \pgfnodealias{\tikz@pp@name{#3}}{yquantbox}%
   }%
}

\protected\def\yquant@draw@pcontrol#1#2{%
   \let\idx=\yquant@draw@@idx@pcontrol%
   \yquant@draw@control{positive}{#1}{#2}%
   \numdef\yquant@draw@@idx@pcontrol{\yquant@draw@@idx@pcontrol+1}%
}

\protected\def\yquant@draw@ncontrol#1#2{%
   \let\idx=\yquant@draw@@idx@ncontrol%
   \yquant@draw@control{negative}{#1}{#2}%
   \numdef\yquant@draw@@idx@ncontrol{\yquant@draw@@idx@ncontrol+1}%
}

\protected\def\yquant@draw@cline{%
   \pgfscope%
      % install the clipping
      \pgfsyssoftpath@setcurrentpath\yquant@circuit@extendcontrolline@clip%
      % and invert it. It is sufficient to cover the current bounding box, as the wire will be drawn between existing operators.
      \ifyquantdebug%
         \pgfsetfillcolor{teal}%
         \pgfsetfillopacity{.3}%
         \pgfusepathqfill%
      \else%
         \begingroup%
            \pgftransformreset%
            \pgfpathrectanglecorners%
               {\pgfqpoint{\pgf@picminx}{\pgf@picminy}}%
               {\pgfqpoint{\pgf@picmaxx}{\pgf@picmaxy}}%
            \pgfseteorule% even-odd to properly invert the clipping
            \pgfusepathqclip%
         \endgroup%
      \fi%
      \edef\cmd{%
         \noexpand\path[/yquant/every control line]%
            \yquant@circuit@extendcontrolline@cmd;%
      }%
      \cmd%
   \endpgfscope%
}

\protected\def\yquant@draw@mline@prep{%
   \eappto\yquant@circuit@extendmultiline@total{%
      \yquant@draw@mline%
         {\unexpanded\expandafter{\yquant@circuit@extendmultiline@clip}}%
         {\yquant@circuit@extendmultiline@cmd}%
   }%
}

\protected\def\yquant@draw@mline#1#2{%
   \pgfscope%
      % install the clipping
      \def\pgfsyssoftpath@thepath{#1}%
      \pgfsyssoftpath@setcurrentpath\pgfsyssoftpath@thepath%
      % and invert it. It is sufficient to cover the current bounding box, as the wire will be drawn between existing operators.
      \ifyquantdebug%
         \pgfsetfillcolor{teal}%
         \pgfsetfillopacity{.3}%
         \pgfusepathqfill%
      \else%
         \begingroup%
            \pgftransformreset%
            \pgfpathrectanglecorners%
               {\pgfqpoint{\pgf@picminx}{\pgf@picminy}}%
               {\pgfqpoint{\pgf@picmaxx}{\pgf@picmaxy}}%
            \pgfseteorule% even-odd to properly invert the clipping
            \pgfusepathqclip%
         \endgroup%
      \fi%
      \path[/yquant/every multi line] #2;%
   \endpgfscope%
}

\protected\def\yquant@draw@alias#1{%
   \pgfnodealias{\tikz@pp@name{#1}}{\tikz@pp@name{#1-0}}%
}

\protected\def\yquant@draw@alias@ctrl#1#2{%
   \pgfnodealias{\tikz@pp@name{#1-#2}}{\tikz@pp@name{#1-#20}}%
}

\protected\def\yquant@draw@wire#1#2{%
   \begingroup%
      \yquant@register@get@typeywire{#1}\wiretype\wireypos\wirelast%
      \unless\ifnum\wiretype=\yquant@register@type@none%
         \edef\wirexprevpos{\expandafter\@firstoffour\wirelast}%
         \ifx0#2%
            \edef\wirexpos{\expandafter\@secondoffour\wirelast}%
         \else%
            \let\wirexpos=\yquant@env@end@xpos%
         \fi%
         \ifdim\wirexpos>\wirexprevpos %
            \edef\wirestyle{\noexpand\tikzset{%
               /yquant/this wire/.style={%
                  /yquant/every wire,%
                  /yquant/every \yquant@register@type@tostring\wiretype\space wire,%
                     \yquant@register@get@style{#1}%
               }, /yquant/this wire%
            }}%
            \wirestyle%
            % load all clippings
            \edef\wireclipping{%
               \unexpanded\expandafter\expandafter\expandafter{%
                  \expandafter\@thirdandfourthoffour\wirelast%
               }%
            }%
            \pgfscope%
               % install the clipping
               \pgfsyssoftpath@setcurrentpath\wireclipping%
               % invert the clipping
               \ifyquantdebug%
                  \pgfsetfillcolor{orange}%
                  \pgfsetfillopacity{.3}%
                  \pgfusepathqfill%
               \else%
                  % We need to access the current bounding box as well as other positions in the local coordinate frame. For this, transform the bounding box to the current frame (though this is expensive). Does this capture rotations correctly?
                  \begingroup%
                     \pgftransforminvert%
                     \pgfpointtransformednonlinear{\pgfqpoint{\pgf@picminx}{\pgf@picminy}}%
                     \global\@tempdima=\pgf@y%
                     \pgfpointtransformednonlinear{\pgfqpoint{\pgf@picmaxx}{\pgf@picmaxy}}%
                     \global\@tempdimb=\pgf@y%
                  \endgroup%
                  % To avoid rendering artifacts at all zoom levels with all renderers, we need to make the clipping region large. Let's try the current bounding box first.
                  % This may be insufficient if there no or a tiny wire label and only registers of a small height. In this case, take ten times the line width or at least 1cm, but don't let it affect the bounding box.
                  \ifdim\dimexpr\@tempdimb-\@tempdima\relax<10\pgflinewidth %
                     \@tempdima=\dimexpr\wireypos-5\pgflinewidth\relax%
                     \@tempdimb=\dimexpr\wireypos+5\pgflinewidth\relax%
                  \fi%
                  \ifdim\dimexpr\@tempdimb-\@tempdima\relax<1cm %
                     \@tempdima=\dimexpr\wireypos-5mm\relax%
                     \@tempdimb=\dimexpr\wireypos+5mm\relax%
                  \fi%
                  \pgfinterruptboundingbox%
                     \pgfpathrectanglecorners%
                        {\pgfqpoint{\dimexpr\wirexprevpos-2\pgflinewidth\relax}%
                                   {\@tempdima}}%
                        {\pgfqpoint{\dimexpr\wirexpos+2\pgflinewidth\relax}%
                                   {\@tempdimb}}%
                  \endpgfinterruptboundingbox%
                  \pgfseteorule% even-odd to properly invert the clipping
                  \pgfusepathqclip%
               \fi%
               % the clip inversion is left to the drawing commands (clip two \pgflinewidth more to avoid renderer artifacts)
               \csname yquant@draw@wire@\wiretype\endcsname{#1}%
            \endpgfscope%
         \fi%
      \fi%
   \endgroup%
}

% quantum wire
\protected\csdef{yquant@draw@wire@\yquant@register@type@q}#1{%
   \edef\cmd{%
      \noexpand\path [/yquant/this wire]%
         (\wirexprevpos,\wireypos) -- (\wirexpos,\wireypos);%
   }%
   \cmd%
}

% classical wire
\protected\csdef{yquant@draw@wire@\yquant@register@type@c}#1{%
   \edef\cmd{%
      \noexpand\path [/yquant/this wire]%
         (\wirexprevpos,\wireypos+2\pgflinewidth)--(\wirexpos,\wireypos+2\pgflinewidth)%
         (\wirexprevpos,\wireypos-2\pgflinewidth)--(\wirexpos,\wireypos-2\pgflinewidth);%
   }%
   \cmd%
}

% quantum-bundle
\protected\csdef{yquant@draw@wire@\yquant@register@type@qs}#1{%
   \edef\cmd{%
      \noexpand\path [/yquant/this wire]%
         (\wirexprevpos,\wireypos+2\pgflinewidth)--(\wirexpos,\wireypos+2\pgflinewidth)%
         (\wirexprevpos,\wireypos)--(\wirexpos,\wireypos)%
         (\wirexprevpos,\wireypos-2\pgflinewidth)--(\wirexpos,\wireypos-2\pgflinewidth);%
   }%
   \cmd%
}

\protected\long\def\yquant@draw@output@group#1{%
   \begingroup%
      \def\idx{0}%
      \yquant@set{#1}%
}

\let\yquant@draw@output@endgroup=\endgroup%

\protected\long\def\yquant@draw@output@single#1#2{%
   \path%
      (\yquant@env@end@xpos, \yquant@register@get@y{#1})%
      node[/yquant/every output,%
           /yquant/every \yquant@register@type@tostring{\yquant@register@get@type{#1}} output] {#2};
   \numdef\idx{\idx+1}%
}

\protected\long\def\yquant@draw@output@multi#1#2#3#4{%
   % extremely similar to \yquant@draw@multiinit
   \@tempdima=-.5\dimexpr\yquant@config@register@sep\relax%
   \dimdef\yquant@draw@multiinit@@min{\yquant@register@get@y{#1}-\@tempdima}%
   \dimdef\yquant@draw@multiinit@@max{\yquant@register@get@y{#2}+\@tempdima}%
   \dimdef\yquant@draw@multiinit@@total{%
      \yquant@draw@multiinit@@max-\yquant@draw@multiinit@@min%
   }%
   \def\pgfdecorationsegmentaspect{0}%
   \let\yquant@register@multi@contiguous=\yquant@draw@multiinit@contiguous%
   \let\pgfdecorationsegmentfromto=\empty%
   #3%
   \edef\pgfdecorationsegmentfromto{\expandafter\@gobble\pgfdecorationsegmentfromto}%
   \path[/yquant/every multi output]%
      (\yquant@env@end@xpos, \yquant@draw@multiinit@@min) --%
      (\yquant@env@end@xpos, \yquant@draw@multiinit@@max)%
      node {#4};%
   \numdef\idx{\idx+1}%
}
% END_FOLD

% BEGIN_FOLD Preparation of drawing a generic shape
% Most drawing operations will be realized through nodes
\let\yquant@draw@callback@box=\@gobble
\let\yquant@draw@callback@wire=\@gobble

\def\yquant@draw@sort#1#2{%
   \yquant@draw@sort@aux#1\relax#2\relax%
      \expandafter\@firstoftwo%
   \else%
      \expandafter\@secondoftwo%
   \fi%
}

\def\yquant@draw@sort@aux#1#2#3\relax#4#5#6\relax{%
   \unless\ifnum#2>#5\relax%
}

% generic shape of an operator
% #1: value
% #2: tikz options that select the correct shape
% #3: positive controls
% #4: negative controls
% #5: targets
\protected\long\def\yquant@draw#1#2#3#4#5{%
   % setup the required macros
   \yquant@circuit@operator{#3}{#4}{#5}%
   \yquant@draw@{#1}{#2}%
}

\protected\long\def\yquant@draw@#1#2{%
   \yquant@sort@clear%
   \def\idx{0}%
   \dimen2=\yquant@config@operator@minimum@width%
   % BEGIN_FOLD register
   \def\do##1{%
      \ifx\yquant@lang@attr@name\empty%
         \let\nodename=\empty%
      \else%
         \edef\nodename{\yquant@lang@attr@name-\idx}%
      \fi%
      \ifyquant@firsttoken\yquant@register@multi{##1}{%
         \yquant@draw@@multi{#1}{#2}{##1}%
      }{%
         \yquant@draw@@single{#1}{#2}{##1}%
      }%
         \ifx\yquant@draw@@multi\yquant@draw@@multiinit%
            % if we draw an initialization (whether multi or single), this possibly affects the minimal x position. All other gates will be shifted so that they cannot extend beyond the minimal position.
            \ifdim\pgf@picminx<\csname\yquant@prefix xmin\endcsname%
               \csxdef{\yquant@prefix xmin}{\the\pgf@picminx}%
            \fi%
         \fi%
         \expandafter%
      \endpgfinterruptboundingbox%
      \expandafter\dimen\expandafter0\expandafter=%
         \the\dimexpr\pgf@picmaxx-\pgf@picminx\relax\relax%
      \ifdim\dimen0>\dimen2 %
         \dimen2=\dimen0 %
      \fi%
      \numdef\idx{\idx+1}%
   }%
   \dolistloop\yquant@circuit@operator@targets%
   % END_FOLD
   \yquant@draw@@controls%
   \yquant@draw@@finalize{#1}{#2}%
}

\protected\def\yquant@draw@@controls@loop#1#2{%
   \ifx\yquant@lang@attr@name\empty%
      \let\nodename=\empty%
   \else%
      \edef\nodename{\yquant@lang@attr@name-#1\idx}%
   \fi%
   \yquant@sort@eadd{%
      \expandafter\noexpand\csname yquant@draw@#1control\endcsname%
         {#2}% register index
         {\nodename}%
   }%
   \unless\ifdefined\yquant@draw@controltype%
      \edef\yquant@draw@controltype{#2}%
   \fi%
   \numdef\idx{\idx+1}%
}

\protected\def\yquant@draw@@controls{%
   \ifyquant@circuit@operator@hasControls%
      \def\idx{0}%
      \forlistloop{\yquant@draw@@controls@loop p}\yquant@circuit@operator@pctrls%
      \def\idx{0}%
      \forlistloop{\yquant@draw@@controls@loop n}\yquant@circuit@operator@nctrls%
      \unless\ifdefined\yquant@draw@controltype%
         \PackageError{yquant.sty}{Assertion failure}%
                      {Internal inconsistency in yquant: Controlled action detected, but no controls were found.}%
      \fi%
   \fi%
}

\protected\long\def\yquant@draw@@finalize#1#2{%
   % We now know the dimensions of all the registers (though we didn't bother with the height of the control knobs [if present], we just assume they are too small to change this).
   \protected\def\idx{}%
   \protected@edef\yquant@draw@append{%
      \noexpand\yquant@draw@group%
         {\the\dimexpr\yquant@circuit@operator@x+.5\dimen2\relax}% mid x position
         \ifyquant@circuit@operator@hasControls%
           \yquant@draw@controltype%
         \else%
           F%
         \fi% if-switch whether controls are present
         {\yquant@attrs@remaining}% custom style
         {#2}% operator style
         {#1}%
   }%
   \yquant@sort\yquant@draw@sort%
   \advance \dimen2 by \yquant@circuit@operator@x\relax%
   \ifyquant@circuit@operator@hasControls%
      % If we draw a control line, all intermediate registers are affected in their position so that the line is never crossed. If the vertical line is instead caused by a multi register, \yquant@draw@@finalize@ctrl will be responsible for advancing only the affected positions.
      \yquant@for \yquant@i := \yquant@circuit@operator@minctrl to \yquant@circuit@operator@maxctrl {%
         \yquant@register@set@x\yquant@i{\the\dimen2}%
      }%
   \fi%
   \def\do##1{%
      \appto\yquant@draw@append{##1}%
      \yquant@draw@@finalize@ctrl##1%
   }%
   \yquant@sort@dolistloop%
   \csxappto{\yquant@prefix draw}{%
      \unexpanded\expandafter{\yquant@draw@append}%
      \noexpand\yquant@draw@endgroup%
         \ifyquant@circuit@operator@hasControls%
            T%
         \else%
            F%
         \fi%
         \ifx\yquant@lang@attr@name\empty%
            {}0%
         \else%
            {\yquant@lang@attr@name}%
            \ifnum\yquant@circuit@operator@numtarget=1 %
               \ifnum\yquant@circuit@operator@numpctrl=1 %
                  \ifnum\yquant@circuit@operator@numnctrl=1 %
                     7%
                  \else%
                     6%
                  \fi%
               \else%
                  \ifnum\yquant@circuit@operator@numnctrl=1 %
                     5%
                  \else%
                     4%
                  \fi%
               \fi%
            \else%
               \ifnum\yquant@circuit@operator@numpctrl=1 %
                  \ifnum\yquant@circuit@operator@numnctrl=1 %
                     3%
                  \else%
                     2%
                  \fi%
               \else%
                  \ifnum\yquant@circuit@operator@numnctrl=1 %
                     1%
                  \else%
                     0%
                  \fi%
               \fi%
            \fi%
         \fi%
   }%
}

\protected\def\yquant@draw@@single#1#2#3{%
   \yquant@sort@eadd{%
      \yquant@draw@single%
         {#3}% register index
         {\nodename}%
   }%
   % determine the actual dimensions by a virtual draw command
   \pgfinterruptboundingbox%
      \yquant@env@virtualize@path%
      \path%
         (0pt, 0pt)%
         node[/yquant/every operator, #2, /yquant/this operator,%
              name prefix=, name suffix=, name=] {#1};%
      \yquant@register@update@height{#3}{\the\pgf@picmaxy}%
      \yquant@register@update@depth{#3}{\the\dimexpr-\pgf@picminy\relax}%
}

\protected\def\yquant@draw@@multi#1#2#3{%
   \yquant@sort@eadd{%
      \yquant@draw@multi%
         #3%
         {\nodename}%
   }%
   % Determining the actual height is a problem - where to store its value? If there are single-register parts, we update the height accordingly; else we just assume that there is always enough space for such a control, since it anyways already spans multiple registers. (TODO?)
   \pgfinterruptboundingbox%
      \yquant@env@virtualize@path%
      \def\yquant@draw@@content{#1}%
      \def\yquant@draw@@style{#2}%
      \let\yquant@register@multi@contiguous=\yquant@draw@@multi@contiguous%
      \@fifthoffive#3%
      \ifdim\pgf@picmaxy=-16000pt %
         % if there was no single contiguous part before, determine the width now
         \path%
            (0pt, 0pt)%
            node[/yquant/every operator, #2, /yquant/this operator,%
                 name prefix=, name suffix=, name=] {#1};%
      \fi%
}

\protected\def\yquant@draw@@multi@contiguous#1#2#3{%
   \ifnum#1=#2 %
      % we only care about single-register parts
      \global\pgf@picmaxy=-16000pt %
      \global\pgf@picminy=16000pt %
      \edef\cmd{%
         \noexpand\path (0pt, 0pt)%
            node[/yquant/every operator, \yquant@draw@@style, /yquant/this operator,%
                 /yquant/operator/multi main=\ifnum#3=1 true\else false\fi,%
                 name prefix=, name suffix=, name=]%
               {\unexpanded\expandafter{\yquant@draw@@content}};%
      }%
      \cmd%
      \yquant@register@update@height{#1}{\the\pgf@picmaxy}%
      \yquant@register@update@depth{#1}{\the\dimexpr-\pgf@picminy\relax}%
   \fi%
}

\protected\def\yquant@draw@@multiinit#1#2#3{%
   \yquant@sort@eadd{%
      \yquant@draw@multiinit%
         #3%
         {\nodename}%
   }%
   % Determining the actual height is a problem - where to store its value? We just assume that there is always enough space for such a control, since it anyway already spans multiple registers. (TODO?)
   \pgfinterruptboundingbox%
      \yquant@env@virtualize@path%
      \path%
         (0pt, 0pt)%
         node[/yquant/every operator, #2, /yquant/every multi label,%
              name prefix=, name suffix=, name=] {#1};%
}

\protected\long\def\yquant@draw@@subcircuit{%
   \yquant@sort@clear%
   \def\idx{0}%
   \dimen2=\yquant@config@operator@minimum@width%
   % BEGIN_FOLD register
   \def\do##1{%
      \ifx\yquant@lang@attr@name\empty%
         \let\nodename=\empty%
      \else%
         \edef\nodename{\yquant@lang@attr@name-\idx}%
      \fi%
      \ifyquant@firsttoken\yquant@register@multi{##1}{%
         \yquant@draw@@subcircuit@multi{##1}%
      }{%
         \yquant@draw@@subcircuit@single{##1}%
      }%
      \dimen0=\dimexpr%
         \csname yquant@env\yquant@circuit@subcircuit@id @xmax\endcsname-%
         \csname yquant@env\yquant@circuit@subcircuit@id @xmin\endcsname%
      \relax\relax%
      \pgfinterruptboundingbox%
         % There will still be some extra width taken by separations or other things related to the box itself.
         \yquant@env@virtualize@path%
         \path%
            (0pt, 0pt)%
            node[/yquant/every operator, /yquant/operators/every subcircuit,%
                 /yquant/operators/every subcircuit box, /yquant/this operator,%
                 /yquant/operators/this subcircuit box,%
                 name prefix=, name suffix=, name=]%
            {\hbox to \dimen0{}};%
         % We also don't update the height here, as the height of the subcircuit is undetermined as long as we did not match inner and outer wires.
         \expandafter%
      \endpgfinterruptboundingbox%
      \expandafter\dimen\expandafter0\expandafter=%
         \the\dimexpr\pgf@picmaxx-\pgf@picminx\relax\relax%
      \ifdim\dimen0>\dimen2 %
         \dimen2=\dimen0 %
      \fi%
      \numdef\idx{\idx+1}%
   }%
   \dolistloop\yquant@circuit@operator@targets%
   % END_FOLD
   \yquant@draw@@controls%
   \yquant@draw@@finalize{}{/yquant/operators/every subcircuit}%
}

\protected\def\yquant@draw@@subcircuit@single#1{%
   \edef\yquant@circuit@subcircuit@param{#1\yquant@list@delim}%
   \yquant@circuit@subcircuit%
   \yquant@sort@eadd{%
      \yquant@draw@subcircuit@single%
         {#1}% register index
         {\yquant@circuit@subcircuit@id}%
         {\nodename}%
   }%
}

\protected\def\yquant@draw@@subcircuit@multi#1{%
   \edef\yquant@circuit@subcircuit@param{\@fifthoffive#1}%
   \yquant@sort@list\yquant@circuit@subcircuit@param\yquant@sort@ascending%
   \yquant@circuit@subcircuit%
   \yquant@sort@eadd{%
      \yquant@draw@subcircuit@multi%
         #1%
         {\yquant@circuit@subcircuit@id}%
         {\nodename}%
   }
}

\protected\long\def\yquant@draw@@output@single#1#2{%
   \path
      (\yquant@env@end@xpos, 0pt)%
      node[/yquant/every output,%
           /yquant/every \yquant@register@type@tostring{\yquant@register@get@type{#1}} output] {#2};%
   \numdef\idx{\idx+1}%
}

\protected\long\def\yquant@draw@@output@multi#1{%
   \path[/yquant/every multi output]%
      (\yquant@env@end@xpos, 0pt) -- (\yquant@env@end@xpos, 1cm)%
      node {#1};%
   \numdef\idx{\idx+1}%
}

\def\yquant@draw@@finalize@ctrl#1{%
   \let\next=\yquant@draw@@finalize@ctrl@single%
   \ifx\yquant@draw@multi#1%
      \let\next=\yquant@draw@@finalize@ctrl@multi%
   \else%
      \ifx\yquant@draw@multiinit#1%
         \let\next=\yquant@draw@@finalize@ctrl@multi%
      \else%
         \ifx\yquant@draw@subcircuit@multi#1%
            \let\next=\yquant@draw@@finalize@ctrl@subcircuit@multi%
         \else%
            \ifx\yquant@draw@subcircuit@single#1%
               \let\next=\yquant@draw@@finalize@ctrl@subcircuit@single%
            \fi%
         \fi%
      \fi%
   \fi%
   \next%
}

\protected\def\yquant@draw@@finalize@ctrl@single#1#2{%
   \unless\ifyquant@circuit@operator@hasControls%
      \yquant@register@set@x#1{\the\dimen2}%
   \fi%
   \eappto\yquant@draw@append{%
      \yquant@draw@callback@wire{#1}%
   }%
}

\def\yquant@draw@@finalize@ctrl@subcircuit@single#1#2#3{%
   \yquant@draw@@finalize@ctrl@single{#1}{#3}%
}

\protected\def\yquant@draw@@finalize@ctrl@singleinit#1#2{%
   \eappto\yquant@draw@append{%
      \yquant@draw@callback@wire{#1}%
   }%
}

\protected\def\yquant@draw@@finalize@ctrl@multi#1#2#3#4#5{%
   \unless\ifyquant@circuit@operator@hasControls{%
      % \yquant@for uses \loop...\repeat and hence redefines \body, which would destroy an outer loop.
      % if we did not draw a control line, the x position has not yet been set. A multi-qubit register might visually extend over multiple registers that are not even part, hence we update them all.
      \yquant@for \yquant@i := #1 to #2 {%
         \yquant@register@set@x\yquant@i{\the\dimen2}%
      }%
   }\fi%
   \let\yquant@register@multi@contiguous=\yquant@draw@@finalize@ctrl@multi@contiguous%
   \ifyquant@circuit@operator@hasControls%
      \ifyquant@config@operator@multi@warn%
         \def\yquant@draw@@finalize@ctrl@multi@contiguous@warn{0}%
      \else%
         \def\yquant@draw@@finalize@ctrl@multi@contiguous@warn{2}%
      \fi%
   \else%
      \def\yquant@draw@@finalize@ctrl@multi@contiguous@warn{2}%
   \fi%
   \cslet{\yquant@prefix finalize@ctrl@draw@appto}\empty%
   #4%
   \eappto\yquant@draw@append{\csname\yquant@prefix finalize@ctrl@draw@appto\endcsname}%
   \csgundef{\yquant@prefix finalize@ctrl@draw@appto}%
}

\protected\def\yquant@draw@@finalize@ctrl@subcircuit@multi@loop#1{%
   \begingroup%
      \edef\tmp{\yquant@draw@callback@wire{#1}}%
      \expandafter%
   \endgroup%
   \expandafter\appto\expandafter\yquant@draw@append\expandafter{\tmp}%
}

\protected\def\yquant@draw@@finalize@ctrl@subcircuit@multi#1#2#3#4#5#6{%
   % there are no contiguous parts here, don't call the normal @multi
   \unless\ifyquant@circuit@operator@hasControls{%
      % \yquant@for uses \loop...\repeat and hence redefines \body, which would destroy an outer loop.
      % if we did not draw a control line, the x position has not yet been set. A multi-qubit register might visually extend over multiple registers that are not even part, hence we update them all.
      \yquant@for \yquant@i := #1 to #2 {%
         \yquant@register@set@x\yquant@i{\the\dimen2}%
      }%
   }\fi%
   \forlistloop\yquant@draw@@finalize@ctrl@subcircuit@multi@loop{#4}%
}

\protected\def\yquant@draw@@finalize@ctrl@multiinit#1#2#3#4#5{%
   % this is called from a do loop itself, so preserve \do (but do not enter grouping)
   \let\yquant@register@multi@contiguous=\yquant@draw@@finalize@ctrl@multi@contiguous%
   \ifyquant@circuit@operator@hasControls%
      \ifyquant@config@operator@multi@warn%
         \def\yquant@draw@@finalize@ctrl@multi@contiguous@warn{0}%
      \else%
         \def\yquant@draw@@finalize@ctrl@multi@contiguous@warn{2}%
      \fi%
   \else%
      \def\yquant@draw@@finalize@ctrl@multi@contiguous@warn{2}%
   \fi%
   \cslet{\yquant@prefix finalize@ctrl@draw@appto}\empty%
   #4%
   \eappto\yquant@draw@append{\csname\yquant@prefix finalize@ctrl@draw@appto\endcsname}%
   \csgundef{\yquant@prefix finalize@ctrl@draw@appto}%
}

\protected\def\yquant@draw@@finalize@ctrl@multi@contiguous#1#2#3{%
   \ifnum\yquant@draw@@finalize@ctrl@multi@contiguous@warn=1 %
      \PackageWarning{yquant.sty}{Ambiguous operation: multiple discontiguous multi-register operations in combination with controls make it hard to visually determine on which registers the gates act on.}%
      % switch the warning off for this group (which is a single operation)
      \yquant@config@operator@multi@warnfalse%
      \def\yquant@draw@@finalize@ctrl@multi@contiguous@warn{2}%
   \else%
      \numdef\yquant@draw@@finalize@ctrl@multi@contiguous@warn{%
         \yquant@draw@@finalize@ctrl@multi@contiguous@warn+1%
      }%
   \fi%
   {% save \body
      \yquant@for \yquant@i := #1 to #2 {{% let inner loop mess up with macros
         \csxappto{\yquant@prefix finalize@ctrl@draw@appto}{%
            \expandafter\yquant@draw@callback@wire\expandafter{\yquant@i}%
         }%
      }}%
   }%
}
% END_FOLD