\pgfdeclaremetadecoration{tikz@internal}{pre}{%
   \state{pre}[width=\pgfkeysvalueof{/pgf/decoration/pre length}+\pgfkeysvalueof{/pgf/decoration/post length}, next state=main]{%
      \appto\tikz@dec@shift{\pgftransformxshift{-\pgfkeysvalueof{/pgf/decoration/post length}}}%
      \tikz@dec@trans%
      \decoration{\pgfkeysvalueof{/pgf/decoration/pre}}%
   }%
   \state{main}[width=\pgfmetadecoratedremainingdistance, next state=final]{%
      \tikz@dec@trans%
      \decoration{\tikz@decoration@name}%
   }%
   \state{final}{%
      \tikz@dec@trans%
      \decoration{\pgfkeysvalueof{/pgf/decoration/post}}%
   }%
}

% BEGIN_FOLD Circuit layout
\ifnum\yquant@compat<2 %
   \pgfqkeys{/yquant}{%
      register/minimum height/.code=%
         {\dimdef\yquant@config@register@minimum@height{.5\dimexpr#1\relax}%
          \let\yquant@config@register@minimum@depth=\yquant@config@register@minimum@height}%
   }
\else%
   \pgfqkeys{/yquant}{%
      register/minimum height/.code=%
         {\dimdef\yquant@config@register@minimum@height{#1}},%
      register/minimum depth/.code=%
         {\dimdef\yquant@config@register@minimum@depth{#1}}%
   }
\fi
\pgfqkeys{/yquant}{%
   register/separation/.code=%
      {\dimdef\yquant@config@register@sep{#1}},%
   operator/minimum width/.code=%
      {\dimdef\yquant@config@operator@minimum@width{#1}},%
   operator/separation/.code=%
      {\dimdef\yquant@config@operator@sep{#1}},%
   operator/multi warning/.is if=%
      yquant@config@operator@multi@warn%
}%
% END_FOLD
% BEGIN_FOLD Register creation
\pgfqkeys{/yquant}{%
   register/default name/.store in=%
      \yquant@config@register@default@name%
}
\ifnum\yquant@compat<2 %
   \pgfqkeys{/yquant}{%
      every label/.style=%
         {shape=yquant-init, anchor=center, align=right, outer xsep=2pt},%
      every multi label/.style=%
         {draw, decoration={gapped brace, mirror, raise=2pt}, decorate}%
   }
\else
   \pgfqkeys{/yquant}{%
      every label/.style=%
         {shape=yquant-init, anchor=center, align=right, outer xsep=2pt,%
          /yquant/operator/if multi={draw, decoration={gapped brace, mirror, raise=2pt}, decorate}}%
    }
\fi
\pgfqkeys{/yquant}{
   every initial label/.style=%
      {anchor=east},%
   every qubit label/.style=%
      {},%
   every cbit label/.style=%
      {},%
   every qubits label/.style=%
      {}%
}
\ifnum\yquant@compat<2 %
   \pgfqkeys{/yquant}{%
      every input label/.style=%
         {}%
   }%
\fi%
% END_FOLD
% BEGIN_FOLD Register outputs
\ifnum\yquant@compat<2 %
   \pgfqkeys{/yquant}{%
      every output/.style=%
         {shape=yquant-output, anchor=west, align=left, outer xsep=2pt},%
      every multi output/.style=%
         {draw, decoration={gapped brace, raise=2pt}, decorate}%
   }
\else
   \pgfqkeys{/yquant}{%
      every output/.style=%
         {shape=yquant-output, anchor=west, align=left, outer xsep=2pt,%
          /yquant/operator/if multi={draw, decoration={gapped brace, raise=2pt}, decorate}}%
   }
\fi
\pgfqkeys{/yquant}{%
   every qubit output/.style=%
      {},%
   every cbit output/.style=%
      {},%
   every qubits output/.style=%
      {}%
}
% END_FOLD
% BEGIN_FOLD General styling
\pgfqkeys{/yquant}{%
   every circuit/.style={%
      every node/.prefix style={transform shape}
   }%
}
\@ifpackagelater{pgf}{2020/09/31}\relax{
   \pgfkeys{/yquant/every circuit/.append style={%
      every label/.prefix style={transform shape=false}%
   }}
}
\pgfqkeys{/yquant}{%
   every wire/.style=%
      {draw},%
   every qubit wire/.style=%
      {},%
   every cbit wire/.style=%
      {},%
   every qubits wire/.style=%
      {},%
   every control line/.style=%
      {draw},%
   every control/.style=%
      {shape=yquant-circle, anchor=center, radius=.5mm},%
   every positive control/.style=%
      {fill=black},%
   every negative control/.style=%
      {draw},%
   every operator/.style=%
      {anchor=center},%
   every multi line/.style=%
      {draw, decoration={snake, amplitude=.25mm, segment length=5pt}, decorate},%
   this operator/.style=%
      {},%
   this control/.style=%
      {},%
   operator style/.style=%
      {/yquant/this operator/.append style={#1}},%
   control style/.style=%
      {/yquant/every control line/.append style={#1},%
       /yquant/this control/.append style={#1}},%
   style/.style=%
      {/yquant/this operator/.append style={#1},%
       /yquant/every control line/.append style={#1},%
       /yquant/this control/.append style={#1}},%
   operator/multi as single/.style=%
      {/yquant/every multi line/.style=/yquant/every control line},%
   operator/if multi/.code=%
      {\ifyquant@config@operator@multi\pgfkeysalso{#1}\fi},%
   circuit/seamless/.is if=yquant@config@circuit@seamless,%
   every post measurement control/.is choice,
   every post measurement control/indirect/.code={%
      \undef\yquant@lang@attr@directcontrol%
   },%
   every post measurement control/direct/.code={%
      \let\yquant@lang@attr@directcontrol=\relax%
   }%
}
% END_FOLD
% BEGIN_FOLD Styles for operators
\ifnum\yquant@compat<2 %
   \pgfqkeys{/yquant}{%
      operators/every barrier/.style=%
         {shape=yquant-line, dashed, draw}%
   }
\else
   \pgfqkeys{/yquant}{%
      operators/every barrier/.style=%
         {shape=yquant-line, dashed, draw, shorten <= -1mm, shorten >= -1mm}
   }
\fi
\pgfqkeys{/yquant}{%
   operators/every box/.style=%
      {shape=yquant-rectangle, draw, align=center, inner xsep=1mm, x radius=2mm, y radius=2.47mm},%
}
\ifnum\yquant@compat<2 %
   \pgfqkeys{/yquant}{%
      operators/every custom gate/.style=%
         {/yquant/operators/subcircuit/frameless, /yquant/register/default name=}%
   }
\else
   \pgfqkeys{/yquant}{%
      operators/every custom gate/.style=%
         {/yquant/operators/subcircuit/seamless}%
   }
\fi
\pgfqkeys{/yquant}{%
   operators/every dmeter/.style=%
      {shape=yquant-dmeter, x radius=2mm, y radius=2mm, draw},
   % every h is implicitly defined during gate declaration
   operators/every inspect/.style=%
      {shape=yquant-output, align=left, outer xsep=.3333em, y radius=2.47mm,%
       /yquant/operator/if multi={draw, decoration={gapped brace, raise=2pt}, decorate}},%
   operators/every measure/.style=%
      {shape=yquant-measure, x radius=4mm, y radius=2.5mm, draw},%
   operators/every measure meter/.style=%
      {draw, -{Latex[length=2.5pt]}},%
   operators/every not/.style=%
      {shape=yquant-oplus, radius=1.3mm, draw},%
   operators/every pauli/.style=%
      {/yquant/operators/every box},%
   operators/every phase/.style=%
      {shape=yquant-circle, radius=.5mm, fill},%
   operators/every slash/.style=%
      {shape=yquant-slash, x radius=.5mm, y radius=.7mm, draw},%
   operators/every subcircuit/.style=%
      {},%
   operators/every subcircuit box/.style=%
      {/yquant/operators/every box},%
   subcircuit box style/.style=%
      {/yquant/operators/every subcircuit box/.append style={#1}},%
   operators/this subcircuit box/.style=%
      {},%
   this subcircuit box style/.style=%
      {/yquant/operators/this subcircuit box/.append style={#1}},%
   operators/subcircuit/frameless/.style=
      {/yquant/operators/this subcircuit box/.append style={draw=none, inner sep=0pt}},%
   operators/subcircuit/seamless/.code=%
      {\pgfkeysalso{/yquant/operators/subcircuit/frameless, /yquant/register/default name=}%
       \letcs\yquant@prevseamless{\yquant@prefix seamless}%
       \yquant@config@circuit@seamlesstrue},
   operators/subcircuit/name mangling/.is choice,%
   operators/subcircuit/name mangling/prefix or discard/.code=%
      {\def\yquant@config@operator@subcircuit@mangling{0}},%
   operators/subcircuit/name mangling/prefix or transparent/.code=%
      {\def\yquant@config@operator@subcircuit@mangling{1}},%
   operators/subcircuit/name mangling/transparent/.code=%
      {\def\yquant@config@operator@subcircuit@mangling{2}},%
   operators/subcircuit/name mangling/discard/.code=%
      {\def\yquant@config@operator@subcircuit@mangling{3}},%
   operators/subcircuit/name mangling reset/.is if=%
      yquant@config@operator@subcircuit@manglingreset,
   operators/every swap/.style=%
      {shape=yquant-swap, radius=.75mm, draw},%
   operators/every wave/.style=%
      {shape=yquant-circle, radius=.5mm, fill},%
   % every x is implicitly defined during gate declaration
   operators/every xx/.style=%
      {shape=yquant-rectangle, radius=.75mm, draw},%
   % every y is implicitly defined during gate declaration
   % every z is implicitly defined during gate declaration
   operators/every zz/.style=%
      {shape=yquant-circle, radius=.5mm, fill},%
}
% END_FOLD
% BEGIN_FOLD Internal styles (undocument, do not use!)
\pgfqkeys{/yquant}{%
   internal/move label/.code=%
      {\yquant@config@operator@position@rightaligntrue},
   internal/no advance/.code=%
      {\unless\ifyquantmeasuring% We don't draw subcircuits during measuring time, but we need their advancement
          \yquant@config@operator@position@advancefalse%
       \fi},
   internal/multi main/.is if=%
      yquant@config@internal@multi@main,%
}
\ifnum\yquant@compat<2 %
   \pgfqkeys{/yquant}{%
      internal/squeeze slash/.code=%
         {\dimdef\yquant@config@operator@sep{.5\dimexpr\yquant@config@operator@sep-\pgflinewidth\relax}%
          \yquant@config@operator@position@advancefalse},%
   }
\else%
   \pgfqkeys{/yquant}{%
      internal/squeeze slash/.code={}%
   }
\fi
% END_FOLD

\def\yquant@config@register@default@name{\regidx}
\def\yquant@config@register@minimum@height{1.5mm}
\def\yquant@config@register@minimum@depth{1.5mm}
\def\yquant@config@register@sep{1mm}
\def\yquant@config@operator@sep{1mm}
\def\yquant@config@operator@minimum@width{5mm}
\newif\ifyquant@config@internal@multi@main%
\yquant@config@internal@multi@maintrue
\newif\ifyquant@config@operator@multi@warn
\yquant@config@operator@multi@warntrue
\newif\ifyquant@config@operator@position@rightalign
\newif\ifyquant@config@operator@position@advance
\yquant@config@operator@position@advancetrue
\newif\ifyquant@config@circuit@seamless
\def\yquant@config@operator@subcircuit@mangling{0}
\newif\ifyquant@config@operator@subcircuit@manglingreset
\yquant@config@operator@subcircuit@manglingresettrue
\newif\ifyquant@config@operator@multi

\protected\def\yquant@config@operator@subcircuit@mangling@set#1{%
   \ifyquant@config@operator@subcircuit@manglingreset%
      \def\yquant@config@operator@subcircuit@mangling{#1}%
   \fi%
}