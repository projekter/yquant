\pgfdeclaremetadecoration{tikz@internal}{pre}{%
   \state{pre}[width=\pgfkeysvalueof{/pgf/decoration/pre length}+\pgfkeysvalueof{/pgf/decoration/post length}, next state=main]{%
      \appto\tikz@dec@shift{\pgftransformxshift{-\pgfkeysvalueof{/pgf/decoration/post length}}}%
      \tikz@dec@trans%
      \decoration{\pgfkeysvalueof{/pgf/decoration/pre}}%
   }%
   \state{main}[width=\pgfmetadecoratedremainingdistance, next state=final]{%
      \tikz@dec@trans%
      \decoration{\tikz@decoration@name}%
   }%
   \state{final}{%
      \tikz@dec@trans%
      \decoration{\pgfkeysvalueof{/pgf/decoration/post}}%
   }%
}

% BEGIN_FOLD Circuit layout
\ifnum\yquant@compat<2 %
   \pgfqkeys{/yquant}{%
      register/minimum height/.code=%
         {\dimdef\yquant@config@register@minimum@height{.5\dimexpr#1\relax}%
          \let\yquant@config@register@minimum@depth=\yquant@config@register@minimum@height}%
   }
\else%
   \pgfqkeys{/yquant}{%
      register/minimum height/.code=%
         {\dimdef\yquant@config@register@minimum@height{#1}},%
      register/minimum depth/.code=%
         {\dimdef\yquant@config@register@minimum@depth{#1}},%
      register/minimum left/.forward to=/yquant/register/minimum height,%
      register/minimum right/.forward to=/yquant/register/minimum depth,%
      register/minimum before/.forward to=/yquant/register/minimum height,%
      register/minimum after/.forward to=/yquant/register/minimum after%
   }
\fi
\def\yquant@config@draw@ensurelayers@addwires#1,main,#2\yquant@stop{%
   \ifstrempty{#2}{%
      % main was the last item in the list, so we needed to add the comma
      \edef\pgf@layerlist{\@gobble#1,wires,main}%
   }{%
      % there were more items in the list, so we have a spurious comma in #2
      \expandafter\yquant@config@draw@ensurelayers@addwires@%
         \expandafter,\pgf@layerlist\yquant@stop%
   }%
}
\def\yquant@config@draw@ensurelayers@addwires@#1,main,#2\yquant@stop{%
   \edef\pgf@layerlist{\@gobble#1,wires,main,#2}%
}
\def\yquant@config@draw@ensurelayers@haswires#1,wires,#2\yquant@stop{%
   \ifstrempty{#2}{%
      % wires not present yet, take action
      \expandafter\yquant@config@draw@ensurelayers@addwires%
         \expandafter,\pgf@layerlist,\yquant@stop%
   }\relax%
}
\def\yquant@config@draw@ensurelayers@addbehindwires#1,wires,#2\yquant@stop{%
   \ifstrempty{#2}{%
      % wires was the last item in the list, so we needed to add the comma (should never happen)
      \edef\pgf@layerlist{\@gobble#1,behindwires,wires}%
      \PackageWarning{yquant.sty}{The wires layer was set on top of the main layer - expect undesirable outcomes.}%
   }{%
      % there were more items in the list, so we have a spurious comma in #2
      \expandafter\yquant@config@draw@ensurelayers@addbehindwires@%
         \expandafter,\pgf@layerlist\yquant@stop%
   }%
}
\def\yquant@config@draw@ensurelayers@addbehindwires@#1,wires,#2\yquant@stop{%
   \edef\pgf@layerlist{\@gobble#1,behindwires,wires,#2}%
}
\def\yquant@config@draw@ensurelayers@hasbehindwires#1,behindwires,#2\yquant@stop{%
   \ifstrempty{#2}{%
      % wires not present yet, take action
      \expandafter\yquant@config@draw@ensurelayers@addbehindwires%
         \expandafter,\pgf@layerlist,\yquant@stop%
   }\relax%
}
\def\yquant@config@draw@ensurelayers{%
   \expandafter\yquant@config@draw@ensurelayers@haswires%
      \expandafter,\pgf@layerlist,wires,\yquant@stop%
   \expandafter\yquant@config@draw@ensurelayers@hasbehindwires%
      \expandafter,\pgf@layerlist,behindwires,\yquant@stop%
   \ifyquant@config@draw@quality%
      % maybe the option was set outside of a tikzpicture, then we need to keep this
      \global\let\endpgfpicture=\yquant@old@endpgfpicture%
      \global\yquant@config@requirelayersfalse%
   \fi%
}

\pgfqkeys{/yquant}{%
   register/separation/.code=%
      {\dimdef\yquant@config@register@sep{#1}},%
   operator/minimum width/.code=%
      {\dimdef\yquant@config@operator@minimum@width{#1}},%
   operator/minimum extent/.forward to=/yquant/operator/minimum width,%
   operator/separation/.code=%
      {\dimdef\yquant@config@operator@sep{#1}},%
   operator/multi warning/.is if=%
      yquant@config@operator@multi@warn,%
   drawing mode/.is choice,%
   drawing mode/quality/.code=%
      {\yquant@config@draw@qualitytrue%
       \pgfkeysalso{/yquant/default background=none}%
       \unless\ifpgfpicture%
          % we do not reset the special layer injection (there may be multiple yquant environments, and only one of them had the "size" drawing mode; then, we still need the layers), unless this is actually set outside of a tikzpicture
          \ifdefined\yquant@old@endpgfpicture% only if we changed before
            \global\let\endpgfpicture=\yquant@old@endpgfpicture%
            \global\yquant@config@requirelayersfalse%
          \fi%
       \fi},%
   drawing mode/size/.code=%
      {\yquant@config@draw@qualityfalse%
       \pgfkeysalso{/yquant/default background=white}%
       \unless\ifyquant@config@requirelayers%
          \global\let\yquant@old@endpgfpicture=\endpgfpicture%
          \gpreto\endpgfpicture\yquant@config@draw@ensurelayers%
       \fi%
       \global\yquant@config@requirelayerstrue},%
   default background/.initial=%
      {none},%
   default fill/.style={%
      fill/.expanded=\pgfkeysvalueof{/yquant/default background}}
}%
% END_FOLD
% BEGIN_FOLD Register creation
\pgfqkeys{/yquant}{%
   register/default name/.store in=%
      \yquant@config@register@default@name,%
   register/default lazy name/.store in=%
      \yquant@config@register@default@lazyname%
}
\ifnum\yquant@compat<2 %
   \pgfqkeys{/yquant}{%
      every label/.style=%
         {shape=yquant-init, anchor=center,%
          align/.expanded=\ifyquanthorz{right}{center}, outer timesep=2pt},%
      every multi label/.style=%
         {draw, decoration/.expanded={gapped brace, raise=2pt, \ifyquanthorz{mirror}{}}, decorate}%
   }
\else
   \pgfqkeys{/yquant}{%
      every label/.style=%
         {shape=yquant-init, anchor=center,%
          align/.expanded=\ifyquanthorz{right}{center}, outer timesep=2pt,%
          /yquant/operator/if multi={draw, decoration/.expanded={gapped brace, raise=2pt, \ifyquanthorz{mirror}{}}, decorate}}%
    }
\fi
\pgfqkeys{/yquant}{
   every initial label/.style=%
      {anchor/.expanded=\ifyquanthorz{east}{south}, /yquant/internal/autorotate init},%
   every qubit label/.style=%
      {},%
   every cbit label/.style=%
      {},%
   every qubits label/.style=%
      {}%
}
\ifnum\yquant@compat<2 %
   \pgfqkeys{/yquant}{%
      every input label/.style=%
         {}%
   }%
\fi%
% END_FOLD
% BEGIN_FOLD Register outputs
\ifnum\yquant@compat<2 %
   \pgfqkeys{/yquant}{%
      every output/.style=%
         {shape=yquant-output, anchor/.expanded=\ifyquanthorz{west}{north},%
          align/.expanded=\ifyquanthorz{left}{center}, outer timesep=2pt,%
          /yquant/internal/autorotate output},%
      every multi output/.style=%
         {draw, decoration/.expanded={gapped brace, raise=2pt, \ifyquanthorz{}{mirror}}, decorate}%
   }
\else
   \pgfqkeys{/yquant}{%
      every output/.style=%
         {shape=yquant-output, anchor/.expanded=\ifyquanthorz{west}{north},%
          align/.expanded=\ifyquanthorz{left}{center}, outer timesep=2pt,%
          /yquant/operator/if multi={draw, decoration/.expanded={gapped brace, raise=2pt, \ifyquanthorz{}{mirror}}, decorate},%
          /yquant/internal/autorotate output}%
   }
\fi
\pgfqkeys{/yquant}{%
   every qubit output/.style=%
      {},%
   every cbit output/.style=%
      {},%
   every qubits output/.style=%
      {}%
}
% END_FOLD
% BEGIN_FOLD General styling
\pgfqkeys{/yquant}{%
   every circuit/.style={%
      every node/.prefix style={transform shape}
   }%
}
\@ifpackagelater{pgf}{2020/09/31}\relax{
   \pgfkeys{/yquant/every circuit/.append style={%
      every label/.prefix style={transform shape=false}%
   }}
}
\pgfqkeys{/yquant}{%
   every wire/.style=%
      {draw},%
   every qubit wire/.style=%
      {},%
   every cbit wire/.style=%
      {},%
   every qubits wire/.style=%
      {},%
   every control line/.style=%
      {draw},%
   every control/.style=%
      {shape=yquant-circle, anchor=center, radius=.5mm},%
   every positive control/.style=%
      {fill=black},%
   every negative control/.style=%
      {draw, /yquant/default fill},%
   every operator/.style=%
      {anchor=center},%
   every multi line/.style=%
      {draw, decoration={snake, amplitude=.25mm, segment length=5pt}, decorate},%
   this operator/.style=%
      {},%
   this control/.style=%
      {},%
   operator style/.style=%
      {/yquant/this operator/.append style={#1}},%
   control style/.style=%
      {/yquant/every control line/.append style={#1},%
       /yquant/this control/.append style={#1}},%
   style/.style=%
      {/yquant/this operator/.append style={#1},%
       /yquant/every control line/.append style={#1},%
       /yquant/this control/.append style={#1}},%
   operator/multi as single/.style=%
      {/yquant/every multi line/.style=/yquant/every control line},%
   operator/if multi/.code=%
      {\ifyquant@config@operator@multi\pgfkeysalso{#1}\fi},%
   circuit/seamless/.is if=yquant@config@circuit@seamless,%
   circuit/orientation/.is choice,%
   circuit/orientation/horizontal/.code={%
      \ifyquant@config@circuitsetup%
         \pgfkeysalso{/yquant/internal/autorotate init/.style={},%
            /yquant/internal/autorotate output/.style={}}%
         \yquant@config@circuit@orientation@horizontal%
      \else%
         \PackageError{yquant.sty}{`circuit/orientation' can only be modified in the environment options or globally.}{Don't modify this within a circuit.}%
      \fi%
   },%
   circuit/orientation/vertical/.code={%
      \ifyquant@config@circuitsetup%
         \yquant@config@circuit@orientation@vertical%
      \else%
         \PackageError{yquant.sty}{`circuit/orientation' can only be modified in the environment options or globally.}{Don't modify this within a circuit.}%
      \fi%
   },%
   horizontal/.forward to=/yquant/circuit/orientation/horizontal,%
   vertical/.code={%
      \pgfkeysalso{/yquant/circuit/orientation/vertical}%
      \unless\ifdim#1 pt=0pt %
         \ifdefined\adjustbox%
            \pgfkeysalso{%
               /yquant/internal/autorotate init/.style={%
                  /tikz/execute at begin node={\adjustbox{rotate=#1}}%
               }, /yquant/internal/autorotate output/.style={%
                  /tikz/execute at begin node={\adjustbox{rotate=-#1}}%
               }%
            }%
         \else%
            \PackageWarning{yquant.sty}{In order to use the automatic rotation of initial labels for vertical circuits, you need to load the `adjustbox' package.}%
         \fi%
      \fi%
   },%
   vertical/.default=0,%
   every post measurement control/.is choice,
   every post measurement control/indirect/.code={%
      \undef\yquant@lang@attr@directcontrol%
   },%
   every post measurement control/direct/.code={%
      \let\yquant@lang@attr@directcontrol=\relax%
   }%
}
% END_FOLD
% BEGIN_FOLD Styles for operators
\ifnum\yquant@compat<2 %
   \pgfqkeys{/yquant}{%
      operators/every barrier/.style=%
         {shape=yquant-line, dashed, draw}%
   }
\else
   \pgfqkeys{/yquant}{%
      operators/every barrier/.style=%
         {shape=yquant-line, dashed, draw, shorten <= -1mm, shorten >= -1mm}
   }
\fi
\pgfqkeys{/yquant}{%
   operators/every text/.style=%
      {shape=yquant-rectangle, align=center, inner timesep=1mm, time radius=2mm, space radius=2.47mm, /yquant/default fill}%
}
\ifnum\yquant@compat<3 %
   \pgfqkeys{/yquant}{%
      % we did not have this style before 0.6, but for the ease of implementation, we just change the direction of inheritance---users of earlier compat versions will never use the style, so this is fully transparent
      operators/every rectangular box/.style=%
         {/yquant/operators/every box},%
      operators/every box/.style=%
         {shape=yquant-rectangle, draw, align=center, inner timesep=1mm, time radius=2mm, space radius=2.47mm, /yquant/default fill}%
   }
\else
   \pgfqkeys{/yquant}{%
      operators/every rectangular box/.style=%
         {shape=yquant-rectangle, draw, align=center, inner timesep=1mm, time radius=2mm, space radius=2.47mm, /yquant/default fill},%
      operators/every box/.style=%
         {/yquant/operators/every rectangular box}%
   }
\fi
\ifnum\yquant@compat<2 %
   \pgfqkeys{/yquant}{%
      operators/every custom gate/.style=%
         {/yquant/operators/subcircuit/frameless, /yquant/register/default name=}%
   }
\else
   \pgfqkeys{/yquant}{%
      operators/every custom gate/.style=%
         {/yquant/operators/subcircuit/seamless}%
   }
\fi
\pgfqkeys{/yquant}{%
   operators/every dmeter/.style=%
      {shape=yquant-dmeter, time radius=2mm, space radius=2mm, draw, /yquant/default fill},
   % every h is implicitly defined during gate declaration
   operators/every inspect/.style=%
      {shape=yquant-output, align/.expanded=\ifyquanthorz{left}{center},
       outer timesep=.3333em, space radius=2.47mm, /yquant/default fill,%
       /yquant/operator/if multi={draw, decoration/.expanded={gapped brace, raise=2pt, \ifyquanthorz{}{mirror}}, decorate}},%
   operators/every measure/.style=%
      {shape=yquant-measure, x radius=4mm, y radius=2.5mm, draw, /yquant/default fill},%
   operators/every measure meter/.style=%
      {draw, -{Latex[length=2.5pt]}},%
   operators/every not/.style=%
      {shape=yquant-oplus, radius=1.3mm, draw, /yquant/default fill},%
   operators/every pauli/.style=%
      {/yquant/operators/every rectangular box},%
   operators/every phase/.style=%
      {shape=yquant-circle, radius=.5mm, fill},%
   operators/every slash/.style=%
      {shape=yquant-slash, x radius=.5mm, y radius=.7mm, draw},%
   operators/every subcircuit/.style=%
      {},%
   operators/every subcircuit box/.style=%
      {/yquant/operators/every rectangular box, fill=none},%
   subcircuit box style/.style=%
      {/yquant/operators/every subcircuit box/.append style={#1}},%
   operators/this subcircuit box/.style=%
      {},%
   this subcircuit box style/.style=%
      {/yquant/operators/this subcircuit box/.append style={#1}},%
   operators/subcircuit/frameless/.style=
      {/yquant/operators/this subcircuit box/.append style={draw=none, inner sep=0pt}},%
   operators/subcircuit/seamless/.code=%
      {\pgfkeysalso{/yquant/operators/subcircuit/frameless, /yquant/register/default name=}%
       \ifdefined\yquant@prefix%
          \csletcs{yquant@prevseamless}{\yquant@prefix seamless}%
       \else%
          \csletcs{yquant@prevseamless}{iffalse}%
       \fi%
       \yquant@config@circuit@seamlesstrue},
   operators/subcircuit/name mangling/.is choice,%
   operators/subcircuit/name mangling/prefix or discard/.code=%
      {\def\yquant@config@operator@subcircuit@mangling{0}},%
   operators/subcircuit/name mangling/prefix or transparent/.code=%
      {\def\yquant@config@operator@subcircuit@mangling{1}},%
   operators/subcircuit/name mangling/transparent/.code=%
      {\def\yquant@config@operator@subcircuit@mangling{2}},%
   operators/subcircuit/name mangling/discard/.code=%
      {\def\yquant@config@operator@subcircuit@mangling{3}},%
   operators/subcircuit/name mangling reset/.is if=%
      yquant@config@operator@subcircuit@manglingreset,
   operators/every swap/.style=%
      {shape=yquant-swap, radius=.75mm, draw},%
   operators/every wave/.style=%
      {shape=yquant-circle, radius=.5mm, fill},%
   % every x is implicitly defined during gate declaration
   operators/every xx/.style=%
      {shape=yquant-rectangle, radius=.75mm, draw, /yquant/default fill},%
   % every y is implicitly defined during gate declaration
   % every z is implicitly defined during gate declaration
   operators/every zz/.style=%
      {shape=yquant-circle, radius=.5mm, fill},%
}
% END_FOLD
% BEGIN_FOLD Internal styles (undocument, do not use!)
\pgfqkeys{/yquant}{%
   internal/move label/.code=%
      {\yquant@config@operator@position@rightaligntrue},
   internal/no advance/.code=%
      {\unless\ifyquantmeasuring% We don't draw subcircuits during measuring time, but we need their advancement
          \yquant@config@operator@position@advancefalse%
       \fi},
   internal/multi main/.is if=%
      yquant@config@internal@multi@main,%
   internal/setup done/.code=%
      {\yquant@config@circuitsetupfalse},%
   internal/autorotate init/.style={},%
   internal/autorotate output/.style={},%
}
\ifnum\yquant@compat<2 %
   \pgfqkeys{/yquant}{%
      internal/squeeze slash/.code=%
         {\dimdef\yquant@config@operator@sep{.5\dimexpr\yquant@config@operator@sep-\pgflinewidth\relax}%
          \yquant@config@operator@position@advancefalse},%
   }
\else%
   \pgfqkeys{/yquant}{%
      internal/squeeze slash/.code={}%
   }
\fi
% END_FOLD

\protected\def\yquant@config@circuit@orientation@horizontal{%
   \let\ifyquanthorz=\@firstoftwo%
   \tikzset{%
      % in principle, these are all /.forward to handlers, but we want them non-accumulative
      time radius/.code={\pgfkeys{/tikz/x radius={##1}}},%
      space radius/.code={\pgfkeys{/tikz/y radius={##1}}},%
      inner timesep/.code={\pgfkeys{/tikz/inner xsep={##1}}},%
      inner spacesep/.code={\pgfkeys{/tikz/inner ysep={##1}}},%
      outer timesep/.code={\pgfkeys{/tikz/outer xsep={##1}}},%
      outer spacesep/.code={\pgfkeys{/tikz/outer ysep={##1}}}%
   }%
   \def\yquant@config@register@minimum@height@default{1.5mm}%
   \def\yquant@config@register@minimum@depth@default{1.5mm}%
   \def\yquant@config@operator@minimum@width@default{5mm}%
   \let\yquant@pgf@picminx=\pgf@picminx%
   \let\yquant@pgf@picmaxx=\pgf@picmaxx%
   \let\yquant@pgf@picminy=\pgf@picminy%
   \let\yquant@pgf@picmaxy=\pgf@picmaxy%
   \let\yquant@pgf@x=\pgf@x%
   \let\yquant@pgf@y=\pgf@y%
   \let\yquant@pgf@pt@x=\pgf@pt@x%
   \let\yquant@pgf@pt@y=\pgf@pt@y%
   \let\yquant@pgfqpoint=\pgfqpoint%
   \let\yquant@coords=\empty%
   \let\pgfshapeclippath=\yquant@pgfshapeclippath@horz%
   \def\yquant@orientation@x{x}%
   \def\yquant@orientation@y{y}%
   \def\yquant@orientation@plus{+}%
   \def\yquant@orientation@minus{-}%
}

\protected\def\yquant@config@circuit@orientation@vertical{%
   \let\ifyquanthorz=\@secondoftwo%
   \tikzset{%
      % in principle, these are all /.forward to handlers, but we want them non-accumulative
      time radius/.code={\pgfkeys{/tikz/y radius={##1}}},%
      space radius/.code={\pgfkeys{/tikz/x radius={##1}}},%
      inner timesep/.code={\pgfkeys{/tikz/inner ysep={##1}}},%
      inner spacesep/.code={\pgfkeys{/tikz/inner xsep={##1}}},%
      outer timesep/.code={\pgfkeys{/tikz/outer ysep={##1}}},%
      outer spacesep/.code={\pgfkeys{/tikz/outer xsep={##1}}}%
   }%
   \def\yquant@config@register@minimum@height@default{2.5mm}%
   \def\yquant@config@register@minimum@depth@default{2.5mm}%
   \def\yquant@config@operator@minimum@width@default{3mm}%
   \let\yquant@pgf@picminx=\pgf@picminy%
   \let\yquant@pgf@picmaxx=\pgf@picmaxy%
   \let\yquant@pgf@picminy=\pgf@picminx%
   \let\yquant@pgf@picmaxy=\pgf@picmaxx%
   \let\yquant@pgf@x=\pgf@y%
   \let\yquant@pgf@y=\pgf@x%
   \let\yquant@pgf@pt@x=\pgf@pt@y%
   \let\yquant@pgf@pt@y=\pgf@pt@x%
   \def\yquant@pgfqpoint##1##2{\pgfqpoint{##2}{##1}}%
   \def\yquant@coords(##1,##2){({##2},{##1})}%
   \let\pgfshapeclippath=\yquant@pgfshapeclippath@vert%
   \def\yquant@orientation@x{y}%
   \def\yquant@orientation@y{x}%
   \def\yquant@orientation@plus{-}%
   \def\yquant@orientation@minus{+}%
}

\let\yquant@pgfshapeclippath@horz=\pgfshapeclippath
\let\yquant@pgfshapeclippath@vert=\pgfshapeclippath
\patchcmd\yquant@pgfshapeclippath@vert{%
   \global\let\pgfshapeclippathhorzresult=\pgfshapeclippathresult%
   \ifcsname pgf@sh@clipvert@\csname pgf@sh@ns@#1\endcsname\endcsname%
      % different clipping in vertical direction
      \pgfsyssoftpath@setcurrentpath\pgfutil@empty%
      \csname pgf@sh@clipvert@\csname pgf@sh@ns@#1\endcsname\endcsname%
      \pgfsyssoftpath@getcurrentpath\pgfshapeclippathresult%
      \pgfprocessround{\pgfshapeclippathresult}{\pgfshapeclippathresult}%
   \fi%
   \global\let\pgfshapeclippathvertresult=\pgfshapeclippathresult%
}{%
   \global\let\pgfshapeclippathvertresult=\pgfshapeclippathresult%
   \ifcsname pgf@sh@clipvert@\csname pgf@sh@ns@#1\endcsname\endcsname%
      % different clipping in vertical direction
      \pgfsyssoftpath@setcurrentpath\pgfutil@empty%
      \csname pgf@sh@clipvert@\csname pgf@sh@ns@#1\endcsname\endcsname%
      \pgfsyssoftpath@getcurrentpath\pgfshapeclippathresult%
      \pgfprocessround{\pgfshapeclippathresult}{\pgfshapeclippathresult}%
   \fi%
   \global\let\pgfshapeclippathhorzresult=\pgfshapeclippathresult%
}{}{%
   \PackageError{yquant.sty}{Unable to path \string\pgfshapeclippath}%
                {Vertical layout will not work.}%
}

\def\yquant@config@register@default@name{\regidx}
\let\yquant@config@register@default@lazyname=\empty
\def\yquant@config@register@minimum@height{\yquant@config@register@minimum@height@default}
\def\yquant@config@register@minimum@height@default{1.5mm}
\def\yquant@config@register@minimum@depth{\yquant@config@register@minimum@depth@default}
\def\yquant@config@register@minimum@depth@default{1.5mm}
\def\yquant@config@register@sep{1mm}
\def\yquant@config@operator@sep{1mm}
\def\yquant@config@operator@minimum@width{\yquant@config@operator@minimum@width@default}
\def\yquant@config@operator@minimum@width@default{5mm}
\newif\ifyquant@config@internal@multi@main%
\yquant@config@internal@multi@maintrue
\newif\ifyquant@config@operator@multi@warn
\yquant@config@operator@multi@warntrue
\newif\ifyquant@config@operator@position@rightalign
\newif\ifyquant@config@operator@position@advance
\yquant@config@operator@position@advancetrue
\newif\ifyquant@config@circuit@seamless
\def\yquant@config@operator@subcircuit@mangling{0}
\newif\ifyquant@config@operator@subcircuit@manglingreset
\yquant@config@operator@subcircuit@manglingresettrue
\newif\ifyquant@config@operator@multi
\newif\ifyquant@config@circuitsetup
\yquant@config@circuitsetuptrue
\yquant@config@circuit@orientation@horizontal
\newif\ifyquant@config@draw@quality
\yquant@config@draw@qualitytrue
\newif\ifyquant@config@requirelayers

\protected\def\yquant@config@operator@subcircuit@mangling@set#1{%
   \ifyquant@config@operator@subcircuit@manglingreset%
      \def\yquant@config@operator@subcircuit@mangling{#1}%
   \fi%
}