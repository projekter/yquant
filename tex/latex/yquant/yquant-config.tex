\pgfdeclaremetadecoration{tikz@internal}{pre}{
   \state{pre}[width=\pgfkeysvalueof{/pgf/decoration/pre length}+\pgfkeysvalueof{/pgf/decoration/post length}, next state=main]{
      \appto\tikz@dec@shift{\pgftransformxshift{-\pgfkeysvalueof{/pgf/decoration/post length}}}
      \tikz@dec@trans
      \decoration{\pgfkeysvalueof{/pgf/decoration/pre}}
   }
   \state{main}[width=\pgfmetadecoratedremainingdistance, next state=final]{
      \tikz@dec@trans
      \decoration{\tikz@decoration@name}
   }
   \state{final}{
      \tikz@dec@trans
      \decoration{\pgfkeysvalueof{/pgf/decoration/post}}
   }
}
\pgfqkeys{/yquant}{
   every circuit/.style={%
      every node/.prefix style={transform shape},%
      every label/.prefix style={transform shape=false}% TODO: no, we don't really want this, but pgf bug #843 requires this if we still want to have `label position` available
   },
   % register settings
   register/default name/.store in=%
      \yquant@config@register@default@name,
   register/minimum height/.code=%
      {\dimdef\yquant@config@register@minimum@height{#1}},
   register/separation/.code=%
      {\dimdef\yquant@config@register@sep{#1}},
   % register label style
   every label/.style=%
      {shape=yquant-text, anchor=center, align=right},
   every initial label/.style=%
      {anchor=east},
   every qubit label/.style=%
      {},
   every cbit label/.style=%
      {},
   every qubits label/.style=%
      {},
   every multi label/.style=%
      {shift={(-.075, 0)}, draw, decoration={gapped brace, mirror}, decorate, /yquant/gapped brace/apply shift,
       every node/.append style={shape=yquant-text, anchor=east, align=right, shift={(-.05, 0)}, pos=-1}},
   % output label styles
   every output/.style=%
      {shape=yquant-text, anchor=west, align=left},
   every qubit output/.style=%
      {},
   every cbit output/.style=%
      {},
   every qubits output/.style=%
      {},
   every multi output/.style=%
      {shift={(.075, 0)}, draw, decoration={gapped brace}, decorate, /yquant/gapped brace/apply shift,
       every node/.append style={shape=yquant-text, anchor=west, align=left, shift={(.05, 0)}, pos=-1}},
   % wire style
   every wire/.style=%
      {draw},
   every qubit wire/.style=%
      {},
   every cbit wire/.style=%
      {},
   every qubits wire/.style=%
      {},
   % operator settings
   operator/separation/.code=%
      {\dimdef\yquant@config@operator@sep{#1}},
   operator/minimum width/.code=%
      {\dimdef\yquant@config@operator@minimum@width{#1}},
   operator/multi warning/.is if=%
      yquant@config@operator@multi@warn,
   % operator style: control
   every control line/.style=%
      {draw},
   every control/.style=%
      {shape=yquant-circle, anchor=center, radius=.5mm},
   every positive control/.style=%
      {fill=black},
   every negative control/.style=%
      {draw},
   % operator style: main part
   every operator/.style=%
      {anchor=center},
   operator/multi main/.is if=%
      yquant@config@operator@multi@main,
   operator/multi as single/.style=%
      {/yquant/every multi line/.style=/yquant/every control line},
   every multi line/.style=%
      {draw},
   % overwriting all styles
   this operator/.style=%
      {},
   this control/.style=%
      {},
   operator style/.style=%
      {/yquant/this operator/.append style={#1}},
   control style/.style=%
      {/yquant/every control line/.append style={#1},
       /yquant/this control/.append style={#1}},
   style/.style=%
      {/yquant/this operator/.append style={#1},
       /yquant/every control line/.append style={#1},
       /yquant/this control/.append style={#1}},
   % different operator appearances
   operators/every box/.style=%
      {shape=yquant-rectangle, draw, align=center, inner xsep=1mm, x radius=2mm, y radius=2.47mm},
   operators/every h/.style=%
      {/yquant/operators/every box},
   operators/every pauli/.style=%
      {/yquant/operators/every box},
   operators/every x/.style=%
      {/yquant/operators/every pauli},
   operators/every y/.style=%
      {/yquant/operators/every pauli},
   operators/every z/.style=%
      {/yquant/operators/every pauli},
%   operators/every subcircuit/.style=%
%      {/yquant/operators/every box, inner ysep=0pt},
   operators/every phase/.style=%
      {shape=yquant-circle, radius=.5mm, fill},
   operators/every zz/.style=%
      {shape=yquant-circle, radius=.5mm, fill, /yquant/operator/multi as single},
   operators/every xx/.style=%
      {shape=yquant-rectangle, radius=.75mm, draw, /yquant/operator/multi as single},
   operators/every slash/.style=%
      {shape=yquant-slash, x radius=.5mm, y radius=.7mm, draw},
   operators/every swap/.style=%
      {shape=yquant-swap, radius=.75mm, draw, /yquant/operator/multi as single},
   operators/every not/.style=%
      {shape=yquant-oplus, radius=1.3mm, draw},
   operators/every measure/.style=%
      {shape=yquant-measure, x radius=4mm, y radius=2.5mm, draw},
   operators/every measure meter/.style=%
      {draw, -{Latex[length=2.5pt]}},
   operators/every dmeter/.style=%
      {shape=yquant-dmeter, x radius=2mm, y radius=2mm, draw},
   operators/every barrier/.style=%
      {shape=yquant-line, dashed, draw},
   operators/every wave/.style=%
      {shape=yquant-circle, radius=.5mm, fill},
   operators/every wave line/.style=%
      {/yquant/every multi line/.append style={decoration={snake, amplitude=.25mm, segment length=5pt}, decorate}, /tikz/label position=left},
   /pgf/decoration/from to/.store in=\pgfdecorationsegmentfromto,
   gapped brace/apply shift/.code={%
      \let\tikz@timer@line=\yquant@gappedbrace@timer%
   },
}

\def\yquant@config@register@default@name{\regidx}
\def\yquant@config@register@minimum@height{3mm}
\def\yquant@config@register@sep{1mm}
\def\yquant@config@operator@sep{1mm}
\def\yquant@config@operator@minimum@width{5mm}
\newif\ifyquant@config@operator@multi@main%
\yquant@config@operator@multi@maintrue
\newif\ifyquant@config@operator@multi@warn
\yquant@config@operator@multi@warntrue
\def\pgfdecorationsegmentfromto{0-1}%

\protected\def\yquant@gappedbrace@extract#1-#2\yquant@sep{%
   \dimdef\from{#1\pgfdecoratedremainingdistance}%
   \dimdef\to{#2\pgfdecoratedremainingdistance}%
}
% This function is set to replace the transformation that changes the fraction value specified in the node property pos by a computation of the correct value for the gappedbrace decoration. It thus requires \pgfdecorationsegmentfromto to be already set up properly. Only if the special position -1 is given, these transformations are applied; else, the default behavior is reproduced (\tikz@timer@line).
\protected\def\yquant@gappedbrace@timer{%
   \ifdim\tikz@time pt=-1pt %
      % first set \pgfdecoratedremainingdistance appropriately
      \pgfpointdiff\tikz@timer@start\tikz@timer@end%
      \pgfmathsqrt@{\dimexpr\pgf@x*\pgf@x/65536+\pgf@y*\pgf@y/65536\relax\@gobbletwo}
      \pgfdecoratedremainingdistance=\pgfmathresult pt %
      % now perform the transformation
      \pgf@xc=\pgfdecorationsegmentaspect\pgfdecoratedremainingdistance%
      \expandafter\forcsvlist\expandafter\yquant@gappedbrace@shiftloop%
         \expandafter{\pgfdecorationsegmentfromto}%
      % and finally let us return the desired position, just ignoring \tikz@time...
      \pgftransformlineattime{\pgfdecorationsegmentaspect}{\tikz@timer@start}{\tikz@timer@end}%
   \else%
      \pgftransformlineattime{\tikz@time}{\tikz@timer@start}{\tikz@timer@end}%
   \fi%
}
\def\yquant@gappedbrace@shiftloop#1{%
   \yquant@gappedbrace@extract#1\yquant@sep%
   \unless\ifdim\pgfdecorationsegmentaspect\pgfdecoratedremainingdistance<\from %
      \unless\ifdim\pgfdecorationsegmentaspect\pgfdecoratedremainingdistance>\to %
         % be careful about arch positions at the border
         \ifdim\dimexpr\to-\from\relax<2\pgfdecorationsegmentamplitude %
            % The arch is larger than the segment. We do not draw a line to it or an end line and place it in the mid of the segment, even if it is too short (this issues one extra \pgfpathmoveto command, but catching this rare case is not worth the effort).
            \edef\pgfdecorationsegmentaspect{\pgfmath@tonumber{\dimexpr%
               .5\dimexpr\from+\to\relax*65536/\pgfdecoratedremainingdistance%
            \relax}}%
         \else%
            % The segment is large enough to cover the whole arch. But maybe we are too close at the border?
            \ifdim\dimexpr\pgf@xc-\pgfdecorationsegmentamplitude\relax<\from %
               \pgf@xc=\dimexpr\from+\pgfdecorationsegmentamplitude\relax%
            \fi%
            \ifdim\dimexpr\pgf@xc+\pgfdecorationsegmentamplitude\relax>\to %
               \pgf@xc=\dimexpr\to-\pgfdecorationsegmentamplitude\relax%
            \fi%
            \edef\pgfdecorationsegmentaspect{\pgfmath@tonumber{\dimexpr%
               \dimexpr\pgf@xc*65536/\pgfdecoratedremainingdistance%
            \relax}}%
         \fi%
         \expandafter\expandafter\expandafter\listbreak%
      \fi%
   \fi%
}
\def\yquant@gappedbrace@loop#1{%
   \yquant@gappedbrace@extract#1\yquant@sep%
   \unless\ifdim\from=0pt %
      \pgfpathmoveto{%
         \pgfqpoint{\from}{.5\pgfdecorationsegmentamplitude}%
      }%
   \fi%
   \unless\ifdim\pgfdecorationsegmentaspect\pgfdecoratedremainingdistance<\from %
      \unless\ifdim\pgfdecorationsegmentaspect\pgfdecoratedremainingdistance>\to %
         % be careful about arch positions at the border
         \ifdim\dimexpr\to-\from\relax<2\pgfdecorationsegmentamplitude %
            % The arch is larger than the segment. We do not draw a line to it or an end line and place it in the mid of the segment, even if it is too short (this issues one extra \pgfpathmoveto command, but catching this rare case is not worth the effort).
            \pgfpathmoveto{%
               \pgfqpoint{\dimexpr.5\dimexpr\from+\to\relax-\pgfdecorationsegmentamplitude\relax}%
                         {.5\pgfdecorationsegmentamplitude}%
            }%
            \edef\to{\the\pgfdecoratedremainingdistance}% to prevent the final line, we do not need "to" any more
         \else%
            % The segment is large enough to cover the whole arch. But maybe we are too close at the border?
            \ifdim\dimexpr\pgf@xc-\pgfdecorationsegmentamplitude\relax<\from %
               \pgf@xc=\dimexpr\from+\pgfdecorationsegmentamplitude\relax%
            \fi%
            \ifdim\dimexpr\pgf@xc+\pgfdecorationsegmentamplitude\relax>\to %
               \pgf@xc=\dimexpr\to-\pgfdecorationsegmentamplitude\relax%
            \fi%
            % Both cases can't occur at the same time in this \else clause.
            \pgfpathlineto{%
               \pgfqpoint{\dimexpr\pgf@xc-\pgfdecorationsegmentamplitude\relax}%
                         {.5\pgfdecorationsegmentamplitude}%
            }%
         \fi%
         \pgfpathcurveto{%
            \pgfqpoint{\dimexpr\pgf@xc-.5\pgfdecorationsegmentamplitude\relax}%
                      {.5\pgfdecorationsegmentamplitude}%
         }{%
            \pgfqpoint{\dimexpr\pgf@xc-.15\pgfdecorationsegmentamplitude\relax}%
                      {.7\pgfdecorationsegmentamplitude}%
         }{%
            \pgfqpoint{\pgf@xc}{\pgfdecorationsegmentamplitude}%
         }%
         \pgfpathcurveto{%
            \pgfqpoint{\dimexpr\pgf@xc+.15\pgfdecorationsegmentamplitude\relax}%
                      {.7\pgfdecorationsegmentamplitude}%
         }{%
            \pgfqpoint{\dimexpr\pgf@xc+.5\pgfdecorationsegmentamplitude\relax}%
                      {.5\pgfdecorationsegmentamplitude}%
         }{%
            \pgfqpoint{\dimexpr\pgf@xc+\pgfdecorationsegmentamplitude\relax}%
                      {.5\pgfdecorationsegmentamplitude}%
         }%
      \fi%
   \fi%
   \unless\ifdim\to=\pgfdecoratedremainingdistance %
      \pgfpathlineto{%
         \pgfqpoint{\to}{.5\pgfdecorationsegmentamplitude}%
      }%
   \fi%
}
% This is a variant of the brace in pathreplacing that allows for holes
\pgfdeclaredecoration{gapped brace}{final}{%
   \state{final}{%
      \pgf@xc=\pgfdecorationsegmentaspect\pgfdecoratedremainingdistance%
      \pgfpathcurveto{%
         \pgfqpoint{.15\pgfdecorationsegmentamplitude}{.3\pgfdecorationsegmentamplitude}%
      }{%
         \pgfqpoint{.5\pgfdecorationsegmentamplitude}{.5\pgfdecorationsegmentamplitude}%
      }{%
         \pgfqpoint{\pgfdecorationsegmentamplitude}{.5\pgfdecorationsegmentamplitude}%
      }%
      \expandafter\forcsvlist\expandafter\yquant@gappedbrace@loop%
         \expandafter{\pgfdecorationsegmentfromto}%
      \pgfpathlineto{%
         \pgfqpoint{\dimexpr\pgfdecoratedremainingdistance-\pgfdecorationsegmentamplitude\relax}%
                   {.5\pgfdecorationsegmentamplitude}%
      }%
      \pgfpathcurveto{%
         \pgfqpoint{\dimexpr\pgfdecoratedremainingdistance-.5\pgfdecorationsegmentamplitude\relax}%
                   {.5\pgfdecorationsegmentamplitude}%
      }{%
         \pgfqpoint{\dimexpr\pgfdecoratedremainingdistance-.15\pgfdecorationsegmentamplitude\relax}%
                   {.3\pgfdecorationsegmentamplitude}%
      }{%
         \pgfqpoint{\pgfdecoratedremainingdistance}{0pt}%
      }%
   }%
}%