\documentclass{scrartcl}

\title{yquant.sty package documentation}
\subtitle{Typesetting quantum circuits in a human-readable language}
\author{Benjamin Desef}

% BEGIN_FOLD Packages
\RequirePackage{fontspec}
\setmainfont{XCharter}[Scale=.98]
\RequirePackage[osf, semibold]{sourcesanspro}
\RequirePackage[math-style=ISO, bold-style=ISO, sans-style=italic]{unicode-math}

\usepackage{xcolor}
\definecolor{darkblue}{RGB}{5, 10, 122}
\usepackage[english]{babel}
\usepackage[colorlinks, linkcolor=darkblue, bookmarksdepth=paragraph]{hyperref}
\usepackage[compat=newest]{yquant}
\usepackage{braket,hyphenat,microtype,hologo,minted,import,ragged2e}
\usetikzlibrary{quotes,fit,shapes.symbols,backgrounds,quantikz}
\useyquantlanguage{qasm}
\usepackage{amsmath,adjustbox,cleveref}
\usepackage[framemethod=tikz]{mdframed}
% END_FOLD

\makeatletter

\long\def\compat#1{\texorpdfstring{\leavevmode\makenote{#1}\penalty10000 }{}}

% BEGIN_FOLD Minted configuration
\begingroup
\catcode`\&=11
% I use Intel Python, which requires using the `activate' command first
\renewcommand{\MintedPygmentize}{activate && pygmentize}
\endgroup

\definecolor{boxGrayBody}{gray}{.94}
\definecolor{boxRedHead} {RGB} {178, 34,  34}
\definecolor{boxRedBody} {RGB} {255, 255, 192}
\definecolor{boxBlueHead}{RGB} {118, 152, 199}
\definecolor{boxBlueBody}{RGB} {242, 247, 255}
\definecolor{mintgreen}  {RGB} {11,  97,  11}
\setminted{frame=none, linenos=false, fontsize=\footnotesize, breaklines, bgcolor=boxGrayBody}
\setmintedinline{bgcolor={}, fontsize=}

\newmintinline[yquant]{text}{}
\newmintinline[tex]{tex}{}
\newmintinline[ebnf]{ebnf}{}
% END_FOLD

% BEGIN_FOLD Block environments (options, code examples, ...)
\def\unskipOpt{\vskip-5mm }
\let\unpatched@minted@inline@iii=\minted@inline@iii%
\def\patched@minted@inline@iii#1{%
   \unpatched@minted@inline@iii{#1}%
   \par%
   \endgroup\vskip-\parskip%
   \noindent\ignorespaces%
}
\newenvironment{option*}[3][]{\vskip5mm%
   \noindent\begin{minipage}[t]{\linewidth}%
      \begingroup%
         \let\minted@inline@iii=\patched@minted@inline@iii%
         \raggedleft%
         \hspace*{-1cm}%
         \phantomsection\label{style:/#2/#3}%
         \texttt{\ifstrempty{#1}\relax{\compat{#1}}\textcolor{gray}{/#2/}\textcolor{mintgreen}{#3}}\hfill default: %
         \tex}{%
   \end{minipage}}
\newenvironment{option}[1][]{\csname option*\endcsname[{#1}]{yquant}}{\csname endoption*\endcsname}

\newsavebox\codeexamplebox
\newenvironment{codeexample}{%
   \VerbatimEnvironment%
   \let\FVB@VerbatimOut\minted@FVB@VerbatimOut
   \let\FVE@VerbatimOut\minted@FVE@VerbatimOut
   \minted@configlang{tex}%
   \minted@fvset
   \begin{VerbatimOut}[codes={\catcode`\^^I=12},firstline,lastline]{\minted@jobname.pyg}%
}{
   \end{VerbatimOut}%
   \minted@langlinenoson%
   \savebox\codeexamplebox{\input \minted@jobname.pyg}%
   \ifdim\wd\codeexamplebox>\dimexpr.5\linewidth-3mm\relax%
      \wd\codeexamplebox=.5\linewidth% squeeze if necessary
   \else%
      \wd\codeexamplebox=\dimexpr\wd\codeexamplebox+3mm\relax% else give some space
   \fi%
   \noindent\begin{minipage}{\wd\codeexamplebox}%
      \centering%
      \usebox\codeexamplebox%
   \end{minipage}%
   \begin{minipage}{\dimexpr\linewidth-\wd\codeexamplebox\relax}%
      \minted@pygmentize{\minted@lang}%
   \end{minipage}%
   \minted@langlinenosoff%
   \par%
}

\newenvironment{codeexample*}{%
   \VerbatimEnvironment%
   \let\FVB@VerbatimOut\minted@FVB@VerbatimOut
   \let\FVE@VerbatimOut\minted@FVE@VerbatimOut
   \minted@configlang{tex}%
   \minted@fvset
   \begin{VerbatimOut}[codes={\catcode`\^^I=12},firstline,lastline]{\minted@jobname.pyg}%
}{
   \end{VerbatimOut}%
   \minted@langlinenoson%
   \begin{adjustbox}{center}
      \input \minted@jobname.pyg %
   \end{adjustbox}\nopagebreak
   \minted@pygmentize{\minted@lang}%
   \minted@langlinenosoff%
   \par%
}

\def\bd@boxextra#1{%
   \node[anchor=north east, overlay] at (current bounding box.north west) {#1\hskip7pt};%
}

\newcommand*\bd@boxintro[5]{%
   \begingroup%
      \setlength{\parskip}{0pt}%
      \setlength{\parindent}{0pt}%
      \edef\tmp{#5}%
      \ifx\tmp\empty%
         \expandafter\@firstoftwo%
      \else%
         \expandafter\@secondoftwo%
      \fi{%
         \begin{mdframed}[style=bdBox, backgroundcolor=#4, linecolor=#2,
                          firstextra=\bd@boxextra{#1}, singleextra=\bd@boxextra{#1}]%
      }{%
         \begin{mdframed}[style=bdBox, backgroundcolor=#4,
                          frametitlefont=\bfseries\sffamily\color{#3},
                          frametitlebackgroundcolor=#2,
                          frametitle={\strut\tmp}, linecolor=#2,
                          firstextra=\bd@boxextra{#1}, singleextra=\bd@boxextra{#1}]%
      }%
      \ignorespaces%
}
\def\bd@boxoutro{%
      \end{mdframed}%
   \endgroup%
}
\iffalse% code folding breaks
   \end{mdframed}
\fi
\mdfdefinestyle{bdBox}{
   skipabove=3pt,
   skipbelow=0pt,
   leftmargin=0pt,
   rightmargin=0pt,
   innerleftmargin=7pt,
   innerrightmargin=7pt,
   innertopmargin=5pt,
   innerbottommargin=5pt,
   splittopskip=17pt,
   frametitlealignment=\raggedright,
   frametitleaboveskip=3pt,
   frametitlebelowskip=1.5pt
}
\newenvironment*{warning}[1][]{%
   \bd@boxintro{\raisebox{\baselineskip-\height-6.5pt}{\includegraphics[width=1cm]{Warning.pdf}}}%
               {boxRedHead}{white}{boxRedBody}{\unexpanded{#1}}%
   \ignorespaces%
}{%
   \bd@boxoutro%
}
\def\example{\@ifnextchar<\example@i{\example@i<>}}
\def\example@i<#1>{\@ifnextchar[{\example@ii<{#1}>}{\example@ii<{#1}>[]}}
\long\def\example@ii<#1>[#2]{%
   \bd@boxintro{\ifstrempty{#1}{}{\ifstrempty{#2}{\compat{#1}}{}}%
                \raisebox{\baselineskip-\height-6.5pt}{%
                   \includegraphics[width=1cm]{Bulb.png}%
               }}{boxBlueHead}{white}{boxBlueBody}%
               {\ifstrempty{#1}{}{\ifstrempty{#2}{}{\unexpanded{\compat{#1}}}}\unexpanded{#2}}%
   \ignorespaces%
}
\let\endexample=\bd@boxoutro
% END_FOLD

\def\TikZ{\textcolor{brown}{Ti\textit kZ}}
\def\pkg#1{\textcolor{brown}{\texttt{#1}}}
\def\gate#1{\hyperref[gate:#1]{\texttt{#1}}}
\def\style#1{\hyperref[style:#1]{\texttt{#1}}}
\def\Yquant{\pkg{yquant}}
\def\ii{{\symup i}}
\def\ee{{\symup e}}
\def\ketbra#1#2{\ket{#1}\!\bra{#2}}

\frenchspacing
\RequirePackage{scrlayer-scrpage,scrlayer-notecolumn}
\setcounter{tocdepth}{\subsectiontocdepth}
\KOMAoptions{DIV=8}
\newcommand*{\the@orig@section}{}
\let\the@orig@section=\section
\renewcommand*{\section}{%
   \clearpage%
   \the@orig@section%
}
\addtokomafont{notecolumn.marginpar}{\footnotesize\RaggedRight}
\recalctypearea

\begin{document}
   \maketitle
   \begin{abstract}
      \noindent This manual introduces \Yquant, a \LaTeX\hyp only package that outputs quantum circuits.
      They are entered using a human\hyp readable language that, even from the source code, allows for a fluent understanding of the logic that underlies the circuit.
      \Yquant{} internally builds on \TikZ{} and can be easily combined with arbitrary \LaTeX{} code.
      More than forty pages of examples complement the formal manual.
   \end{abstract}
   \tableofcontents
   
   \section{Introduction}
      This document outlines the scope and usage of the \Yquant{} package.
      It contains both a reference and a huge number of examples.
      \Yquant{} is a package that makes typesetting quantum circuits easy; the package is available on CTAN.
      This beta version~0.4 \emph{should} be stable and interfaces are not very likely to change in an incompatible way in the future.
      \compat{New in 0.4}Sometimes, backwards\hyp incompatible changes are required or advisable, in which case a compatibility setting will allow to revert back to the old behavior (rather, to maximize compatibility, this is an opt\hyp in setting: unless you choose the new behavior, you will get the old one).
      Please do report all issues and desirable additions.
      
      \subsection{How to read the manual}
         The probably fastest way to start using \Yquant{} is by just scanning through the examples in \cref{sec:examples}.
         A more formal description of the \Yquant{} grammar and its fundamental concepts can be found in \cref{sec:grammar}.
         If your desire is to change the appearance of \Yquant{} elements, use the configuration reference in \cref{sec:config}.
         The full list of all available gates is provided in \cref{sec:gates}.
         Finally, you may find that \Yquant{} \emph{almost} does what you want, but there is some final tweak that you cannot achieve\dots.
         Then, have a look at \cref{sec:tikz} (or \cref{sec:alt}).
      
      \subsection{Installation}
         The recommended way of installation is through CTAN.
         A direct installation from this Git repository to obtain the latest additions and features is be possible by just cloning it to a path visible to your \TeX{} compiler.
         While the repository may contain new additions, they are not thoroughly tested until they end up on CTAN; features that are not documented in this manual are entirely unreliable.
         For example, you may put them in the same directory as your document (if you just want to give a try), or you may extract them to \texttt{tex/latex/yquant} in your local \texttt{texmf} (followed by an update of the file name database).
      
      \subsection{Purpose of \Yquant, alternatives}\label{sec:alt}
         \Yquant{} is the acronym for ``yet another quantum circuit package.''
         This highlights the fact that nothing that this package provides cannot be achieved by other means.
         In particular, there are at least the following methods to typeset quantum circuits in \TeX.
         \begin{itemize}
            \item Use some external program to draw them and include the output via \tex!\includegraphics!.
            \item Use either \TeX's own drawing capabilities (the \tex!picture! environment) or other drawing packages such as \TikZ{} or \pkg{pstricks}.
            \item Use a package specifically designed to draw quantum circuits (if you feel some other package should be mentioned here, please file an issue):
               \begin{itemize}
                  \item \pkg{qasm} is probably the first of them (in terms of age).
                     It was developed to typeset the circuits found in Nielsen and Chuang's famous \emph{Quantum Computation and Quantum Information} book.
                     \pkg{qasm} consists of a Python~2 script (\texttt{qasm2circ}) that reads a quantum circuit written in a very intuitive language: declare names for your qubits, perform gates on them in each line.
                     \texttt{qasm2circ} converts those circuits into \TeX{} files that internally make use of the \pkg{xy} package to display the output.
                     Consequently, the user is restricted to the set of features that \pkg{qasm} directly offers (which is small).
                     Changes to the output, while possible, will be overwritten if \texttt{qasm2circ} is run again.
                     \pkg{qasm} output often looks sub\hyp optimal do to the fact that, e.g., rectangles are made up of four lines that do not properly connect and give a crumbly general feeling.
                     
                     Note that since version~0.3, \Yquant{} understands \pkg{qasm} syntax, see \cref{sec:foreign:qasm}.
                     
                     Maintenance status: last update of \pkg{qasm} in 2005. Also, \pkg{xy} was last updated in 2013, and the script is not compatible out\hyp of\hyp the\hyp box with Python~3, though an automatic conversion should work.
                  \item \pkg{qcircuit} is probably the most\hyp widely used package.
                     It provides commands that make it much easier to create quantum circuits using the \pkg{xy} package.
                     Its syntax therefore is grid\hyp oriented; inferring what a circuit does or locating a gate in the code can be tough.
                     This is particularly true for multi\hyp qubit gates.
                     Additionally, the \tex!\xymatrix! syntax is also somewhat cryptic.
                     \pkg{qcircuit} provides some flexibility within the limits of \pkg{xy} as to configuring the output.
                     
                     Maintenance status: last update in 2018; and remember this is \pkg{xy} based, with last update in 2013.
                  \item \pkg{quantikz} is a relatively recent package that, following the same grid\hyp based approach as \pkg{qcircuit}, instead builds on \TikZ{} as a backend.
                     As a consequence, it provides the full flexibility of customization that \TikZ{} offers, where hardly anything cannot be done.
                     It also reduces burdens of the \pkg{xy} syntax.
                     However, the disadvantages of the grid\hyp based syntax still remain.
                     
                     Maintenance status: last update in 2020; the underlying \TikZ{} is actively maintained again by now.
                  \item \pkg{qpic} follows the approach of \pkg{qasm}: It makes use of an external Python program that reads the quantum circuits in an own language and converts them into \TikZ{} commands.
                     The language \pkg{qpic} follows is much more powerful than \pkg{qasm}'s.
                     The disadvantage that modifications in the output code will not remain after running the Python script again is mitigated by the possibility to define own \TeX{} macros.
                     Being an external program, \pkg{qpic}'s intrinsic set of features (including, e.g., vertically set circuits) are huge.
                     However, the language \pkg{qpic} uses cannot be understood without a detailed study of the manual, it appears to have been designed with the aim to minimize the length of command names.
                     A disadvantage of external programs is that the amount of space gates need is not accessible by the script; hence, manual intervention may be required.
                     
                     Maintenance status: last update in 2020; the underlying \TikZ{} is actively maintained, and the script is compatible with Python~3.
               \end{itemize}
         \end{itemize}
      
      \subsection{License}
          This work may be distributed and/or modified under the conditions of the \LaTeX{} Project Public License, either version~1.3c of this license or (at your option) any later version.
          The latest version of this license is in
          \begin{center}
             \url{http://www.latex-project.org/lppl.txt}
          \end{center}
          and version~1.3c or later is part of all distributions of LaTeX version 2005/12/01 or later.
   
   \section{Basic elements of \Yquant}\label{sec:grammar}
      \Yquant, as some of the aforementioned packages, builds on \TikZ.
      Its basic syntax is similar to \pkg{pgfplots}: Start a \tex!tikzpicture! environment (perhaps passing some options); inside, start a \tex!yquant! environment.
      
      Inside the \tex!yquant! environment, \TeX{} will now understand the \Yquant{} language---so \Yquant{} falls into the same category as \pkg{qasm} and \pkg{qpic}, providing a human\hyp readable language for the specification of the circuit that is not fixed to the actual layout.
      
      However, \Yquant{} is a \TeX\hyp only package (actually, \hologo{LaTeX2e}, but not \hologo{LaTeX3}) that requires no external script to run---so it also falls into the same category as \pkg{qcircuit} and \pkg{quantikz}.
      
      Since it runs entirely within \TeX, you can at any time interject \Yquant{} code with arbitrary \TeX{} or \TikZ{} code (though if it is ``too arbitrary,'' you may need to restart the \Yquant{} interpreter).
      
      \subsection{General usage}
         \begin{minted}{tex}
% preamble: \usepackage[compat=<version>]{yquant}
\begin{tikzpicture}% tikz options possible
   % tikz commands go here
   \begin{yquant}% yquant options possible. Watch the newlines!
      % yquant and tikz commands go here
   \end{yquant}
   % tikz commands go here
\end{tikzpicture}
         \end{minted}
         
         Note that \Yquant{} depends on \pkg{etoolbox}, \TikZ, \pkg{trimspaces}, and \pkg{xkeyval}.
         Additionally, it requires a moderately recent version of \hologo{LaTeX2e}, using either \hologo{LuaLaTeX}, or (untested), \hologo{pdfLaTeX} or \hologo{XeLaTeX}.
         
         \begin{warning}[Optional arguments]
            The optional arguments for the \tex!yquant! environment have to appear \emph{on the same line} as the environment itself.
            If you want to put the arguments into a new line, it is crucial to mask the line break by putting a comment symbol after the environment: \tex!\begin{yquant}%!.
            Without this comment, \Yquant{} will detect your line break (this is one of the few places in \TeX{} where line breaks and spaces are different) and assume that the expression in square brackets instead provides arguments for the following operation! \\
            Finally note that in (non\hyp fragile) \pkg{beamer} frames, this discrimination between spaces and new lines does not work; the optional arguments will always be counted for the environment, not for the gate.
            In this case, you can either declare the frame as fragile or (recommended) introduce a blank line between the environment and the options for the first gate.
         \end{warning}
         
         \begin{warning}[\compat{New in 0.4}Compatibility mode]
            Sometimes, continued development shows that certain choices of interfaces, configuration, or behavior are less ideal than originally thought.
            In other cases, bugs are detected and fixed.
            Both may lead to a change in the look of circuits developed with a previous version of \Yquant{} or even---though this should rarely, if ever, happen, and should be filed as a bug---prevent compilation in the new version.
            
            For this reason, \Yquant{} offers a compatibility key that is highly recommended to be specified as a package option.
            This allows certain features which are expected to break old layout or functionality to revert to their previous behavior.
            Every feature affected by the \texttt{compat} key is documented in this manual.
            Once a compatibility version is selected in a document, it cannot be changed any more.
            Compatibility versions will include the major and minor, but not the patch level version number.
            Bugs that clearly violated behavior described in this manual will be fixed without a possibility to revert back.
            Changes that are not supposed to result in a (more than marginally) different result will not be included in the compatibility layer.
            If you find this to be wrong in a particular case, please file a bug report.
            
            When starting a new document, it is recommended to leave out the compatibility key at first compilation.
            \Yquant{} will then issue a warning from which you can infer the recommended setting, corresponding to the current version.
            You should then pass the appropriate version to the \tex!\usepackage! command.
            For example, this manual corresponds to \tex!\usepackage[compat=0.4]{yquant}!.
            
            Allowed values for \texttt{compat} are \texttt{newest} (discouraged), which equals \texttt{0.4}, and \texttt{0.3} (default).
         \end{warning}
      
      \subsection{Starred vs. unstarred environment}
         You may choose to use either the \tex!yquant! or the \tex!yquant*! environment.
         The former one requires you to define all your registers before you use them (though you may decide to define a register after some operations on \emph{different} registers, but before its first usage).
         
         The starred form additionally supports the use of undeclared registers: it basically declares a registers upon its first usage.
         This will always be a \texttt{qubit} register; but if you use the corresponding attribute and the first usage is an \gate{init} command, you may overwrite this.
         \compat{New in 0.2}Subcircuits always use the unstarred form.
         
         \compat{New in 0.1.1}Additionally, if you refer to the index $i$ of a vector register of length $L < i$, this register will automatically be enlarged to $i \coloneq L$.
         It is also possible to convert a scalar register into a vector register in this manner.
         To enlarge a register in the unstarred environment, you must precede the number of registers to be added in the second declaration by a plus sign.
         Note that in this manner, you may even create discontiguous vectors.
         
         \emph{This might be a good point to proceed to the examples \cref{sec:examples}.}
      
      \subsection[Formal syntax]{\compat{Enhanced in 0.1.2, 0.1.1}Formal syntax}
         Every \Yquant{} command has the same structure (described here in EBNF syntax):
         \begin{minted}{ebnf}
Command = { Arguments }, ?command?, [ Value ], [ RegisterList ], Controls, ";";
Arguments = "[", ?pgfkeys?, "]";
Value = "{", ?TeX code?, "}";
Controls = [ "|", [ RegisterSingleList ] ], [ "~", [ RegisterSingleList ] ];

RegisterList = (RegisterSingle | RegisterMulti), [ ",", RegisterList ];
RegisterSingleList = RegisterSingle, [ ",", RegisterSingleList ];

RegisterSingle = RegisterSingleNoRange | RegisterRange;
RegisterSingleNoRange = ?name?, [ "[", IndexMultiList, "]" ];
RegisterMulti = "(", ( RegisterMultiNoRange | [ "*" ], RegisterRange ), ")";
RegisterMultiNoRange = [ "*" ], ?name?, [ "[", IndexSingleMainList, "]" ];
RegisterRange = [ RegisterUnique ], "-", [ RegisterUnique ];
RegisterUnique = ?name?, [ "[", ?number?, "]" ];

IndexMultiList = IndexMulti, [ ",", IndexMultiList ];
IndexSingleList = IndexSingle, [ ",", IndexSingleList ];
IndexSingleMainList = [ "*" ], IndexSingle, [ ",", IndexSingleMainList ];
IndexMulti = IndexSingle | ( "(", IndexSingle, ")" );
IndexSingle = ?number? | ( [ ?number? ], "-", [ ?number? ] );
         \end{minted}
         Note that \Yquant{} is quite tolerant with respect to whitespaces.
         Virtually every comma in the EBNF notation may consist of an arbitrary (including zero) number of whitespaces.
         Not all combinations that can be constructed by this grammar are actually allowed semantically; but it would make the grammar too verbose to spell this out in detail.
         Deviations are noted in this manual.
         
         Valid values for
         \begin{BVerbatim}[commandchars=\\\{\}]
\PYG{n+ni}{?command?}
         \end{BVerbatim}
         \space (case\hyp insensitive) are documented in a \cref{sec:gates}.
         We use
         \begin{BVerbatim}[commandchars=\\\{\}]
\PYG{n+ni}{?pgfkeys?}
         \end{BVerbatim}
         \space to describe any valid content passed to the \tex!\pgfkeys! macro (rather, \tex!\yquantset! is invoked with some subtleties); and by
         \begin{BVerbatim}[commandchars=\\\{\}]
\PYG{n+ni}{?name?}
         \end{BVerbatim}
         \space we denote any valid register name.
         Register names must not contain any of the control literals used before (semicolon, comma, parentheses, square brackets, dash, pipe, tilde, beginning star); and you should avoid using special \TeX{} characters.
         Note that for performance reasons, \Yquant{} does not check whether a register name is valid or not, but expect to either see unintended output or not\hyp so\hyp helpful error messages if you choose an invalid name.
         \begin{BVerbatim}[commandchars=\\\{\}]
\PYG{n+ni}{?number?}
         \end{BVerbatim}
         \space is a decimal integer larger or equal to zero (in the context of register creation, strictly larger; in this context, it may also contain a leading
         \begin{BVerbatim}[commandchars=\\\{\}]
\PYG{l+s+s2}{\PYGZdq{}+\PYGZdq{}}
         \end{BVerbatim}
         ).
         
      \subsection{Registers}\label{sec:registers}
         Every quantum circuit is structured by means of \emph{registers}.
         A register has a \emph{type} that specifies how its wire is drawn, and that may even change during its lifetime.
         At the moment, \Yquant{} supports four types:
         \begin{enumerate}
            \item \gate{qubit} is the most common type, used for a quantum register.
               It corresponds to a single line.
            \item \gate{cbit} is a classical register, which can be either declared from the beginning or arises by using measurements.
               It corresponds to a double line.
            \item \gate{qubits} is a ``quantum bundle,'' i.e., a bunch of quantum registers that are always addressed in a group as a single register.
               Operations between bundles of the same length should be interpreted as transversal.
               It corresponds to a triple line.
               An alternative (and more common) representation is to use the \gate{qubit} type and a \gate{slash} gate at its very beginning.
            \item \gate{nobit} is the most obscure type, corresponding to a non\hyp existing wire.
               Mostly, this register type arises by using the \gate{discard} command.
               However, it can also be directly declared, which on rare occasions might be necessary (its type can then be changed by means of an \gate{init} or \gate{setwire} pseudo\hyp gate).
               If you want to declare a register only at a certain horizontal position in the circuit, consider using the \texttt{after} argument instead.
         \end{enumerate}
         
         Registers must be declared before they can be used (though in the \tex!yquant*! environment, this declaration may be implicit, creating a \gate{qubit} register).
         
         Registers can have a vector character, i.e., not only a \emph{name}, but also an \emph{index} (or, in the declaration, a \emph{length}).
         The index (zero\hyp based) or length is specified in square brackets following the name, which closely mimics the OpenQASM language.
         
         \compat{New in 0.1.1}Vector registers may be non\hyp contiguous: Whenever you create a bunch of registers, it is put at the bottom of the circuit.
         If you later on again create registers of the same name---either implicitly in the \tex!yquant*! environment, or explicitly by preceding the length of the vectors entries to be added by a plus, as in \yquant!qubit a[+3];!---they will be put to what is \emph{now} the bottom of the circuit, even if some other registers are interspersed.
         
         Registers are referenced---i.e., used in operations---by their name and index.
         If the latter is omitted, all indices of the register are targeted.
         Multiple registers can be referenced by joining their names in a comma\hyp separated list, or by means of a range specifier: give the name of the first (topmost), a dash, and the last (bottom\hyp most) register.
         Both are inclusive.
         In a range specifier, omitting the start name means that the range begins at the first known register; omitting the end name means that the range ends at the last known (at the moment of its use) register.
         Omitting both indicates a range over all known registers.
         
         \compat{New in 0.1.1}It is also possible to use comma\hyp separated lists and ranges within the indices themselves, so that, e.g., \yquant!a[0, 2, 5-], b[-2]! will target the zeroth and second index of \yquant!a!; the remaining indices of \yquant!a! starting from five; and the first three indices of \yquant!b!.
         However, if you use an \emph{outer} range (i.e., a range between indices of registers with different names), the initial and final register of the range must be unique, i.e., either you omit the index (targeting the first or last register with the given name) or specify a single one.
         
         \begin{warning}[Ranges and discontiguous registers]
            Assume a configuration in which the vector register \yquant!a! begins with one qubit, then the single register \yquant!b! follows, and after that \yquant!a! is continued with another qubit.
            
            The range \yquant!a-b! will target \yquant!a[0]! and \yquant!b[0]!, but not \yquant!a[1]!.
            As \yquant!a! is used as the initial register in the range without an explicit index specification, \Yquant{} automatically translates this into \yquant!a[0]!, while \yquant!b!, being used as the final register, is automatically translated into the last register of name \yquant!b! (which here happens to be \yquant!b[0]!).
            Ranges between different register names (outer ranges) are \emph{visual} ranges, i.e., they refer to the top\hyp to\hyp bottom order that is visible.
            Consequently, the register \yquant!a[1]! is left out since it is visually below the others.
            
            Likewise, the range \yquant!b-a! will target \yquant!b[0]! and \yquant!a[1]!.
            
            Ranges within indices are \emph{logical} ranges.
            Hence, \yquant!a!, \yquant!a[-]!, \yquant!a[0-]!, \yquant!a[-1]!, and \yquant!a[0-1]! are all equivalent: they all refer to the registers \yquant!a[0]! and \yquant!a[1]!, but never to \yquant!b!, regardless of any visual position.
         \end{warning}
         
         All that was said so far refers to the operation being carried out on each of the registers \emph{individually}, i.e., producing several copies of the operation.
         This is different from using the operation multiple times on the individual single registers only with regard to the vertical positioning: if specified as a register list with one operation, all copies of the operation will be aligned at the same vertical position (as if an \gate{align} command had been carried out before).
         \begin{warning}
            It is forbidden (in the sense of ``not useful,'' but \Yquant{} does not check for this) to list the same register multiple times (explicitly or via ranges) in one operation.
         \end{warning}
         
         Instead of copies of single\hyp register operations, one might want to carry out a multi\hyp register operation.
         In this case, the desired list of registers (comma separated, range, or both) must be surrounded by parentheses.
         It is possible to mix single- and multi\hyp register operations arbitrarily.
         In an index list, you may also choose to surround only certain indices with parenthesis, provided the whole register is not already a multi\hyp register.
         \begin{warning}
            Note that some gates, such as the \gate{swap} gate, always require (semantically, not grammatically) multi\hyp register operations.
            \compat{Changed in 0.1.2}The number of constituents is not stipulated; while a \gate{swap} gate with more than two targets is no longer well\hyp defined, other registers such as \gate{zz} may still be useful.
            \Yquant{} will prevent you from using a gate in a multi\hyp qubit setting when it may only be used for single registers.
         \end{warning}
         Typically, multi\hyp register operations should only be carried out on adjacent registers---but sometimes, one might want to carry out a multi\hyp qubit operation on a visually discontiguous set of registers (which, due to a particular quantum computer topology, might even be physically feasible).
         \compat{New in 0.1.2}\Yquant{} supports these discontiguous operations explicitly.
         It will draw a \emph{main} part of the gate at the first contiguous slice of registers in the target list---you may select another register for this part by preceding the name or index with a star (which, contrary to the simplified grammar, may only occur \emph{once} in a target specification).
         All other contiguous slices of target registers will be drawn in a \emph{subordinate style} for this gate.
         Finally, all slices will be connected by a single vertical line with the style \style{/yquant/every multi line}.
         \compat{New in 0.2}Subcircuits, will always span the full region from the first to the last register specified in a multi\hyp qubit gate.
         This is due to the fact that they may contain arbitrary ancilla registers which may be positioned somewhere in between the parts that actually constitute the subcircuit---surrounding this with a scattered set of connected subcircuit boxed would look quite unpleasant.
         \clearpage
         \begin{warning}[\compat{Improved in 0.1.2}Discontiguous targets and control lines]
            A control line extends from the very first to the very last affected register in an operation.
            A sub\hyp gate line that is used for discontiguous registers will only span the range of a multi\hyp register.
            This distinction becomes crucial if you want to carry out a \emph{controlled} operation on more than one multi\hyp register, where at least one is discontiguous.
            Without the controls, the separate multi\hyp registers could be identified, since no connecting vertical line extends between them (unless, which you should strictly avoid, they are intertwined).
            However, with the controls, the control line will make it hard (for some gates, impossible) to visually distinguish the connected parts.
            \Yquant{} will kindly provide a warning in this case.
            You may choose to suppress this warning using the boolean key \style{/yquant/operator/multi warning}.
         \end{warning}
         \begin{warning}
            There is no established style for discontiguous gates.
            Note that at the moment, main and subordinate style coincide for all gates except for the \gate{measure} gate with a value.
            In order to still make it possible to visually distinguish discontiguous multi\hyp register gates operating on slices of a single register from just a bunch of single\hyp register gates that are executed in a parallel manner, if controls are present, \Yquant's default vertical line style for the former case is a wavy line instead of a straight one.
            Still, the meaning of this should probably be explained.
            Please feel free to submit issues or pull requests with propositions of how default styles or alternative subordinate gate shapes may additionally help to mitigate the problem.
         \end{warning}
      
      \subsection{Arguments}
         Every command may take one or multiple arguments.
         Those are specified in square brackets that precede the command itself.
         The content of those square brackets is essentially fed to a \tex!\pgfkeys!-like macro.
         The default path is set appropriately such that the arguments of the command can be accessed without path specifiers.
         If the key is not a valid argument for the command or a global argument and it is not given by an absolute path, it is searched for in the \texttt{/yquant} namespace.
         If it cannot be found there, it is passed to \style{/yquant/operator style}.
         
         Note that commands may have required arguments.
         If a required argument is missing, an error will be issued.
         
         The \texttt{value} attribute can alternatively be given inside curly brackets after the command name and before the register specification.
         This has the advantage that special characters such as a closing square bracket need not be escaped.
         If both alternatives are present, the value inside curly brackets takes precedence and a warning is issued.
      
      \subsection{Controls}
         Lots of gates may have controls, i.e., they are only to be executed if some other gate is set or unset.
         The former case is called a \emph{positive control}, the latter one a \emph{negative control}.
         Those are indicated by filled and empty circles on the control registers and a vertical line that joins the registers that belong together.
         
         The gate specification is followed by the list of target registers.
         By then writing a pipe (``\yquant!|!''), the list of positive controls is introduced; this mimics the mathematical syntax ``conditioned on'' for probabilities or ``given'' for sets.
         If there are no positive controls, the list may be empty or, together with the pipe, omitted.
         Preceded by a tilde (``\yquant!~!''), the list of negative controls then follows; this mimics the syntax of many programming languages that denote logical negation by a tilde.
         If there are no negative controls, the list may be empty or, together with the pipe, omitted.
         
      \subsection[Importing circuit from files]{\compat{New in 0.2}Importing circuits from files}\label{sec:import}
         \Yquant{} provides a simple way to import circuits that are stored in external files.
         The macro \tex!\yquantimport! can be used in three different contexts:
         \begin{itemize}
            \item Outside of a \TikZ{} picture environment. \\
               Here, \tex!\yquantimport[<options>]{<filename>}! will be equivalent to
               \begin{minted}{tex}
\begin{tikzpicture}
   \begin{yquant}[<options>]
      % the content of <filename> goes here
   \end{yquant}
\end{tikzpicture}
               \end{minted}
               The starred form, \tex!\yquantimport*[<options>]{<filename>}!, instead inserts the starred \Yquant{} environment.
               Note that the options are always \Yquant{} options; if you want to pass \TikZ{} options, you will have to create the picture environment by yourself or change the option path to the correct one (\texttt{/tikz/.cd}).
            \item Inside a \TikZ{} picture environment, but outside of a \Yquant{} environment. \\
               This is the same as before, just that no extra picture environment will be added.
            \item Inside both a \TikZ{} picture environment and a \Yquant{} environment. \\
               The file will be inserted directly into the environment.
               \Yquant's parser is automatically restarted after this.
               The content will always be put in a \TeX{} group; if additional options are provided, \Yquant{} also inserts a \TikZ{} \tex!scope! and executes \tex!\yquantset{<options>}! directly after the scope.
               If \tex!\yquantimport! is used, the content will be read as if the containing environment was an unstarred one; if \tex!\yquantimport*! is used, the content will be read as if the containing environment was a starred one.
         \end{itemize}
         Note that \Yquant{} internally uses plain \TeX's \tex!\import! command (i.e., \tex!\@@import! in \LaTeX).
         However, when the \pkg{import} package is loaded, it uses \\ \tex!\subimport{\yquantimportpath}{<filename>}!, where \tex!\yquantimportpath! defaults to \texttt{./}---so by changing this, files from other folders may be imported which by themselves again include other files, and the relative path resolution will work.
         
         Note that you may in particular import the content of a \gate{subcircuit}.
      
      \subsection{Defining own gates}
         \begin{warning}[Scope]
            All gate declarations are always global.
         \end{warning}
         
         \compat{New in 0.2.1}If you want to define a gate that corresponds to a single \gate{box} gate with a certain pre\hyp defined content, you may use the macro\\
         \tex!\yquantdefinebox{<name>}[<style>]{<content>}!, which is far more efficient than the much more general \tex!\yquantdefinegate! introduced below.
         It works in the following way:
         \begin{itemize}
            \item It creates a new gate with name \texttt{<name>} that can be accessed as all the other build\hyp in gates.
               Note that \texttt{<name>} is case\hyp insensitive and may not contain spaces.
               Special characters are allowed if \TeX{} can cope with them (i.e., no comment signs, no unbalanced braces, no backslashes...).
            \item It creates a style \style{/yquant/operators/every <name>} and assigns the optional \texttt{<style>} to it.
               If no style is provided, the default style will inherit from \style{/yquant/operators/every box}.
            \item It defines \texttt{<content>} to be the value that is written into the box.
               This \texttt{<content>} is expanded in a protected manner at the time of gate declaration.
               You may need to prefix fragile macros by \tex!\protect!.
         \end{itemize}
         
         \compat{New in 0.2}Sometimes, you may wish to define gates that are more than just a single box---perhaps a succession of multiple gates or even multi\hyp register gates with individual operations on the input registers.
         \Yquant{} provides a simple macro that allows this.
         The macro \tex!\yquantdefinegate{<name>}[<style>]{<content>}! works in the following way:
         \begin{itemize}
            \item It creates a new gate with name \texttt{<name>} that can be accessed as all the other built\hyp in gates.
               Note that \texttt{<name>} is case\hyp insensitive and may not contain spaces.
               Special characters are allowed if \TeX{} can cope with them (i.e., no comment signs, no unbalanced braces, no backslashes...).
            \item It creates a style \style{/yquant/operators/every <name>} and assigns the optional \texttt{<style>} to it.
               If no style is provided, the default style will inherit from \style{/yquant/operators/every custom gate}.
               This will make the gate ``seamless,'' i.e., avoid highlighting the fact that this is a custom gate.
            \item It defines a macro that contains \texttt{<content>} (expanded in a protected manner) and that will be inserted as a subcircuit whenever this gate is invoked.
               This in particular means that if you use \hyperref[sec:import]{\tex!\yquantimport!} within the gate, the file will only be loaded once at the time of declaration.
         \end{itemize}
         When the gate is later drawn, the styles are invoked in the following order---remember custom gates are implemented by means of subcircuits---:
         \begin{enumerate}
            \item \style{/yquant/every operator}
            \item \style{/yquant/operators/every <name>}
            \item \style{/yquant/operators/every subcircuit box}
            \item \style{/yquant/this operator}
            \item \style{/yquant/operators/this subcircuit box}
         \end{enumerate}
         Gates defined in this way can only make use of the default gates or other custom gates.
         They do not accept custom arguments, and it is not possible to declare own, custom shapes in this way (though other predefined shapes may be used).
         If they are used in a multi\hyp qubit manner, they will never be split into contiguous slices (but their content will be, so if you use the default style that turns off the box, the only way to notice this is that intermediate unaffected registers will not be allowed to have gates visually within the rectangle that would bound the custom gate).
         
         \begin{warning}[Redefining existing gates]
            The above macros will issue an error if the gate already exists.
            You can use \tex!\yquantredefinebox! or \tex!\yquantredefinegate! to overwrite existing gate definitions.
            Note that this will overwrite \emph{any} gate, even the built\hyp in ones.
            
            Generally, it is discouraged to make use of this possibility.
            For custom gates, if you redefine a gate as a box which was previously a general subcircuit\hyp based gate, the macro that contains the subcircuit will still be held in memory.
            Overwriting built\hyp in gates will not clear the attributes associated to this gate (though required attributes will no longer be required afterwards).
            Again, this is not a problem but prevents \Yquant{} from issuing potentially helpful error message if such a---now meaningless---attribute is used.
            
            Finally, once a built\hyp in gate is overwritten, it cannot be restored.
            In particular, the register creation pseudo\hyp gates \gate{qubit}, \gate{cbit}, \gate{qubits}, and \gate{nobit} perform some magic that cannot be mimicked with custom gates.
         \end{warning}
         
         More advanced declaration of custom gates requires the use of backend macros.
         Refer to \texttt{yquant-lang.tex} for this.
         For the declaration of custom shapes, see \texttt{yquant-shapes.tex} for examples.
      
   \section{Configuration}\label{sec:config}
      \Yquant{} uses \pkg{pgfkeys} to control its options, which are located in the path \texttt{/yquant}.
      The following list contains all options and styles that are recognized, apart from gate arguments.
      Those are listed together with their operations.
      
      \subsection{Circuit layout}\unskipOpt
         \begin{option}[Changed in 0.4]{register/minimum height}!1.5mm!
            \Yquant{} automatically determines the height (extent from wire to top boundary) of a register as the height of the largest operation.
            This might be too small for two reasons:
            \begin{itemize}
               \item if the register is used only with small gates (e.g., only as a control, or as a swap), and it does not have a label (or one containing only x\hyp height letters).
               \item if you manually turned off height calculation or multi\hyp extent calculation for a large gate.
                  \Yquant{} will then not consider the vertical extent of this gate, which might consequently lead to undesirable overlaps.
            \end{itemize}
            This key provides an easy alleviation of the problem by requiring a minimal height for every register.
            As the value of this key is relevant at the time of register declaration, it can also be changed for each register individually.
            
            Note that this key is affected by the \texttt{compat} setting.
            Before version \texttt{0.4}, there was no \style{/yquant/register/minimum depth} key.
            In this compatibility setting, passing the value $x$ to this key will set both height and depth to $\frac x2$.
            The default for $x$ is then \texttt{3mm}.
         \end{option}
         
         \begin{option}[New in 0.4]{register/minimum depth}!1.5mm!
            \emph{see} \style{/yquant/register/minimum height}
            
            This key allows to specify a minimum depth (extent from wire to bottom boundary) for a register.
            
            Note that this key is affected by the \texttt{compat} setting.
            Before version \texttt{0.4}, this key will not be available.
         \end{option}
         
         \begin{option}{register/separation}!1mm!
            This key controls the amount of vertical space that is inserted between two successive registers.
            Half of this value is also the length that multi\hyp\gate{init} or multi\hyp\gate{output} braces extend beyond the mid position of the register.
         \end{option}
         
         \begin{option}{operator/minimum width}!3mm!
            \Yquant{} automatically determines the width of an operator according to its content.
            However, single\hyp letter boxes are among the most common operators, and giving them slightly different widths would result in a very uneven spacing, as \Yquant{} does not use a grid layout but stacks the operators horizontally one after each other.
            Hence, this key provides a minimum width that will be set for every operator.
            This does not imply that the \emph{visual} appearance (i.e., the \texttt{x radius} key) is enlarged, but that operators of a smaller actual width will be centered in a virtual box of the minimum width.
         \end{option}
         
         \begin{option}{operator/separation}!1mm!
            This key controls the amount of horizontal space that is inserted between two successive operators.
         \end{option}
         
         \begin{option}[New in 0.1.2]{operator/multi warning}!true!
            If this key is \texttt{true}, a warning is displayed whenever more than a single multi\hyp register gate, where at least one is discontiguous, are employed together with controls.
            Even if a visual distinction between control and multi\hyp qubit line may be possible (depending on the style in use), they will overlap and produce unaesthetic output.
            You may disable this warning globally, on a per\hyp circuit, or even on a per\hyp gate basis.
         \end{option}
      
      \subsection{Register creation}\unskipOpt
         \begin{option}{register/default name}!\regidx!
            The printed name that is used by default if a new register is created explicitly (\gate{qubit}, \gate{cbit}, \gate{qubits}; not used for \gate{nobit} or for implicit declarations) and no value is specified.
            The following macros are available:
            \begin{itemize}
               \item \tex!\reg! contains your name to identify this register.
               \item \tex!\idx! contains the index (zero\hyp based) of the current register within a vector register.
               \item \tex!\regidx! expands to \tex!\reg! if the register is of length one, and to \tex!\reg[\idx]! else.
               \item \tex!\len! contains the length of the current register vector.
            \end{itemize}
         \end{option}
         
         \begin{option}{every label}!shape=yquant-text, anchor=center, align=right!
            This style is installed for every single register name label (i.e., upon creation and when used with the \gate{init} gate).
            The default style allows to use line breaks in the labels.
         \end{option}
         
         \begin{option}{every initial label}!anchor=east!
            This style is installed for every single register name label at the left border of the circuit.
            Hence, it is only used for the \gate{init} gate if in the \tex!yquant*! environment, the gate occurs for a new register (which allows to override the default register type).
         \end{option}
         
         \begin{option}{every qubit label}!!
            This style is installed for every single register name label of a register of type \gate{qubit}.
         \end{option}
         
         \begin{option}{every cbit label}!!
            This style is installed for every single register name label of a register of type \gate{cbit}.
         \end{option}
         
         \begin{option}{every qubits label}!!
            This style is installed for every single register name label of a register of type \gate{qubits}.
         \end{option}
         
         \begin{option}[Changed in 0.1.2]{every multi label}!shift={(-.075, 0)}, draw, decoration={gapped brace, mirror}, decorate, /yquant/gapped brace/apply shift, every node/.append style={shape=yquant-text, anchor=east, align=right, shift={(-.05, 0)}, pos=-1}!
            This style is installed for every register name label that is attached to a multi\hyp qubit register by means of the \gate{init} gate.
            \Yquant{} additionally inserts a straight line that connects the topmost and the bottom\hyp most register at their left ends.
            The default style turns this line into a brace and places the description at the appropriate position.
            The decoration \texttt{gapped brace} allows to additionally specify the regions in which a line should be drawn by using the \texttt{/tikz/decoration/from to} key, which expects a comma\hyp separated list of dimension ranges, and which is automatically populated by \Yquant.
            Since it may happen that the arch of the brace needs to be shifted from the value specified in \texttt{/tikz/decoration/aspect} (else, it would be drawn into a gap), the special key \texttt{/yquant/gapped brace/apply shift} is installed, which transforms the \texttt{pos=-1} specification into a position that corresponds to the actual \texttt{aspect} value.
         \end{option}
         
         \begin{option}[New in 0.2]{every input label}!!
            This style is installed for every register name label in a \gate{subcircuit} when the register is an input (or input and output) register.
         \end{option}
      
      \subsection{Register outputs}\unskipOpt
         \begin{option}{every output}!shape=yquant-text, anchor=west, align=left!
            This style is installed for every \gate{output} label at the end of the circuit.
            The default style allows to use line breaks in the labels.
         \end{option}
         
         \begin{option}{every qubit output}!!
            This style is installed for every \gate{output} label of a register of type \gate{qubit}.
         \end{option}
         
         \begin{option}{every cbit output}!!
            This style is installed for every \gate{output} label of a register of type \gate{cbit}.
         \end{option}
         
         \begin{option}{every qubits output}!!
            This style is installed for every \gate{output} label of a register of type \gate{qubits}.
         \end{option}
         
         \begin{option}[Changed in 0.1.2]{every multi output}!shift={(.075, 0)}, draw, decoration={gapped brace}, decorate, /yquant/gapped brace/apply shift, every node/.append style={shape=yquant-text, anchor=west, align=left, shift={(.05, 0)}, pos=-1}!
            This style is installed for every \gate{output} label that is attached to a multi\hyp qubit register.
            \Yquant{} additionally inserts a straight line that connects the topmost and the bottom\hyp most register at their right ends.
            The default style turns this line into a brace and places the description at the appropriate position.
            See \style{/yquant/every multi label} for a more detailed explanation.
         \end{option}
      
      \subsection{General styling}\unskipOpt
         \begin{option}[Changed in 0.4, 0.1.2]{every circuit}!every node/.prefix style={transform shape}!
            Style that is installed for every \tex!yquant! and \tex!yquant*! environment, as if it had been given as an option.
            The style's default path is, as with all other styles, \texttt{/tikz}.
            The style is re\hyp applied for every subcircuit.
            The default style will make all nodes (which in particular means, all gates) respect outer canvas transformations.
            
            The default requires at least \TikZ{} version~3.1.6a.
            Consider updating if you use an older version or add \tex!every label/.prefix style={transform shape=false}! to this style, which undoes the effect for \texttt{label}s (see \TikZ{} bug \href{https://github.com/pgf-tikz/pgf/issues/843}{\#843}).
         \end{option}
         
         \begin{option}{every wire}!draw!
            This style is installed whenever a wire is drawn.
         \end{option}
         
         \begin{option}{every qubit wire}!!
            This style is installed whenever a wire for a register of type \gate{qubit} is drawn.
         \end{option}
         
         \begin{option}{every cbit wire}!!
            This style is installed whenever a wire for a register of type \gate{cbit} is drawn.
         \end{option}
         
         \begin{option}{every qubits wire}!!
            This style is installed whenever a wire for a register of type \gate{qubits} is drawn.
         \end{option}
         
         \begin{option}{every control line}!draw!
            This style is used to draw the vertical control line that connects controlled operations and their controls.
         \end{option}
         
         \begin{option}{every control}!shape=yquant-circle, anchor=center, radius=.5mm!
            This style is used to draw the node for a control, both positive and negative.
         \end{option}
         
         \begin{option}{every positive control}!fill=black!
            This style is installed for every positive control (i.e., one that conditions on the register being in state $\ket1$ or $1$).
         \end{option}
         
         \begin{option}{every negative control}!draw!
            This style is installed for every negative control (i.e., one that conditions on the register being in state $\ket0$ or $0$).
         \end{option}
         
         \begin{option}{every operator}!anchor=center!
            This style is installed for every gate (and also pseudo\hyp gates such as the \gate{slash} operator) that acts on one or multiple registers.
         \end{option}
         
         \begin{option}[New in 0.1.2]{every multi line}!draw, decoration={snake, amplitude=.25mm, segment length=5pt}, decorate!
            This style is used to draw the vertical line that connects discontiguous slices of sub\hyp gates.
         \end{option}
         
         \begin{option}{this operator}!!
            This style is appended to the current style installed for an operator; it should be used only locally to overwrite any global configuration effect.
         \end{option}
         
         \begin{option}{this control}!!
            This style is appended to the current style installed for a control; it should be used only locally to overwrite any global configuration effect.
         \end{option}
         
         \begin{option}{operator style}!/yquant/this operator/.append style={#1}!
            This is a shorthand that can be used to modify the appearance of the current operator.
         \end{option}
         
         \begin{option}{control style}!/yquant/every control line/.append style={#1}, /yquant/this control/.append style={#1}!
            This is a shorthand that can be used to modify the appearance of the current control and its associated line.
         \end{option}
         
         \begin{option}{style}!/yquant/operator style={#1}, /yquant/control style={#1}!
            This is a shorthand that modifies the appearance of both the current operator and any controls or control lines.
         \end{option}
         
         \begin{option}[New in 0.1.2]{operator/multi as single}!/yquant/every multi line/.style=/yquant/every control line!
            This style is automatically set for certain gates such as the \gate{swap} or the \gate{zz} gate.
            For those gates, neighboring registers will be treated as discontiguous; and this style will enforce their connecting line to have the style used by control lines.
            
            The default \style{/yquant/every multi line} is a wavy line; this allows to distinguish discontiguous multi\hyp qubit gates from multiple single\hyp qubit gates when using controls.
            Still, some gates have such an established appearance that---despite being logically misleading---we rather use the same style as for a control line.
         \end{option}
         
         \begin{option}[New in 0.4]{circuit/seamless}!false!
            The value of this setting determines whether the current circuit is drawn in a ``seamless'' state.
            The key \style{/yquant/operator/separation} will control the amount of padding with which a wire starts or ends before the first or after the last gate.
            By turning on the seamless state, this padding is suppressed.
            Using outputs or giving an initial value at the register declaration brings the corresponding padding back.
            Usually, this setting is intended only for subcircuits.
            Direct access is discouraged, as it will persist in subcircuits.
            Only access it via \style{/yquant/operators/subcircuit/seamless}.
         \end{option}
      
      \subsection{Styles for operators}\unskipOpt
         \begin{option}[Changed in 0.4]{operators/every barrier}!shape=yquant-line, dashed, draw, shorten <= -1mm, shorten >= -1mm!
            This style is installed for every \gate{barrier} pseudo\hyp gate, i.e., the one that is used to explicitly denote a separation between ``before'' and ``after'' within the circuit.
            
            Note that the \texttt{shorten} keys are only present in the default style if you specify at least the compatibility version \texttt{0.4}.
         \end{option}
      
         \begin{option}{operators/every box}!shape=yquant-rectangle, draw, align=center, inner xsep=1mm, x radius=2mm, y radius=2.47mm!
            This style is installed for every \gate{box} operator.
         \end{option}
         
         \begin{option}[Changed in 0.4\\New in 0.2]{operators/every custom gate}!/yquant/operators/subcircuit/seamless, /yquant/register/default name=!
            This style is by default installed for every user\hyp defined gate.
            User\hyp defined gates are implemented via subcircuits; this style suppresses the box that surrounds the subcircuit and by default suppresses all register names.
            This allows a seamless integration of the gate/subcircuit into the main circuit, without putting particular emphasis to the fact that what was defined as the custom gate indeed belongs together.
         \end{option}
         
         \begin{option}{operators/every dmeter}!shape=yquant-dmeter, x radius=2mm, y radius=2mm, fill=white, draw!
            This style is installed for every \gate{dmeter} gate.
            The \texttt{yquant-dmeter} shape consists of a rectangle whose right side is replaced by a circle, resembling the letter ``D.''
         \end{option}
         
         \begin{option}{operators/every h}!/yquant/operators/every box!
            This style is installed for every \gate{h} (Hadamard) operator.
         \end{option}
         
         \begin{option}{operators/every measure}!shape=yquant-measure, x radius=4mm, y radius=2.5mm, draw!
            This style is installed for every \gate{measure} gate.
            The \texttt{yquant-measure} shape is a rectangle that contains a ``meter'' symbol.
            It allows for a text to be put inside (e.g., a basis), which then shifts the meter symbol accordingly.
         \end{option}
         
         \begin{option}{operators/every measure meter}!draw, -{Latex[length=2.5pt]}!
            This style is applied to the path that resembles the ``meter'' symbol that is drawn by the \texttt{yquant-measure} shape.
            Due to the default style, the \TikZ{} library \pkg{arrows.meta} is automatically loaded with \Yquant.
         \end{option}
         
         \begin{option}{operators/every not}!shape=yquant-oplus, radius=1.3mm, draw!
            This style is installed for every \gate{not} or \gate{cnot} gate (which are synonyms, and actually do the same as the Pauli~$\sigma_x$ gate).
            The \texttt{yquant-oplus} shape resembles the addition\hyp modulo\hyp two symbol $\oplus$.
         \end{option}
         
         \begin{option}{operators/every pauli}!/yquant/operators/every box!
            This style is installed for every Pauli operator, i.e., \gate{x}, \gate{y}, and \gate{z}.
         \end{option}
         
         \begin{option}{operators/every phase}!shape=yquant-circle, radius=.5mm, fill!
            This style is installed for every \gate{phase} gate $\ketbra00 + \ee^{\ii\phi} \ketbra11$.
         \end{option}
         
         \begin{option}{operators/every slash}!shape=yquant-slash, x radius=.5mm, y radius=.7mm, draw!
            This style is installed for every \gate{slash} pseudo\hyp gate, i.e., the one that is used to indicate that a single register line actually denotes multiple registers.
         \end{option}
         
         \begin{option}[New in 0.2]{operators/every subcircuit}!!
            This style is installed for every \gate{subcircuit}.
            Note that all styles given here will also apply to every element in the subcircuit; in a way, this is an addition to \style{/yquant/every circuit} (which is also again put into effect at the beginning of a subcircuit).
         \end{option}
         
         \begin{option}[Changed in 0.4\\New in 0.2]{operators/every subcircuit box}!/yquant/operators/every box, inner xsep=0pt!
            This style is installed for every \gate{subcircuit}.
            Note that in contrast to all other styles such as \style{/yquant/operators/every subcircuit} or \style{/yquant/this operator}, this style is only applied to the ``container'' node of the subcircuit, but not to the elements in the subcircuit themselves.
         \end{option}
         
         \begin{option}[New in 0.2]{subcircuit box style}!/yquant/operators/every subcircuit box/.append style={#1}!
            This is a shorthand to append styles to the subcircuit box only.
         \end{option}
         
         \begin{option}[New in 0.2]{operators/this subcircuit box}!!
            This style is appended to the current style installed for the \gate{subcircuit}, but will not apply to its contents.
            Additionally, this style will be reset to an empty style at the beginning of each subcircuit, so that it really only applies to exactly the subcircuit box it is explicitly specified on, not to nested subcircuit boxes.
         \end{option}
         
         \begin{option}[New in 0.2]{this subcircuit box style}!/yquant/operators/this subcircuit box/.append style={#1}!
            This is a shorthand to append styles to the current subcircuit box only.
         \end{option}
         
         \begin{option}[New in 0.4]{operators/subcircuit/frameless}!/yquant/operators/this subcircuit box/.append style={draw=none, inner ysep=0pt}!
            This is a shorthand style that removes the frame and additional vertical separation for the current subcircuit.
         \end{option}
         
         \begin{option}[New in 0.4]{operators/subcircuit/seamless}!/yquant/operators/subcircuit/frameless, /yquant/circuit/seamless!
            This option carries out multiple actions that are responsible to let the current subcircuit appear in a ``seamless'' state:
            \begin{itemize}
               \item It calls \style{/yquant/operators/subcircuit/frameless}.
               \item It sets \style{/yquant/circuit/seamless} to \texttt{true}.
               \item It ensures that \style{/yquant/circuit/seamless} is reset within the subcircuit, so that it does not propagate to nested subcircuits.
            \end{itemize}
         \end{option}
         
         \begin{option}{operators/every swap}!shape=yquant-swap, radius=.75mm, draw!
            This style is installed for every \gate{swap} gate that interchanges two qubits.
            The \texttt{yquant-swap} shape consists of a single cross.
         \end{option}
         
         \begin{option}[New in 0.1.2]{operators/every wave}!shape=yquant-circle, radius=.5mm, fill!
            This style is installed for every \gate{correlate} gate.
         \end{option}
         
         \begin{option}{operators/every x}!/yquant/operators/every pauli!
            This style is installed for every Pauli operator $\sigma_x$, i.e., \gate{x}.
         \end{option}
         
         \begin{option}{operators/every xx}!shape=yquant-rectangle, radius=.75mm, draw!
            This style is installed for every \gate{xx} gate in symmetrized notation ($\ketbra{++}{++} + \ketbra{+-}{+-} + \ketbra{-+}{-+} - \ketbra{--}{--}$).
         \end{option}
         
         \begin{option}{operators/every y}!/yquant/operators/every pauli!
            This style is installed for every Pauli operator $\sigma_y$, i.e., \gate{y}.
         \end{option}
         
         \begin{option}{operators/every z}!/yquant/operators/every pauli!
            This style is installed for every Pauli operator $\sigma_z$, i.e., \gate{z}.
         \end{option}

         \begin{option}{operators/every zz}!shape=yquant-circle, radius=.5mm, fill!
            This style is installed for every \gate{zz} gate (aka \textsc{CPhase}) in symmetrized notation ($\ketbra{00}{00} + \ketbra{01}{01} + \ketbra{10}{10} - \ketbra{11}{11}$).
         \end{option}
         
   \section{Doing the impossible}\label{sec:tikz}
      \Yquant{} will almost certainly never be able to do everything an author has in mind.
      Sometimes, there is the need to draw something non\hyp standard, and this cannot be implemented in the \Yquant{} language.
      However, since \Yquant{} is a layer on top of \TikZ, it should be very hard to find something (meaningful) that cannot be done by combining the power of both packages.
      
      \subsection{Mixing \texorpdfstring{\Yquant}{yquant} and \texorpdfstring{\TikZ}{TikZ} code}
         Before or after any gate, you may interrupt the \Yquant{} instructions to perform arbitrary \TikZ{} path operations.
         After every such operation, \Yquant{} will automatically restart its parser so that you can fluently jump between \Yquant{} and \TikZ{} code.
         You can even interject arbitrary \TeX{} code (or, say, low\hyp level \pkg{pgf} commands); however, then, \Yquant{} is not able to restart its parser.
         For this reason, after the last command in a block of \TeX{} commands, you must issue \tex!\yquant!, which then re\hyp enables the \Yquant{} language.
      
      \subsection{Accessing gates in \texorpdfstring{\TikZ}{TikZ}}
         The feature to perform arbitrary \TikZ{} operations is powerful in itself, but would be of limited use were there no way to access the elements in the quantum circuit.
         \Yquant{} provides a global attribute \texttt{name} that can be assigned to every gate.
         All quantum operations are in fact \TikZ{} nodes, and the name you give to them then becomes a \TikZ{} name, which you can easily reference to get the coordinates of a particular operator.
         Note that the name you specify is only available if a single register is targeted.
         The name is suffixed by \tex!-\idx!, where \tex!\idx! refers to the (zero\hyp based) index of the operation ordered from top to bottom (i.e., if an operator acts on two qubits and should be named \texttt{op}, the topmost operator will be available as \texttt{op-0} and the second as \texttt{op-1}).
         Multiple slices in a discontiguous multi\hyp register are additionally suffixed by \texttt{-s\textit{<slice index>}}.
         All controls are also named, suffixed by \tex!-p\idx! or \tex!-n\idx! for positive and negative controls (i.e., the topmost positive control of the previous operator will be available as \texttt{op-p0}).
         Counters for target registers, positive, and negative controls are all independent.
         Finally, you can even access names within a subcircuit, provided you give a name to the subcircuit.
         All nodes in the subcircuit will then have the name \texttt{\textit{<subcircuit name>}-\textit{<name specified in the subcircuits>}}.
         For nested subcircuits, you will get multiple prefixes.
      
      \subsection{Shapes and the drawing pipeline}
         All \Yquant{} shapes have the anchors available you would typically expect from a \TikZ{} shape of the given outline.
         The \texttt{center} anchor will be aligned to the wire.
         \compat{Changed in 0.1.2}In addition to the normal paths implemented by \TikZ{} shapes, those fit for \Yquant{} must additionally implement clipping paths, a \Yquant{} addition to \TikZ{} shapes.
         Such a path has to provide the ``clipping outline,'' i.e., anything that should not contain register or control lines.
         There may be a difference between horizontal and vertical clipping outlines.
         To understand clipping paths, \Yquant's drawing pipeline needs to be explained.
         \begin{itemize}
            \item In a first run---this is what happens directly at the position where you type the gate command---\Yquant{} will ``virtually'' draw the gates in order to determine their dimensions and calculate register heights.
               The actual drawing commands are written to a macro (this is the cause that some macros must be preceded by \tex!\protect! if used in a gate \texttt{value}).
            \item When \tex!\end{yquant}! is encountered, the actual drawing commands are executed.
            \item Unless the operation changes the wire type or style, do the following (first two items for every register at which an operator node has to be created).
               \begin{itemize}
                  \item Create the operator node at the appropriate position.
                  \item Call \tex!\pgfshapeclippath! on the newly created node.
                     This will first determine whether the node was stroked; if not, \tex!\pgflinewidth! is set to zero.
                     Then, it will call the horizontal clipping path, which is supposed to create some soft path commands.
                     Those soft path commands are collected in a macro on a per\hyp register basis and the soft path is cleared.
                     The same happens for the vertical clipping path, which is collected in a macro on a per\hyp operation basis.
                  \item If control lines or multi lines are to be drawn, the vertical clipping path commands are now executed and installed as an inverted\footnote{Inverting the clipping means that instead of drawing only \emph{within} the clipping path (which corresponds to the gates), we only draw \emph{outside}. However, as there is no direct support for this, we invert by exploiting the even\hyp odd rule. If you specify a register multiple times, whether as target, control, or mixed, funny effects can be expected, as the clipping region is inverted multiple times. Note that using a register more than once is always an error, but \Yquant{} does not check for it due to the high overhead.} clipping path.
                  \item Control lines and multi lines are drawn (in this order) from one to the next \texttt{center} anchor.
                     Due to the clipping commands, this will create a perfect connection with the shape of the gate, but even transparent gates are possible without the lines being visible.
               \end{itemize}
            \item If the operation changes the wire type or style, or if there is no operation left on this register, the following is done.
               \begin{itemize}
                  \item Load the clipping paths accumulated for all the gates acting on this register and install the inverted clipping.
                  \item Draw the wire as one continuous line from where the last wire ended (or the beginning of the circuit) to the center of the last gate, or to the common end position for all wires of the circuit.
                  \item Remove the clipping paths stored so far on this register, apart from the clipping on the last gate (which will be needed again if this was not the end of the circuit).
               \end{itemize}
         \end{itemize}
      
      \subsection{Overwriting the height and depth calculation}
         \Yquant{} automatically takes care of calculating the height and depth of all registers, so that their final vertical positions are chosen without overlap.
         This is almost always advisable, but it has some weaknesses:
         \begin{itemize}
            \item If you specify a multi\hyp register gate, say, extending for three registers and this requires a certain height and depth, where should this be accounted for?
               \compat{New in 0.4}\Yquant{} is able to handle these situations by first determining all heights and depths that can safely be attributed to individual registers.
               After that, it checks for all multi\hyp register gates: Is the space from the top of the first to the bottom of the last register enough to hold the multi\hyp qubit gate?
               If not, it evenly distributes the additional required space to all registers that are visually within the range of this multi\hyp register.
               
               This will fail to produce good results only in two cases:
               \begin{itemize}
                  \item If you place \texttt{label}s on the gate, those are outside of the gate---and typically, either below or above.
                     Hence, the additional extent stemming from them should \emph{not} be equally distributed among all registers, but either to the height of the first or the depth of the last one.
                     Currently, \Yquant{} is unable to detect this (and, considering the fact that you can place \texttt{label}s at any angle, this is not an easy problem to solve except for special cases).
                  \item If you make use of a discontiguous \gate{init} gate with a large vertical extent, \Yquant{} will correctly allocate space as if the gate's content were placed in the vertical center.
                     However, if there is no way to put the arch of the brace at the middle, as the register at this position is excluded from the gate, the content will be shifted---but only after calculating the extent.
                     Hence, the automatically calculated vertical positions will be unsuitable.
               \end{itemize}
            \item Sometimes, there is more space available than \Yquant{} thinks because you already discarded some wire.
               \Yquant{} does not keep track of whether the wires below or above a gate are actually visible at this position---which might not even be known at the time the gate command is issued.
               Hence, you may choose to draw ``within'' the other, invisible wire.
         \end{itemize}
         
         \compat{New in 0.4}In these certain special cases, you may want to turn off the automatic calculation for one particular gate.
         Note that you may then, depending on the situation, obtain results with overlapping gates.
         You can use the keys \style{/yquant/register/minimum height} and \style{/yquant/register/minimum depth} when declaring the relevant register to manually specify a larger desired value, but you have to experiment with regard to what this value is.
         
         The global attribute \texttt{overlay} (conveniently overwriting \TikZ's \texttt{overlay} key, which should not be used for gates) can take the values
         \begin{itemize}
            \item \texttt{true} (default, combines \texttt{multi}, \texttt{height}, and \texttt{depth}),
            \item \texttt{multi} (short \texttt{m}),
            \item \texttt{height} (short \texttt{ht}, \texttt{h}),
            \item \texttt{depth} (short \texttt{dp}, \texttt{d}),
            \item \texttt{single} (short \texttt{s}, equals \texttt{height} and \texttt{depth}), and
            \item \texttt{false} (useless).
         \end{itemize}
         It disables the calculation of the selected vertical extent for this particular gate.
         (In fact, \texttt{multi}, \texttt{height}, \texttt{depth}, and \texttt{single} accept boolean values.)
   
   \section{Reference: Gates and operations}\label{sec:gates}
      This section lists all operations \Yquant{} currently understands.
      It also details all arguments that can be given to customize the operation, apart from \texttt{name} and \texttt{overlay}, which are always available.
      Note that the \tex![value=<value>]! attribute can (and should) alternatively be given as a braced expression that follows the name of the register.
      
      \subsection[\texorpdfstring{\yquant!addstyle!}{addstyle}]{\compat{New in 0.1.2}\yquant!addstyle!}\label{gate:addstyle}
         Syntax: \yquant!setstyle <target>;! \\
         This is an invisible pseudo\hyp gate that immediately changes the \TikZ{} style with which the register lines of all target registers are drawn.
         It adds to the styles that are already installed.
         Use \gate{setstyle} to replace styles.
         It may not span multiple registers and does not allow for controls.
         
         \emph{Possible attributes:}
         \begin{itemize}
            \item \yquant![value=<styles>]! (required) \\
               Denotes the new styles; this should be a string that could be passed to \tex!\tikzset!.
         \end{itemize}
      
      \subsection{\texorpdfstring{\yquant!align!}{align}}\label{gate:align}
         Syntax: \yquant!align <target>;! \\
         This is an invisible pseudo\hyp gate that enforces all affected registers to share a common horizontal position for their next gate, which is determined by the largest position of all gates involved.
         It may not span multiple registers and does not allow for controls.
         
         \emph{Possible attributes:} none
      
      \subsection{\texorpdfstring{\yquant!barrier!}{barrier}}\label{gate:barrier}
         Syntax: \yquant!barrier <target>;! \\
         This is a pseudo\hyp gate that denotes some physical barrier that ensures execution with a specific timing; it is basically a visible version of the \hyperref[gate:align]{\yquant!align!} gate, denoted by a vertical line.
         It may span multiple registers, but does not allow for controls.
         The style \style{/yquant/operators/every barrier} is installed.
         
         \emph{Possible attributes:} none
         
      \subsection{\texorpdfstring{\yquant!box!}{box}}\label{gate:box}
         Syntax: \yquant!box <target> | <pcontrol> ~ <ncontrol>;! \\
         This is a generic register of a rectangular shape that can be filled with arbitrary content.
         It may span multiple registers and allows for controls.
         The style \style{/yquant/operators/every box} is installed.
         
         \emph{Possible attributes:}
         \begin{itemize}
            \item \yquant![value=<value>]! \\
               Denotes the content of the box.
         \end{itemize}
      
      \subsection{\texorpdfstring{\yquant!cbit!}{cbit}}\label{gate:cbit}
         Syntax: \yquant!cbit <name>[<len>];! \\
         Declares a register of type \texttt{cbit}.
         
         \emph{see \hyperref[gate:qubit]{\yquant!qubit!}}
      
      \subsection[\texorpdfstring{\yquant!correlate!}{correlate}]{\compat{New in 0.1.2}\yquant!correlate!}\label{gate:correlate}
         Syntax: \yquant!correlate <target>;! \\
         This is a pseudo\hyp gate that indicates a correlation (usually a Bell\hyp state) present between the multi\hyp registers involved.
         This gate should span multiple registers and does not allow for controls.
         
         \emph{Possible attributes:} none
      
      \subsection{\texorpdfstring{\yquant!cnot!}{cnot}}\label{gate:cnot}
         Syntax: \yquant!cnot <target> | <pcontrol> ~ <ncontrol>;! \\
         This is a synonym for the \gate{not} gate.
         Note that despite its name, controls are not mandatory and also here, the style \style{/yquant/operators/every not} is installed.
      
      \subsection{\texorpdfstring{\yquant!discard!}{discard}}\label{gate:discard}
         Syntax: \yquant!discard <target>;! \\
         This is an invisible pseudo\hyp gate that changes the type of all target registers to \gate{nobit}, i.e., no line will be drawn for them.
         This has effect already for the outgoing line of the last visible gate on the target registers.
         The gate may not span multiple registers and does not allow for controls.
         To change a register type on\hyp the\hyp fly into something different from \gate{nobit}, use the \compat{Changed in 0.1.2}\gate{settype} pseudo\hyp gate.
         
         \emph{Possible attributes:} none
      
      \subsection{\texorpdfstring{\yquant!dmeter!}{dmeter}}\label{gate:dmeter}
         Syntax: \yquant!dmeter <target>;! \\
         This is a measurement gate, denoted by a ``D'' shape.
         It changes the type of all targets involved.
         It may span multiple registers, but does not allow for controls.
         The style \style{/yquant/operators/every dmeter} is installed.
         
         \emph{Possible attributes:}
         \begin{itemize}
            \item \yquant![value=<value>]! \\
               Allows to specify a text that will be included inside the gate, possible enlarging its width.
               For outside texts, use \TikZ{} \texttt{label}s instead.
            \item \yquant![type=<qubit|cbit|qubits>]! \\
               Allows to specify the type into which the affected targets are converted.
               Default is \gate{cbit}.
         \end{itemize}
      
      \subsection{\texorpdfstring{\yquant!h!}{h}}\label{gate:h}
         Syntax: \yquant!h <target> | <pcontrol> ~ <ncontrol>;! \\
         This is a Hadamard gate, $\bigl( \ketbra00 + \ketbra01 + \ketbra10 - \ketbra11 \bigr) \fracslash \sqrt2$, denoted by a rectangle that contains the letter $H$.
         It may not span multiple registers, but allows for controls. \\
         The style \style{/yquant/operators/every h} is installed.
         
         \emph{Possible attributes:} none
      
      \subsection{\texorpdfstring{\yquant!hspace!}{hspace}}\label{gate:hspace}
         Syntax: \yquant!hspace <target>;! \\
         This is an invisible pseudo\hyp gate that inserts a certain amount of white space into all target registers.
         It may not span multiple registers and does not allow for controls.
         
         \emph{Possible attributes}:
         \begin{itemize}
            \item \yquant![value=<dim>]! (required) \\
               Gives the amount of white space that is to be inserted. Must be a valid \TeX{} dimension.
         \end{itemize}
      
      \subsection{\texorpdfstring{\yquant!init!}{init}}\label{gate:init}
         Syntax: \yquant!init <target>;! \\
         This is a pseudo\hyp gate that (re)initializes a registers to a given state.
         It may span multiple registers, but does not allow for controls.
         The style \style{/yquant/every label} is installed.
         Note that this pseudo\hyp gate, unlike all others, behaves differently if it is the first operation acting on a register: in this case, it does not increment the horizontal position, but uses the space available to the left; and the style \style{/yquant/every initial label} is installed additionally.
         For multiple registers, the style \style{/yquant/every multi label} is installed at the end, and a path is constructed that extends from the left end of the first to the left end of the last register in the multi\hyp register compound.
         
         \emph{Possible attributes:}
         \begin{itemize}
            \item \yquant![type=<qubit|cbit|qubits>]! \\
               Allows to specify the type into which the affected target registers are converted.
               Default is the type of the first target register that is different from \gate{nobit}, or \gate{qubit} if they all are \gate{nobit}.
               The style \texttt{/yquant/every <type> label} is installed additionally.
            \item \yquant![value=<value>]! (required) \\
               Denotes the label that is printed to the left of the wire.
               
               Inside the value, \tex!\idx! expands to the current index within the register list.
         \end{itemize}
      
      \subsection{\texorpdfstring{\yquant!measure!}{measure}}\label{gate:measure}
         Syntax: \yquant!measure <target>;! \\
         This is a measurement gate, denoted by a rectangle with a meter symbol.
         It changes the type of all targets involved.
         It may span multiple registers, but does not allow for controls.
         The style \style{/yquant/operators/every measure} is installed.
         
         \emph{Possible attributes:}
         \begin{itemize}
            \item \yquant![type=<qubit|cbit|qubits>]! \\
               Allows to specify the type into which the affected targets are converted.
               Default is \gate{cbit}.
            \item \yquant![value=<value>]! \\
               Allows to specify a text that will be included at the bottom of the rectangle (which will shift the meter symbol upwards accordingly).
               For outside texts, use \TikZ{} \texttt{label}s instead.
         \end{itemize}
         
      \subsection{\texorpdfstring{\yquant!nobit!}{nobit}}\label{gate:nobit}
         Syntax: \yquant!nobit <name>[<len>];! \\
         Declares a register of type \gate{nobit}.
         The \texttt{<name>} must be a self\hyp chosen name for the register which was not previously used as a register name in this \tex!yquant! environment.
         Names are case\hyp insensitive.
         The register can be made into a vector register by specifying \texttt{<len>} (default \tex!1!).
         
         \clearpage
         \emph{Possible attributes:}
         \begin{itemize}
            \item \compat{New in 0.2}\yquant![out]! or \yquant![ancilla]! (required in subcircuits) \\
               \emph{see \gate{qubit}}
         \end{itemize}
      
      \subsection{\texorpdfstring{\yquant!not!}{not}}\label{gate:not}
         Syntax: \yquant!not <target> | <pcontrol> ~ <ncontrol>;! \\
         This is a \textsc{not} gate, $\ketbra01 + \ketbra10$, denoted by the $\oplus$ symbol.
         It may not span multiple registers, but allows for controls.
         Due to its common usage, the synonymous gate \hyperref[gate:cnot]{\yquant!cnot!} is provided.
         The style \style{/yquant/operators/every not} is installed.
         
         \emph{Possible attributes:} none
      
      \subsection{\texorpdfstring{\yquant!output!}{output}}\label{gate:output}
         Syntax: \yquant!output <target>;! \\
         This is a pseudo\hyp gate that allows to write some text at the very end of the register line.
         It may only be specified once per register.
         It may span multiple registers, but does not allow for controls.
         The style \style{/yquant/every output} is installed, and also the style \texttt{/yquant/every <type> output}, where \texttt{<type>} is the type of the affected register (at the time of printout).
         For outputs on multiple registers, the style \style{/yquant/every multi output} is installed instead of the two previously mentioned ones; and additionally, a path is constructed that extends from the first to the last register in the multi\hyp register compound.
         
         \emph{Possible attributes:}
         \begin{itemize}
            \item \yquant![value=<value>]! (required) \\
               Denotes the text that is to be printed.
               Inside the value, \tex!\idx! expands to the current index within the register list.
         \end{itemize}
      
      \subsection{\texorpdfstring{\yquant!phase!}{phase}}\label{gate:phase}
         Syntax: \yquant!phase <name> | <pcontrol> ~ <ncontrol>;! \\
         This is a phase gate, $\ketbra00 + \ee^{\ii\phi} \ketbra11$, denoted by a filled circle.
         It may not span multiple registers, but allows for controls.
         The style \style{/yquant/operators/every phase} is installed.
         
         \emph{Possible attributes:}
         \begin{itemize}
            \item \yquant![value=<value>]! (required) \\
               Denotes the angle $\phi$ that is to be printed together with the gate.
               Position and appearance can be influenced by setting the position of \TikZ{} labels, as this is internally used.
               Note that at the moment, it is not possible to change any label options on a gate\hyp type basis, only locally or fully globally (\TikZ{} feature request \href{https://github.com/pgf-tikz/pgf/issues/811}{\#811}).
         \end{itemize}
         
      \subsection{\texorpdfstring{\yquant!qubit!}{qubit}}\label{gate:qubit}
         Syntax: \yquant!qubit <name>[<len>];! \\
         Declares a register of type \texttt{qubit}.
         The \texttt{<name>} must be a self\hyp chosen name for the register which was not previously used as a register name in this \tex!yquant! environment.
         Names are case\hyp insensitive.
         The register can be made into a vector register by specifying \texttt{<len>} (default \tex!1!).
         
         \emph{Possible attributes:}
         \begin{itemize}
            \item \yquant![after=<regname>]! \\
               If given, the register will start not at the left of the circuit but instead at the position at which the last gate in the register \texttt{<regname>} ended.
               
               This attribute may not be given in combination with \yquant![in]! or \yquant![inout]!.
            \item \compat{New in 0.2}\yquant![in]!, \yquant![out]!, \yquant![inout]!, or \yquant![ancilla]! \\
               Default: \yquant![ancilla]! for top\hyp level circuits (do not change there); \yquant![inout]! for subcircuits.
               
               Determines how a subcircuit interacts with its parent circuit.
               
               Registers declared with the \yquant![ancilla]! attribute are available only to the subcircuit; they cannot be connected to an outside wire.
               
               Registers declared with the \yquant![in]! or \yquant![inout]! attribute will expect an outer wire of the same type to be present and will then be identical with this outer wire.
               Any changes applied to the wire within the subcircuit automatically also happen on the associated outer wire.
               If the attribute is \yquant![in]!, the wire will automatically be discarded at the end of the subcircuit (and hence also in the outer circuit, where it may be re\hyp initialized).
               This is different from applying the \gate{discard} gate in that the wire will still extend until the end of the subcircuit and may thus receive proper \gate{output}s.
               
               Registers declared with the \yquant![out]! attribute will expect a discarded outer wire to be present, which will be initialized to a \texttt{qubit} at the beginning of the subcircuit, and from then on be identical with the outer wire.
            \item \yquant![value=<value>]! \\
               Denotes the label that is printed to the left of the wire.
               If the value is omitted, the default is used (\style{/yquant/register/default name}, preinitialized to \tex!\regidx!).
               
               Inside the value, \tex!\reg! expands to \texttt{<name>}, \tex!\len! expands to \texttt{<len>}, \tex!\idx! expands to the current index within the vector register ($0 \leq \text{\tex!\idx!} < \text{\texttt{<len>}}$), and \tex!\regidx! expands to \tex!\reg! if \texttt{<len>} is one, or to \tex!\reg[\idx]! else.
         \end{itemize}
      
      \subsection{\texorpdfstring{\yquant!qubits!}{qubits}}\label{gate:qubits}
         Syntax: \yquant!qubits <name>[<len>];! \\
         Declares a register of type \texttt{qubits}.
         
         \emph{see \hyperref[gate:qubit]{\yquant!qubit!}}
      
      \subsection[\texorpdfstring{\yquant!setstyle!}{setstyle}]{\compat{New in 0.1.2}\yquant!setstyle!}\label{gate:setstyle}
         Syntax: \yquant!setstyle <target>;! \\
         This is an invisible pseudo\hyp gate that immediately changes the \TikZ{} style with which the register lines of all target registers are drawn.
         It replaces all previous styles.
         Use \gate{addstyle} to accumulate styles.
         It may not span multiple registers and does not allow for controls.
         
         \emph{Possible attributes:}
         \begin{itemize}
            \item \yquant![value=<styles>]! (required) \\
               Denotes the new styles; this should be a string that could be passed to \tex!\tikzset!.
         \end{itemize}
         
      \subsection[\texorpdfstring{\yquant!settype!}{settype}]{\compat{New in 0.1.2}\yquant!settype!}\label{gate:settype}
         Syntax: \yquant!settype <target>;! \\
         This is an invisible pseudo\hyp gate that immediately changes the type of the targets registers, taking effect with the output line extending from the last drawn gate.
         It may not span multiple registers and does not allow for controls.
         
         \emph{Possible attributes:}
         \begin{itemize}
            \item \yquant![value=<qubit|cbit|qubits>]! (required) \\
               Denotes the new type that is assigned to all registers.
               To change the type to \gate{nobit}, use the \hyperref[gate:discard]{\yquant!discard!} pseudo\hyp gate instead.
         \end{itemize}
      
      \subsection[\texorpdfstring{\yquant!setwire!}{setwire}]{\compat{Removed in 0.4\\Deprecated in 0.1.2}\yquant!setwire!}\label{gate:setwire}
         \emph{Use \gate{settype} instead.}
         
         This gate is only available if a compatibility version before \texttt{0.4} is chosen.
      
      \subsection{\texorpdfstring{\yquant!slash!}{slash}}\label{gate:slash}
         Syntax: \yquant!slash <target>;! \\
         This is a pseudo\hyp gate used to denote that a single line actually represents multiple registers.
         It is drawn as a short slash through the line of the register.
         Note that this gate, in contrast to all others, is positioned on the line extending from the last gate or the initialization line of the registers and does not advance the register's horizontal position.
         The style \style{/yquant/operators/every slash} is installed.
         
         \emph{Possible attributes:} none
      
      \subsection[\texorpdfstring{\yquant!subcircuit!}{subcircuit}]{\compat{New in 0.2}\yquant!subcircuit!}\label{gate:subcircuit}
         Syntax: \yquant!subcircuit <target>;! \\
         This is a subcircuit gate which inserts independent quantum circuits at the current position within the circuit.
         It may span multiple registers, but is never split into contiguous slices.
         It allows for controls and may change the type of any target involved, depending on the particular subcircuit.
         The style \style{/yquant/operators/every subcircuit} is installed.
         
         \emph{Possible attributes:}
         \begin{itemize}
            \item \compat{New in 0.4}\yquant!frameless! \\
               This \style{/yquant/operators/subcircuit/frameless} style is activated with this shorthand.
            \item \compat{New in 0.4}\yquant!seamless! \\
               The \style{/yquant/operators/subcircuit/seamless} style is activated with this shorthand (implies \yquant!frameless!).
            \item \yquant!value=<subcircuit>! (required) \\
               Denotes the content of the subcircuit.
               It is specified in the usual syntax of \Yquant.
               Note that, regardless of the outer environment, a subcircuit always implicitly uses the unstarred form, i.e., you must declare every register explicitly before its first usage.
               This is to make sure that the interface of the circuit, i.e., which registers are taken as input and/or output parameters and in which order, is not accidentally mistaken.
               
               The mapping between input and output registers is trivial for single\hyp qubit uses.
               For multi\hyp qubit uses, it works in the following way---in short, it matches in visual order.
               You declare input and output registers by using the appropriate attributes on the \gate{qubit}, \gate{cbit}, \gate{qubits} (or even \gate{nobit}) gates.
               The list of all non\hyp ancillas, from the topmost to the bottom\hyp most, forms the list of parameter registers of the subcircuit.
               This is exactly the number of registers that must be supplied within one multi\hyp qubit target.
               Also within the multi\hyp qubit target, we sort all registers from the topmost to the bottom\hyp most (in the order as they visually appear, not the order in which they are entered).
               Those two lists of equal length are then mapped $1:1$ to each other.
               Intermixing with ancillas is possible at every position and will lead to a vertical shift of the wires, until all registers, inner and outer, can be displayed flawlessly.
               
               As subcircuits follow the same rules as ordinary circuits, it is possible to mix them with arbitrary \TeX{} code, and also to access named gates within the subcircuit---but note that named gates in the outer circuit cannot be accessed (at least unless you play with the \texttt{name prefix} key in \TikZ).
               In order to access inner nodes from the outer circuit, the subcircuit itself must be named; the inner nodes are then prefixed by the name of the subcircuit and a dash.
               
               It is possible to nest subcircuits arbitrarily.
         \end{itemize}
      
      \subsection{\texorpdfstring{\yquant!swap!}{swap}}\label{gate:swap}
         Syntax: \yquant!swap <targets> | <pcontrol> ~ <ncontrol>;! \\
         This is the two\hyp qubit \textsc{swap} gate $\ketbra{00}{00} + \ketbra{01}{10} + \ketbra{10}{01} + \ketbra{11}{11}$ that exchanges two qubits.
         It is denoted by crosses at the affected registers which are connected by a control line.
         It may span multiple registers (in fact, it should always span exactly two registers, though \Yquant{} does not enforce this), and it allows for controls.
         However, refrain from combining \emph{multiple} two\hyp qubit targets \emph{together} with controls.
         The control line will extend from the first to the last of all registers involved in the operation, so that it is impossible to discern visually which registers should actually be swapped.
         Using multiple swaps without controls in one operation is fine, as well as a single controlled swap.
         The style \style{/yquant/operators/every swap} is installed.
         
         \emph{Possible attributes:} none
      
      \subsection{\texorpdfstring{\yquant!x!}{x}}\label{gate:x}
         Syntax: \yquant!x <target> | <pcontrol> ~ <ncontrol>;! \\
         This is a Pauli $\sigma_{\symup x}$ gate $\ketbra01 + \ketbra10$, denoted by a rectangle that contains the letter $X$.
         It may not span multiple registers, but allows for controls. \\
         The style \style{/yquant/operators/every x} is installed.
         
         \emph{Possible attributes:} none
         
      \subsection{\texorpdfstring{\yquant!xx!}{xx}}\label{gate:xx}
         Syntax: \yquant!xx <targets>;! \\
         This is a symmetric flip gate, denoted by joined open squares.
         It should span multiple registers and it allows for controls.
         The same warnings as for the \gate{swap} gate apply.
         The style \style{/yquant/operators/every xx} is installed.
         
         \emph{Possible attributes:} none
      
      \subsection{\texorpdfstring{\yquant!y!}{y}}\label{gate:y}
         Syntax: \yquant!y <target> | <pcontrol> ~ <ncontrol>;! \\
         This is a Pauli $\sigma_{\symup y}$ gate $-\ii\ketbra01 + \ii\ketbra10$, denoted by a rectangle that contains the letter $Y$.
         It may not span multiple registers, but allows for controls. \\
         The style \style{/yquant/operators/every y} is installed.
         
         \emph{Possible attributes:} none
      
      \subsection{\texorpdfstring{\yquant!z!}{z}}\label{gate:z}
         Syntax: \yquant!z <target> | <pcontrol> ~ <ncontrol>;! \\
         This is a Pauli $\sigma_{\symup z}$ gate $\ketbra00 - \ketbra11$, denoted by a rectangle that contains the letter $Z$.
         It may not span multiple registers, but allows for controls. \\
         The style \style{/yquant/operators/every z} is installed.
         
         \emph{Possible attributes:} none
         
      \subsection{\texorpdfstring{\yquant!zz!}{zz}}\label{gate:zz}
         Syntax: \yquant!zz <targets>;! \\
         This is a symmetric phase gate, denoted by joined filled circles.
         It should span multiple registers, but does not allow for controls.
         The same warnings as for the \gate{swap} gate apply.
         The style \style{/yquant/operators/every zz} is installed.
         
         \emph{Possible attributes:} none
         
   
   \section{Examples}\label{sec:examples}
      This section will contain lots of examples.
      On the left\hyp hand side, the output is given, while the code to construct the example is on the right.
      All examples that are provided originate from the examples supplied with \pkg{qasm}, \pkg{qcircuit}, and \pkg{quantikz}.
      We will essentially follow their manuals example\hyp by\hyp example, which gives a nice comparison in how to achieve the given feature using these packages and \Yquant{} instead.
      All examples of course require inclusion of the \Yquant{} package with newest compatibility in the preamble, and some also require \pkg{braket}.
      
      \subsection{\pkg{qasm} documentation}
      \begingroup%
         \yquantset{register/default name=$\ket{\reg_{\idx}}$}%
         The \pkg{qasm} documentation most often names the registers in the way $\ket{\mathrm{register}_{\mathrm{index}}}$.
         This can be achieved by writing
         \begin{minted}{tex}
qubit {$\ket{<name>_{\idx}}$} <name>[<len>];
         \end{minted}
         but if you want to realize this naming scheme for all circuits in your document, it is more convenient to say
         \begin{minted}{tex}
\yquantset{register/default name=$\ket{\reg_{\idx}}$}
         \end{minted}
         in the preamble, as is done here.
         
         \compat{New in 0.3}Note that \Yquant{} also directly supports the \pkg{qasm} syntax, see \cref{sec:foreign:qasm}.
         
         \begin{example}[test1 (create an EPR pair)]
            \begin{codeexample}
\begin{tikzpicture}
   \begin{yquant}
      qubit q[2];
      
      h q[0];
      cnot q[1] | q[0];
   \end{yquant}
\end{tikzpicture}
            \end{codeexample}
         \end{example}
         
         \clearpage
         \begin{example}<Updated in 0.1.1>[test2 (simple teleportation circuit)]
            \begin{codeexample}
\begin{tikzpicture}
   \begin{yquant}
      qubit q[3];
      
      h q[1];
      cnot q[2] | q[1];
      cnot q[1] | q[0];
      h q[0];
      measure q[0-1];
      
      z q[2] | q[1];
      x q[2] | q[0];
   \end{yquant}
\end{tikzpicture}
            \end{codeexample}
         \end{example}
         
         \begin{example}[test3 (swap circuit)]
            \begin{codeexample}
\begin{tikzpicture}
   \begin{yquant}
      qubit q[2];
      
      cnot q[1] | q[0];
      cnot q[0] | q[1];
      cnot q[1] | q[0];
   \end{yquant}
\end{tikzpicture}
            \end{codeexample}
         \end{example}
         
         \clearpage
         \begin{example}<Updated in 0.1.1>[test4 (quantum fourier transform on three qubits)]
            \begin{codeexample}
\begin{tikzpicture}
   \begin{yquant}
      qubit j[3];
      
      h j[0];
      box {$S$} j[0] | j[1];
      box {$T$} j[0] | j[2];
      h j[1];
      box {$S$} j[1] | j[2];
      h j[2];
      swap (j[0, 2]);
   \end{yquant}
\end{tikzpicture}
            \end{codeexample}
         \end{example}
         
         \begin{example}[test5 (demonstrate arbitrary qubit matrix ops)]
            \begin{codeexample}
% \usepackage{amsmath}
\begin{tikzpicture}
   \begin{yquant}
      qubit j[2];
      
      box {$\begin{bmatrix}
         e^{i \alpha} & 0 \\
         0 & e^{-i \alpha}
      \end{bmatrix}$} j[1] | j[0];
      box {$\begin{bmatrix}
         \cos\theta & -\sin\theta \\
         \sin\theta & \cos\theta
      \end{bmatrix}$} j[0];
   \end{yquant}
\end{tikzpicture}
            \end{codeexample}
         \end{example}
            
         \clearpage
         \begin{example}<Updated in 0.1.1>[test6 (demonstrate multiple-qubit controlled single-q-gates)]
            \begin{codeexample}
\begin{tikzpicture}
   \begin{yquant}
      qubit j[4];
      
      cnot j[2] | j[0, 1];
      x j[0];
      box {$U$} j[1] | j[0, 2-3];
      h j[2];
      measure j[3];
   \end{yquant}
\end{tikzpicture}
            \end{codeexample}
         \end{example}
         
         \begin{example}<Updated in 0.1.1>[test7 (measurement of operator with correction)]
            \begin{codeexample}
\begin{tikzpicture}
   \begin{yquant}
      qubit q[2];
      
      h q[0];
      box {$U$} q[1] | q[0];
      h q[0];
      measure q[0];
      box {$V$} q[1] | q[0];
   \end{yquant}
\end{tikzpicture}
            \end{codeexample}
         \end{example}
         
         \clearpage
         \begin{example}[test8 (stage in simplification of quantum teleportation)]
            \begin{codeexample}
\begin{tikzpicture}
   \begin{yquant}
      qubit {$\ket{q_0} = \ket\psi$} q[1];
      qubit {$\ket{q_{\idx}} = \ket0$} q[+2];
      
      h q[1];
      cnot q[1] | q[0];
      cnot q[2] | q[1];
      cnot q[1] | q[0];
      h q[0];
      cnot q[2] | q[1];
      z q[0] | q[2];
   \end{yquant}
\end{tikzpicture}
            \end{codeexample}
            Note that we left out two Hadamards at the end.
            
            Another way to provide various initial values in a single command is by performing case discrimination on \tex!\idx!, for example in the following manner:
            \begin{minted}{tex}
qubit {$\ket{q_{\idx}} = \ifcase\idx\relax \ket\psi \else \ket0 \fi$} q[3];
            \end{minted}
         \end{example}
         
         \begin{example}[test9 (two-qubit gate circuit implementation of Toffoli)]
            \begin{codeexample}
\begin{tikzpicture}
   \begin{yquant}
      qubit q[3];
      
      box {$\sqrt X$} q[2] | q[1];
      cnot q[1] | q[0];
      box {$\sqrt X^\dagger$} q[2] | q[1];
      cnot q[1] | q[0];
      box {$\sqrt X$} q[2] | q[0];
   \end{yquant}
\end{tikzpicture}
            \end{codeexample}
         \end{example}
         
         \clearpage
         \begin{example}<Alternative in 0.1.2\\Updated in 0.1.1>[test10 (multi-qubit gates also demonstrates use of classical bits)]
            \begin{codeexample}
\begin{tikzpicture}
   \begin{yquant}
      qubit {$\ket{q_0}$} q;
      cbit {$c_1$} c;
      qubit {$\ket{q_2}$} q[+1];
      
      h q[0];
      box {$U$} (q[0], c);
      box {$S$} q[1];
      box {$U$} (c, q[1]);
   \end{yquant}
\end{tikzpicture}
            \end{codeexample}
            Instead of a discontiguous vector register, we could also have used three scalar registers.
            The labels chosen for \pkg{qasm} do not fit well to the indices \Yquant{} assigns.
            We might also have used a three\hyp register vector and used the \gate{settype} pseudo\hyp gate to immediately change the second register into a classical one, which would give indices matching the labels---but still, the registers would have a common name, which would make this a very unnatural approach.
         \end{example}
         
         \begin{example}<Updated in 0.1.1>[test11 (user-defined multi-qubit ops)]
            \begin{codeexample}
\begin{tikzpicture}
   \begin{yquant}
      qubit q[3];
      
      h q[0];
      box {$U_{f(x)}$} (q[0, 1]);
      h q[1];
      box {$U_{f(x, y)}$} (q);
   \end{yquant}
\end{tikzpicture}
            \end{codeexample}
            Here we used the fact that a vector register can also be addressed as a whole.
            Instead of \tex!(q)!, we could have also written, e.g., \tex!(q[0]-q[2])! or \tex!(q[0-2])!, or enumerated all sub\hyp registers in a comma\hyp separated list.
         \end{example}
         
         \clearpage
         \begin{example}<Updated in 0.1.1>[test12 (multi-qubit controlled multi-qubit operations)]
            \begin{codeexample}
\begin{tikzpicture}
   \begin{yquant}
      qubit q[3];
      
      h q[0];
      box {$U$} (q[1-2]) | q[0];
      h q[0];
      box {$V$} (q[0-1]) | q[2];
   \end{yquant}
\end{tikzpicture}
            \end{codeexample}
         \end{example}
         
         \begin{example}[test13 (three-qubit phase estimation circuit with QFT and controlled-U)]
            \begin{codeexample*}
\begin{tikzpicture}
   \begin{yquant}
      qubit {$\ket{j_{\idx}} = \ket0$} j[3];
      qubit s[2];
      
      h j;
      box {$U^4$} (s) | j[0];
      box {$U^2$} (s) | j[1];
      box {$U$} (s) | j[2];
      h j[0];
      box {$S$} j[1] | j[0];
      h j[1];
      box {$T$} j[2] | j[0];
      box {$S$} j[2] | j[1];
      h j[2];
      measure j;
   \end{yquant}
\end{tikzpicture}
            \end{codeexample*}
         \end{example}
         
         \begingroup%
            \mdfapptodefinestyle{bdBox}{%
               leftmargin=-.12\linewidth,
               rightmargin=-.12\linewidth
            }%
            \begin{example}[test14 (three-qubit FT QEC circuit with syndrome measurement)]
               \begin{codeexample*}
\begin{tikzpicture}
   \begin{yquant}
      qubit q[3];
      qubit {$\ket{s_{\idx}} = \ket0$} s[2];
      cbit {$c_{\idx} = 0$} c[2];
      
      h s[0];
      cnot s[1] | s[0];
      cnot s[0] | q[0];
      cnot s[1] | q[1];
      cnot s[1] | s[0];
      h s[0];
      measure s;
      cnot c[0] | s[0];
      cnot c[1] | s[1];
      discard s; % to suppress wires extending until re-initialization
      
      init {$\ket0$} s;
      h s[0];
      cnot s[1] | s[0];
      cnot s[0] | q[1];
      cnot s[1] | q[2];
      cnot s[1] | s[0];
      h s[0];
      measure s;
      
      box {Process\\Syndrome} (s, c);
      box {$\symcal R$} (q) | s, c;
   \end{yquant}
\end{tikzpicture}
               \end{codeexample*}
            \end{example}
         \endgroup%
         
         \clearpage
         \begin{example}<Updated in 0.1.1>[test15 (``D-type'' measurement)]
            \begin{codeexample}
\begin{tikzpicture}
   \begin{yquant}
      qubit {$\ket{q_0} = \ket\psi$} q;
      qubit {$\ket{q_1} = \ket+$} q[+1];
      
      zz (q);
      dmeter {$H Z_\theta$} q[0];
   \end{yquant}
\end{tikzpicture}
            \end{codeexample}
         \end{example}
            
         \begin{example}<Updated in 0.1.2, 0.1.1>[test16 (example from Nielsen paper on cluster states)]
            \begin{codeexample}
\begin{tikzpicture}
   \begin{yquant}
      qubit {$\ket{q_{\idx}} = \ket\psi$} q[2];
      qubit {$\ket{q_2} = \ket\phi$} q[+1];
      qubit {$\ket{q_3} = \ket0$} q[+1];
      
      zz (q[1], q[2]);
      align q;
      cnot q[3] | q[2];
      slash q[0];
      dmeter q[3];
      discard q[2];
   \end{yquant}
\end{tikzpicture}
            \end{codeexample}
            We needed to include an \gate{align} pseudo\hyp gate to put the slash at the desired position.
            Usually, this would be sufficient to put the \gate{cnot} and the \gate{slash} gate directly under each other, as it is in the \pkg{qasm} example.
            However, the \gate{slash} gate is special in that it does not need horizontal space and is put with only half of the usual operator separation into the circuit (for this reason, it can be put at the beginning of a wire without creating weird shifts with respect to the ``unslashed'' registers---it is put in the initial line that every wire even without an operation has).
            Hence, you should normally only use the \gate{slash} gate as the very first gate in a circuit.
            It is possible, but not worthwhile to construct the exact same appearance as in the \pkg{qasm} example.
            Note that \gate{discard} currently just drops the wire directly after the last operation.
         \end{example}
         
         \begin{example}<Updated in 0.1.1>[test17 (example from Nielsen paper on cluster states)]
            \begin{codeexample}
\begin{tikzpicture}
   \begin{yquant}
      qubit {$\ket{q_0} = \ket\psi$} q;
      qubit {$\ket{q_{\idx}} = \ket+$} q[+2];
      qubit {$\ket{q_3} = \ket\phi$} q[+1];
      
      zz q[(0-1), (2-3)];
      zz (q[1-2]);
      dmeter {$H$} q[1-2];
   \end{yquant}
\end{tikzpicture}
            \end{codeexample}
            This example shows how the multi\hyp qubit delimiter (the parenthesis) can even be used within indices.
         \end{example}
         
         \begin{example}<Updated in 0.1.2>[test18 (multiple-control bullet op)]
            \begin{codeexample}
\begin{tikzpicture}
   \begin{yquant}
      qubit {$\ket{q_{\idx}} = \ket{\ifcase\idx\relax \psi \or + \or + \or \phi \fi}$} q[4];
      
      zz (q);
      dmeter {$H$} q[1, 2];
   \end{yquant}
\end{tikzpicture}
            \end{codeexample}
            This gate is probably a generalization of \gate{zz}, $\symbb1 - 2 \ketbra{1\dotsm1}{1\dotsm1}$, and indeed since version~0.1.2, we can use \gate{zz} for this purpose.
            This time, we used the case distinction method in the initialization, as already alluded to before.
         \end{example}
      \endgroup
   
      \clearpage
      \subsection{\pkg{qcircuit} documentation}
      \begingroup%
         \yquantset{operator/separation=2mm}
         For a better orientation, we use the same section headings as the \pkg{qcircuit} manual.
         The manual uses unnamed registers a lot; often, we will use the \tex!yquant*! environment to make things more concise.
         As \pkg{qcircuit} uses a bit larger separation between the operators than \Yquant's default, we globally say\\*\tex!\yquantset{operator/separation=2mm}!.
         
         \subsubsection{I. Introduction}
            \begin{example}<Updated in 0.4, 0.1.1>%
               \begin{codeexample*}
\begin{tikzpicture}[baseline=(current bounding box.center)]
   \begin{yquant}[operators/subcircuit/frameless]
      nobit q[3];
      
      subcircuit {
         [out]
         qubit {} q[3];
         box {$U$} q[2] | q[0, 1];
      } (q);
      discard -;
      
      [draw=none]
      box {$=$} (-);
      
      subcircuit {
         [out]
         qubit {} q[3];
         box {$V$} q[2] | q[1];
         cnot q[1] | q[0];
         box {$V^\dagger$} q[2] | q[1];
         cnot q[1] | q[0];
         box {$V$} q[2] | q[0];
      } (q);
      discard -;
   \end{yquant}
\end{tikzpicture}
               \end{codeexample*}
               The best way to realize circuit equalities is with the help of \texttt{frameless} subcircuits.
               The \texttt{[frameless]} attribute can either be specified on each subcircuit individually or, as done here, globally via the corresponding style.
               In this way, we can specify all subcircuits individually.
               All wires that are identical in the circuits must be outer wires; but in fact, we do not want them to be visible outside of the subcircuits.
               Consequently, we initialize them in the outer circuit as \gate{nobit}s, declare them with the \texttt{[out]} attribute, and discard them right after the subcircuit.
               It is important not to discard them within the subcircuit, as the wires would then not extend to the same horizontal final position.
               The equality sign is realized very easily by means of a \gate{box} gate that spans all registers and from which we remove the border.
            \end{example}
      
            \begin{example}
               \begin{codeexample*}
\begin{tikzpicture}
   \begin{yquant}
      qubit {$\ket\psi$} a;
      qubit {$\ket0$} b[2];
      
      h b[1];
      cnot b[0] | b[1];
      cnot b[0] | a;
      h a;
      align a, b;
      measure a;
      measure b[0];
      
      x b[1] | b[0];
      z b[1] | a;
      
      discard a;
      discard b[0];
      output {$\ket\psi$} b[1];
   \end{yquant}
\end{tikzpicture}
               \end{codeexample*}
            \end{example}
            
            \begin{example}<Updated in 0.4>
               \begin{codeexample*}
% \usetikzlibrary{quotes}
\begin{tikzpicture}
   \begin{yquant}
      qubit {} msg[3];
      nobit syndrome[3];
      
      [this subcircuit box style={dashed, "Syndrome Measurement"}]
      subcircuit {
         qubit {} msg[3];
         [out]
         qubit {$\ket0$} syndrome[3];
         
         cnot syndrome[0] | msg[0];
         cnot syndrome[0] | msg[1];
         cnot syndrome[1] | msg[1];
         cnot syndrome[1] | msg[2];
         cnot syndrome[2] | msg[0];
         cnot syndrome[2] | msg[2];
         
         dmeter {$M_{\symbol{\numexpr`a+\idx}}$} syndrome;
      } (msg[-2], syndrome[-2]);
      
      ["Recovery"]
      box {$\mathcal R$} (msg) | syndrome;
      discard syndrome;
   \end{yquant}
\end{tikzpicture}
               \end{codeexample*}
               The example demonstrates how to put a description next to a gate.
               In general, those descriptions should be realized using the \TikZ{} feature \texttt{label}.
               Using the \TikZ{} library \texttt{quotes}, the label is most easily specified.
               Since the label is not part of the valid arguments and also cannot be found in the \texttt{/yquant} path, it is automatically passed to \style{/yquant/operator style}.
               
               To enclose a part of the circuit by a rectangle, we use a subcircuit.
               We define the incoming \gate{qubit}s in the outer circuit, they will have the default attribute \texttt{[inout]}; the syndrome registers, which are created only in the subcircuit enter as \gate{nobit}s and consequently have the \texttt{[out]} attribute.
               It is important to note that both the \texttt{dashed} style as well as the \texttt{label} (here with quoted syntax) are specified only inside \style{/yquant/this subcircuit box style}.
               This ensures that they are not also attached to every single gate in the subcircuit.
               
               Then we see how to apply an operation to multiple registers in parallel while using the \tex!\idx! macro to still give them a different text.
               Since \tex!\idx! gives a numerical index (zero\hyp based), we exploit the ASCII code (actually, this document is compiled in Unicode mode...) to turn this into a letter.
               
               Note that it could have become necessary to pass the \texttt{overlay} attribute to the recovery gate, as it is a multi\hyp register gate with a \texttt{label}, meaning that \Yquant{} cannot reliably distribute its total vertical extent over its constituent registers.
               However, as the $\symcal R$ together with the label in total were not higher than the three\hyp qubit gate would have been anyway, this was not necessary here.
               In generally, don't use \texttt{overlay} unless necessary, as this drawback is expected to be fixed in a future version at least to some degree.
               
               Finally, we will give a similar circuit by using the \TikZ{} interface instead of subcircuits, this time also showing how we can change the shape of the measurement gate to one as in the \pkg{qcircuit} manual:
               
               \begin{codeexample*}
% \usetikzlibrary{fit, quotes}
\begin{tikzpicture}
   \begin{yquant}
      qubit {} msg[3];
      [name=inits]
      qubit {$\ket0$} syndrome[3];
      
      [name=scnot0]
      cnot syndrome[0] | msg[0];
      cnot syndrome[0] | msg[1];
      cnot syndrome[1] | msg[1];
      cnot syndrome[1] | msg[2];
      cnot syndrome[2] | msg[0];
      cnot syndrome[2] | msg[2];
      [name=smeas, shape=yquant-rectangle, rounded corners=.45em]
      dmeter {$M_{\symbol{\numexpr`a+\idx}}$} syndrome;
      ["Recovery"]
      box {$\symcal R$} (msg) | syndrome;
      discard syndrome;
   \end{yquant}
   \node[draw, dashed, fit=(inits-2) (scnot0-p0) (smeas-2), "Syndrome Measurement"] {};
\end{tikzpicture}
               \end{codeexample*}
               We name several elements that visually form the enclosing rectangle; then, we use the \TikZ{} library \texttt{fit} to put a node around them all.
               Any gate can be given a custom shape; here, we use a \texttt{yquant-rectangle}, which is the analogue to \TikZ's \texttt{rectangle} and thus supports the standard \texttt{rounded corners} style.
            \end{example}
            
         \subsubsection{IV. Simple Quantum Circuits}
            \begin{example}
               \begin{codeexample}
\begin{tikzpicture}
   \begin{yquant*}
      x q;
   \end{yquant*}
\end{tikzpicture}
               \end{codeexample}
            \end{example}
            
            \clearpage
            \paragraph{A. Wires and gates}\leavevmode
               \begin{example}
                  \begin{codeexample}
\begin{tikzpicture}
   \begin{yquant*}
      h a;
      align a, b;
      z a;
      x b;
      h a;
   \end{yquant*}
\end{tikzpicture}
                  \end{codeexample}
               \end{example}
               
               \begin{example}<Updated in 0.1.2>
                  \begin{codeexample}
\begin{tikzpicture}
   \begin{yquant*}
      box {$U$} a;
      box {$U^\dagger$} b;
      setstyle {->} -;
   \end{yquant*}
\end{tikzpicture}
                  \end{codeexample}
                  \Yquant{} allows to change wire styles by means of the \gate{setstyle} and \gate{addstyle} pseudo\hyp gates.
                  Here, we use the gate on all wires in order to set an arrow style.
                  Note that arrowheads are actually very special in two respects:
                  \begin{itemize}
                     \item \Yquant{} draws continuous wires for as long as possible.
                        In this example, the wire path extends from the very left to the end of the circuit; \compat{Changed in 0.1.2}\Yquant{} does not draw a wire \emph{to} the gate and then a separate one \emph{from} the gate to the next or the end.
                        The only way to force \Yquant{} to draw multiple wires is to change the wire style or type mid\hyp circuit.
                        For example, by saying \yquant!addstyle {} -;!, all wire paths will be separated at the current position, which \emph{in theory} allows to draw arrowsheads on intermediate wires.
                     \item \emph{In practice}, this will not work due to the clipping commands that \Yquant{} installs.
                        Every wire extends from the center of the left to the center of the right gate, and the gate's shape acts as a clipping path.
                        Consequently, though the arrowhead is drawn, it is actually drawn at the center of the gate instead of the \texttt{west} anchor and then clipped away (unless the gate is small, in which case you might still see some fragments of the arrowhead).
                  \end{itemize}
                  Thus, it is currently not possible to use arrowheads on intermediate wires.
                  If you really need to do this (say, for only a single gate), you may experiment with the \TikZ{} \texttt{shorten} keys, which allow you to manually reduce the length of the wire, but the amount of reduction must be hand\hyp computed for every gate.
                  If you need this more often, consider filing a feature request.
               \end{example}
               
            \paragraph{B. CNOT and other controlled single qubits gates}\leavevmode
               \begin{example}
                  \begin{codeexample}
\begin{tikzpicture}
   \begin{yquant*}
      cnot a[1] | a[0];
      cnot a[0] | a[1];
   \end{yquant*}
\end{tikzpicture}
                  \end{codeexample}
               \end{example}
               
               \begin{example}<Updated in 0.1.1>
                  \begin{codeexample}
\begin{tikzpicture}
   \begin{yquant*}
      cnot q[2] | q[0];
      cnot q[0] | q[1-3];
      box {$U$} q[0] | q[2];
   \end{yquant*}
\end{tikzpicture}
                  \end{codeexample}
               \end{example}
               
            \paragraph{C. Vertical wires}\leavevmode
               \begin{example}<Updated in 0.1.1>
                  \begin{codeexample}
\begin{tikzpicture}
   \begin{yquant*}
      box {$U_{\protect\the\numexpr\idx+1}$} q[0, 2] | q[1];
   \end{yquant*}
\end{tikzpicture}
                  \end{codeexample}
                  There is no direct support for this construction, but as with the initialization of a vector registers, \Yquant{} allows to access the macro \tex!\idx! within an operator \texttt{value}.
                  This macro follows the same rules as the name suffix, i.e., it assigns indices (zero\hyp based) to the target registers in top\hyp to\hyp bottom order, regardless of which order was specified in the target list.
                  Since we instead want a one\hyp based subscript, we need to add one.
                  Note that if you want to output \tex!\idx! directly or within an unexpandable expression, you don't need to take any action.
                  However, here, \tex!\the! is expandable; and since \Yquant{} needs to process all its output twice (first in order to determine the vertical spacing, second to actually typeset), you must manually take care that the command is \emph{not} expanded prematurely by inserting \tex!\protect!.
                  Had you not done this, the subscript would have been ``$1$'' for both operators.
                  Note this is not the case if this macro is used upon creation of a register (as is evident by the fact that the previous examples that used \tex!\ifcase! within the value did not need to say \tex!\protect\ifcase ... \protect\or ... \protect\fi!), this is internally handled quite differently.
               \end{example}
               
            \paragraph{D. Labeling input and output states}\leavevmode
               \begin{example}
                  \begin{codeexample}
\begin{tikzpicture}
   \begin{yquant*}
      qubit {$\ket1$} q[2];
      cnot q[0] | q[1];
      output {$\ket\idx$} q;
   \end{yquant*}
\end{tikzpicture}
                  \end{codeexample}
               \end{example}
               
               \begin{example}<Updated in 0.4, 0.1.1>
                  \begin{codeexample}
\begin{tikzpicture}
   \begin{yquant*}
      {
         \yquantset{every multi label/.style={every node/.style={anchor=east, midway}}}
         init {$\ket{0^k}$} (a[-1]);
      }
      init {$\ket\psi$} (b[-1]);
      qubit {\ifcase\idx\relax$A$\or$B$\fi} c[2];
      [every multi label/.style={every node/.style={anchor=east, midway, shift={(-.3, 0)}}}]
      init {$\ket\psi$} (c);
   \end{yquant*}
\end{tikzpicture}
                  \end{codeexample}
                  Here, three different styles for the initialization of multi\hyp qubit labels are used.
                  The second one (using a curly brace) corresponds to the default.
                  It is overwritten for the first qubit, and to make this modification local, this is done in a group.
                  The third qubit pair uses an overall label and additionally individual labels on the lines.
                  This is achieved by some trickery: the individual labels are given as initialization labels on the register; the global label is given as an \gate{init} multi\hyp qubit gate.
               \end{example}
               
         \subsubsection{V. More Complicated Circuits: Multiple Qubit gates and Beyond}
            \paragraph{A. Multiple qubit gates}\leavevmode
               \begin{example}<Updated in 0.1.1>
                  \begin{codeexample}
\begin{tikzpicture}
   \begin{yquant*}
      box {$U^\dagger$} (a[-2]);
   \end{yquant*}
\end{tikzpicture}
                  \end{codeexample}
               \end{example}
               
               \begin{example}<Updated in 0.4>
                  \begin{codeexample}
\yquantdefinebox{dots}[inner sep=0pt]{$\dots$}
\begin{tikzpicture}
   \begin{yquant}
      qubit {} a;
      cbit {} b;
      [register/minimum height=0pt, register/minimum depth=0pt]
      nobit ellipsis;
      qubit {} c;
      
      dots ellipsis;
      box {$U$} (a, b, ellipsis, c);
      dots ellipsis;
   \end{yquant}
\end{tikzpicture}
                  \end{codeexample}
                  This demonstrates how a register of type \gate{nobit} might even be useful if the register is never used and no subcircuits are involved.
                  Note how we overwrite the default minimum height and depth setting for this register only.
                  Additionally, we for the first time define our own gate, which we call \texttt{dots}.
                  As we define our own style, it does not inherit from \style{/yquant/operators/every box}; hence, we only need to overwrite the \texttt{inner sep} coming from \TikZ's defaults.
               \end{example}
               
               \clearpage
               \begin{example}<Updated in 0.1.2>
                  \begin{codeexample}
\begin{tikzpicture}
   \begin{yquant*}
      box {$\symcal F$} (a[-1]);
      box {$\symcal G$} (a[0, 2]);
   \end{yquant*}
\end{tikzpicture}
                  \end{codeexample}
                  This demonstrates \Yquant's capabilities of discontiguous multi\hyp qubit gates.
                  \Yquant{} automatically splits multi\hyp qubit gates into slices of directly adjacent wires (which, for the $\symcal G$ case, are the single wires \yquant!a[0]! and as second slice \yquant!a[2]!).
               \end{example}
               
            \paragraph{B. Measurements and classical bits}\leavevmode
               \begin{example}
                  \begin{codeexample}
\begin{tikzpicture}
   \begin{yquant*}
      measure a;
      dmeter {$\chi$} b;
      measure {$\ket{\xi_\pm}$} c;
   \end{yquant*}
\end{tikzpicture}
                  \end{codeexample}
                  The ``tab'' and ``measure'' type are not supported yet.
                  Extracting a meter symbol on its own will not be supported.
                  If you are interested in the code, have a look at \texttt{yquant-shapes.tex} and search for the \texttt{yquant-measure} shape.
               \end{example}
               
               \begin{example}<Updated in 0.4, 0.1.2>
                  \begin{codeexample}
\begin{tikzpicture}
   \begin{yquant*}
      [shape=yquant-rectangle, rounded corners=.45em]
      measure {Codebit} a;
      box {$\chi$} b | a;
      discard a;
      measure b;
   \end{yquant*}
\end{tikzpicture}
                  \end{codeexample}
                  Rectangles with rounded corners are not a specific style, but since the \texttt{yquant-rectangle} shape internally uses \tex!\pgfpathrectangle!, the ordinary \TikZ{} option can be used (also, an inset specification can control how much the corner is rounded).
               \end{example}
               
               \clearpage
               \begin{example}
                  \begin{codeexample}
\begin{tikzpicture}
   \begin{yquant}
      qubit {$\ket\psi$} q;
      
      measure {$\ket{\xi_\pm}$} q;
   \end{yquant}
\end{tikzpicture}
                  \end{codeexample}
               \end{example}
               
               \begin{example}
                  \begin{codeexample}
\begin{tikzpicture}
   \begin{yquant*}
      dmeter {Bell} (a[0, 1]);
      discard a;
   \end{yquant*}
\end{tikzpicture}
                  \end{codeexample}
               \end{example}
               
               \begin{example}<Updated in 0.1.2, 0.1.1>
                  \begin{codeexample}
\begin{tikzpicture}
   \begin{yquant*}
      measure {$\ket{\xi_\mp}$} (a[-1, 3]);
   \end{yquant*}
\end{tikzpicture}
                  \end{codeexample}
                  Multi\hyp qubit gates (including measurements) on non\hyp adjacent registers are properly supported.
                  As explained in \cref{sec:registers}, there is one main and multiple subordinate gate in such a discontiguous multi\hyp qubit operation (though at the moment, the \gate{measure} gates with text is the only gate that makes this distinction).
                  In our case, the main part contains the measurement symbol and the text, while the subordinate gates only contain the text.
                  By default, \Yquant{} uses the first slice as main part, but you may influence this by preceding what you want to be ``main'' by a star:
                  
                  \begin{codeexample}
\begin{tikzpicture}
   \begin{yquant*}
      measure {$\ket{\xi_\mp}$} (a[-1, *3]);
   \end{yquant*}
\end{tikzpicture}
                  \end{codeexample}
               \end{example}
               
            \clearpage
            \paragraph{C. Non-gate inserts, forcing space, and swap}\leavevmode
               \begin{example}<Updated in 0.1.1>
                  \begin{codeexample}
\begin{tikzpicture}
   \begin{yquant*}
      [name=sw]
      swap (a[0-1]);
      [draw=none]
      box {$X$} a[1];
      slash b;
      box {$H^{\otimes n}$} b;
      \node[anchor=199] at (sw-0.north) {Defective Circuit};
   \end{yquant*}
\end{tikzpicture}
                  \end{codeexample}
                  Here, the intermediate text was inserted by using a \gate{box} without drawing.
                  Another way would be to use an \gate{init} command, although this is semantically wrong (probably).
               \end{example}
               
               \begin{example}
                  \begin{codeexample}
\begin{tikzpicture}
   \begin{yquant*}
      qubit {} a;
      [name=ypos]
      qubit {} b[3];
      
      cnot a | b[0];
      [name=left]
      cnot a | b[1];
      hspace {7mm} -;
      [name=right]
      cnot a | b[2];
   \end{yquant*}
   \path (left |- ypos-0) -- (right |- ypos-1) node[midway] {$\dots$};
\end{tikzpicture}
                  \end{codeexample}
                  Note how the register range \texttt{-} was used to denote all registers.
                  We positioned the dots by first naming the relevant registers, so that the vertical position is at the coordinates \texttt{ypos-0} and \texttt{ypos-1}; and then, we also named the \gate{cnot} gates, so that we are able to discern the horizontal position.
               \end{example}
               
            \clearpage
            \paragraph{D. Barriers}\leavevmode
               \begin{example}<Updated in 0.1.1>
                  \begin{codeexample}
\begin{tikzpicture}
   \begin{yquant*}
      x a[0, 1];
      barrier (a);
   \end{yquant*}
\end{tikzpicture}
                  \end{codeexample}
               \end{example}
               Now the \pkg{qcircuit} manual lists three circuits with barriers at different positions.
               They cannot be drawn with \Yquant; however, since neither of them is a valid circuit (no indication whether the control is positive or negative), this is of no concern.
               
            \paragraph{E. How to control anything}\leavevmode
               \begin{example}<Updated in 0.1.1>
                  \begin{codeexample}
\begin{tikzpicture}
   \begin{yquant*}
      zz (a[0, 2]);
      cnot a[1] ~ a[0];
      zz (a[2, 3]);
      h a[3] | a[0] ~ a[1];
      measure a[2, 3];
      box {$U$} (a[0, 1]) | a[3] ~ a[2];
      discard a[2, 3];
   \end{yquant*}
\end{tikzpicture}
                  \end{codeexample}
                  Note that it is not possible to draw a control to measurement (the measurement operations are explicitly defined not to accept controls): Either the measurement is performed or not (which transforms the register type), but a measurement conditioned on a quantum state is not possible.
                  In principle, one could think of a measurement conditioned on a classical register (in which case the register type cannot change, as maybe the state stays quantum; the measurement operation then is similar to a complete dephasing).
                  If there is need for this, please file a feature request.
                  But note the relatively common (though unsupported by \Yquant{} at the moment) use of control lines that directly go from the measurement operator to the controlled operation; in this case, the opposite direction would be meant by using the same notation.
               \end{example}
               
         \subsubsection{VI. Bells and Whistles: Tweaking Your Diagram to Perfection}
            For options how to configure the circuits, refer to \cref{sec:config}.
            
            \clearpage
            \paragraph{A. Spacing}\leavevmode
               \begin{example}<Updated in 0.1.1>
                  \begin{codeexample}
\begin{tikzpicture}
   \begin{yquant*}
      swap (a[0, 1]);
      box {$T^\dagger$} a[2] | a[1];
      swap (a[0, 1]);
   \end{yquant*}
\end{tikzpicture}
                  \end{codeexample}
               \end{example}
               
            \paragraph{B. Labeling}\leavevmode
               \begin{example}
                  \begin{codeexample}
\begin{tikzpicture}
   \begin{yquant}[every initial label/.style={anchor=south east}]
      qubit {\rlap{\hskip2mm $a$}} a;
      qubit {\rlap{\hskip2mm $b$}} b;
      hspace {5mm} -;
      
      measure b;
      align -;
      measure a;
      output {$A$} a;
      output {$B$} b;
   \end{yquant}
\end{tikzpicture}
                  \end{codeexample}
                  Measurement with vertical output are not supported (yet).
                  Repositioning the initial labels needs some care and manual fine\hyp tuning.
               \end{example}
               
            \clearpage
            \paragraph{C. Grouping}\leavevmode
               \begin{example}<Updated in 0.4>
                  \begin{codeexample*}
% \usetikzlibrary{fit}
\begin{tikzpicture}
   \begin{yquant*}[register/separation=3mm]
      cnot a[2] | a[0];
      cnot a[2] | a[1];
      [name=left]
      h a[0, 1];
      z a[2];
      cnot a[1] | a[0];
      [name=righttop]
      h a[0];
      [name=rightbot]
      h a[1] | a[2];
      
      hspace {2mm} -;
   \end{yquant*}
   \node[draw, dashed, fit=(left-0) (left-1) (righttop) (rightbot-0)] {};
\end{tikzpicture}
                  \end{codeexample*}
                  Note that \tex!\begin{yquant*}! must not be followed by a line break (unless masked by \tex!%!) if options follow.
                  Also note that here, we cannot make use of a subcircuit due to the very last control, which would then control an inner gate of said subcircuit---but they are not exposed.
               \end{example}
      \endgroup
      
      \clearpage
      \subsection{\pkg{quantikz} documentation}
      \begingroup
         \yquantset{operator/separation=4mm}%
         Again, our section headings will be the same as in the \pkg{quantikz} manual.
         And since \pkg{quantikz} also has even more space between the gates, we globally say \tex!\yquantset{operator/separation=4mm}!.
         
         \subsubsection{II. A single wire}
            \begin{example}
               \begin{codeexample*}
\begin{tikzpicture}[label position=north east, every label/.style={inner sep=1pt}]
   \begin{yquant}
      qubit {$\ket0$} a;
      
      phase {$\alpha$} a;
      h a;
      phase {$\beta$} a;
      h a;
      phase {$\gamma$} a;
      
      [every output/.append style={align=center}]
      output {Arbitrary\\pure state} a;
   \end{yquant}
\end{tikzpicture}
               \end{codeexample*}
               The captions of \gate{phase} commands are internally implemented using \TikZ{} \texttt{label}s.
               At the moment, it is not possible to change any label options on a gate-type basis, only locally or fully globally (\TikZ{} feature request \href{https://github.com/pgf-tikz/pgf/issues/811}{\#811}).
            \end{example}
            
            \clearpage
            \paragraph{A. Measurements}\leavevmode
               \begin{example}
                  \begin{codeexample}
% \usetikzlibrary{quotes}
\begin{tikzpicture}
   \begin{yquant*}
      ["$0$"]
      measure a;
      discard a;
      
      init {} a;
      dmeter {$1$} a;
      discard a;
   \end{yquant*}
\end{tikzpicture}
                  \end{codeexample}
                  Other measurement shapes are not supported at the moment.
               \end{example}
            
            \paragraph{B. Wires and arrows}\leavevmode
               \begin{example}<Updated in 0.1.2>
                  \begin{codeexample*}
% \usetikzlibrary{quotes}
\begin{tikzpicture}
   \begin{yquant}[operator/separation=1cm, every label/.append style={align=center}]
      qubit {$\ket0$\\initial state} a;
      
      [draw=none]
      box {$X$} a;
      
      ["$0$/$1$", type=qubit]
      measure a;
      
      addstyle {->} a;
      output {$\ket1$} a;
      \yquantset{operator/separation=5mm}
   \end{yquant}
\end{tikzpicture}
                  \end{codeexample*}
                  This example demonstrates how to instruct the \gate{measure} gate to use a different output type than the standard \gate{cbit}.
                  
                  In general, any macros that are used within a \TikZ{} path or a \Yquant{} operation must not be fragile, or must be preceded with \tex!\protect!.
                  In this example, \tex!\\! is a robust command (at least in newer kernels), so protection is not required.
                  Since it may occur quite frequently that \Yquant{} is used within a \tex!center! environment or in \tex!\centering! mode (in which \tex!\\! is still fragile), \Yquant{} takes care of this (it actually robustifies \tex!\@centercr!, which is the meaning of \tex!\\! in these surroundings).
                  
                  In order to change the style of an individual wire, we use \gate{addstyle}.
                  To make the final line shorter, we change the operator separation by issuing \tex!\yquantset! at the end.
               \end{example}
               
               \begin{example}<New in 0.4>
                  \begin{codeexample}
\begin{tikzpicture}
   \begin{yquant}
      qubit {$\ket0$} q;
      [name=h]
      h q;
      discard q;
      \path[/yquant/every wire, /yquant/every qubit wire, -Stealth] (h) -| ++(1cm, -.5cm) node[below] {trash};
   \end{yquant}
\end{tikzpicture}
                  \end{codeexample}
                  Here, we use an ordinary \tex!\path! command to reproduce the ``trash'' line.
                  This time, we chose to use the appropriate styles as \Yquant{} itself would do it instead of just saying \tex!\draw! without the options, which would also have worked.
               \end{example}
               
               \begin{example}<Updated in 0.4>
                  \begin{codeexample}
% \usetikzlibrary{quotes}
\begin{tikzpicture}
   \begin{yquant*}
      qubit {$\ket0^{\otimes n}$} a;
      ["north east:3" {font=\protect\footnotesize, inner sep=0pt}]
      slash a;
      h a;
   \end{yquant*}
\end{tikzpicture}
                  \end{codeexample}
                  Again, you see an example of how some commands need to be \tex!\protect!ed when used in \Yquant{} options, and that you can indeed exploit all features of the \texttt{quotes} library.
               \end{example}
               
               \begin{example}
                  \begin{codeexample}
\begin{tikzpicture}
   \begin{yquant}
      qubits {$\ket0^{\otimes n}$} a;
      h a;
   \end{yquant}
\end{tikzpicture}
                  \end{codeexample}
               \end{example}
            
         \subsubsection{III. Multiple Qubits}
            \begin{example}
               \begin{codeexample*}
\begin{tikzpicture}
   \begin{yquant}
      qubit {$\ket0$} a;
      qubit {$\ket0$} b;
      
      h a, b;
      cnot b | a;
      box {$U$} a ~ b;
      zz (a, b);
      
      [after=a]
      qubit {} c;
      
      swap (a, c);
      box {$U$} c | a ~ b;
   \end{yquant}
\end{tikzpicture}
               \end{codeexample*}
               This example demonstrates the use of the \texttt{after} argument that instructs the register creation to begin the register only after the current position of another register that already exists.
               Note that this argument will always make the wire begin \emph{at the right end} of the last gate of the referenced register; however, if---as is the case here---this gate is shorter than \style{/yquant/operator/minimum width}, this might not coincide with the visual right end.
            \end{example}
            
            \clearpage
            \begin{example}<Updated in 0.1.2, 0.1.1>
               \begin{codeexample*}
\begin{tikzpicture}
   \begin{yquant*}
      [name=c]
      cnot a[0, 2] | a[1];
      [name=m]
      measure a[1];
      discard a[2];
      \path[/yquant/every wire, /yquant/every qubit wire] (c-1) -- (m.center |- c-1);
      box {$U$} a[0] | a[1];
      discard a[1];
   \end{yquant*}
\end{tikzpicture}
               \end{codeexample*}
               It is not possible for the double control line to directly exit the measurement gate.
               Also, if you discard a gate, this will prevent it from exiting from its last gate.
               \Yquant{} will not allow you (apart from manual drawing) to extend the wire to some arbitrary position, then drop it.
               But of course, as done here, you can always resort to the full power of \TikZ.
               Still, this is inferior to a wire drawn by \Yquant, as it does not use clippings: the connection with the \gate{cnot} gate may not be accurate; in particular, if the wire is of a different color or if you need to draw classical or bundle wires, the connection will become unpleasant.
            \end{example}
            
            \begin{example}<Updated in 0.1.1>
               \begin{codeexample}
\begin{tikzpicture}
   \begin{yquant*}
      cnot a[1, 2] | a[0];
      measure a[1], a[2];
      box {$U$} a[0] | a[1];
      box {$V$} a[0] | a[2];
      discard a[1]-;
   \end{yquant*}
\end{tikzpicture}
               \end{codeexample}
            \end{example}
            
            \Yquant{} doesn't offer anything comparable to the new \tex!\ctrlbundle! command; and as the bundle lines are spaced much more tightly in \Yquant, this would not really make sense.
            
         \clearpage
         \subsubsection{IV. Operating on many Qubits}
            \begin{example}<Updated in 0.1.1>
               \begin{codeexample}
\begin{tikzpicture}
   \begin{yquant*}
      init {$\ket0^{\otimes3}$} (a[-2]);
      
      box {$U$} (a[1-2]) | a[0];
      measure a[0];
      discard a[0];
      output {$\ket{11}$} (a[1-2]);
   \end{yquant*}
\end{tikzpicture}
               \end{codeexample}
               Multi\hyp qubits inputs are possible using the \gate{init} command.
               The text assigned to a register declaration is always for an individual register.
            \end{example}
            
            \begin{example}<Updated in 0.4>
               \begin{codeexample}
\begin{tikzpicture}
   \begin{yquant*}[register/minimum height=6mm, register/minimum depth=6mm]
      hspace {7.5mm} a;
      h a;
      hspace {7.5mm} a;
      [x radius=1cm]
      box {$U$} (b, c);
      cnot b | a;
      measure a;
      discard a;
   \end{yquant*}
\end{tikzpicture}
               \end{codeexample}
               \Yquant{} does not use a grid layout: operators are stacked next to each other.
               Therefore, there is no automatic centering of a column, though it could be emulated using hand-crafted \gate{hspace} commands, as was done here (the Hadamard gate uses the \style{/yquant/operator/minimum width}, which is \tex!5mm!, while the large box has a width of \tex!2cm!, so that we need two \tex!7.5mm! spacings at the end, as the \gate{hspace} pseudo\hyp gate only inserts exactly the space you give, but not additional [twice] \style{/yquant/operator/separation}, as would be the case for a hypothetical zero\hyp width gate).
               In fact, we don't even need the second \gate{hspace}, since the two\hyp qubit \gate{cnot} will automatically enforce correct alignment.
            \end{example}
            
            \begin{example}
               \begin{codeexample*}
\begin{tikzpicture}[braced/.style={decoration={brace, #1, pre=moveto, pre length=-1pt, post=moveto, post length=-1mm}, decorate}, inner/.style={font=\footnotesize}]
   \begin{yquant}[register/separation=3mm]
      [name=a]
      qubit {} a[4];
      [x radius=1cm, name=u]
      box {$U$} (a[1-3]) | a[0];
   \end{yquant}
   \draw[braced]
      ([xshift=2pt] a-1 -| u.west) -- ([xshift=2pt] a-2 -| u.west)
      node[inner, anchor=west, xshift=1pt, pos=.55] {$x$};
   \draw[braced=mirror]
      ([xshift=-2pt] a-1 -| u.east) -- ([xshift=-2pt] a-2 -| u.east)
      node[inner, anchor=east, xshift=-1pt, pos=.55] {$x$};
   \node[inner, anchor=west] at (a-3 -| u.west) {\smash{$y$}};
   \node[inner, anchor=east] at (a-3 -| u.east) {\smash{$y \oplus f(x)$}};
\end{tikzpicture}
               \end{codeexample*}
               There is no simple way to draw \emph{within} a gate, unless this gate is a proper quantum circuit itself.
               Instead, here the intricate parts were reproduced using \TikZ: first, we make sure we assign a name to every relevant coordinate.
               Then we use some \TikZ{} styles to draw the braces and nodes at the intersection of these coordinates.
               Here, we also make use of the \texttt{moveto} decoration transformation that comes with \Yquant{} and that allows to enlarge the braces slightly for a good overall appearance.
               Finally, as $y$ has a much smaller height than $y \oplus f(x)$, we make sure this does not affect the vertical positioning; and also, as $x$ has no ascender, we need to slightly position it off\hyp mid for a good look.
            \end{example}
            
            \clearpage
            \begin{example}
               \begin{codeexample}
\begin{tikzpicture}
   \begin{yquant*}
      box {$\sqrt Z$} a;
      box {$X$} b;
      box {$\sqrt{\mathrm{SWAP}}$} (a, b);
      measure a;
   \end{yquant*}
\end{tikzpicture}
               \end{codeexample}
               This time, we did not artificially discard the lines.
            \end{example}
            
            \begin{example}
               \begin{codeexample}
\begin{tikzpicture}
   \begin{yquant}
      qubit {$c_{\idx}$} c[4];
      box {M\\A\\J} (c[-2]);
      box {M\\A\\J} (c[1-]);
   \end{yquant}
\end{tikzpicture}
               \end{codeexample}
            \end{example}
            
            \paragraph{A. Different connections}\leavevmode
               \begin{example}
                  \begin{codeexample}
\begin{tikzpicture}
   \begin{yquant}[register/default name=]
      qubit a;
      cbit b;
      nobit c;
      cbit d;
      qubit e;
      box {$U$} (-);
      settype {qubit} c;
   \end{yquant}
\end{tikzpicture}
                  \end{codeexample}
                  This example for the first time demonstrates the declaration of a non\hyp existing register and the \gate{settype} pseudo\hyp gate that acts as a zero\hyp width, no\hyp content \gate{init} gate.
               \end{example}
            
         \clearpage
         \subsubsection{V. Slicing}
            \begin{example}
               \begin{codeexample}
\begin{tikzpicture}
   \begin{yquant}
      qubit {} a[3];
      h a[0];
      [red, thick, label=step]
      barrier (a);
      cnot a[1] | a[0];
      measure a[0];
      discard a[0];
      cnot a[2] | a[1];
      h a[1];
   \end{yquant}
\end{tikzpicture}
               \end{codeexample}
               There is nothing like a \texttt{slice all} keyword, as \Yquant's underlying layout is not grid\hyp based.
               Changing the style of slice captions simply means providing label options.
               This time, we used the \texttt{label} key instead of the shorter syntax provided by the \texttt{quotes} library, which is of course also possible.
            \end{example}
            
            \begin{example}
               \begin{codeexample*}
% \usetikzlibrary{quotes}
\begin{tikzpicture}[every label/.style={rotate=40, anchor=south west}]
   \begin{yquant}[operators/every barrier/.append style={blue, thick}]
      qubit {} a[3];
      h a[0];
      ["slice 1"]
      barrier (-);
      cnot a[1] | a[0];
      ["slice 2"]
      barrier (-);
      measure a[0];
      discard a[0];
      cnot a[2] | a[1];
      ["slice 3"]
      barrier (-);
      h a[1];
   \end{yquant}
\end{tikzpicture}
               \end{codeexample*}
            \end{example}
            
            \begin{example}
               \begin{codeexample*}
% \usetikzlibrary{quotes}
\begin{tikzpicture}[label distance=4mm]
   \begin{yquant}[operators/every barrier/.append style={red, thick, shorten <= -2mm, shorten >= -2mm}]
      qubit {$\ket0$} a;
      h a;
      ["1"]
      barrier (a);
      phase {[label distance=0pt]$\varphi$} a;
      ["2"]
      barrier (a);
      h a;
      ["3"]
      barrier (a);
      output {$\cos\frac\varphi2 \ket0 - i\sin\frac\varphi2 \ket1$};
   \end{yquant}
\end{tikzpicture}
               \end{codeexample*}
               Usually, the \texttt{shorten} keys do not have any effect on \Yquant{} operations, since the latter are all made up of nodes.
               However, the \texttt{yquant-line} shape explicitly takes care of correctly handling them.
               It is the only one that does so.
               Since barriers usually end quite closely to the wires---and the default \texttt{dashed} style may make this worse---the \texttt{shorten}ing may often prove useful.
               Note that if the barriers are enlarged by means of negative \texttt{shorten}ings, this will not affect the bounding box and you must take care of appropriately shifting labels.
               The internal register height calculations might be inconsistent for multi\hyp register barriers with \texttt{shorten}ing: While \Yquant{} takes care of enlarging the registers so that there is enough space for placing the \gate{barrier} with its \emph{original} (single\hyp register) height, it actual height of the registers is only known at the second stage in calculation; but applying \texttt{shorten}ings after this stage would require another iteration of height calculation.
               Hence, multi\hyp register barriers that are enlarged by a lot will probably look bad unless you add manual spacing to the appropriate registers.
               Also note that we used much larger magnitudes in order to achieve a similar appearance as in \pkg{quantikz}.
               To avoid that the large distance also affects the \gate{phase} gate badly, we locally reset the distance; for this, there are two ways.
               The easiest one is to make use of the fact that the value of the \gate{phase} gate is passed directly as \texttt{label} argument, so that we can locally reset the distance.
               The other possibility would be to write
               \begin{minted}[linenos=false, fontsize=\footnotesize, bgcolor=boxGrayBody]{tex}
   {
      \yquantset{/tikz/label distance=0pt}
      phase {$\varphi$} a;
   }
               \end{minted}
               since due to the aforementioned lack of support for a style that sets the options in \TikZ, we must manually use a (grouped) \tex!\yquantset! instruction for this.
               Note that whenever you change a \TikZ{} style in a \Yquant{} environment, use the \tex!\yquantset! macro, \emph{not} \tex!\tikzset! or \tex!\pgfkeys!.
               Not only will the latter two not automatically restart the parser (so that you would have to issue \tex!\yquant! after their use), but \Yquant{} has to process all its content twice in order to properly determine the register height.
               Only \tex!\yquantset! will be properly captured and re\hyp issued at the correct position when the content is actually typeset.
               Had we written \tex!\tikzset{label distance=0pt} \yquant!, no effect at all would have been visible, since this command would only have taken effect in the first (invisible) round when \Yquant{} determines heights.
            \end{example}
            
            \Yquant{} does not provide a mechanism for vertical labels, but you may of course just insert line breaks at appropriate positions (and set the \texttt{align} property of the labels).
         
         \clearpage
         \subsubsection{VI. Spacing}
            \paragraph{A. Local adjustment}\leavevmode
               \begin{example}
                  \begin{codeexample}
\begin{tikzpicture}
   \begin{yquant}[register/default name=]
      [register/minimum height=2cm]
      qubit a;
      qubit {\vbox to 1cm{}} b;
      qubit c;
      
      h a;
      x b-;
      hspace {2cm} -;
      x a;
      z b-;
      h a;
      z b-;
   \end{yquant}
\end{tikzpicture}
                  \end{codeexample}
                  At the moment, the distance between registers is calculated by \Yquant{} automatically.
                  We show various possibilities to intervene in this example.
                  The first is to locally, upon creation of the register, reset \style{/yquant/register/minimum height} to a different value.
                  The second is to artificially enlarge the label that the qubit initializer takes.
                  However, both approaches enlarge the \emph{height} of the registers, i.e., add half of the specified amount to the top and the bottom.
               \end{example}
               
               \begin{example}
                  \begin{codeexample*}
\begin{tikzpicture}
   \begin{yquant*}
      [x radius=1cm]
      x a;
      box {\hbox to 1cm{\hfil$X$\hfil}} a;
      hspace {1cm} a;
      x a;
      discard a;
   \end{yquant*}
\end{tikzpicture}
                  \end{codeexample*}
                  Here, we demonstrate two possibilities to enlarge a box: The first is by specifying its size in terms of the \texttt{x radius} or \texttt{y radius} keys beforehand.
                  Those values serve as minimum sizes and would be extended if the text extended beyond the box.
                  The second option is to just enlarge the text artificially by explicitly putting it into a fixed\hyp width box.
                  Note that in the first case, the \emph{radius} is specified, i.e., the half\hyp width, while in the second case, it is the \emph{total} width (both times modulo the inner separation).
                  Also note that the \style{/yquant/operator/minimum width} style is unsuitable for the given task: it would not change the visual width, only what \Yquant{} assumes its width to be.
               \end{example}
               
            \paragraph{B. Global Adjustment}\leavevmode
               \begin{example}
                  \begin{codeexample*}
\begin{tikzpicture}
   \begin{yquant*}[operator/separation=1cm]
      h a;
      phase {$\beta$} a;
      h a;
   \end{yquant*}
\end{tikzpicture}
                  \end{codeexample*}
               \end{example}
               
               \begin{example}
                  \begin{codeexample}
\begin{tikzpicture}
   \begin{yquant*}[register/minimum height=0pt]
      x a[0, 2];
      zz (a[0, 1]);
      x a[0];
      h b;
   \end{yquant*}
\end{tikzpicture}
                  \end{codeexample}
                  By default, \Yquant{} will use the height that is required by the individual gates, but at least \style{/yquant/register/minimum height} (which defaults to \tex!3mm!).
                  Only manually reducing the default height will produce the cramped spacing displayed here.
               \end{example}
               
            \clearpage
            \paragraph{C. Alignment}\leavevmode
               \begin{example}
                  \begin{codeexample}
\begin{tikzpicture}
   \begin{yquant*}
      x a[0];
      cnot a[1] | a[0];
   \end{yquant*}
\end{tikzpicture}
$=$
\begin{tikzpicture}
   \begin{yquant*}
      cnot a[1] | a[0];
      x a;
   \end{yquant*}
\end{tikzpicture}
                  \end{codeexample}
                  Not specifying anything for the vertical alignment will lead to the common \TikZ{} problem: the baseline will be at the bottom, which is particularly bad in this case due to the missing $X$ gate.
                  The \style{/yquant/register/minimum height} key does not help here, since it only affects \Yquant's internal handling, but not the bounding box calculated by \TikZ.
                  In the first example of the \pkg{qcircuit} documentation, we demonstrated how the desired task can easily be achieved in terms of \texttt{baseline}s.
                  We will now do the same with scopes instead.\vskip2mm 
                  \begin{codeexample*}
\begin{tikzpicture}[/yquant/register/minimum height=5mm]
   \begin{yquant*}
      x a[0];
      cnot a[1] | a[0];
   \end{yquant*}
   \path (current bounding box.east |- 0, 0) ++(1, 0) coordinate (shift);
   \begin{scope}[shift=(shift)]
      \begin{yquant*}
         cnot a[1] | a[0];
         x a;
      \end{yquant*}
   \end{scope}
   \node at (current bounding box) {$=$};
\end{tikzpicture}
                  \end{codeexample*}
                  Here, we increased the minimum height so that in the left circuit despite the absence of the $X$ gate, the second register has the same separation.
                  We used the meta\hyp node \texttt{current bounding box} to avoid the need to manually specify hard\hyp coded positions.
               \end{example}
               
               \subparagraph{1. Perfecting Vertical Alignment}\leavevmode
                  \begin{example}
                     \begin{codeexample*}
\begin{tikzpicture}[baseline=(W)]
   \begin{yquant}
      qubit {$\ket0$} anc;
      [name=W]
      qubit {} x;
      h x;
      cnot x | anc;
   \end{yquant}
\end{tikzpicture} $\equiv$ \begin{tikzpicture}[baseline=(W)]
   \begin{yquant}
      [name=W]
      qubit {} x;
      h x;
      x x;
   \end{yquant}
\end{tikzpicture}
                     \end{codeexample*}
                     While here, we opted for the most logical choice to name the declaration of the register, a name put to any operation on the desired register would also serve the purpose (as long as this operation is vertically symmetrical, which subcircuits may not be).
                  \end{example}
                  
                  \clearpage
                  \begin{example}
                     \begin{codeexample*}
\begin{tikzpicture}[baseline={([yshift=.2cm]W)}]
   \begin{yquant}
      qubit {$\ket0$} anc;
      [name=W]
      qubit {} x;
      h x;
      cnot x | anc;
   \end{yquant}
\end{tikzpicture} $\equiv$
\begin{tikzpicture}[baseline={([yshift=.2cm]new)}]
   \begin{yquant}
      [name=new]
      qubit {} x;
      h x;
      x x;
   \end{yquant}
\end{tikzpicture}
                     \end{codeexample*}
                     Of course, you may also use the features of the \TikZ{} library \texttt{calc} to achieve the same shift.
                  \end{example}
            
            \paragraph{D. Scaling}\leavevmode
               \begin{example}
                  \begin{codeexample*}
\begin{tikzpicture}[scale=1.5, every label/.append style={scale=1.5}]
   \begin{yquant*}
      h a;
      phase {$\beta$} a;
      h a;
   \end{yquant*}
\end{tikzpicture}
                  \end{codeexample*}
                  Here, we first scaled the circuit itself.
                  The default style for \style{/yquant/every circuit} sets the \texttt{transform shape} key for every node (which means any gate), so that those are also scaled.
                  This is explicitly undone for \texttt{label}s due to \TikZ{} bug \href{https://github.com/pgf-tikz/pgf/issues/843}{\#843}, so the scaling must be specified explicitly for the \texttt{label}s.
                  % TODO Note that unless you use a very recent version of \TikZ{} in which bug \href{https://github.com/pgf-tikz/pgf/issues/843}{\#843} was fixed, you might need to add \tex!every label/.prefix style={transform shape=false}! to this style, which undoes the effect for \texttt{label}s.
               \end{example}
         
         \subsubsection{VII. Typesetting}
            \paragraph{A. Global Styling}\leavevmode
               \begin{example}
                  \begin{codeexample*}
% \usetikzlibrary{quotes}
\begin{tikzpicture}
   \begin{yquant*}[operators/every h/.append style={fill=red!20}]
      h a;
      phase {$\beta$} a;
      h a;
      ["$\ket\pm$" above right]
      measure a;
   \end{yquant*}
\end{tikzpicture}
                  \end{codeexample*}
                  Instead of setting \style{/yquant/operators/every h}, we could also have changed \style{/yquant/operators/every box}.
                  Had we used \style{/yquant/every operator}, then the measurement would also have changed.
                  Again, due to a \TikZ{} limitation, it is not possible to change the position of labels on a per\hyp style basis, only by using \texttt{label} options or a global setting.
               \end{example}
               
               \begin{example}
                  \begin{codeexample*}
\begin{tikzpicture}[thick]
   \begin{yquant*}[every operator/.prefix style={fill=white}]
      cnot a[1] | a[0];
      box {$R_z(-\theta\fracslash2)$} a[1];
      cnot a[1] | a[0];
      box {$R_z(\theta\fracslash2)$} a[1];
      measure a[1];
   \end{yquant*}
\end{tikzpicture}
                  \end{codeexample*}
                  As the ``thin'' style is the default, we present the opposite.
                  By default, all operators are transparent; we changed this by giving all of them a white background color (but as a style \emph{prefix}, so that, e.g., black fillings overwrite this).
                  Contrary to \pkg{quantikz}, this also fills the \gate{cnot}s.
                  If you only want to fill certain operators, you have to selectively target them using their styles.
               \end{example}
               
               \begin{example}
                  \begin{codeexample*}
\begin{tikzpicture}
   \begin{yquant}[operators/every box/.append style={fill=white}]
      qubit {} j[3];
      box {$J_{12}$} (-j[1]);
      box {$J_{13}$} (j[0, 2]);
      box {$J_{23}$} (j[1]-);
   \end{yquant}
\end{tikzpicture}
                  \end{codeexample*}
                  \Yquant{} properly splits discontiguous multi\hyp qubit operations as of version~0.1.2.
               \end{example}
            
            \paragraph{B. Per-Gate Styling}\leavevmode
               \begin{example}
                  \begin{codeexample}
% \usetikzlibrary{quotes}
\begin{tikzpicture}
   \begin{yquant*}
      [fill=red!20, font=\color{cyan}]
      h a;
      [green]
      phase {[green]$\beta$} a;
      ["$\ket\pm$"]
      measure a;
      discard a;
   \end{yquant*}
\end{tikzpicture}
                  \end{codeexample}
               \end{example}
               
               \clearpage
               \begin{example}
                  \begin{codeexample*}
% \usetikzlibrary{shapes.symbols, fit}
\begin{tikzpicture}
   \begin{yquant}
      qubit {} data;
      qubit {$\ket0$} anc1[2];
      
      h data;
      cnot anc1 | data;
      [after=data]
      qubit {$\ket0$} anc2[2];
      [name=box, draw=none]
      box {\phantom{noise}} (data, anc1);
      cnot anc2[0] | data;
      cnot anc2 | anc1[0];
      cnot anc2[1] | anc1[1];
      measure anc2;
   \end{yquant}
   \node[starburst, cyan, fill=yellow, draw=red,
         line width=2pt, inner xsep=-4pt, inner ysep=-5pt, fit=(box)] {noise};
\end{tikzpicture}
                  \end{codeexample*}
                  \TikZ{} shapes cannot simply be used with \Yquant.
                  Any \Yquant{} shape must be aware of the keys \texttt{x radius} and \texttt{y radius} that control its width and height.
                  Additionally, \Yquant{} shapes must implement clipping paths.
                  Those objects, which are a \Yquant{} addition to \TikZ{} allow \Yquant{} to properly clip wires and vertical lines to the shape of the gate.
                  \Yquant{} draws its elements sequentially; hence, a wire that comes into an operator will be hidden by anything the operator draws on top of it; but outgoing wires will in turn draw on the operator (modulo clipping).
                  To avoid the issues, we construct an invisible box operator and name it; \emph{outside} of the \tex!yquant! environment, we \texttt{fit} the special \TikZ{} shape on top of it.
               \end{example}
               
            \paragraph{C. Boxing/Highlighting Parts of a Circuit}\leavevmode
               \begin{example}
                  \begin{codeexample*}
% \usetikzlibrary{quotes}
\begin{tikzpicture}
   \begin{yquant*}
      h a;
      cnot b | a;
      [name=left]
      h -;
      cnot b | a;
      [name=right]
      h -;
      cnot b|a;
      h b;
   \end{yquant*}
   \node[fit=(left-0) (left-1) (right-0) (right-1),
         draw, inner sep=6pt, "reversed c-\textsc{not}"] {};
\end{tikzpicture}
                  \end{codeexample*}
                  In this case, since version~0.2, a subcircuit may provide a similar experience:
                  \begin{codeexample*}
% \usetikzlibrary{quotes}
\begin{tikzpicture}
   \begin{yquant*}
      h a;
      cnot b | a;
      [this subcircuit box style={inner sep=6pt, "reversed c-\textsc{not}"}]
      subcircuit {
         qubit {} x;
         qubit {} y;
         h -;
         cnot y | x;
         h -;
      } (-);
      cnot b | a;
      h b;
   \end{yquant*}
\end{tikzpicture}
                  \end{codeexample*}
                  Here, we used the key \style{/yquant/this subcircuit box style} to influence only the style of the subcircuit box itself instead of providing global options that apply to every object in the subcircuit (you wouldn't want the label be assigned to every single gate).
               \end{example}
               
               \clearpage
               \begin{example}
                  \begin{codeexample*}
% \usetikzlibrary{quotes}
\begin{tikzpicture}
   \begin{yquant*}
      h a;
      [this subcircuit box style={draw, dashed, rounded corners, fill=blue!20, inner xsep=6pt, inner ysep=10pt, "\textsc{swap}" below}, register/default name=]
      subcircuit {
         qubit a;
         qubit b;
         cnot b | a;
         cnot a | b;
         cnot b | a;
      } (a-b);
      h b;
   \end{yquant*}
\end{tikzpicture}
                  \end{codeexample*}
                  Since version~0.2, fully enclosing a bunch of operations (with no controls extending to some inner component) is possible by means of \gate{subcircuit}s.
                  Before, this had to be done using named operations and layers.
                  Note that here we used the style \style{/yquant/this subcircuit box style} to assign a styling that only applies to the box containing the subcircuit, but not to the inner gates---which would have happened had we just given the arguments to the subcircuit directly.
               \end{example}
               
               \Yquant{} does not support the fancy nearest\hyp neighbor swap gate that \pkg{quantikz} has.
               It would however not be very difficult to implement this particular shape and make it available.
               Maybe even a multi\hyp swap gate using the \texttt{knots} library would be possible.
               
         \clearpage
         \subsubsection{VIII. Otherwise undocumented features}
            \begin{example}
               \begin{codeexample*}
% \usetikzlibrary{quantikz}
\begin{tikzpicture}
   \begin{yquant}[register/default name=]
      qubit a;
      [name=wave, register/minimum height=1cm]
      nobit wave;
      qubit b;
      qubit c;
      
      h a, b;
      box {$U$} c | a;
      [draw=none]
      box {$\dots$} a, b-;
      box {$U^k$} c | b;
      h a, b;
   \end{yquant}
   \node[wave, fit=(wave) (current bounding box.east |- wave), inner ysep=.5pt, inner xsep=0pt] {};
\end{tikzpicture}
               \end{codeexample*}
               Here, we included \pkg{quantikz}, which provides the \texttt{wave} shape, then introduced a register that will contain this wave (and enlarged it sufficiently).
               After the circuit is drawn, we \texttt{fit} the wave along.
               Since the name assigned to a register without any text actually is of a \texttt{coordinate} shape, we need to enlarge the height of the wave by providing a slightly increased \texttt{inner ysep}.
               Additionally, \pkg{quantikz} sets a negative \texttt{inner xsep}, which is probably required for its grid layout; but \Yquant{} positions exactly, so we also need to reset this.
            \end{example}
         
         \clearpage
         \subsubsection{X. Troubleshooting}
            \begin{example}
               \begin{codeexample*}
\begin{tikzpicture}
   \begin{yquant}
      qubit {$\ket0$} a[2];
      box {$\begin{pmatrix}
               \alpha & \beta \\
               \beta & -\alpha
            \end{pmatrix}$} a[0];
      cnot a[1] | a[0];
      box {$U_{\protect\the\numexpr\idx+1}$} a;
   \end{yquant}
\end{tikzpicture}
               \end{codeexample*}
            \end{example}
      \endgroup
   
   \section{Foreign language support}\label{sec:foreign}
      \Yquant{} is built in various modules, so that it is not hard to use the quantum circuit rendering backend, but expose a different language frontend.
      Since version~0.3, \Yquant{} not only understands its own language, but also others.
      
      \subsection{qasm}\label{sec:foreign:qasm}
         By saying \tex!\useyquantlanguage{qasm}! in the preamble after loading \Yquant{} itself, the parser for \pkg{qasm} (not OpenQASM) is loaded.
         It provides the environment \tex!qasm! as well as the macro \tex!\qasmimport!, which works similarly to \tex!\yquantimport! (but does not accept additional options).
         
         \subsubsection{Language specification}
            The \pkg{qasm} language is not formally defined, but an overview is provided at \href{https://web.archive.org/web/20050410022847/https://www.media.mit.edu/quanta/qasm2circ/#spec}{the archived website of \texttt{qasm2circ}}.
            The \Yquant{} implementation is designed to be compatible with the original parser, with the following exceptions:
            \begin{itemize}
               \item In \pkg{qasm}, lines could begin in an arbitrary manner; the first whitespace followed by the first valid command were then the instruction.
                  Contrary to this, \Yquant's parser always expects a line to start with a valid gate (preceded by arbitrary whitespaces), a comment, or to be empty.
               \item In \pkg{qasm}, user\hyp defined gates will be drawn in a box unless they contain the text \texttt{\textbackslash dmeter}, and they will be recognized as measurement gates if they contain \texttt{\textbackslash meter} or \texttt{\textbackslash dmeter}.
                  Contrary to this, \Yquant's parser expect the gates to \emph{start} with one of the \emph{macros} \tex!\meter!, \tex!\dmeter!, or \tex!\dmeterwide!.
                  Using these macros \emph{within} the content of a gate does not make sense from the point of view that in \Yquant, gates are nodes with shapes, so either the full gate has a particular shape or it does not, but not only parts of it.
               \item The \texttt{space} gate is supposed to produce a horizontal whitespace without a gate.
                  In \Yquant's implementation, you have to discard the wire if you want to reproduce this behavior; \texttt{space} and \texttt{nop} are equivalent.
            \end{itemize}
            
            The default \pkg{qasm} style defines several macros that can be used in gates.
            \Yquant{} makes \tex!\m! (matrix; requires \pkg{amsmath}) and \tex!\txt! (switch to text mode) available within the \pkg{qasm} environment.
            
            Do not expect \Yquant's output to match the one of \pkg{qasm} exactly.
            \Yquant{} is not grid based, so that commands such as \texttt{nop} don't even make sense.
            They are implemented for compatibility reasons and will produce a fixed horizontal space of the operator minimum width plus one separation, which might or might not be accurate.
            
            Note that whatever you write between \tex!\begin{qasm}! and \tex!\end{qasm}! is essentially treated as verbatim; only where the specification says so (in the definition of a new gate and in the optional third command to the register definition), it is interpreted as \TeX{} markup.
            Consequently, in \pkg{beamer}, any frame containing these environments must be given the \texttt{verbatim} option.
         
         \subsubsection{Configuration}
            Loading the \pkg{qasm} language interpreter will define several new configuration keys.
            For all the gates, it will use the keys defined in \cref{sec:config}, and it additionally provides the following:
            
            \begin{option}{operators/every s}!/yquant/operators/every box!
               This style is installed for every \texttt{s} operator.
            \end{option}
            
            \begin{option}{operators/every t}!/yquant/operators/every box!
               This style is installed for every \texttt{t} operator.
            \end{option}
            
            \begin{option}{operators/every utwo}!/yquant/operators/every box!
               This style is installed for every \texttt{Utwo} operator.
            \end{option}
            
            \begin{option*}{qasm}{zero}!\qasm@ket0!
               The content of this macro is used as the initialization content whenever the \texttt{zero} gate is invoked.
            \end{option*}
            
            \begin{option*}{qasm}{register/default qubit name}!\qasm@ket{#1}!
               This macro is invoked with a single parameter (the name of a qubit register) and gives back what is printed as the name of the register (will be in math mode automatically).
            \end{option*}
            
            \begin{option*}{qasm}{register/default qubit name value}!\qasm@ket{#1} = \qasm@ket{#2}!
               This macro is invoked with two parameters (the name of a qubit register and its initial value) and gives back what is printed as the name of the register (will be in math mode automatically).
            \end{option*}
         
         \subsubsection{Examples}
            The unaltered\footnote{Only in \texttt{test14.qasm}, \tex!\cal! was replaced by \tex!\symcal!---and this is only necessary as this manual was set with \pkg{unicode-math}. In traditional mode, even this would not be necessary.} \texttt{.qasm} files provided from \href{https://www.media.mit.edu/quanta/qasm2circ/}{the \texttt{qasm2circ} page} were stored in the subfolder \texttt{qasm} relative to this manual's \TeX{} file.
            The following command is then used to print all of them:
            \begin{minted}{tex}
% preamble:
% \usepackage{yquant,import}
% \useyquantlanguage{qasm}
\def\yquantimportpath{qasm/}
\foreach \circuitno in {1, ..., 18} {
   \paragraph{Circuit \#\circuitno}
      \begin{center}
         \qasmimport{test\circuitno.qasm}
      \end{center}
}
            \end{minted}
            
            \def\yquantimportpath{qasm/}
            \foreach \circuitno in {1, ..., 18} {
               \paragraph{Circuit \#\circuitno}\leavevmode\nopagebreak\par\nopagebreak
                  \begin{adjustbox}{center}
                     \qasmimport{test\circuitno.qasm}
                  \end{adjustbox}
            }
      
   \section{Wishlist}
      This section contains some thoughts on future improvements and features.
      \begin{itemize}
         \item Support for more other languages. \\
            Since version~0.3, \Yquant{} understands \pkg{qasm}.
            It would be nice if \Yquant{} could also understand \texttt{OpenQASM} correctly.
            The way \texttt{OpenQASM} treats operations is a bit different from \Yquant{} and also \pkg{qasm}.
            Some machinery would be required to automatically carry out transversal gates; apart from that everything should be available in \Yquant.
         \item Vertical layout. \\
            Sometimes, long quantum circuits on a portrait page can be better represented in a vertical layout.
            Also if lots of explanations are to be added, this becomes problematic in the horizontal version.
            In principle, \Yquant's approach could allow for a simple key switch that changes horizontal to vertical.
            Currently, this is largely unsupported by all quantum circuit packages except for \pkg{qpic}.
      \end{itemize}
   
   \section{Changelog}
      \subsection*{2020-03-15: Version 0.1}
         Initial release
      
      \subsection*{2020-03-22: Version 0.1.1}
         Complete rewrite of the register name parser.
         \Yquant{} now understands comma\hyp separated lists and ranges in indices, and also is far more tolerant with respect to whitespaces. \\
         \Yquant{} now also supports non\hyp contiguous vector registers and allows to add new registers into an already existing vector that is not the last register, and also in the unstarred mode.
      
      \subsection*{2020-04-11: Version 0.1.2}
         Introduce \gate{setstyle} and \gate{addstyle} pseudo\hyp gates that allow to style individual wires; rename \gate{setwire} to \gate{settype} (the old name is still available and shows a deprecation warning). \\
         Complete rewrite of the way \Yquant{} draws wires; projection anchors are removed in favor of clipping paths.
         This allows perfect connections between gates and wires, even if the (rather rectangular) wire lines meets with nonplanar shapes, while still preserving the possibility of transparent wires. \\
         \Yquant{} now also properly draws non\hyp contiguous multi\hyp qubit operations. \\
         New gate: \gate{correlate}.
         Various bug fixes.
      
      \subsection*{2020-06-02: Version 0.2}
         Introduce \gate{subcircuit}; required rewriting how \Yquant{} internally positions vertically.
         Provide simple macros to load circuits (or parts) from a file and to declare own custom gates.
      
      \subsection*{2020-06-07: Version 0.2.1}
         Introduce a macro to declare a lightweight custom gate, which is only a single box with custom content.
         
      \subsection*{2020-06-13: Version 0.3}
         Introduce support for the \pkg{qasm} language.
         
      \subsection*{2020-07-11: Version 0.3.1}
         Add legacy support for very old \TikZ{} versions such as the one used on the arXiv.
      
      \subsection*{2020-08-24: Version 0.3.2}
         Fix \href{https://github.com/projekter/yquant/issues/5}{\#5}: Can't draw circuits with more than 9 qubits.
      
      \subsection*{2020-10-27: Version 0.3.3}
         Fix \href{https://github.com/projekter/yquant/issues/6}{\#6}: shorten doesn't work for 2-qubit barriers. This fixes a bug in how the \texttt{shorten} keyword worked on \gate{barrier}s, which may require re\hyp assessing your chosen values.
%END_FOLD
\end{document}